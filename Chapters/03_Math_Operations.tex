\chapter{Binary Mathematics Operations}\label{ch03}
\section{Binary Addition}

Adding binary numbers is a simple task similar to the longhand addition of decimal numbers. As with decimal numbers, the bits are added one column at a time, from right to left. Unlike decimal addition, there is little to memorize in the way of an ``Addition Table,'' as seen in Table \ref{MO:tab:binary_addition_table}

\begin{table}[H]
  \sffamily
  \newcommand{\head}[1]{\textcolor{white}{\textbf{#1}}}    
  \begin{center}
    \rowcolors{2}{gray!10}{white} % Color every other line a light gray
    \begin{tabular}{ c c c | c c } \hline
      \multicolumn{3}{c|}{\textbf{Inputs}} & \multicolumn{2}{c}{\textbf{Outputs}} \\
      \hline
      \rowcolor{black!75}
      \head{Carry In} & \head{Augend} & \head{Addend} & \head{Sum} & \head{Carry Out} \\
      \hline
      0        & 0      & 0      & 0   & 0 \\
      0        & 0      & 1      & 1   & 0 \\
      0        & 1      & 0      & 1   & 0 \\
      0        & 1      & 1      & 0   & 1 \\
      1        & 0      & 0      & 1   & 0 \\
      1        & 0      & 1      & 0   & 1 \\
      1        & 1      & 0      & 0   & 1 \\
      1        & 1      & 1      & 1   & 1 \\ \hline
    \end{tabular}
  \end{center}
  \caption{Addition Table}
  \label{MO:tab:binary_addition_table}
\end{table} 

Just as with decimal addition, two binary integers are added one column at a time, starting from the \ac{LSB} (the right-most bit in the integer): 

\begin{binDisp}[commandchars=~\[\]]
      1001101
     +~underline[0010010]
      1011111
\end{binDisp}

When the sum in one column includes a carry out, it is added to the next column to the left (again, like decimal addition). Consider the following examples:

\begin{binDisp}[commandchars=~\[\]]
       11  1  <--Carry Bits
      1001001
     +~underline[0011001]
      1100010
\end{binDisp}

\begin{binDisp}[commandchars=~\[\], samepage=true]
         11   <--Carry Bits
      1000111
     +~underline[0010110]
      1011101
\end{binDisp}

The ``ripple-carry'' process is simple for humans to understand, but it causes a significant problem for designers of digital circuits. Consequently, ways were developed to carry a bit to the left in an electronic adder circuit and that is covered in Section \ref{CL:sec:adders_and_subtractors}, page \pageref{CL:sec:adders_and_subtractors}.

Binary numbers that include a fractional component are added just like binary integers; however, the radix points must align so the augend and addend may need to be padded with zeroes on either the left or the right. Here is an example: 

\begin{binDisp}[commandchars=~\[\]]
       111 1    <--Carry Bits
      1010.0100
     +~underline[0011.1101]
      1110.0001
\end{binDisp}

\subsection{Overflow Error}
\label{MO:sub:overflow_error}

One problem circuit designers must consider is a carry out bit in the \ac{MSB} (left-most bit) in the answer. Consider the following:

\begin{binDisp}[commandchars=~\[\]]
      11 11    <--Carry Bits
      10101110
     +~underline[11101101]
     110011011
\end{binDisp}

This example illustrates a significant problem for circuit designers. Suppose the above calculation was done with a circuit that could only accommodate eight data bits. The augend and addend are both eight bits wide, so they are fine; however, the sum is nine bits wide due to the carry out in the \ac{MSB}. In an eight-bit circuit (that is, a circuit where the devices and data lines can only accommodate eight bits of data), the carry out bit would be dropped since there is not enough room to accommodate it.

The result of a dropped bit cannot be ignored. The example problem above, when calculated in decimal, is $ 174_{10} + 237_{10} = 411_{10} $. If, though, the \ac{MSB} carry out is dropped, then the answer becomes $ 155_{10} $, which is, of course, incorrect. This type of error is called an \emph{Overflow Error}, and a circuit designer must find a way to correct overflow. One typical solution is to simply alert the user that there was an overflow error. For example, on a handheld calculator, the display may change to something like \emph{-E-} if there is an error of any sort, including overflow. 

\subsection{Sample Binary Addition Problems}
\label{MO:sub:sample_binary_addition_problems}
The following table lists several binary addition problems that can be used for practice.

\begin{table}[H]
  \sffamily
  \newcommand{\head}[1]{\textcolor{white}{\textbf{#1}}}    
  \begin{center}
    \rowcolors{2}{gray!10}{white} % Color every other line a light gray
    \begin{tabular}{ S S S }
      \hline
      \rowcolor{black!75}
      {\head{Augend}} & {\head{Addend}} & {\head{Sum}}     \\
      \hline
      10110.   & 11101.   & 110011.   \\ 
      111010.  & 110011.  & 1101101.  \\
      1011.    & 111000.  & 1000011.  \\ 
      1101001. & 11010.   & 10000011. \\ 
      1010.111 & 1100.001 & 10111.000 \\ 
      101.01   & 1001.001 & 1110.011  \\ 
      \hline
    \end{tabular}
  \end{center}
  \caption{Binary Addition Problems}
  \label{MO:tab:binary_addition_problems}
\end{table} 

\section{Binary Subtraction}
\label{MO:sec:binary_subtraction}
\subsection{Simple Manual Subtraction}
\label{MO:sub:simple_manual_subtraction}

Subtracting binary numbers is similar to subtracting decimal numbers and uses the same process children learn in primary school. The minuend and subtrahend are aligned on the radix point, and then columns are subtracted one at a time, starting with the least significant place and moving to the left. If the subtrahend is larger than the minuend for any one column, an amount is ``borrowed'' from the column to the immediate left. Binary numbers are subtracted in the same way, but it is important to keep in mind that binary numbers have only two possible values: zero and one. Consider the following problem: 

\begin{binDisp}[commandchars=~\[\]]
      10.1
     -~underline[01.0]
      01.1
\end{binDisp}

In this problem, the \ac{LSB} column is $ 1 - 0 $, and that equals one. The middle column, though, is $ 0 - 1 $, and one cannot be subtracted from zero. Therefore, one is borrowed from the most significant bit, so the problem in middle column becomes $ 10 - 1 $. (Note: do not think of this as ``ten minus one'' - remember that this is binary so this problem is ``one-zero minus one,'' or two minus one in decimal) The middle column is $ 10 - 1 = 1 $, and the \ac{MSB} column then becomes $ 0 - 0 = 0 $. 

The radix point must be kept in alignment throughout the problem, so if one of the two operands has too few places it is padded on the left or right (or both) to make both operands the same length. As an example, subtract: $ 101101.01 - 1110.1 $: 

\begin{binDisp}[commandchars=~\[\], samepage=true]
      101101.01
     -~underline[001110.10]
       11110.11
\end{binDisp}

There is no difference between decimal and binary as far as the subtraction process is concerned. In each of the problems in this section the minuend is greater than the subtrahend, leading to a positive difference; however, if the minuend is less than the subtrahend, the result is a negative number and negative numbers are developed in the next section of this chapter. 

Table \ref{MO:tab:binary_subtraction_problems} includes some subtraction problems for practice: 

\begin{table}[H]
  \sffamily
  \newcommand{\head}[1]{\textcolor{white}{\textbf{#1}}}    
  \begin{center}
    \rowcolors{2}{gray!10}{white} % Color every other line a light gray
    \begin{tabular}{ S S S }
      \hline
      \rowcolor{black!75}
      {\head{Minuend}} & {\head{Subtrahend}} & {\head{Difference}}     \\
      \hline
      1001011.      & 0111010.    & 10001.   \\ 
      100010.       & 010010.     & 10000.  \\
      101110110.    & 11001010.   & 10101100.  \\ 
      1110101.      & 111010.     & 111011. \\ 
      11011010.1101 & 101101.1    & 10101101.0101 \\ 
      10101101.1    & 1101101.101 & 111111.111  \\ 
      \hline
    \end{tabular}
  \end{center}
  \caption{Binary Subtraction Problems}
  \label{MO:tab:binary_subtraction_problems}
\end{table} 

\subsection{Representing Negative Binary Numbers Using Sign-and-Magnitude}
\label{MO:sub:representing_negative_sign_magnitude}

\marginpar{Sign-and-magnitude was used in early computers since it mimics real number arithmetic, but has been replaced by more efficient negative number systems in modern computers.} Binary numbers, like decimal numbers, can be both positive and negative. While there are several methods of representing negative binary numbers; one of the most intuitive is using \emph{sign-and-magnitude}, which is essentially the same as placing a ``–'' in front of a decimal number. With the sign-and-magnitude system, the circuit designer simply designates the \ac{MSB} as the \emph{sign bit} and all others as the magnitude of the number. When the sign bit is one the number is negative, and when it is zero the number is positive. Thus, $ -5_{10} $ would be written as $ 1101_2 $. 

Unfortunately, despite the simplicity of the sign-and-magnitude approach, it is not very practical for binary arithmetic, especially when done by a computer. For instance, negative five ($ 1101_2 $) cannot be added to any other binary number using standard addition technique since the sign bit would interfere. As a general rule, errors can easily occur when bits are used for any purpose other than standard place-weighted values; for example, $ 1101_2 $ could be misinterpreted as the number $ 13_{10} $ when, in fact, it is meant to represent $ -5 $. To keep things straight, the circuit designer must first decide how many bits are going to be used to represent the largest numbers in the circuit, add one more bit for the sign, and then be sure to never exceed that bit field length in arithmetic operations. For the above example, three data bits plus a sign bit would limit arithmetic operations to numbers from negative seven ($ 1111_2 $) to positive seven ($ 0111_2 $), and no more. 

This system also has the quaint property of having two values for zero. If using three magnitude bits, these two numbers are both zero: $ 0000_2 $ (positive zero) and $ 1000_2 $ (negative zero). 

\subsection{Representing Negative Binary Numbers Using Signed Complements }
\label{MO:sub:representing_negative_sign_complement}

\subsubsection{About Complementation}
\label{MO:subsub:about_complementation}

Before discussing negative binary numbers, it is important to understand the concept of complementation. To start, recall that the \emph{radix} (or base) of any number system is the number of ciphers available for counting; the decimal (or base-ten) number system has ten ciphers ($ 0, 1, 2, 3, 4, 5, 6, 7, 8, 9 $) while the binary (or base-two) number system has two ciphers ($ 0, 1 $). By definition, a number plus its complement equals the radix (this is frequently called the \emph{radix complement}). For example, in the decimal system four is the radix complement of six since $ 4 + 6 = 10 $. Another type of complement is the \emph{diminished radix complement}, which is the complement of the radix minus one. For example, in the decimal system six is the diminished radix complement of three since $ 6 + 3 = 9 $ and nine is equal to the radix minus one.

\paragraph{Decimal.} In the decimal system the radix complement is usually called the \emph{tens} complement since the radix of the decimal system is ten. Thus, the tens complement of eight is two since $ 8 + 2 = 10 $. The diminished radix complement is called the \emph{nines} complement in the decimal system. As an example, the nines complement of decimal eight is one since $ 8 + 1 = 9 $ and nine is the diminished radix of the decimal system. 

To find the nines complement for a number larger than one place, the nines complement must be found for each place in the number. For example, to find the nines complement for $ 538_{10} $, find the nines complement for each of those three digits, or $ 461 $. The easiest way to find the tens complement for a large decimal number is to first find the nines complement and then add one. For example, the tens complement of $ 283 $ is $ 717 $, which is calculated by finding the nines complement, $ 716 $, and then adding one.

\paragraph{Binary.} Since the radix for a binary number is $ 10_2 $, (be careful! this is not ten, it is one-zero in binary) the diminished radix is $ 1_2 $. The diminished radix complement is normally called the \emph{ones complement} and is obtained by reversing (or ``flipping'') each bit in a binary number; so the ones complement of $ 100101_2 $ is $ 011010_2 $. 

The radix complement (or \emph{twos complement}) of a binary number is found by first calculating the ones complement and then adding one to that number. The ones complement of $ 101101_2 $ is $ 010010_2 $, so the twos complement is $ 010010_2 + 1_2 = 010011_2 $.  

\subsubsection{Signed Complements} 
\label{MO:subsub:signed_complement}
In circuits that use binary mathematics, a circuit designer can opt to use ones complement for negative numbers and designate the most significant bit as the sign bit; and, if so, the other bits are the magnitude of the number. This is similar to the \emph{sign-and-magnitude} system discussed on page \pageref{MO:sub:representing_negative_sign_magnitude}. By definition, when using ones complement negative numbers, if the most significant bit is zero, then the number is positive and the magnitude of the number is determined by the remaining bits; but if the most significant bit is one, then the number is negative and the magnitude of the number is determined by calculating the ones complement of the number. Thus: $ 0111_2 = +7_{10} $, and $ 1000_2 = -7_{10} $ (the ones complement for $ 1000_2 $ is $ 0111_2 $). Table \ref{MO:tab:ones_complement} may help to clarify this concept: 

\begin{table}[H]
  \sffamily
  \newcommand{\head}[1]{\textcolor{white}{\textbf{#1}}}    
  \begin{center}
    \rowcolors{2}{gray!10}{white} % Color every other line a light gray
    \begin{tabular}{ S c c } 
      \hline
      \rowcolor{black!75}
      {\head{Decimal}} & {\head{Positive}} & {\head{Negative}} \\
      \hline
      0 & 0000 & 1111 \\
      1 & 0001 & 1110 \\
      2 & 0010 & 1101 \\
      3 & 0011 & 1100 \\
      4 & 0100 & 1011 \\
      5 & 0101 & 1010 \\
      6 & 0110 & 1001 \\
      7 & 0111 & 1000 \\
      \hline
    \end{tabular}
  \end{center}
  \caption{Ones Complement}
  \label{MO:tab:ones_complement}
\end{table} 

In a four-bit binary number, any decimal number from $ -7 $ to $ +7 $ can be represented; but, notice that, like the sign-and-magnitude system, there are two values for zero, one positive and one negative. This requires extra circuitry to test for both values of zero after subtraction operations.

To simplify circuit design, a designer can opt to use twos complement negative numbers and designate the most significant bit as the sign bit so the other bits are the number's magnitude. To use twos complement numbers, if the most significant bit is zero, then the number is positive and the magnitude of the number is determined by the remaining bits; but if the most significant bit is one, then the number is negative and the magnitude of the number is determined by taking the twos complement of the number (that is, the ones complement plus one). Thus: $ 0111 = 7 $, and $ 1001 = -7 $ (the ones complement of $ 1001 $ is $ 0110 $, and $ 0110 + 1 = 0111 $). Table \ref{MO:tab:twos_complement} may help to clarify this concept:

\begin{table}[H]
  \sffamily
  \newcommand{\head}[1]{\textcolor{white}{\textbf{#1}}}    
  \begin{center}
    \rowcolors{2}{gray!10}{white} % Color every other line a light gray
    \begin{tabular}{ S c c } 
      \hline
      \rowcolor{black!75}
      {\head{Decimal}} & {\head{Positive}} & {\head{Negative}} \\
      \hline
      0 & 0000 & 10000 \\
      1 & 0001 & 1111 \\
      2 & 0010 & 1110 \\
      3 & 0011 & 1101 \\
      4 & 0100 & 1100 \\
      5 & 0101 & 1011 \\
      6 & 0110 & 1010 \\
      7 & 0111 & 1001 \\
      8 & N/A & 1000 \\
      \hline
    \end{tabular}
  \end{center}
  \caption{Twos Complement}
  \label{MO:tab:twos_complement}
\end{table} 

The twos complement removes that quirk of having two values for zero. Table \ref{MO:tab:twos_complement} shows that zero is either $ 0000 $ or $ 10000 $; but since this is a four-bit number the initial one is discarded, leaving $ 0000 $ for zero whether the number is positive or negative. Also, $ 0000 $ is considered a positive number since the sign bit is zero. Finally, notice that $ 1000 $ is $ -8 $ (ones complement of $ 1000 $ is $ 0111 $, and $ 0111 + 1 = 1000 $). \marginpar{Programmers reading this book may have wondered why the maximum/minimum values for various types of variables is asymmetrical.}This means that binary number systems that use a twos complement method of designating negative numbers will be asymmetrical; running, for example, from $ -8 $ to $ +7 $. A twos complement system still has the same number of positive and negative numbers, but zero is considered positive, not neutral.

\marginpar{All modern computer systems use radix (or twos) complements to represent negative numbers.} One other quirk about the twos complement system is that the decimal value of the binary number can be quickly calculated by assuming the sign bit has a negative place value and all other places are added to it. For example, in the negative number $ 1010_2 $, if the sign bit is assumed to be worth $ -8 $ and the other places are added to that, the result is $ -8+2 $, or $ -6 $; and $ -6 $ is the value of $ 1010_2 $ in a twos complement system.

\subsubsection{About Calculating the Twos Complement}
\label{MO:subsub:about_calculating_twos_complement}

In the above section, the twos (or radix) complement is calculated by finding the ones complement of a number and then adding one. For machines, this is the most efficient method of calculating the twos complement; but there is a method that is much easier for humans to use to find the twos complement of a number. Start with the \ac{LSB} (the right-most bit) and then read the number from right to left. Look for the first one and then invert every bit to the left of that one. As an example, the twos complement for $ 1010\underline{10}_2 $ is formed by starting with the least significant bit (the zero on the right), and working to the left, looking for the first one, which is in the second place from the right. Then, every bit to the left of that one is inverted, ending with: $ 0101\underline{10}_2 $ (the two \acp{LSB} are underlined to show that they are the same in both the original and twos complement number).

Table \ref{MO:tab:example_twos_comp} displays a few examples:

\begin{table}[H]
  \sisetup{parse-numbers = false}
  \sffamily
  \newcommand{\head}[1]{\textcolor{white}{\textbf{#1}}}    
  \begin{center}
    \rowcolors{2}{gray!10}{white} % Color every other line a light gray
    \begin{tabular}{ r l } 
      \hline
      \rowcolor{black!75}
      {\head{Number}} & {\head{Twos Complement}} \\
      \hline
      0110100   & 1001100   \\
      11010     & 00110     \\
      001010    & 110110    \\
      1001011   & 0110101   \\
      111010111 & 000101001 \\
      \hline
    \end{tabular}
  \end{center}
  \caption{Example Twos Complement}
  \label{MO:tab:example_twos_comp}
\end{table} 

\subsection{Subtracting Using the Diminished Radix Complement }
\label{MO:sub:subtracting_using_diminished_radix}

When thinking about subtraction, it is helpful to remember that $ A - B $ is the same as $ A + (-B) $. Computers can find the complement of a particular number and add it to another number much faster and easier than attempting to create separate subtraction circuits. Therefore, subtraction is normally carried out by adding the complement of the subtrahend to the minuend.

\subsubsection{Decimal}
\label{MO:subsub:decimal_subtraction_with_diminished_radix}

\marginpar{This method is commonly used by stage performers who can subtract large numbers in their heads. While it seems somewhat convoluted, it is fairly easy to master.} It is possible to subtract two decimal numbers by adding the nines complement, as in the following example:

\begin{binDisp}[commandchars=~\[\]]
      735
     -~underline[142]
\end{binDisp}

Calculate the nines complement of the subtrahend: $ 857 $ (that is $ 9-1 $, $ 9-4 $, and $ 9-2 $). Then, add that nines complement to the minuend: 

\begin{binDisp}[commandchars=~\[\]]
      735
     +~underline[857]
     1592
\end{binDisp}

The initial one in the sum (the thousands place) is dropped so the number of places in the answer is the same as for the two addends, leaving $ 592 $. Because the diminished radix used to create the subtrahend is one less than the radix, one must be added to the answer; giving $ 593 $, which is the correct answer for $ 735-142 $.

\subsubsection{Binary}
\label{MO:subsub:binary_subtraction_with_diminished_radix}

The diminished radix complement (or ones complement) of a binary number is found by simply ``flipping'' each bit. Thus, the ones complement of $ 11010 $ is $ 00101 $. Just as in decimal, a binary number can be subtracted from another by adding the diminished radix complement of the subtrahend to the minuend, and then adding one to the sum. Here is an example: 

\begin{binDisp}[commandchars=~\[\]]
      101001
     -~underline[011011]
\end{binDisp}

Add the ones complement of the subtrahend:

\begin{binDisp}[commandchars=~\[\]]
      101001
     +~underline[100100]
     1001101
\end{binDisp}

The most significant bit is discarded so the solution has the same number of bits as for the two addends. This leaves $ 001101_2 $ and adding one to that number (because the diminished radix is one less than the radix) leaves $ 1110_2 $. In decimal, the problem is $ 41-27=14 $.

Often, diminished radix subtraction circuits are created such that they use \emph{end around} carry bits. In this case, the most significant bit is carried around and added to the final sum. If that bit is one, then that increases the final answer by one, and the answer is a positive number. If, though, the most significant bit is zero, then there is no end around carry so the answer is negative and must be complemented to find the true value. Either way, the correct answer is found. 

Here is an example:

\begin{binDisp}[commandchars=~\[\]]
      0110  (6)
     -~underline[0010  (2)]
\end{binDisp}

Solution:

\begin{binDisp}[commandchars=~\[\]]
      0110  (6)
     +~underline[1101  (-2 in ones complement)]
     10011
         1  (End-around carry the MSB)
     =0100  (4)
\end{binDisp}

Answer: 4 (since there was an end-around carry the solution is a positive number). Here is a second example:

\begin{binDisp}[commandchars=~\[\], samepage=true]
      0010  (2)
     -~underline[0110  (6)]
\end{binDisp}

Solution:

\begin{binDisp}[commandchars=~\[\], samepage=true]
      0010  (2)
     +~underline[1001  (-6 in ones complement)]
      1011  (No end-around carry, so ones complement)
     =0100  (-4: no end-around carry so negative answer)
\end{binDisp}

Because the diminished radix (or ones) complement of a binary number includes that awkward problem of having two representations for zero, this form of subtraction is not used in digital circuits; instead, the radix (or twos) complement is used (this process is discussed next). It is worth noting that subtracting by adding the diminished radix of the subtrahend and then adding one is awkward for humans, but complementing and adding is a snap for digital circuits. In fact, many early mechanical calculators used a system of adding complements rather than having to turn gears backwards for subtraction. 

\subsection{Subtracting Using the Radix Complement}
\label{MO:sub:subtracting_using_radix_complement}

\subsubsection{Decimal}
\label{MO:subsub:decimal_subtraction_with_radix_complement}

The radix (or tens) complement of a decimal number is the nines complement plus one. Thus, the tens complement of $ 7 $ is $ 3 $; or ($ (9-7)+1 $) and the tens complement of $ 248 $ is $ 752 $ (find the nines complement of each place and then add one to the complete number: $ 751 + 1 $). It is possible to subtract two decimal numbers using the tens complement, as in the following example: 

\begin{binDisp}[commandchars=~\[\]]
     735
    -~underline[142]
\end{binDisp}

Calculate the tens complement of the subtrahend, $ 142 $, by finding the nines complement for each digit and then adding one to the complete number: $ 858 $ (that is $ 9-1 $, $ 9-4 $, and $ 9-2+1 $). Then, add that tens complement number to the original minuend: 

\begin{binDisp}[commandchars=~\[\]]
     735
    +~underline[858]
    1593
\end{binDisp}

The initial one in the answer (the thousands place) is dropped so the answer has the same number of decimal places as the addends, leaving $ 593 $, which is the correct answer for $ 735-142 $.

\subsubsection{Binary}
\label{MO:subsub:binary_subtraction_with_radix_complement}

To find the radix (or twos) complement of a binary number, each bit in the number is ``flipped'' (making the ones complement) and then one is added to the result. Thus, the twos complement of $ 11010_2 $ is $ 00110_2 $ (or $ (00101_2)+1_2 $). Just as in decimal, a binary number can be subtracted from another by adding the radix complement of the subtrahend to the minuend. Here's an example: 

\begin{binDisp}[commandchars=~\[\]]
     101001
    -~underline[011011]
\end{binDisp}

Add the twos complement of the subtrahend:

\begin{binDisp}[commandchars=~\[\]]
     101001
    +~underline[011011]
    1001110
\end{binDisp}

The most significant bit is discarded so the solution has the same number of bits as for the two addends. This leaves $ 001110_2 $ (or $ 14_{10} $). Converting all of this to decimal, the original problem is $ 41-27=14 $.

Here are two worked out examples:

Calculate $ 0110_2 - 0010_2 $ (or $ 6_{10} - 2_{10} $):

\begin{binDisp}[commandchars=~\[\]]
     0110  (6)
    -~underline[0010  (2)]
\end{binDisp}

Solution:

\begin{binDisp}[commandchars=~\[\]]
     0110  (6)
    +~underline[1110  (-2 in twos complement)]
    10100  (Discard the MSB, the answer is 4)
\end{binDisp}

Calculate $ 0010_2 - 0110_2 $ (or $ 2_{10} - 6_{10} $)

\begin{binDisp}[commandchars=~\[\]]
     0010  (2)
    -~underline[0110  (6)]
\end{binDisp}

Solution:

\begin{binDisp}[commandchars=~\[\]]
     0010  (2)
    +~underline[1010  (-6 in twos complement)]
     1100
     0100  (Twos complement of the sum, -4)
\end{binDisp}

\subsection{Overflow}
\label{MO:sub:overflow}

One caveat with signed binary numbers is that of overflow, where the answer to an addition or subtraction problem exceeds the magnitude which can be represented with the allotted number of bits. Remember that the sign bit is defined as the most significant bit in the number. For example, with a six-bit number, five bits are used for magnitude, so there is a range from $ 00000_2 $ to $ 11111_2 $, or $ 0_{10} $ to $ 31_{10} $. If a sign bit is included, and using twos complement, numbers as high as $ 011111_2 $ ($ +31_{10} $) or as low as $ 100000_2 $ ($ -32_{10} $) are possible. However, an addition problem with two signed six-bit numbers that results in a sum greater than $ +31_{10} $ or less than $ -32_{10} $ will yield an incorrect answer. As an example, add $ 17_{10} $ and $ 19_{10} $ with signed six-bit numbers: 

\begin{binDisp}[commandchars=~\[\]]
     010001  (17)
    +~underline[010011  (19)]
     100100
\end{binDisp}

The answer ($ 100100_2 $), interpreted with the most significant bit as a sign, is equal to $ -28_{10} $, not $ +36_{10} $ as expected. Obviously, this is not correct. The problem lies in the restrictions of a six-bit number field. Since the true sum ($ 36 $) exceeds the allowable limit for our designated bit field (five magnitude bits, or $ +31 $), it produces what is called an overflow error. Simply put, six places is not large enough to represent the correct sum if the \ac{MSB} is being used as a sign bit, so whatever sum is obtained will be incorrect. A similar error will occur if two negative numbers are added together to produce a sum that is too small for a six-bit binary field. As an example, add $ -17_{10} $ and $ -19_{10} $: 

\begin{binDisp}[commandchars=~\[\]]
     -17 = 101111
     -19 = 101101

     101111  (-17)
    +~underline[101101  (-19)]
    1011100
\end{binDisp}

The solution as shown: $ 011100_2 $ = $ +28_{10} $. (Remember that the most significant bit is dropped in order for the answer to have the same number of places as the two addends.) The calculated (incorrect) answer for this addition problem is $ 28 $ because true sum of $ -17 + -19 $ was too small to be properly represented with a five bit magnitude field.

Here is the same overflow problem again, but expanding the bit field to six magnitude bits plus a seventh sign bit. In the following example, both $ 17 + 19 $ and $ (-17) + (-19) $ are calculated to show that both can be solved using a seven-bit field rather than six-bits. 

Add 17 + 19:

\begin{binDisp}[commandchars=~\[\], samepage=true]
     0010001  (17)
    +~underline[0010011  (19)]
     0100100  (36)
\end{binDisp}

Add (-17) + (-19):

\begin{binDisp}[commandchars=~\[\], samepage=true]
     -17 = 1101111
     -19 = 1101101

     1101111  (-17)
    +~underline[1101101  (-19)]
    11011100  (-36)
\end{binDisp}

The correct answer is only found by using bit fields sufficiently large to handle the magnitude and sign bits in the sum.

\subsubsection{Error Detection}
\label{MO:subsub:error_detection}

Overflow errors in the above problems were detected by checking the problem in decimal form and then comparing the results with the binary answers calculated. For example, when adding $ +17 $ and $ +19 $, the answer was supposed to be $ +36 $, so when the binary sum was $ -28 $, something had to be wrong. Although this is a valid way of detecting overflow errors, it is not very efficient, especially for computers. After all, the whole idea is to reliably add binary numbers together and not have to double-check the result by adding the same numbers together in decimal form. This is especially true when building logic circuits to add binary quantities: the circuit must detect an overflow error without the supervision of a human who already knows the correct answer.

The simplest way to detect overflow errors is to check the sign of the sum and compare it to the signs of the addends. Obviously, two positive numbers added together will give a positive sum and two negative numbers added together will give a negative sum. With an overflow error, however, the sign of the sum is always opposite that of the two addends: $ (+17) + (+19) = -28 $ and $ (-17) + (-19) = +28 $. By checking the sign bits an overflow error can be detected. 

It is not possible to generate an overflow error when the two addends have opposite signs. The reason for this is apparent when the nature of overflow is considered. Overflow occurs when the magnitude of a number exceeds the range allowed by the size of the bit field. If a positive number is added to a negative number then the sum will always be closer to zero than either of the two added numbers; its magnitude must be less than the magnitude of either original number, so overflow is impossible.

\section{Binary Multiplication}
\label{MO:sec:binary_multiplication}

\subsection{Multiplying Unsigned Numbers}
\label{MO:sub:multiplying_unsigned_numbers}

Multiplying binary numbers is very similar to multiplying decimal numbers. There are only four entries in the Binary Multiplication Table:

\begin{table}[H]
  \sffamily
  \newcommand{\head}[1]{\textcolor{white}{\textbf{#1}}}    
  \begin{center}
    \rowcolors{1}{gray!10}{white} % Color every other line a light gray
    \begin{tabular}{ c } 
      $ 0X0=0 $   \\
      $ 0X1=0 $   \\
      $ 1X0=0 $   \\
      $ 1X1=1 $
    \end{tabular}
  \end{center}
  \caption{Binary Multiplication Table}
  \label{MO:tab:binary_multiplication}
\end{table} 

To multiply two binary numbers, work through the multiplier one number at a time (right-to-left) and if that number is one, then shift left and copy the multiplicand as a partial product; if that number is zero, then shift left but do not copy the multiplicand (zeros can be used as placeholders if desired). When the multiplying is completed add all partial products. This sounds much more complicated than it actually is in practice and is the same process that is used to multiply two decimal numbers. Here is an example problem.

\begin{binDisp}[commandchars=~\[\]]
        1011  (11)
      X ~underline[1101  (13)]
        1011
       0000
      1011
     ~underline[1011          ]
    10001111  (143)
\end{binDisp}

\subsection{Multiplying Signed Numbers}
\label{MO:sub:multiplying_signed_numbers}

The simplest method used to multiply two numbers where one or both are negative is to use the same technique that is used for decimal numbers: multiply the two numbers and then determine the sign from the signs of the original numbers: if those signs are the same then the result is positive, if they are different then the result is negative. Multiplication by zero is a special case where the result is always zero.

The multiplication method discussed above works fine for paper-and-pencil; but is not appropriate for designing binary circuits. Unfortunately, the mathematics for binary multiplication using an algorithm that can become an electronic circuit is beyond the scope of this book. Fortunately, though, \acp{IC} already exist that carry out multiplication of both signed and floating-point numbers, so a circuit designer can use a pre-designed circuit and not worry about the complexity of the multiplication process.

\section{Binary Division}
\label{MO:sec:binary_division}

Binary division is accomplished by repeated subtraction and a right shift function; the reverse of multiplication. The actual process is rather convoluted and complex and is not covered in this book. Fortunately, though, \acp{IC} already exist that carry out division of both signed and floating-point numbers, so a circuit designer can use a pre-designed circuit and not worry about the complexity of the division process.

\section{Bitwise Operations}
\label{MO:sec:bitwise_operations}

It is sometimes desirable to find the value of a given bit in a byte. For example, if the \ac{LSB} is zero then the number is even, but if it is one then the number is odd. To determine the ``evenness'' of a number, a bitwise mask is multiplied with the original number. As an example, imagine that it is desired to know if $ 1001010_2 $ is even, then:

\begin{binDisp}[commandchars=~\[\]]
         1001010  <- Original Number
    BitX ~underline[0000001]  <- "Evenness" Mask
         0000000
\end{binDisp}

The bits are multiplied one position at a time, from left-to-right. Any time a zero appears in the mask that bit position in the product will be zero since any number multiplied by zero yields zero. When a one appears in the mask, then the bit in the original number will be copied to the solution. In the given example, the zero in the least significant bit of the top number is multiplied with one and the result is zero. If that \ac{LSB} in the top number had been one then the \ac{LSB} in the result would have also been one. Therefore, an ``even'' original number would yield a result of all zeros while an odd number would yield a one.

\section{Codes}
\label{MO:sec:codes}

\subsection{Introduction}
\label{MO:sub:codes_introduction}

Codes are nothing more than using one system of symbols to represent another system of symbols or information. Humans have used codes to encrypt secret information from ancient times. However, digital logic codes have nothing to do with secrets; rather, they are only concerned with the efficient storage, retrieval, and use of information. 

\subsubsection{Morse Code}
\label{MO:subsub:morse_code}

As an example of a familiar code, Morse code changes letters to electric pulses that can be easily transmitted over a radio or telegraph wire. Samuel Morse's code uses a series of dots and dashes to represent letters so an operator at one end of the wire can use electromagnetic pulses to send a message to some receiver at a distant end. Most people are familiar with at least one phrase in Morse code: \emph{SOS}. Here is a short sentence in Morse: 

\begin{binDisp}[commandchars=~\[\]]
  -.-. --- -.. . ...   .- .-. .   ..-. ..- -.
   c    o   d  e  s    a   r  e    f    u  n
\end{binDisp}

\subsubsection{Braille Alphabet}
\label{MO:subsub:braille_alphabet}

As one other example of a commonly-used code, in $ 1834 $ Louis Braille, at the age of $ 15 $, created a code of raised dots that enable blind people to read books. For those interested in this code, the Braille alphabet can be found at \url{http://braillebug.afb.org/braille_print.asp}. 

\subsection{Computer Codes}
\label{MO:sub:computer_codes}

The fact is, computers can only work with binary numbers; that is how information is stored in memory, how it is processed by the \ac{CPU}, how it is transmitted over a network, and how it is manipulated in any of a hundred different ways. It all boils down to binary numbers. However, humans generally want a computer to work with words (such as email or a word processor), ciphers (such as a spreadsheet), or graphics (such as photos). All of that information must be encoded into binary numbers for the computer and then decoded back into understandable information for humans. Thus, binary numbers stored in a computer are often codes used to represent letters, programming steps, or other non-numeric information. 

\subsubsection{ASCII}
\label{MO:subsub:ascii}

Computers must be able to store and process letters, like those on this page. At first, it would seem easiest to create a code by simply making \lstinline[columns=fixed]|A=1|, \lstinline[columns=fixed]|B=2|, and so forth. While this simple code does not work for a number of reasons, the idea is on the right track and the code that is actually used for letters is similar to this simple example. 

In the early $ 1960 $s, computer scientists came up with a code they named \ac{ASCII} and this is still among the most common ways to encode letters and other symbols for a computer. If the computer program knows that a particular spot in memory contains binary numbers that are actually ASCII-coded letters, it is a fairly easy job to convert those codes to letters for a screen display. For simplicity, \ac{ASCII} is usually represented by hexadecimal numbers rather than binary. For example, the word \emph{Hello} in \ac{ASCII} is: $ 048 \; 065 \; 06C \; 06C \; 06F $.

\ac{ASCII} code also has a predictable relationship between letters. For example, capital letters are exactly $ 20_{16} $ higher in ASCII than their lower-case version. Thus, to change a letter from lower-case to upper-case, a programmer can add $ 20_{16} $ to the \ac{ASCII} code for the lower-case letter. This can be done in a single processing step by using what is known as a \emph{bit-wise \textsf{AND}} on the bit representing $ 20_{16} $ in the \ac{ASCII} code's binary number.

An \ac{ASCII} chart using hexadecimal values is presented in Table \ref{MO:tab:ascii_table}. The most significant digit is read across the top row and the least significant digit is read down the left column. For example, the letter \emph{A} is $ 41_{16} $ and the number \emph{6} is $ 36_{16} $.

\begin{table}[H]
  \sisetup{parse-numbers = false}
  \sffamily
  \newcommand{\head}[1]{\textcolor{white}{\textbf{#1}}}    
  \begin{center}
    \rowcolors{2}{gray!10}{white} % Color every other line a light gray
    \begin{tabular}{ c c c c c c c c c } 
      \hline
      \rowcolor{black!75}
      & {\head{0}} & {\head{1}} & {\head{2}} & {\head{3}}
      & {\head{4}} & {\head{5}} & {\head{6}} & {\head{7}} \\
      \hline  
      \cellcolor{black!75}\head{0} & {NUL} & {DLE} & {} & {0} & {@} & 
      {P} & {'} & {p} 
      \\
      \cellcolor{black!75}\head{1} & {SOH} & {DC1} & {!} & {1} & {A} & 
      {Q} & {a} & {q} 
      \\
      \cellcolor{black!75}\head{2} & {STX} & {DC2} & {''} & {2} & {B} & 
      {R} & {b} & {r} 
      \\
      \cellcolor{black!75}\head{3} & {ETX} & {DC3} & {\#} & {3} & {C} & 
      {S} & {c} & {s} 
      \\
      \cellcolor{black!75}\head{4} & {EOT} & {DC4} & {\$} & {4} & {D} & 
      {T} & {d} & {t} 
      \\
      \cellcolor{black!75}\head{5} & {ENQ} & {NAK} & {\%} & {5} & {E} & 
      {U} & {e} & {u} 
      \\
      \cellcolor{black!75}\head{6} & {ACK} & {SYN} & {\&} & {6} & {F} & 
      {V} & {f} & {v} 
      \\
      \cellcolor{black!75}\head{7} & {BEL} & {ETB} & {'} & {7} & {G} & 
      {W} & {g} & {w} 
      \\
      \cellcolor{black!75}\head{8} & {BS} & {CAN} & {(} & {8} & {H} & 
      {X} & {h} & {x}
      \\
      \cellcolor{black!75}\head{9} & {HT} & {EM} & {)} & {9} & {I} & 
      {Y} & {i} & {y} 
      \\
      \cellcolor{black!75}\head{A} & {LF} & {SUB} & {*} & {:} & {J} & 
      {Z} & {j} & {z} 
      \\
      \cellcolor{black!75}\head{B} & {VT} & {ESC} & {+} & {;} & {K} & 
      $ [ $ & {k} & \{ 
      \\
      \cellcolor{black!75}\head{C} & {FF} & {FS} & {,} & $ < $ & {L} & 
      \textbackslash & {l} & {$ \arrowvert $} 
      \\
      \cellcolor{black!75}\head{D} & {CR} & {GS} & {-} & {=} & {M} & 
      $ ] $ & {m} & {\}} 
      \\
      \cellcolor{black!75}\head{E} & {SO} & {RS} & {.} & $ > $ & {N} & 
      $ \wedge $ & {n} & $ \sim $ 
      \\
      \cellcolor{black!75}\head{F} & {SI} & {US} & {/} & {?} & {O} & 
      {\_} & {o} & {DEL}       
      \\
      \hline
    \end{tabular}
  \end{center}
  \caption{ASCII Table}
  \label{MO:tab:ascii_table}
\end{table}

\marginpar{Teletype operators from decades past tell stories of sending 25 or more $ 07_{16} $ codes (ring the bell) to a receiving terminal just to irritate another operator in the middle of the night.} \ac{ASCII} $ 20_{16} $ is a space character used to separate words in a message and the first two columns of ASCII codes (where the high-order nibble are zero and one) were codes essential for teletype machines, which were common from the $ 1920 $s until the $ 1970 $s. The meanings of a few of those special codes are given in Table \ref{MO:tab:ascii_symbols}.

\begin{table}[H]
  \sisetup{parse-numbers = false}
  \sffamily
  \newcommand{\head}[1]{\textcolor{white}{\textbf{#1}}}    
  \begin{center}
    \rowcolors{1}{gray!10}{white} % Color every other line a light gray
    \begin{tabular}{ c l } 
      \hline \hline % The gray in the first row overwrites the first hline - so do it twice.
      {NUL} & {All Zeros (a ``null'' byte)} \\
      {SOH} & {Start of Header} \\
      {STX} & {Start of Text} \\
      {ETX} & {End of Text} \\
      {EOT} & {End of Transmission} \\
      {ENQ} & {Enquire (is the remote station on?)} \\
      {ACK} & {Acknowledge (the station is on)} \\
      {BEL} & {Ring the terminal bell (get the operator's attention)} \\
      \hline  
    \end{tabular}
  \end{center}
  \caption{ASCII Symbols}
  \label{MO:tab:ascii_symbols}
\end{table}

Table \ref{MO:tab:ascii_practice} contains a few phrases in both plain text and ASCII for practice.

\begin{table}[H]
  \sisetup{parse-numbers = false}
  \sffamily
  \newcommand{\head}[1]{\textcolor{white}{\textbf{#1}}}    
  \begin{center}
    \rowcolors{2}{gray!10}{white} % Color every other line a light gray
    \begin{tabular}{ c l } 
      \hline
      \rowcolor{black!75}
      {\head{Plain Text}} & {\head{ASCII}} \\
      \hline    
      {codes are fun} & {63 6f 64 65 73 20 61 72 65 20 66 75 6e} \\
      {This is ASCII} & {54 68 69 73 20 69 73 20 41 53 43 49 49} \\
      {365.25 days} & {33 36 35 2e 32 35 20 64 61 79 73} \\
      {It's a gr8 day!} & {49 74 27 73 20 61 20 67 72 38 20 64 61 79 21} \\
      \hline  
    \end{tabular}
  \end{center}
  \caption{ASCII Practice}
  \label{MO:tab:ascii_practice}
\end{table}
% Pull Quote - Marginal Note - Sidebar

While the ASCII code is the most commonly used text representation, it is certainly not the only way to encode words. Another popular code is \ac{EBCDIC} (pronounced like ``Eb See Deck''), which was invented by IBM in $ 1963 $ and has been used in most of their computers ever since.

Since the early 2000's, computer programs have begun to use Unicode character sets, which are similar to ASCII but multiple bytes are combined to expand the number of characters available for non-English languages like Cyrillic.

\subsubsection{Binary Coded Decimal (BCD)}
\label{MO:subsub:binary_coded_decimal}

It is often desirable to have numbers coded in such a way that they can be easily translated back and forth between decimal (which is easy for humans to manipulate) and binary (which is easy for computers to manipulate). \ac{BCD} is the code used to represent decimal numbers in binary systems. \ac{BCD} is useful when working with decimal input (keypads or transducers) and output (displays) devices.

There are, in general, two types of \ac{BCD} systems: non-weighted and weighted. Non-weighted codes are special codes devised for a single purpose where there is no implied relationship between one value and the next. As an example, $ 1001 $ could mean one and $ 1100 $ could mean two in some device. The circuit designer would create whatever code meaning is desired for the application. 

Weighted \ac{BCD} is a more generalized system where each bit position is assigned a ``weight,'' or value. These types of \ac{BCD} systems are far more common than non-weighted and are found in all sorts of applications. Weighted \ac{BCD} codes can be converted to decimal by adding the place value for each position in exactly the same way that Expanded Positional Notation is used for to covert between decimal and binary numbers. As an example, the weights for the Natural \ac{BCD} system are $ 8-4-2-1 $. (These are the same weights used for binary numbers; thus the name ``natural'' for this system.) The code $ 1001_{BCD} $ is converted to decimal like this: 

\begin{align}
  1001_{BCD} &= (1X8)+(0X4)+(0X2)+(1X1) \\
  \nonumber
  &= (8)+(0)+(0)+(1) \\
  \nonumber
  &= 9_{10}
\end{align}

Because there are ten decimal ciphers ($ 0, 1, 2, 3, 4, 5, 6, 7, 8, 9 $), it requires four bits to represent all decimal digits; so most \ac{BCD} code systems are four bits wide. In practice, only a few different weighted \ac{BCD} code systems are commonly used and the most common are shown in Table \ref{MO:tab:bcd_systems}. 

\begin{table}[H]
  \sffamily
  \newcommand{\head}[1]{\textcolor{white}{\textbf{#1}}}    
  \begin{center}
    \rowcolors{2}{gray!10}{white} % Color every other line a light gray
    \begin{tabular}{ c c c c c } 
      \hline
      \rowcolor{black!75}
      {\head{Decimal}} & {\head{8421 (Natural)}} & \head{2421} 
      & \head{Ex3} & \head{5421} \\
      \hline    
      0 & 0000 & 0000 & 0011 & 0000 \\
      1 & 0001 & 0001 & 0100 & 0001 \\
      2 & 0010 & 0010 & 0101 & 0010 \\
      3 & 0011 & 0011 & 0110 & 0011 \\
      4 & 0100 & 0100 & 0111 & 0100 \\
      5 & 0101 & 1011 & 1000 & 1000 \\
      6 & 0110 & 1100 & 1001 & 1001 \\
      7 & 0111 & 1101 & 1010 & 1010 \\
      8 & 1000 & 1110 & 1011 & 1011 \\
      9 & 1001 & 1111 & 1100 & 1100 \\
      \hline  
    \end{tabular}
  \end{center}
  \caption{BCD Systems}
  \label{MO:tab:bcd_systems}
\end{table}

\marginpar{Remember that BCD is a code system, not a number system; so the meaning of each combination of four-bit codes is up to the designer and will not necessarily follow any sort of binary numbering sequence.}The name of each type of \ac{BCD} code indicates the various place values. Thus, the $ 2421 $ \ac{BCD} system gives the most significant bit of the number a value of two, not eight as in the natural code. The \emph{Ex3} code (for ``Excess 3'') is the same as the natural code, but each value is increased by three (that is, three is added to the natural code). 

In each of the \ac{BCD} code systems in Table \ref{MO:tab:bcd_systems} there are six unused four-bit combinations; for example, in the \emph{Natural} system the unused codes are: $ 1010, 1011, 1100, 1101, 1110, $ and $ 1111 $. Thus, any circuit designed to use \ac{BCD} must include some sort of check to ensure that if unused binary values are accidentally input into a circuit it does not create an undefined outcome.

Normally, two \ac{BCD} codes, each of which are four bits wide, are packed into an eight-bit byte in order to reduce wasted computer memory. Thus, the packed \ac{BCD} $ 0111 0010 $ contains two BCD numbers: $ 72 $. In fact, a single 32-bit word, which is common in many computers, can contain $ 8 $ \ac{BCD} codes. It is a trivial matter for software to either pack or unpack \ac{BCD} codes from a longer word.

It is natural to wonder why there are so many different ways to code decimal numbers. Each of the \ac{BCD} systems shown in Table \ref{MO:tab:bcd_systems} has certain strengths and weaknesses and a circuit designer would choose a specific system based upon those characteristics.

\paragraph{Converting between BCD and Other Systems.} One thing that makes \ac{BCD} so useful is the ease of converting from \ac{BCD} to decimal. Each decimal digit is converted into a four-bit \ac{BCD} code, one at a time. Here is $ 37_{10} $ in Natural \ac{BCD}:

\begin{binDisp}
     0011 0111
       3    7
\end{binDisp}

It is, generally, very easy to convert Natural \ac{BCD} to decimal since the \ac{BCD} codes are the same as binary numbers. Other \ac{BCD} systems use different place values, and those require more thought to convert (though the process is the same). The place values for \ac{BCD} systems other than Natural are indicated in the name of the system; so, for example, the $ 5421 $ system would interpret the number $ 1001_{BCD5421} $ as:

\begin{align}
  1001_{BCD5421} &= (1X5)+(0X4)+(0X2)+(1X1) \\
  \nonumber
  &= (5)+(0)+(0)+(1) \\
  \nonumber
  &= 6_{10}
\end{align}

Converting from decimal to \ac{BCD} is also a rather simple process. Each decimal digit is converted to a four-bit \ac{BCD} equivalent. In the case of Natural \ac{BCD} the four-bit code is the binary equivalent to the decimal number, other weighted \ac{BCD} codes would be converted with a similar process.

\begin{binDisp}
      2    4    5
    0010 0100 0101
\end{binDisp}

%TODO If codes are separated from this chapter then the Double-Dabble system could be explained in some detail.

To convert binary to \ac{BCD} is no trivial exercise and is best done with an automated process. The normal method used is called the \emph{Shift Left and Add Three} algorithm (or, frequently, \emph{Double-Dabble}). The process involves a number of steps where the binary number is shifted left and occasionally three is added to the resulting shift. Wikipedia (\url{https://en.wikipedia.org/wiki/Double_dabble}) has a good explanation of this process, along with some examples.

Converting \ac{BCD} to any other system (like hexadecimal) is most easily done by first converting to binary and then to whatever base is desired. Unfortunately, converting \ac{BCD} to binary is not as simple as concatenating two \ac{BCD} numbers; for example, $ 0100 0001 $ is 41 in \ac{BCD}, but those two \ac{BCD} numbers concatenated, $ 01000001 $, is $ 65 $ in binary. One way to approach this type of problem is to use the reverse of the \emph{Double-Dabble} process: \emph{Shift Right and Subtract Three}. As in converting binary to \ac{BCD}, this is most easily handled by an automated process.

\paragraph{Self-Complementing.} The Excess-3 code (called \emph{Ex3} in the table) is self-complementing; that is, the nines complement of any decimal number is found by complementing each bit in the Ex3 code. As an example, find the nines complement for $ 127_{10} $:

\begin{table}[H]
  \sisetup{parse-numbers = false}
  \sffamily
  \newcommand{\head}[1]{\textcolor{white}{\textbf{#1}}}    
  \begin{center}
    %\rowcolors{2}{gray!10}{white} % Color every other line a light gray
    \begin{tabular}{ c | c l } 
      \hline
      1 & $ 127_{10} $ & {Original Number} \\
      2 & $ 0100 \; 0101 \; 1010_{Ex3} $ & {Convert 127 to Excess 3} \\
      3 & $ 1011 \; 1010 \; 0101_{Ex3} $ & {Ones Complement} \\
      4 & $ 872_{10} $ & {Convert to Decimal} \\
      \hline  
    \end{tabular}
  \end{center}
  \caption{Nines Complement for 127}
  \label{MO:tab:nines_complement}
\end{table}

Thus, $ 872_{10} $, is the nines complement of $ 127_{10} $. It is a powerful feature to be able to find the nines complement of a decimal number by simply complementing each bit of its Ex3 \ac{BCD} representation. 

\paragraph{Reflexive.} Some \ac{BCD} codes exhibit a reflexive property where each of the upper five codes are complementary reflections of the lower five codes. For example, $ 0111_{Ex3} $ (4) and $ 1000_{Ex3} $ (5) are complements, $ 0110_{Ex3} $ (3) and $ 1001_{Ex3} $ (6) are complements, and so forth. The reflexive property for the $ 5421 $ code is different. Notice that the codes for zero through four are the same as those for five through nine, except for the \ac{MSB} (zero for the lower codes, one for the upper codes). Thus, $ 0000_{5421} $ (zero) is the same as $ 1000_{5421} $ (five) except for the first bit, $ 0001_{5421} $ (one) is the same as $ 1001_{5421} $ (six) except for the first bit, and so forth. Studying Table \ref{MO:tab:bcd_systems} should reveal the various reflexive patterns found in these codes.

\paragraph{Practice.} Table \ref{MO:tab:bcd_practice} shows several decimal numbers in various \ac{BCD} systems which can be used for practice in converting between these number systems.

\begin{table}[H]
  \sffamily
  \newcommand{\head}[1]{\textcolor{white}{\textbf{#1}}}    
  \begin{center}
    \rowcolors{2}{gray!10}{white} % Color every other line a light gray
    {\small         
      \begin{tabular}{ c c c c c } 
        \hline
        \rowcolor{black!75}
        {\head{Dec}} & {\head{8421}} & \head{2421} 
        & \head{Ex3} & \head{5421} \\
        \hline    
        57  & 0101 1110      & 10111 1101     
            & 1000 1010      & 1000 1010 \\
        79  & 0111 1001      & 1101 1111      
            & 1010 1100      & 1010 1100 \\
        
        28  & 0010 1000      & 0010 1110      
            & 0101 1011      & 0010 1011 \\
        421 & 0100 0010 0001 & 0100 0010 0001 
            & 0111 0101 0100 & 0100 0010 0001 \\
        
        903 & 1001 0000 0011 & 1111 0000 0011 
            & 1100 0011 0110 & 1100 0000 0011 \\
        \hline  
      \end{tabular}
    }  % End small font size
  \end{center}
  \caption{BCD Practice}
  \label{MO:tab:bcd_practice}
\end{table}

\paragraph{Adding BCD Numbers.} \ac{BCD} numbers can be added in either of two ways. Probably the simplest is to convert the \ac{BCD} numbers to binary, add them as binary numbers, and then convert the sum back to \ac{BCD}. However, it is possible to add two \ac{BCD} numbers without converting. When two \ac{BCD} numbers are added such that the result is less than ten, then the addition is the same as for binary numbers:

\begin{binDisp}[commandchars=~\[\]]
     0101  (5)
    +~underline[0010  (2)]
     0111  (7)
\end{binDisp}

However, four-bit binary numbers greater than $ 1001_2 $ (that is: $ 9_{10} $) are invalid \ac{BCD} codes, so adding two \ac{BCD} numbers where the result is greater than nine requires a bit more effort:

\begin{binDisp}[commandchars=~\[\]]
     0111  (7)
    +~underline[0101  (5)]
     1100  (12 -- not valid in BCD)
\end{binDisp}

When the sum is greater than nine, then six must be added to that result since there are six invalid binary codes in \ac{BCD}.

\begin{binDisp}[commandchars=~\[\]]
     1100  (12 -- from previous addition)
    +~underline[0110  (6)]
   1 0010  (12 in BCD)
\end{binDisp}

When adding two-digit \ac{BCD} numbers, start with the \ac{LSN}, the right-most nibble, then add the nibbles with carry bits from the right. Here are some examples to help clarify this concept:

\begin{binDisp}[commandchars=~\[\], samepage=true]
     0101 0010  (52)
    +~underline[0011 0110  (36)]
     1000 1000  (88 in BCD)
\end{binDisp}

\begin{binDisp}[commandchars=~\[\], samepage=true]
     0101 0010  (52)
    +~underline[0101 0110  (56)]
     1010 1000  (1010, MSN, invalid BCD code)
    +~underline[0110 0000  (Add 6 to invalid code)]
   1 0000 1000  (108 in BCD)
\end{binDisp}

\begin{binDisp}[commandchars=~\[\], samepage=true]
     0101 0101  (55)
    +~underline[0101 0110  (56)]
     1010 1011  (both nibbles invalid BCD code)
    +~underline[0000 0110  (Add 6 to LSN)]
     1011 0001  (1 carried over to MSN)
    +~underline[0110 0000  (Add 6 to MSN)]
   1 0001 0001  (111 in BSD)
\end{binDisp}

\paragraph{Negative Numbers.} \ac{BCD} codes do not have any way to store negative numbers, so a sign nibble must be used. One approach to this problem is to use a sign-and-magnitude value where a sign nibble is prefixed onto the \ac{BCD} value. By convention, a sign nibble of $ 0000 $ makes the \ac{BCD} number positive while $ 1001 $ makes it negative. Thus, the \ac{BCD} number $ 0000 0010 0111 $ is $ 27 $, but $ 1001 0010 0111 $ is $ -27 $. 

A more mathematically rigorous, and useful, method of indicating negative \ac{BCD} numbers is use the tens complement of the \ac{BCD} number since \ac{BCD} is a code for decimal numbers, exactly like the twos complement is used for binary numbers. It may be useful to review Section \ref{MO:subsub:about_complementation} on page \pageref{MO:subsub:about_complementation} for information about the tens complement. In the \emph{Natural BCD} system the tens complement is found by adding one to the nines complement, which is found by subtracting each digit of the original \ac{BCD} number from nine. Here are some examples to clarify this concept:

\begin{binDisp}[commandchars=~\[\]]
    0111 (7 in BCD)
    0010 (2, the 9's complement of 7 since 9-7=2)
    0011 (3, the 10's complement of 7, or 2+1)
\end{binDisp}

\begin{binDisp}[commandchars=~\[\]]
    0010 0100 (24 in BCD)
    0111 0101 (75, the 9's complement of 24)
    0111 0110 (76, the 10's complement of 24)
\end{binDisp}

Also, some \ac{BCD} code systems are designed to easily create the tens complement of a number. For example, in the $ 2421 $ \ac{BCD} system the tens complement is found by nothing more than inverting the \ac{MSB}. Thus, three is the tens complement of seven and in the $ 2421 $ \ac{BCD} system $ 0011_{BCD2421} $ is the tens complement of $ 1101_{BCD2421} $, a difference of only the \ac{MSB}. Therefore, designers creating circuits that must work with negative \ac{BCD} numbers may opt to use the $ 2421 $ \ac{BCD} system.

\paragraph{Subtracting BCD Numbers.} A \ac{BCD} number can be subtracted from another by changing it to a negative number and adding. Just like in decimal, $ 5 - 2 $ is the same as $ 5 + (-2) $. Either a nines or tens complement can be used to change a \ac{BCD} number to its negative, but for this book, the tens complement will be used. If there is a carry-out bit then it can be ignored and the result is positive, but if there is no carry-out bit then answer is negative so the magnitude must be found by finding the tens complement of the calculated sum. Compare this process with subtracting regular binary numbers. Here are a few examples:

\begin{minipage}{\linewidth} % This keeps the block on the same page
\begin{binDisp}[commandchars=~\[\]]

    7 - 3 = 4

          0111  (7 in BCD)
         +~underline[0111  (add the 10's complement of 3)]
          1110  (invalid BCD code)
         +~underline[0110  (add 6 to invalid BCD code)]
        1 0100  (4 - drop the carry bit)

\end{binDisp}
\end{minipage}

\begin{minipage}{\linewidth} % This keeps the block on the same page
\begin{binDisp}[commandchars=~\[\]]

    7 - 9 = -2

          0111  (7 in BCD)
         +~underline[0001  (10's complement of 9)]
          1000  (valid BCD code)
          0010  (10's complement)

\end{binDisp}
\end{minipage}

\begin{minipage}{\linewidth} % This keeps the block on the same page
\begin{binDisp}[commandchars=~\[\]]

    32 - 15 = 17

          0011 0010  (32 in BCD)
         +~underline[1000 0101  (10's complement of 15)]
          1011 0111  (MSB is invalid BCD code)
         +~underline[0110 0000  (add 6 to MSB)]
        1 0001 0111  (17, drop the carry bit)

\end{binDisp}
\end{minipage}

\begin{minipage}{\linewidth} % This keeps the block on the same page
\begin{binDisp}[commandchars=~\[\]]

    427 - 640 = -213

          0100 0010 0111  (427 in BCD)
         +~underline[0011 0110 0000  (10's complement of 640)]
          0111 1000 0111  (no invalid BCD code)
          0010 0001 0011  (10's complement)

\end{binDisp}
\end{minipage}

\begin{minipage}{\linewidth} % This keeps the block on the same page

\begin{binDisp}[commandchars=~\[\]]

    369 - 532 = -163

          0011 0110 1001  (369 in BCD)
         +~underline[0100 0110 1000  (10's complement of 532)]
          0111 1100 0001  (two invalid BCD codes)
         +~underline[0000 0110 0110  (add 6 to invalid codes)]
          1000 0011 0111  (sum)
          0001 0110 0011  (10's complement)

\end{binDisp}
\end{minipage}

Here are some notes on the last example: adding the \ac{LSN} yields $ 1001+1000=10001 $. The initial one is ignored, but this is an invalid \ac{BCD} code so this byte needs to be corrected by adding six to it. Then the result of that addition includes an understood carry into the next nibble after the correction is applied. In the same way, the middle nibble was corrected: $ 1100+0110=10011 $ but the initial one in this answer is carried to the \ac{MSN} and added there.

\subsubsection{Gray Code}
\label{MO:subsub:gray_code}

% Draw a Gray Code Wheel
% Cite: http://tex.stackexchange.com/questions/56176/handling-of-wrapfig-pictures-in-latex
\begin{wrapfigure}{r}{0.3\textwidth}
  \caption{Optical Disc}
  \label{MO:fig:gray_code_disc}
  \centering

  \xdef\IntRad{2}
  \xdef\Rad{.5}

  \newcommand{\Sector}[2][]{%
    \draw[#1] (22.5:#2) arc (22.5:0:#2)
    --(#2+\Rad,0) arc (0:22.5:#2+\Rad)
    -- cycle ;
  }

\begin{tikzpicture}[scale=0.40]

  \foreach \Loop [count=\j from 0] in {%
    {white,,,white,white,,,white,white,,,white,white,,,white},
    {,,white,white,white,white,,,,,white,white,white,white,,},
    {white,white,white,white,white,white,white,white,,,,,,,,},
    {white,white,white,white,,,,,,,,,white,white,white,white}}
  {\foreach \col [count=\i from 0] in \Loop {%
      \begin{scope}[rotate={22.5*\i}]
      \Sector[fill=\col]{\IntRad+\j*\Rad} ;   
      \end{scope}
    }
  }

  \begin{scope}[rotate=11.75]
  \draw[fill=white] (\IntRad-.2,-.5*\Rad) rectangle (\IntRad+.2+4*\Rad,.5*\Rad) ;

  \foreach \col [count=\i from 0, evaluate=\i as \j using 0.5+\i]
  in {white,black,white,white} {%
    \draw[fill=\col] (\IntRad+\j*\Rad,0) circle (.25*\Rad) ;
  }
  \end{scope}
\end{tikzpicture}
\end{wrapfigure}

It is often desirable to use a wheel to encode digital input for a circuit. As an example, consider the tuning knob on a radio. The knob is attached to a shaft that has a small, clear disk which is etched with a code, similar to Figure \ref{MO:fig:gray_code_disc}. As the disk turns, the etched patterns pass or block a laser beam from reaching an optical sensor, and that pass/block pattern is encoded into binary input. 

One of the most challenging aspects of using a mechanical device to encode binary is ensuring that the input is stable. As the wheel turns past the light beam, if two of the etched areas change at the same time (thus, changing two bits at once), it is certain that the input will fluctuate between those two values for a tiny, but significant, period of time. For example, imagine that the encoded circuit changes from $ 1111 $ to $ 0000 $ at one time. Since it is impossible to create a mechanical wheel precise enough to change those bits at exactly the same moment in time (remember that the light sensors will ``see'' an input several million times a second), as the bits change from $ 1111 $ to $ 0000 $ they may also change to $ 1000 $ or $ 0100 $ or any of dozens of other possible combinations for a few microseconds. The entire change may form a pattern like $ 1111-0110-0010-0000 $ and that instability would be enough to create havoc in a digital circuit. 

The solution to the stability problem is to etch the disk with a code designed in such a way that only one bit changes at a time. The code used for that task is the Gray code. Additionally, a Gray code is cyclic, so when it reaches its maximum value it can cycle back to its minimum value by changing only a single bit. In Figure \ref{MO:fig:gray_code_disc}, each of the concentric rings encodes one bit in a four-bit number. Imagine that the disk is rotating past the fixed laser beam reader \textemdash the black areas (``blocked light beam'') change only one bit at a time, which is characteristic of a Gray code pattern.

It is fairly easy to create a Gray code from scratch. Start by writing two bits, a zero and one: 

\begin{binDisp}[commandchars=~\[\]]
     0
     1
\end{binDisp}

Then, reflect those bits by writing them in reverse order underneath the original bits:

\begin{binDisp}[commandchars=~\[\]]
     0
     1
   -----
     1
     0
\end{binDisp}

Next, prefix the top half of the group with a zero and the bottom half with a one to get a two-bit Gray code. 

\begin{binDisp}
     00
     01
     11
     10  (2-bit Gray code)
\end{binDisp}

Now, reflect all four values of the two-bit Gray code.

\begin{binDisp}
     00
     01
     11
     10
   ------
     10
     11
     01
     00
\end{binDisp}

Next, prefix the top half of the group with a zero and the bottom half with a one to get a three-bit Gray code.

\begin{binDisp}
     000
     001
     011
     010
     110
     111
     101
     100  (3-bit Gray code)
\end{binDisp}

Now, reflect all eight values of a three-bit Gray code.

\begin{binDisp}
     000
     001
     011
     010
     110
     111
     101
     100
   -------
     100
     101
     111
     110
     010
     011
     001
     000
\end{binDisp}

Finally, prefix the top half of the group with a zero and the bottom half with a one to get a four-bit Gray code. 

\begin{binDisp}
     0000
     0001
     0011
     0010
     0110
     0111
     0101
     0100
     1100
     1101
     1111
     1110
     1010
     1011
     1001
     1000  (four-bit Gray code)
\end{binDisp}

The process of reflecting and prefixing can continue indefinitely to create a Gray code of any desired bit length. Of course, Gray code tables are also available in many different bit lengths. Table \ref{MO:tab:gray_codes} contains a two-bit, three-bit, and four-bit Gray code:

\begin{table}[H]
  \sffamily
  \newcommand{\head}[1]{\textcolor{white}{\textbf{#1}}}    
  \begin{center}
    \rowcolors{2}{gray!10}{white} % Color every other line a light gray
    \begin{tabular}{ c c c } 
      \hline
      \rowcolor{black!75}
      {\head{2-Bit Code}} & {\head{3-Bit Code}} & \head{4-Bit Code} \\ 
      \hline    
      00  & 000 & 0000 \\     
      01  & 001 & 0001 \\     
      11  & 011 & 0011 \\     
      10  & 010 & 0010 \\     
      & 110 & 0110 \\     
      & 111 & 0111 \\     
      & 101 & 0101 \\     
      & 100 & 0100 \\     
      &     & 1100 \\     
      &     & 1101 \\     
      &     & 1111 \\     
      &     & 1110 \\     
      &     & 1010 \\     
      &     & 1011 \\     
      &     & 1001 \\     
      &     & 1000 \\     
      \hline  
    \end{tabular}
  \end{center}
  \caption{Gray Codes}
  \label{MO:tab:gray_codes}
\end{table}

