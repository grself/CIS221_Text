%*******************************************************
% Declaration
%*******************************************************
\refstepcounter{dummy}
\pdfbookmark[0]{Style Guide}{styleguide}
\chapter*{Style Guide}
\thispagestyle{empty}

\begin{itemize}

  \item \textsc{Numbers}. 
  
  \begin{itemize}
    \item All numbers must be enclosed in a math block.
    \item Spell out numbers up to ten; thus, not ``1s,'' but ones.
    \item Larger numbers with a suffix, like ``s,'' do not use an apostrophe: not ten's, but tens.
  \end{itemize}

  \item \textsc{Variables}. 
  
  \begin{itemize}
    \item \textsc{Single-Digit}. Single digit variable names, like ``Q,'' are enclosed in a math block.
    \item \textsc{Long}. Long variable names, like \emph{i\_clk}, are enclosed in an \textbackslash emph\{\} block.
  \end{itemize}

  \item \textsc{Figures and Tables}.
  
  \begin{itemize}
    \item Figures and Tables use an [H] specification
    \item Captions go at the end of the table block so it is printed under the table to match figures and listings. (Figures and Listings place the caption under by default)
  \end{itemize}
  
  \item \textsc{Inline Expressions}. Use \lstinline[columns=fixed]|\lstinline[columns=fixed]|foo|| for in-line Boolean expressions, equations, and Verilog commands.

  \item \textsc{Verilog Commands}. Use all caps and  \lstinline[columns=fixed]|\lstinline[columns=fixed]|foo|| for Verilog commands (like CASE).

  \item \textsc{Lists}. Items in a list should have an introductory phrase using title capitalization and set in small caps. If a long variable name or Verilog command starts a list item then use the formatting appropriate for that type of item rather than a list item.
  
  \item \textsc{Gates}. Use San Seriff font and ALL CAPS for gates (like \textsf{AND} or \textsf{OR})
  
  \item \textsc{True/False}. The words \emph{True} and \emph{False} are capitalized and in an \lstinline[columns=fixed]|\emph{}| block.

\item The word: \textit{Logisim-evolution}
\item Inputs/Outputs/Single Bits: \emph{D2In} (Ctrl-Sh-E)
\item Ports: input data ports are labeled with a single letter from the beginning of the alphabet, output data ports are labeled with a single letter from the end of the alphabet, and other signals use descriptive labels
\item ICs and Gates: \textsf{OR} (Ctrl-Sh-A)
\item The word ``Flip-Flop'' is hyphenated and each part is capitalized in a title.


\end{itemize}

\section{Captions and Labels}
%  All lables are the same as the captions, but lowercase and no punctuation. 
%  Also, all spaces are filled with underscores.
%  Occasionally, labels may also need a bit of verbiage to differentiate 
%  them from each other, for example, there will be lots of ``example'' 
%  sections. In that case, put additional verbiage at the end of the label.
%  Labels are referenced book-wide, not just chapter-wide. So prefix 
%  the 2-digit chapter number at the start of the label (Appendicies 
%  will use ``ap''). Also, use the following label designations:
%	ch: -- chapter
%	sec: -- section
%	subsec: -- sub-section
%	para: -- paragraph
%	tab: -- table
%	eq: -- equation
%	fig: -- figure
%	lst: -- code listing
%	soln: -- the solution to an equation
%	A full label would look something like: \label{03:tab:truth_table_for_or}
%  The number of degrees for label position within an image node are:
%	E=0
%	NE=45
%	N=90
%	NW=135
%	W=180
%	SW=225
%	S=270
%	SE=315

