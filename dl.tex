% ****************************************************************
% Digital Logic
% ****************************************************************
\RequirePackage{fix-cm} % fix some latex issues see: http://texdoc.net/texmf-dist/doc/latex/base/fixltx2e.pdf
\documentclass[ twoside,openright,titlepage,numbers=noenddot,headinclude,
                footinclude=true,cleardoublepage=empty,abstractoff, 
                BCOR=5mm,fontsize=11pt,american]{scrreprt}

% ****************************************************************
% Note: Make all your adjustments in here
% ****************************************************************
%*****
% classicthesis-config.tex 
%*****

\PassOptionsToPackage{utf8}{inputenc}
	\usepackage{inputenc}

%*****
% 1. Configure classicthesis
%*****

\PassOptionsToPackage{eulerchapternumbers,listings,%drafting,%
					 pdfspacing,floatperchapter,%linedheaders,%
					 subfig,beramono,eulermath,parts}{classicthesis}                                        
%*****
% Available options for classicthesis.sty 
% (see ClassicThesis.pdf for more information):
% drafting
% parts nochapters linedheaders
% eulerchapternumbers beramono eulermath pdfspacing minionprospacing
% tocaligned dottedtoc manychapters
% listings floatperchapter subfig
%*****

%*****
% 2. Personal data and user ad-hoc commands
%*****
\newcommand{\myTitle}{Exploring Digital Logic\xspace}
\newcommand{\mySubtitle}{With Logisim-Evolution}
\newcommand{\myDegree}{}
\newcommand{\myName}{George Self\xspace}
\newcommand{\myProf}{}
\newcommand{\myOtherProf}{}
\newcommand{\mySupervisor}{}
\newcommand{\myFaculty}{}
\newcommand{\myDepartment}{Computer Information Systems\xspace}
\newcommand{\myUni}{Cochise College\xspace}
\newcommand{\myLocation}{Arizona\xspace}
\newcommand{\myTime}{July 2018\xspace}
\newcommand{\myVersion}{Edition 6.0\xspace}

%*****
% Setup, finetuning, and useful commands
%*****
\newcounter{dummy} % necessary for correct hyperlinks (to index, bib, etc.)
\newlength{\abcd} % for ab..z string length calculation
\providecommand{\mLyX}{L\kern-.1667em\lower.25em\hbox{Y}\kern-.125emX\@}
\newcommand{\ie}{i.\,e.}
\newcommand{\Ie}{I.\,e.}
\newcommand{\eg}{e.\,g.}
\newcommand{\Eg}{E.\,g.}
\newcommand{\blankpage}{ % Create a blank page at the end of the document
  \newpage
  \thispagestyle{empty}
  \mbox{}
  \newpage
  }
% The following creates a function named maxwidth that permits
% me to set a maximum width for images. If the natural width of
% the image is less than maxwidth then the image is rendered at
% its natural size, else scaled to maxwidth.
\makeatletter
\def\maxwidth#1{\ifdim\Gin@nat@width>#1 #1\else\Gin@nat@width\fi}
\makeatother

\newcommand{\Le}{\textit{Logisim-evolution }}


%*****
% 3. Loading some handy packages
%*****

%*****
% Packages with options that might require adjustments
%*****

% sets up various typographic and hyphenation rules
\PassOptionsToPackage{american}{babel}
\usepackage{babel}                  

% manage both inline and block quotes
\usepackage{csquotes}

% I don't use a bibliography 
%\PassOptionsToPackage{%
%  %backend=biber, %instead of bibtex
%	backend=bibtex8,bibencoding=ascii,%
%	language=auto,%
%	style=numeric-comp,%
%    %style=authoryear-comp, % Author 1999, 2010
%    %bibstyle=authoryear,dashed=false, % dashed: substitute rep. author with ---
%    sorting=nyt, % name, year, title
%    maxbibnames=10, % default: 3, et al.
%    %backref=true,%
%    natbib=true % natbib compatibility mode (\citep and \citet still work)
%}{biblatex}
%    \usepackage{biblatex}

\PassOptionsToPackage{fleqn}{amsmath}  % math environments
    \usepackage{amsmath}

%*****
% General useful packages
%*****

%\PassOptionsToPackage{T1}{fontenc} % T2A for cyrillics
%    \usepackage{fontenc}     
\usepackage{textcomp} % fix warning with missing font shapes
\usepackage{scrhack} % fix warnings when using KOMA with listings package
\usepackage{xspace} % to get the spacing after macros right  
\usepackage{mparhack} % get marginpar right
\usepackage{fixltx2e} % fixes some LaTeX stuff 
\PassOptionsToPackage{printonlyused,smaller}{acronym} 
    \usepackage{acronym} % nice macros for handling all acronyms in the thesis
    \renewcommand*{\aclabelfont}[1]{\acsfont{#1}}
\usepackage[paperheight=11in,paperwidth=8.5in]{geometry}
\usepackage{shorttoc} % generate brief version of the table of contents
%*****

%*****
% 4. Setup floats: tables, (sub)figures, and captions
%*****

\usepackage{tabularx} % better tables
    \setlength{\extrarowheight}{3pt} % increase table row height
\newcommand{\tableheadline}[1]{\multicolumn{1}{c}{\spacedlowsmallcaps{#1}}}
\newcommand{\myfloatalign}{\centering} 
\usepackage{caption}
\captionsetup{font=small} % format=hang,
\usepackage{subfig}  
%*****

%*****
% 5. Setup code listings - formatted for Verilog listings
%*****

\usepackage{listings} 
%\lstset{emph={trueIndex,root},emphstyle=\color{BlueViolet}}%\underbar} % for special keywords
\lstset{language=Verilog,%[LaTeX]Tex,
    morekeywords={PassOptionsToPackage,selectlanguage},
    keywordstyle=\color{RoyalBlue},%\bfseries,
    basicstyle=\small\ttfamily,
    %identifierstyle=\color{NavyBlue},
    commentstyle=\color{Green}\ttfamily,
    stringstyle=\rmfamily,
    numbers=left,
    numberstyle=\scriptsize,%\tiny
    stepnumber=2,
    numbersep=8pt,
    showstringspaces=false,
    breaklines=true,
    %frameround=ftff,
    frame=lines,
    captionpos=b,  % put captions at the bottom of the listing
    aboveskip=.75\baselineskip,
    belowskip=.75\baselineskip
    %abovecaptionskip=.75\baselineskip
    %belowcaptionskip=.75\baselineskip
    %frame=L
} 
%*****

%*****
% 6. PDFLaTeX, hyperreferences and citation backreferences
%*****

%*****
% Using PDFLaTeX
%*****
\PassOptionsToPackage{pdftex,hyperfootnotes=false,pdfpagelabels}{hyperref}
    \usepackage{hyperref}  % backref linktocpage pagebackref
\pdfcompresslevel=9
\pdfadjustspacing=1 
\PassOptionsToPackage{pdftex}{graphicx}
    \usepackage{graphicx} 
 
%*****
% Hyperreferences
%*****

\hypersetup{%
    %draft, % = no hyperlinking at all (useful in b/w printouts)
    colorlinks=true, linktocpage=true, pdfstartpage=3, pdfstartview=FitV,%
    % uncomment the following line if you want to have black links (e.g., for printing)
    %colorlinks=false, linktocpage=false, pdfstartpage=3, pdfstartview=FitV, pdfborder={0 0 0},%
    breaklinks=true, pdfpagemode=UseNone, pageanchor=true, pdfpagemode=UseOutlines,%
    plainpages=false, bookmarksnumbered, bookmarksopen=true, bookmarksopenlevel=1,%
    hypertexnames=true, pdfhighlight=/O,%nesting=true,%frenchlinks,%
    urlcolor=webbrown, linkcolor=RoyalBlue, citecolor=webgreen, %pagecolor=RoyalBlue,%
    %urlcolor=Black, linkcolor=Black, citecolor=Black, %pagecolor=Black,%
    pdftitle={\myTitle},%
    pdfauthor={\textcopyright\ \myName, \myUni, \myFaculty},%
    pdfsubject={},%
    pdfkeywords={},%
    pdfcreator={pdfLaTeX},%
    pdfproducer={LaTeX with hyperref and classicthesis}%
}   

%*****
% Setup autoreferences
%*****

\makeatletter
\@ifpackageloaded{babel}%
    {%
       \addto\extrasamerican{%
			\renewcommand*{\figureautorefname}{Figure}%
			\renewcommand*{\tableautorefname}{Table}%
			\renewcommand*{\partautorefname}{Part}%
			\renewcommand*{\chapterautorefname}{Chapter}%
			\renewcommand*{\sectionautorefname}{Section}%
			\renewcommand*{\subsectionautorefname}{Section}%
			\renewcommand*{\subsubsectionautorefname}{Section}%     
                }%
       \addto\extrasngerman{% 
			\renewcommand*{\paragraphautorefname}{Absatz}%
			\renewcommand*{\subparagraphautorefname}{Unterabsatz}%
			\renewcommand*{\footnoteautorefname}{Fu\"snote}%
			\renewcommand*{\FancyVerbLineautorefname}{Zeile}%
			\renewcommand*{\theoremautorefname}{Theorem}%
			\renewcommand*{\appendixautorefname}{Anhang}%
			\renewcommand*{\equationautorefname}{Gleichung}%        
			\renewcommand*{\itemautorefname}{Punkt}%
                }%  
            \providecommand{\subfigureautorefname}{\figureautorefname}%             
    }{\relax}
\makeatother

%*****
% Setup drawing environment
%*****

\usepackage[svgnames,table]{xcolor}
\usepackage{tikz}
\usetikzlibrary{circuits.logic.US,circuits.logic.IEC,circuits.ee.IEC,shapes.geometric}
\usepackage{circuitikz}
\usepackage{tikz-timing} % Timing Diagrams
\usetikztiminglibrary[new={char=Q,reset char=R}]{counters}
\usepackage{rotating} % Rotate an image
\usetikzlibrary{calc} % Do some math in tikz

%*****
% This enables enumerated lists using first, second, etc.
%*****
\usepackage{moreenum}

%*****
% Used to create framed paragraphs, like "interest boxes"
%*****
\usepackage{tcolorbox}

%*****
% Used to enable strike-through text
%*****
\usepackage[normalem]{ulem}

%*****
% The Creative Commons License Package
%*****
\usepackage[type={CC},modifier={zero},version={1.0}]{doclicense}

%*****
% This package permits me to align a table column on a decimal point
%*****
\usepackage{siunitx}

%*****
% Enable certain special chars in verbatim, like underlines
%*****
\usepackage{fancyvrb}

% define specific fancyvrb environments for use in my document
\DefineVerbatimEnvironment
{binDisp}{Verbatim} % Display binary math problems, no line numbering
{fontfamily=courier}

\DefineVerbatimEnvironment
{lineDisp}{Verbatim} % Display verbatim with line numbering
{fontfamily=courier,
 numbers=left}

%*****
% To draw 7-segment displays
%*****
\usepackage{SevenSeg} 

%*****
% Used for wrapping figures
%*****
\usepackage{float}
\usepackage{wrapfig}
\restylefloat{figure}

%*****
% Used for merging cells in tables, creating a slash in a cell, 
% adjusting the width of a table to fit the text column
%*****
\usepackage{multicol}
\usepackage{multirow}
\usepackage{slashbox}
\usepackage{adjustbox}

%*****
% I used this to just find the width of a line in cm (so I can create
% appropriately sized graphics). Load the package here and then copy the
% other line in a separate paragraph where you need the size displayed.
%*****
\usepackage{layouts}
%textwidth in cm: \printinunitsof{cm}\prntlen{\textwidth}

%*****
% 7. Last calls before the bar closes
%*****

%*****
% Development Stuff
%*****
\listfiles
%\PassOptionsToPackage{l2tabu,orthodox,abort}{nag}
%   \usepackage{nag}
%\PassOptionsToPackage{warning, all}{onlyamsmath}
%   \usepackage{onlyamsmath}

%*****
% Last, but not least...
%*****
\usepackage{classicthesis} 


% ****************************************************************
% Begin Document
% ****************************************************************

\begin{document}
\frenchspacing
\raggedbottom
\selectlanguage{american}
\pagenumbering{roman}
\pagestyle{plain}

% ****************************************************************
% Frontmatter
% ****************************************************************

%\include{FrontBackmatter/DirtyTitlepage}
%*******************************************************
% Titlepage
%*******************************************************
\begin{titlepage}
% Original Title Page code follows...
    % if you want the titlepage to be centered, uncomment and fine-tune the line below (KOMA classes environment)
    \begin{addmargin}[-1cm]{-3cm}
    \begin{center}
        \Large  

        \hfill

        \vfill

        \begingroup
            \color{Maroon}\spacedallcaps{\myTitle} \\ \bigskip
            \color{Maroon}\emph{\mySubtitle} \\ \bigskip
        \endgroup

        \spacedlowsmallcaps{\myName}

        \vfill

%        \includegraphics[width=6cm]{gfx/TFZsuperellipse_bw} \\ \medskip

%        \mySubtitle \\ \medskip   
        %\myDegree \\
        %\myDepartment \\                            
        %\myFaculty \\
        %\myUni \\ \bigskip

        \myTime\ -- \myVersion

        \vfill                      

    \end{center}  
  \end{addmargin}       
\end{titlepage}   
\include{FrontBackmatter/Titleback}
%\cleardoublepage\include{FrontBackmatter/Dedication}
%\cleardoublepageThis is my excellent book
%\cleardoublepage\include{FrontBackmatter/Abstract}
%\cleardoublepage\include{FrontBackmatter/Publications}
%\cleardoublepage\include{FrontBackmatter/Acknowledgments}
\pagestyle{scrheadings}
\cleardoublepage\include{FrontBackmatter/Contents}

% ****************************************************************
% Start of Part 1: Theory
% ****************************************************************

\cleardoublepage\pagenumbering{arabic}
\cleardoublepage
\ctparttext{\textsc{Digital Logic}, as most computer science studies, depends on a foundation of theory. This part of the book concerns the theory of digital logic and includes binary mathematics, gate-level logic, Boolean algebra, and simplifying Bool\-ean expressions. An understanding of these foundational conceptsis essential before attempting to design complex   combinational and sequential logic circuits.}
\part{Theory}
\chapter{Introduction}\label{ch01}
\section{Preface}
\subsection{Introduction to the Study of Digital Logic}

Digital logic is the study of how electronic devices make decisions. It functions at the lowest level of computer operations: bits that can either be ``on'' or ``off'' and groups of bits that form ``bytes'' and ``words'' that control physical devices. The language of digital logic is Boolean algebra, which is a mathematical model used to describe the logical function of a circuit; and that model can then be used to design the most efficient device possible. Finally, various simple devices, such as adders and registers, can be combined into increasingly complex circuits designed to accomplish advanced decision-making tasks.

\subsection{Introduction to the Author}

I have worked with computers and computer controlled systems for more than 30 years. I took my first programming class in 1976; and, several years later, was introduced to digital logic while taking classes to learn how to repair computer systems. For many years, my profession was to work on computer systems, both as a repair technician and a programmer, where I used the principles of digital logic daily. I then began teaching digital logic classes at Cochise College and was able to share my enthusiasm for the subject with Computer Information Systems students. Over the years, I have continued my studies of digital logic in order to improve my understanding of the topic; I also enjoy building logic circuits on a simulator to solve interesting challenges. It is my goal to make digital logic understandable and to also ignite a lifelong passion for the subject in students.

\subsection{Introduction to This Book}

This book has two goals:

1. \textsc{Audience}. Many, perhaps most, digital logic books are designed for third or fourth year electronics engineering or computer science students and presume a background that includes advanced mathematics and various engineering classes. For example, it is possible to find a digital logic book that discusses topics like physically building circuits from discrete components and then calculating the heat rise of those circuits while operating at maximum capacity. This book, though, was written for students in their second year of a Computer Information Systems program and makes no assumptions about prior mathematics and engineering classes.

2. \textsc{Cost}. Most digital logic books are priced at \$150 (and up) but this book is published under a Creative Commons license and, though only a tiny drop in the proverbial textbook ocean, is hoped to keep the cost of books for at least one class as low as possible.

Following are the features for the various editions of this book:

\begin{enumerate}
  \item 2012. Prior to 2012, handouts were given to students as they were needed during class; however, it was in this year that the numerous disparate documents were assembled into a cohesive book and printed by a professional printing company.
  \item 2013. A number of complex circuits were added to the book, including a Hamming Code generator/checker, which is used for error detection, and a \gls{cpu} using discrete logic gates.
  \item 2014. New material on Mealy and Moore State Machines was included, but the major change was in the laboratory exercises where five Verilog labs were added to the ten gate-level simulation labs.
  \item 2015. New information was added about adding/subtracting \gls{bcd} numbers and representing floating point numbers in binary form; and all of the laboratory exercises were re-written in Verilog. Also, the book was reorganized and converted to \LaTeX for printing.
  \item 2018. The labs were re-written using \textit{Logisim-evolution} because students find that system easier to understand than iVerilog.
\end{enumerate}

This book was written with \LaTeX \: using TeXstudio. The source for this book is available at GITHUB, \url{http://bit.ly/2w6qU2C}, and anyone is welcomed to fork the book and develop their own version. 

% Add some vertical space
\medskip

% This is a "sidebar" box
\begin{tcolorbox}[colback=blue!5!white,colframe=blue!75!black]
  % Upper half of box: my "title" area
  \textcolor{blue}{DISCLAIMER}
  % Lower half of the box: the content
  \tcblower
  \textsf{I wrote, edited, illustrated, and published this book myself. While I did the best that I could, there are, no doubt, errors. I apologize in advance if anything presented here is factually erroneous; I'll correct those errors as soon as they are discovered. I'll also correct whatever typos I overlooked, despite TeXstudio's red squiggly underlines trying to tell me to check my work. --George Self}
\end{tcolorbox}

% Add some vertical space
\medskip

\subsection{About the Creative Commons License}

This book is being released under the Creative Commons 0 license, which is the same as public domain. That permits people to share, remix, or even rewrite the work so it can be used to help educate students wherever they are studying digital logic. 


\section{About Digital Logic}

\subsection{Introduction}

Digital logic is the study of how logic is used in digital devices to complete tasks in fields as diverse as communication, business, space exploration, and medicine (not to mention everyday life). This definition has two main components: logic and digital. \emph{Logic} is the branch of philosophy that concerns making reasonable judgment based on sound principles of inference. It is a method of problem solving based on a linear, step-by-step procedure. \emph{Digital} is a system of mathematics where only two possible values exist: \emph{True} and \emph{False} (usually represented by 1 and 0). While this approach may seem limited, it actually works quite nicely in computer circuits where \emph{True} and \emph{False} can be easily represented by the presence or absence of voltage.

Digital logic is not the same as programming logic, though there is some relationship between them. A programmer would use the constructs of logic within a high-level language, like Java or C++, to get a computer to complete some task. On the other hand, an engineer would use digital logic with hardware devices to build a machine, like an alarm clock or calculator, which executes some task. In broad strokes, then, programming logic concerns writing software while digital logic concerns building hardware. 

Digital logic may be divided into two broad classes: 

\begin{enumerate}

  \item \textsc{Combinational Logic}, in which the outputs are determined solely by the input states at one particular moment with no memory of prior states. An example of a combinational circuit is a simple adder where the output is determined only by the values of two inputs.

  \item \textsc{Sequential Logic}, in which the outputs depend on both current and prior inputs, so some sort of memory is necessary. An example of a sequential circuit is a counter where a sensed new event is added to some total contained in memory.

\end{enumerate}

Both combinational and sequential logic are developed in this book, along with complex circuits that require both combinational and sequential logic.

\subsection{A Brief Electronics Primer}

Electricity is nothing more than the flow of electrons from point \emph{A} to point \emph{B}. Along the way, those electrons can be made to do work as they flow through various devices. Electronics is the science (and, occasionally, art) of using tiny quantities of electricity to do work. As an example, consider the circuit schematic illustrated in Figure \ref{IN:fig:simple_lamp_circuit}:

\begin{figure}[htb]
  \myfloatalign
  \begin{tikzpicture} [circuit logic US, scale=1.00]
  % make all path lines (the node shapes) a little thicker
  \tikzstyle{every path}=[line width=0.20mm]  
    
  % Draw the figure
  \draw
    (0,0)
    to[battery,label=$B_1$] (0,2) % The voltage source
    to[short] (2,2)
    to[lamp,label=$Lmp_1$] (2,0) % The lamp
    to[ospst,label=$Sw_1$] (0.25,0) % The switch
    to[short] (0,0)
  ;    
  \end{tikzpicture}
  \caption{Simple Lamp Circuit}
  \label{IN:fig:simple_lamp_circuit}
\end{figure}

In this diagram, battery \emph{B1} is connected to lamp \emph{Lmp1} through switch \emph{Sw1}. When the switch is closed, electrons will flow from the negative battery terminal, through the switch and lamp, and back to the positive battery terminal. As the electrons flow through the lamp's filament it will heat up and glow: light! 

\begin{figure}[htb]
  \myfloatalign
  \begin{tikzpicture} [circuit logic US, scale=1.00]
  % make all path lines (the node shapes) a little thicker
  \tikzstyle{every path}=[line width=0.20mm]  
  
  % Draw the figure
  \draw
    (4,1) node[npn,rotate=90,label=$Q_1$,anchor=E](npn) {}
    
    % From origin to switch to Q1 base
    (0,0) 
    to[ospst,label=$Sw_1$] (2,0) % Switch
    to[short] (3.25,0)
    to (npn.base)
    
    (npn.emitter)
    to[short] (4.5,1)
    to[battery,l_=$B_1$] (4.5,3.5) % Battery
    to[short] (4.5,5)
    to[short] (0,5)
    to[resistor,l=$R_1$] (0,0) % Resistor
    to (0,0)
    
    % Internal loop from Q1 Collector to Lamp
    (npn.collector) 
    to[short] (2,1)
    to[short] (2,3.5)
    to[lamp,label=$Lmp_1$] (4.25,3.5) % Lamp
    to(4.5,3.5)    
  ;    
  \end{tikzpicture}
  \caption{Transistor-Controlled Lamp Circuit}
  \label{IN:fig:xistor_lamp_circuit}
\end{figure}

A slightly more complex circuit is illustrated in Figure \ref{IN:fig:xistor_lamp_circuit}. In this case, a transistor, \emph{Q1}, has been added to the circuit. When switch \emph{Sw1} is closed, a tiny current can flow from the battery, through the transistor's emitter (the connection with the arrow) to its base, through the switch, through a resistor \emph{R1}, and back to the positive terminal of the battery. However, that small current turns on transistor \emph{Q1} so a much larger current can also flow from the battery, through the collector port, to lamp \emph{Lmp1}, and back to the battery. The final effect is the same for both circuits: close a switch and the lamp turns on. However, by using a transistor, the lamp can be controlled by applying any sort of positive voltage to the transistor's base; so the lamp could be controlled by a switch, as illustrated, or by the output of some other electronic process, like a photo-electric cell sensing that the room is dark.

Using various electronic components, like transistors and resistors, digital logic ``gates'' can be constructed, and these become the building blocks for complex logic circuits. Logic gates are more thoroughly covered in a later chapter, but one of the fundamental logic gates is an \textsf{OR} gate, and a simplified schematic diagram for an \textsf{OR} gate is in Figure \ref{IN:fig:xistor_or_gate}. In this circuit, any voltage present at \emph{Input A} or \emph{Input B} will activate the transistors and develop voltage at the output. 

\begin{figure}[htb]
  \myfloatalign
  \begin{tikzpicture} [circuit logic US, scale=1.00]
  % make all path lines (the node shapes) a little thicker
  \tikzstyle{every path}=[line width=0.20mm]  
  
  % Draw the figure
  \draw
    (2,3) node[npn,label={[label distance=0.4cm]120:$Q_1$},anchor=B](npn1) {}
    (2,1) node[npn,label={[label distance=0.4cm]120:$Q_2$},anchor=B](npn2) {}
    (3.75,0.225) node[cground](ground) {}
    (5.0,4.25) node[circ,label={0:out}] (out) {}
    (0.0,3) node[circ,label={180:A}] (in_a) {}
    (0.0,1) node[circ,label={180:B}] (in_b) {}
    
    % Ground to Q2 Emitter
    (ground) 
    to (npn2.emitter)
    (0,1) 
    to[R, l_=$R_2$] (2,1)
    (0,3) 
    to[R, l_=$R_1$] (2,3)
    % Ground to Q1 Emitter
    (ground)
    to[short] (3.75,2.225)
    to (npn1.emitter)
    % Q2 collector to short
    (npn2.collector)
    to[short] (4.25,1.75)
    to[short] (4.25,3.75)
    % Q1 collector to R3
    (npn1.collector)
    to[short] (4.25,3.75)
    to[R, l_=$R_3$] (4.25,6.5)
    % short to out
    (4.25,4.25)
    to (out)
  ;    
  \end{tikzpicture}
  \caption{Simple OR Gate Using Transistors}
  \label{IN:fig:xistor_or_gate}
\end{figure}

Figure \ref{IN:fig:nmos_or_gate}\footnote{This schematic diagram was created by Ram\'{o}n Jaramillo and found at \url{http://www.texample.net/tikz/examples/or-gate/}} shows a more complete \textsf{OR} gate created with what are called ``N-Channel metal-oxide semiconductor'' (or \emph{nMOS}) transistors. The electronic functioning of this circuit is beyond the scope of this book (and is in the domain of electronics engineers), but the important point to keep in mind is that this book concerns building electronic circuits designed to accomplish a physical task rather than write a program to control a computer's processes. 

\begin{figure}[bth]
  \myfloatalign
  % Title: OR gate
  % Author: Ramón Jaramillo.
  \begin{tikzpicture}[scale=2] 
  \draw[color=black, thick]
  % Name of MOSFET transistors left to right
  % Left to right in baseline: nmos1, nmos2, nmos3, nmo4 & nmos5.
  % Drawing the transistor and nodes using relative coordinates
  (0,0) node[nigfete] (nmos1) {}
  (1,0) node[nigfete] (nmos2) {}
  (nmos1.S) to (nmos2.S)
  % Connections to V_DD
  to [short,*-o] ++(0,-0.5) {} node[below=2mm] {$V_{DD}$}
  (1,1) node[pfet,yscale=-1] (pmos1) {} % yscale=-1 is mandatory to drawing
  % MOSFETs with the right sense.
  (1,2) node[pigfete,yscale=-1] (pmos2) {}
  (nmos2.G) to [short,-*] (pmos1.G)
  (pmos1.D) to (nmos2.D)  
  (nmos1.G) -- ++(0,1.5) -- ++(0,0.75)
  (pmos2.G) to [short,-*] ++(-1,0){}
  to [short,-o] ++(-0.5,0) {} node[above=3mm] {$A$}
  (pmos1.G) to [short,-o] ++(-1.5,0) {} node[above=3mm] {$B$}
  (pmos1.B) -- ++(0.2,0) -| ++(0,1.15) to [short,-*]++(-0.21,0)
  (pmos1.S) -- (pmos2.D) {}
  % Inverter
  (2,0) node[nigfete](nmos3) {}
  (2,1) node[pigfete, yscale=-1] (pmos3) {}
  (nmos3.G) to (pmos3.G)
  (nmos3.D) to (pmos3.D)
  % Gate OR Output
  (3,0) node[nigfete](nmos4){}
  (3,1) node[pigfete, yscale=-1] (pmos4) {}
  (nmos4.G) to (pmos4.G)
  (nmos4.D) to (pmos4.D)
  %%%
  (4,0) node[nigfete] (nmos5) {}
  (4,1) node[pigfete, yscale=-1] (pmos5) {}
  (nmos5.G) to (pmos5.G)
  (nmos5.D) to (pmos5.D)
  % Connection to V_DD
  (nmos3.S) to [short,-*] (nmos4.S) -- (nmos5.S)
  to [short,*-o] ++(0,-0.5) {} node[below=2mm] {$V_{DD}$}
  % Connection to V_DD
  (pmos3.S) -- (pmos4.S) -- (pmos5.S)
  to [short,*-o] ++(0,1.5) {} node[above=2mm] {$V_{SS}$}
  (pmos2.S) to [short,-o] ++(0,0.5) {} node[above=2mm] {$V_{SS}$}
  %(3,-0.4) node[circ] {}
  % Connections between nodes
  (nmos1.D) -| (0,0.5) to [short,-o] (1,0.5) {}
  (1,0.5) to  [short,*-*] ++(0.5,0) {}
  (2,0.5) to  [short,*-*] ++(0.5,0) {}
  (3,0.5) to  [short,*-*] ++(0.5,0) {}
  (4,0.5) to  [short,*-o] ++(0.5,0) {} node[right=3mm] {$C=A+B$}
  ;
  \end{tikzpicture}
  \caption{OR Gate}
  \label{IN:fig:nmos_or_gate}  
\end{figure}
  
Early electronic switches took the form of vacuum tubes, but those were replaced by transistors which were much more energy efficient and physically smaller. Eventually, entire transistorized circuits, like the \textsf{OR} gate illustrated in Figure \ref{IN:fig:nmos_or_gate}, were miniaturized and placed in a single \gls{ic}, sometimes called a ``chip,'' smaller than a postage stamp. 

\Glspl{ic} make it possible to produce smaller, faster, and more powerful electronics devices. For example, the original ENIAC computer, built in 1946, occupied more than 1800 square feet of floor space and required 150KW of electricity, but by the 1980s integrated circuits in hand-held calculators were more powerful than that early computer. Integrated circuits are often divided into four classes: 

\begin{enumerate}
  \item Small-scale integration with fewer than 10 transistors
  \item Medium-scale integration with 10-500 transistors
  \item Large-scale integration with 500-20,000 transistors
  \item Very large-scale integration with 20,000-1,000,000 transistors
\end{enumerate}

Integrated circuits are designed by engineers who use software written specifically for that purpose. While the intent of this book is to afford students a foundation in digital logic, those who pursue a degree in electronics engineering, software engineering, or some related field, will need to study a digital logic language, like iVerilog.

\section{Boolean Algebra}

\subsection{History}

The Greek philosopher Aristotle founded a system of logic based on only two types of propositions: \emph{True} and \emph{False}. His bivalent (two-mode) definition of truth led to four foundational laws of logic: the Law of Identity (\emph{A is A}); the Law of Non-contradiction (\emph{A is not non-A}); the Law of the Excluded Middle (\emph{either A or non-A}); and the Law of Rational Inference. These laws function within the scope of logic where a proposition is limited to one of two possible values, like \emph{True} and \emph{False}; but they do not apply in cases where propositions can hold other values.

The English mathematician George Boole (1815-1864) sought to give symbolic form to Aristotle's system of logic. Boole wrote a treatise on the subject in 1854, titled \emph{An Investigation of the Laws of Thought, on Which Are Founded the Mathematical Theories of Logic and Probabilities}, which codified several rules of relationship between mathematical quantities limited to one of two possible values: \emph{True} or \emph{False}, $ 1 $ or $ 0 $. His mathematical system became known as \emph{Boolean Algebra}.

All arithmetic operations performed with Boolean quantities have but one of two possible outcomes: $ 1 $ or $ 0 $. There is no such thing as $ 2 $ or $ -1 $ or $ \frac{1}{3} $ in the Boolean world and numbers other than $ 1 $ and $ 0 $ are invalid by definition. Claude Shannon of MIT recognized how Boolean algebra could be applied to on-and-off circuits, where all signals are characterized as either \emph{high} ($ 1 $) or \emph{low} ($ 0 $), and his 1938 thesis, titled \emph{A Symbolic Analysis of Relay and Switching Circuits}, put Boole's theoretical work to use in land line telephone switching, a system Boole never could have imagined.

While there are a number of similarities between Boolean algebra and real-number algebra, it is important to bear in mind that the system of numbers defining Boolean algebra is severely limited in scope: $ 1 $ and $ 0 $. Consequently, the ``Laws'' of Boolean algebra often differ from the ``Laws'' of real-number algebra, making possible such Boolean statements as $ 1 + 1 = 1 $, which would be considered absurd for real-number algebra. 

\subsection{Boolean Equations}

Boolean algebra is a mathematical system that defines a series of logical operations performed on a set of variables. The expression of a single logical function is a \emph{Boolean Equation} \marginpar{Sometimes Boolean Equations are called \textsc{Switching Equations}} that uses standardized symbols and rules. Boolean expressions are the foundation of digital circuits.

Binary variables used in Boolean algebra are like variables in regular algebra except that they can only have two values: one or zero. Boolean algebra includes three primary logical functions: \textsf{AND}, \textsf{OR}, and \textsf{NOT}; and five secondary logical functions: \marginpar{XNOR is sometimes called \textsc{equivalence} and Buffer is sometimes called \textsc{transfer}.} \textsf{NAND}, \textsf{NOR}, \textsf{XOR}, and \textsf{XNOR}, and Buffer. A Boolean equation defines an electronic circuit that provides a relationship between input and output variables and takes the form of:

\begin{align}
\label{01:eq:simple_boolean_equation}
C &= A * B 
\end{align}

where \emph{A} and \emph{B} are binary input variables that are related to the output variable \emph{C} by the function \textsf{AND} (denoted by an asterisk).

In Boolean algebra, it is common to speak of \emph{truth}. This term does not mean the same as it would to a philosopher, though its use is based on Aristotelian philosophy where a statement was either \emph{True} or \emph{False}. In Boolean algebra as used in electronics, \emph{True} commonly means ``voltage present'' (or ``$ 1 $'') while \emph{False} commonly means ``voltage absent'' (or ``$ 0 $''), and this can be applied to either input or output variables. It is common to create a \emph{Truth Table} for a Boolean equation to indicate which combination of inputs should evaluate to a \emph{True} output and Truth Tables are used very frequently throughout this book.

\section{About This Book}

This book is organized into two main parts:

\begin{enumerate}
  \item \textsc{Theory}. Chapters two through six cover the foundational theory of digital logic. Included are chapters on binary mathematics, Boolean algebra, and simplifying Boolean expressions using tools like Karnaugh maps and the Quine-McCluskey method.
  
  \item \textsc{Practice}. Chapters seven through nine expand the theory of digital logic into practical applications. Covered are combinational and sequential logic circuits, and then simulation of various physical devices, like elevators.
\end{enumerate}

There is also an accompanying lab manual where \textit{Logisim-evolution} is used to build digital logic circuits. By combining the theory presented in this book along with the practical application presented in the lab manual it is hoped that students gain a thorough understanding of digital logic.

By combining the theoretical background of binary mathematics and Boolean algebra with the practical application of building logic devices, digital logic becomes understandable and useful.


\chapter{Foundations of Binary Mathematics}\label{ch02}
\section{Introduction to Number Systems}
\subsection{Background}

The expression of numerical quantities is often taken for granted, which is both a good and a bad thing in the study of electronics. It is good since the use and manipulation of numbers is familiar for many calculations used in analyzing electronic circuits. On the other hand, the particular system of notation that has been taught from primary school onward is not the system used internally in modern electronic computing devices and learning any different system of notation requires some re-examination of assumptions.

It is important to distinguish the difference between numbers and the symbols used to represent numbers. A number is a mathematical quantity, usually correlated in electronics to a physical quantity such as voltage, current, or resistance. There are many different types of numbers, for example:

\begin{itemize}
  \item Whole Numbers: $ 1, 2, 3, 4, 5, 6, 7, 8, 9 ... $
  \item Integers: $ -4, -3, -2, -1, 0, 1, 2, 3, 4 ...  $
  \item Rational Numbers: $ -5.3, 0, \frac{1}{3}, 6.7 $
  \item Irrational Numbers: $\pi $ (approx. $ 3.1416 $), e (approx. $ 2.7183 $), and the square root of any prime number 
  \item Real Numbers: (combination of all rational and irrational numbers) 
  \item Complex Numbers: $ 3 - j4 $
\end{itemize}

Different types of numbers are used for different applications in electronics. As examples:

\begin{itemize}
  \item Whole numbers work well for counting discrete objects, such as the number of resistors in a circuit. 
  \item Integers are needed to express a negative voltage or current. 
  \item Irrational numbers are used to describe the charge/discharge cycle of electronic objects like capacitors.
  \item Real numbers, in either fractional or decimal form, are used to express the non-integer quantities of voltage, current, and resistance in circuits.
  \item Complex numbers, in either rectangular or polar form, must be used rather than real numbers to capture the dual essence of the magnitude and phase angle of the current and voltage in alternating current circuits.
\end{itemize}

There is a difference between the concept of a ``number'' as a measure of some quantity and ``number'' as a means used to express that quantity in spoken or written communication. A way to symbolically denote numbers had to be developed in order to use them to describe processes in the physical world, make scientific predictions, or balance a checkbook. The written symbol that represents some number, like how many apples there are in a bin, is called a \emph{cipher} and in western mathematics, the commonly-used ciphers are $ 0, 1, 2, 3, 4, 5, 6, 7, 8, $ and $ 9 $.

\subsection{Binary Mathematics}
\label{MF:sub:binary_mathematics}
Binary mathematics is a specialized branch of mathematics that concerns itself with a number system that contains only two ciphers: zero and one. It would seem to be very limiting to use only two ciphers; however, it is much easier to create electronic devices that can differentiate between two voltage levels rather than the ten that would be needed for a decimal system.

\subsection{Systems Without Place Value}
\label{MF:sub:sysems_without_place_value}
\paragraph{Hash Marks.} One of the earliest cipher systems was to simply use a hash mark to represent each quantity. For example, three apples could be represented like this: \textbar \textbar \textbar. Often, five hash marks were ``bundled'' to aid in the counting of large quantities, so eight apples would be represented like this: \sout{\textbar \textbar \textbar \textbar} \textbar \textbar \textbar. 

\paragraph{Roman Numerals.} The Romans devised a system that was a substantial improvement over hash marks, because it used a variety of ciphers to represent increasingly large quantities. The notation for one is the capital letter \emph{I}. The notation for $ 5 $ is the capital letter \emph{V}. Other ciphers, as listed in Table \ref{MF:tab:roman}, possess increasing values:

\begin{table}[H]
  \rowcolors{1}{gray!10}{white}
  \begin{center}
    \begin{tabular}{ c r } \hline
      I & 1 \\
      V & 5 \\
      X & 10 \\
      L & 50 \\
      C & 100 \\
      D & 500 \\
      M & 1000 \\ \hline
    \end{tabular}
  \end{center}
  \caption{Roman Numerals}
  \label{MF:tab:roman}
\end{table}

If a cipher is accompanied by a second cipher of equal or lesser value to its immediate right, with no ciphers greater than that second cipher to its right, the second cipher's value is added to the total quantity. Thus, \emph{VIII} symbolizes the number $ 8 $, and \emph{CLVII} symbolizes the number $ 157 $. On the other hand, if a cipher is accompanied by another cipher of lesser value to its immediate left, that other cipher's value is subtracted from the first. In that way, \emph{IV} symbolizes the number $ 4 $ (\emph{V} minus \emph{I}), and \emph{CM} symbolizes the number $ 900 $ (\emph{M} minus \emph{C}). The ending credit sequences for most motion pictures contain the date of production, often in Roman numerals. For the year $ 1987 $, it would read: \emph{MCMLXXXVII}. To break this numeral down into its constituent parts, from left to right:

\begin{center}
  $ (M = 1000) + (CM = 900) + (LXXX = 80) + (VII = 7) $
\end{center}

Large numbers are very difficult to denote with Roman numerals; and the left vs. right (or subtraction vs. addition) of values can be very confusing. Adding and subtracting two Roman numerals is also very challenging, to say the least. Finally, one other major problem with this system is that there is no provision for representing the number zero or negative numbers, and both are very important concepts in mathematics. Roman culture, however, was more pragmatic with respect to mathematics than most, choosing only to develop their numeration system as far as it was necessary for use in daily life.

\subsection{Systems With Place Value}
\label{MF:sub:systems_with_place_value}
\paragraph{Decimal Numeration.} The Babylonians developed one of the most important ideas in numeration: cipher position, or place value, to represent larger numbers. Instead of inventing new ciphers to represent larger numbers, as the Romans had done, they re-used the same ciphers, placing them in different positions from right to left to represent increasing values. This system also required a cipher that represents zero value, and the inclusion of zero in a numeric system was one of the most important inventions in all of mathematics (many would argue zero was the single most important human invention, period). The decimal numeration system uses the concept of place value, with only ten ciphers ($ 0, 1, 2, 3, 4, 5, 6, 7, 8, $ and $ 9 $) used in ``weighted'' positions to symbolize numbers.

Each cipher represents an integer quantity, and each place from right to left in the notation is a multiplying constant, or weight, for the integer quantity. For example, the decimal notation ``$ 1206 $'' may be broken down into its constituent weight-products as such:

\begin{center}
  $ 1206 = (1 X 1000) + (2 X 100) + (0 X 10) + (6 X 1) $
\end{center}

Each cipher is called a ``digit'' in the decimal numeration system, and each weight, or place value, is ten times that of the place to the immediate right. So, working from right to left is a ``ones'' place, a ``tens'' place, a ``hundreds'' place, a ``thousands'' place, and so on.

While the decimal numeration system uses ten ciphers, and place-weights that are multiples of ten, it is possible to make a different numeration system using the same strategy, except with fewer or more ciphers.

\paragraph{Binary Numeration.} The binary numeration system uses only two ciphers and the weight for each place in a binary number is two times as much as the place to its right. Contrast this to the decimal numeration system that has ten different ciphers and the weight for each place is ten times the place to its right. The two ciphers for the binary system are zero and one, and these ciphers are arranged right-to-left in a binary number, each place doubling the weight of the previous place. The rightmost place is the ``ones'' place; and, moving to the left, is the ``twos'' place, the ``fours'' place, the ``eights'' place, the ``sixteens'' place, and so forth. For example, the binary number $ 11010 $ can be expressed as a sum of each cipher value times its respective weight:

\begin{center}
  $ 11010 = (1 X 16) + (1 X 8) + (0 X 4) + (1 X 2) + (0 X 1) $
\end{center}

The primary reason that the binary system is popular in modern electronics is because it is easy to represent the two cipher states (zero and one) electronically; if no current is flowing in the circuit it represents a binary zero while flowing current represents a binary one. Binary numeration also lends itself to the storage and retrieval of numerical information: as examples, magnetic tapes have spots of iron oxide that are magnetized for a binary one or demagnetized for a binary zero and optical disks have a laser-burned pit in the aluminum substrate representing a binary one and an unburned spot representing a binary zero.

Digital numbers require so many bits to represent relatively small numbers that programming or analyzing electronic circuitry can be a tedious task. However, anyone working with digital devices soon learns to quickly count in binary to at least $ 11111 $ (that is decimal $ 31 $). Any time spent practicing counting both up and down between zero and $ 11111 $ will be rewarded while studying binary mathematics, codes, and other digital logic topics. Table \ref{MF:tab:bin_dec_conversion} will help in memorizing binary numbers:

\begin{table}[H]
  \sffamily
  \newcommand{\head}[1]{\textcolor{white}{\textbf{#1}}}
  \begin{center}
    \begin{tabular}{|cc|cc|cc|cc|} 
      \hline
      \rowcolor{black!75}
      \head{Bin} & \head{Dec} & \head{Bin} & \head{Dec} & \head{Bin} & 
      \head{Dec} & \head{Bin} & \head{Dec} \\
      \hline 
      0   & 0 & 1000 & 8  & 10000 & 16 & 11000 & 24 \\ 
      1   & 1 & 1001 & 9  & 10001 & 17 & 11001 & 25 \\ 
      10  & 2 & 1010 & 10 & 10010 & 18 & 11010 & 26 \\ 
      11  & 3 & 1011 & 11 & 10011 & 19 & 11011 & 27 \\ 
      100 & 4 & 1100 & 12 & 10100 & 20 & 11100 & 28 \\ 
      101 & 5 & 1101 & 13 & 10101 & 21 & 11101 & 29 \\ 
      110 & 6 & 1110 & 14 & 10110 & 22 & 11110 & 30 \\ 
      111 & 7 & 1111 & 15 & 10111 & 23 & 11111 & 31 \\ 
      \hline
    \end{tabular} 
  \end{center}
  \caption{Binary-Decimal Conversion}
  \label{MF:tab:bin_dec_conversion}
\end{table}

\paragraph{Octal Numeration.} The octal numeration system is place-weighted with a base of eight. Valid ciphers include the symbols $ 0, 1, 2, 3, 4, 5, 6, $ and $ 7 $. These ciphers are arranged right-to-left in an octal number, each place being eight times the weight of the previous place. For example, the octal number $ 4270 $ can be expressed, just like a decimal number, as a sum of each cipher value times its respective weight:

\begin{center}
  $4270 = (4 X 512) + (2 X 64) + (7 X 8) + (0 X 1)$
\end{center}

\paragraph{Hexadecimal Numeration.} The hexadecimal numeration system is place-weighted with a base of sixteen.\marginpar{The word ``hexadecimal'' is a combination of ``hex'' for six and ``decimal'' for ten} There needs to be ciphers for numbers greater than nine so English letters are used for those values. Table \ref{MF:tab:hexadecimal_numbers} lists hexadecimal numbers up to decimal 15:

\begin{table}[H]
  \sffamily
  \newcommand{\head}[1]{\textcolor{white}{\textbf{#1}}}  
  \begin{center}
    \begin{tabular}{|cc|cc|} 
      \hline
      \rowcolor{black!75}
      \head{Hex} & \head{Dec} & \head{Hex} & \head{Dec} \\ 
      \hline
      0 & 0 & 8 & 8  \\ 
      1 & 1 & 9 & 9  \\ 
      2 & 2 & A & 10 \\ 
      3 & 3 & B & 11 \\ 
      4 & 4 & C & 12 \\ 
      5 & 5 & D & 13 \\ 
      6 & 6 & E & 14 \\ 
      7 & 7 & F & 15 \\ 
      \hline
    \end{tabular} 
  \end{center}
  \caption{Hexadecimal Numbers}
  \label{MF:tab:hexadecimal_numbers}  
\end{table}

Hexadecimal ciphers are arranged right-to-left, each place being $ 16 $ times the weight of the previous place. For example, the hexadecimal number $ 13A2 $ can be expressed, just like a decimal number, as a sum of each cipher value times its respective weight:

\begin{center}
  $13A2 = (1 X 4096) + (3 X 256) + (A X 16) + (2 X 1)$
\end{center}

% Begin Sidebar Box
\begin{tcolorbox}[colback=blue!5!white,colframe=blue!75!black]
  % Upper half of box: my "title" area
  \textcolor{blue}{\textbf{Maximum Number Size}}
  % Lower half of the box: the content
  \tcblower
  It is important to know the largest number that can be represented with a given number of cipher positions. For example, if only four cipher positions are available then what is the largest number that can be represented in each of the numeration systems? With the crude hash-mark system, the number of places IS the largest number that can be represented, since one hash mark ``place'' is required for every integer step. For place-weighted systems, however, the answer is found by taking the number base of the numeration system ($ 10 $ for decimal, $ 2 $ for binary) and raising that number to the power of the number of desired places. For example, in the decimal system, a five-place number can represent $ 10^{5} $, or $ 100,000 $, with values from zero to $ 99,999 $. Eight places in a binary numeration system, or $ 2^{8} $, can represent $ 256 $ different values, $ 0 - 255 $.
\end{tcolorbox}
% End Sidebar Box

\subsection{Summary of Numeration Systems}
\label{MF:sub:summary_of_numeration_systems}
Table \ref{MF:tab:counting_to_twenty} counts from zero to twenty using several different numeration systems:

\begin{table}[H]
  \sffamily
  \newcommand{\head}[1]{\textcolor{white}{\textbf{#1}}}
  \begin{center}
    \rowcolors{2}{gray!10}{white}
    \begin{tabular}{lcccccc} 
      \hline
      \rowcolor{black!75}
      \head{Text} & \head{Hash Marks} & \head{Roman} & \head{Dec} & \head{Bin} 
      & \head{Oct} & \head{Hex} \\ 
      \hline
      Zero      & n/a                           & n/a   & 0  & 0     & 0  & 0 \\ 
      One       & $\mid$                        & I     & 1  & 1     & 1  & 1 \\ 
      Two       & $\mid\mid$                    & II    & 2  & 10    & 2  & 2 \\ 
      Three     & $\mid\mid\mid$                & III   & 3  & 11    & 3  & 3 \\ 
      Four      & $\mid\mid\mid\mid$            & IV    & 4  & 100   & 4  & 4 \\ 
      Five      & \sout{$\mid\mid\mid\mid$} & V     & 5  & 101   & 5  & 5 \\ 
      Six       & \sout{$\mid\mid\mid\mid$}
      $\mid$                                    & VI    & 6  & 110   & 6  & 6 \\ 
      Seven     & \sout{$\mid\mid\mid\mid$}
      $\mid\mid$                                & VII   & 7  & 111   & 7  & 7 \\ 
      Eight     & \sout{$\mid\mid\mid\mid$}
      $\mid\mid\mid$                            & VIII  & 8  & 1000  & 10 & 8 \\ 
      Nine      & \sout{$\mid\mid\mid\mid$}
      $\mid\mid\mid\mid$                        & IX    & 9  & 1001  & 11 & 9 \\ 
      Ten       & \sout{$\mid\mid\mid\mid$}
      \sout{$\mid\mid\mid\mid$}             & X     & 10 & 1010  & 12 & A \\ 
      Eleven    & \sout{$\mid\mid\mid\mid$}
      \sout{$\mid\mid\mid\mid$}
      $\mid$                                    & XI    & 11 & 1011  & 13 & B \\ 
      Twelve    & \sout{$\mid\mid\mid\mid$}
      \sout{$\mid\mid\mid\mid$}
      $\mid\mid$                                & XII   & 12 & 1100  & 14 & C \\ 
      Thirteen  & \sout{$\mid\mid\mid\mid$}
      \sout{$\mid\mid\mid\mid$}
      $\mid\mid\mid$                            & XIII  & 13 & 1101  & 15 & D \\ 
      Fourteen  & \sout{$\mid\mid\mid\mid$}
      \sout{$\mid\mid\mid\mid$}
      $\mid\mid\mid\mid$                        & XIV   & 14 & 1110  & 16 & E \\ 
      Fifteen   & \sout{$\mid\mid\mid\mid$}
      \sout{$\mid\mid\mid\mid$}
      \sout{$\mid\mid\mid\mid$}             & XV    & 15 & 1111  & 17 & F \\ 
      Sixteen   & \sout{$\mid\mid\mid\mid$}
      \sout{$\mid\mid\mid\mid$}
      \sout{$\mid\mid\mid\mid$}
      $\mid$                                    & XVI   & 16 & 10000 & 20 & 10 \\ 
      Seventeen & \sout{$\mid\mid\mid\mid$}
      \sout{$\mid\mid\mid\mid$}
      \sout{$\mid\mid\mid\mid$}
      $\mid\mid$                                & XVII  & 17 & 10001 & 21 & 11 \\ 
      Eighteen  & \sout{$\mid\mid\mid\mid$}
      \sout{$\mid\mid\mid\mid$}
      \sout{$\mid\mid\mid\mid$}
      $\mid\mid\mid$                            & XVIII & 18 & 10010 & 22 & 12 \\ 
      Nineteen  & \sout{$\mid\mid\mid\mid$}
      \sout{$\mid\mid\mid\mid$}
      \sout{$\mid\mid\mid\mid$}
      $\mid\mid\mid\mid$                        & XIX   & 19 & 10011 & 23 & 13 \\ 
      Twenty    & \sout{$\mid\mid\mid\mid$}
      \sout{$\mid\mid\mid\mid$}
      \sout{$\mid\mid\mid\mid$}
      \sout{$\mid\mid\mid\mid$}             & XX    & 20 & 10100 & 24 & 14 \\ 
      \hline
    \end{tabular} 
  \end{center}
  \caption{Counting To Twenty}
  \label{MF:tab:counting_to_twenty}
\end{table}

% Begin Sidebar Box
\medskip % Add a bit of space above the box
\begin{tcolorbox}[colback=blue!5!white,colframe=blue!75!black]
  % Upper half of box: my "title" area
  \textcolor{blue}{\textbf{Numbers for Computer Systems}}
  % Lower half of the box: the content
  \tcblower
  An interesting footnote for this topic concerns one of the first electronic digital computers: ENIAC. The designers of the ENIAC chose to work with decimal numbers rather than binary in order to emulate a mechanical adding machine; unfortunately, this approach turned out to be counter-productive and required more circuitry (and maintenance nightmares) than if they had they used binary numbers. ``ENIAC contained $ 17,468 $ vacuum tubes, $ 7,200 $ crystal diodes, $ 1,500 $ relays, $ 70,000 $ resistors, $ 10,000 $ capacitors and around $ 5 $ million hand-soldered joints''\footnote{\url{http://en.wikipedia.org/wiki/Eniac}}. Today, all digital devices use binary numbers for internal calculation and storage and then convert those numbers to/from decimal only when necessary to interface with human operators.
\end{tcolorbox}
% End Sidebar Box

\subsection{Conventions}
\label{MF:sub:conventions}
Using different numeration systems can get confusing since many ciphers, like ``$ 1 $,'' are used in several different numeration systems. Therefore, the numeration system being used is typically indicated with a subscript following a number, like $ 11010_{2} $ for a binary number or $ 26_{10} $ for a decimal number. The subscripts are not mathematical operation symbols like superscripts, which are exponents; all they do is indicate the system of numeration being used. By convention, if no subscript is shown then the number is assumed to be decimal.\marginpar{In this book, subscripts are normally used to make it clear whether the number is binary or some other system.}

Another method used to represent hexadecimal numbers is the prefix $ 0x $. This has been used for many years by programmers who work with any of the languages descended from C, like C++, C\#, Java, JavaScript, and certain shell scripts. Thus, $ 0x1A $ would be the hexadecimal number $ 1A $.

One other commonly used convention for hexadecimal numbers is to add an \emph{h} (for \emph{hexadecimal}) after the number. This is used because that is easier to enter with a keyboard than to use a subscript and is more intuitive than using a $ 0x $ prefix. Thus, $ 1A_{16} $ would be written $ 1Ah $. In this case, the \emph{h} only indicates that the number $ 1A $ is hexadecimal; it is not some sort of mathematical operator.

Occasionally binary numbers are written with a $ 0b $ prefix; thus $ 0b1010 $ would be $ 1010_{2} $, but this is a programmer's convention not often found elsewhere.

\section{Converting Between Radices}
\label{MF:sec:converting_between_bases}

\subsection{Introduction}
\label{MF:sub:introduction_converting_between_bases}

\marginpar{The radix of a system is also commonly called its ``base.''}The number of ciphers used by a number system (and therefore, the place-value multiplier for that system) is called the \emph{radix} for the system. The binary system, with two ciphers (zero and one), is radix two numeration, and each position in a binary number is a \emph{b}inary dig\emph{it} (or \emph{bit}). The decimal system, with ten ciphers, is radix-ten numeration, and each position in a decimal number is a \emph{digit}. When working with various digital logic processes it is desirable to be able to convert between binary/octal/decimal/hexadecimal radices.

\subsection{Expanded Positional Notation}
\label{MF:sub:expanded_positional_notation}
\textsc{Expanded Positional Notation} is a method of representing a number in such a way that each position is identified with both its cipher symbol and its place-value multiplier. For example, consider the number $ 347_{10} $:

\begin{equation}
347_{10}=(3X10^2)+(4X10^1)+(7X10^0)
\end{equation}

The steps to use to expand a decimal number like $ 347 $ are found in Table \ref{MF:tab:expand_347}.

\begin{table}[H]
  \sffamily
  \newcommand{\head}[1]{\textcolor{white}{\textbf{#1}}}      
  \begin{center}
    \rowcolors{2}{gray!10}{white} % Color every other line a light gray
    \begin{tabular}{ p{7cm} l } 
      \hline
      \rowcolor{black!75}
      \head{Step} & \head{Result} \\ 
      \hline 
      Count the number of digits in the number. & Three Digits  \\ 
      Create a series of $ (\_ X\_) $ connected by plus signs such that there is one set for each of the digits in the original number. & $ ( \_X\_ ) + ( \_X\_ ) + ( \_X\_ ) $  \\ \hline
      Fill in the digits of the original number on the left side of each set of parenthesis. & $ ( 3X\_ ) + ( 4X\_ ) + ( 7X\_ ) $  \\ 
      Fill in the radix (or base number) on the right side of each parenthesis. & $ ( 3X10 ) + ( 4X10 ) + ( 7X10 ) $  \\ 
      Starting on the far right side of the expression, add an exponent (power) for each of the base numbers. The powers start at zero and increase to the left. & $ ( 3X10^2 ) + ( 4X10^1 ) + ( 7X10^0 ) $ \\
      \hline
    \end{tabular} 
  \end{center}
  \caption{Expanding a Decimal Number}
  \label{MF:tab:expand_347}
\end{table}

Additional examples of expanded positional notation are: 

\begin{equation}
2413_{10}=(2x10^3)+(4X10^2)+(1X10^1)+(3X10^0)
\end{equation} 

\begin{equation}
1052_8=(1X8^3)+(0X8^2)+(5X8^1)+(2X8^0)  
\end{equation}

\begin{equation}
139_{16}=(1X16^2)+(3X16^1)+(9X16^0)
\end{equation}

The above examples are for positive decimal integers; but a number with any radix can also have a fractional part. In that case, the number's integer component is to the left of the radix point (called the ``decimal point'' in the decimal system), while the fractional part is to the right of the radix point. For example, in the number $ 139.25_{10} $, $ 139 $ is the integer component while $ 25 $ is the fractional component. If a number includes a fractional component, then the expanded positional notation uses increasingly negative powers of the radix for numbers to the right of the radix point. Consider this binary example: $ 101.011_2 $. The expanded positional notation for this number is:

\begin{equation}
101.011_2=(1X2^2)+(0X2^1)+(1X2^0)+(0X2^{-1})+(1X2^{-2})+(1X2^{-3})
\end{equation}

Other examples are:

\begin{equation}
526.14_{10}=(5X10^2)+(2X10^1)+(6X10^0)+(1X10^{-1})+(4X10^{-2})
\end{equation}

\begin{equation}
65.147_8=(6X8^1)+(5X8^0)+(1X8^{-1})+(4X8^{-2})+(7X8^{-3})
\end{equation}

\begin{equation}
D5.3A_{16}=(13X16^1)+(5X16^0)+(3X16^{-1})+(10X16^{-2})
\end{equation}

When a number in expanded positional notation includes one or more negative radix powers, the radix point is assumed to be to the immediate right of the ``zero'' exponent term, but it is not actually written into the notation. Expanded positional notation is useful in converting a number from one base to another. 

\subsection{Binary to Decimal}
\label{MF:sub:binary_to_decimal}
To convert a number in binary form to decimal, start by writing the binary number in expanded positional notation, calculate the values for each of the sets of parenthesis in decimal, and then add all of the values. For example, convert $ 1101_2 $ to decimal:

\begin{align}
  1101_2 &= (1X2^3)+(1X2^2)+(0X2^1)+(1X2^0) \\
  \nonumber
  &= (8)+(4)+(0)+(1) \\
  \nonumber
  &= 13_{10}
\end{align}

Binary numbers with a fractional component are converted to decimal in exactly the same way, but the fractional parts use negative powers of two. Convert binary $ 10.11_2 $ to decimal: 

\begin{align}
  10.11_2 &= (1X2^1)+(0X2^0)+(1X2^{-1})+(1X2^{-2}) \\
  \nonumber
  &= (2)+(0)+(\frac{1}{2})+(\frac{1}{4}) \\
  \nonumber
  &= 2+.5+.25 \\
  \nonumber
  &= 2.75_{10}
\end{align}

Most technicians who work with digital circuits learn to quickly convert simple binary integers to decimal in their heads. However, for longer numbers, it may be useful to write down the various place weights and add them up; in other words, a shortcut way of writing expanded positional notation. For example, convert the binary number $ 11001101_2 $ to decimal: 

\begin{verbatim}
     Binary Number:  1 1 0 0 1 1 0 1
                     - - - - - - - - 
        (Read Down)  1 6 3 1 8 4 2 1 
                     2 4 2 6     
                     8 
\end{verbatim}

A bit value of one in the original number means that the respective place weight gets added to the total value, while a bit value of zero means that the respective place weight does not get added to the total value. Thus, using the example above this paragraph, the binary number $ 11001101_2 $ is converted to: $ 128+64+8+4+1 $, or $ 205_{10} $.

\bigskip

% Begin Sidebar Box
\begin{tcolorbox}[colback=blue!5!white,colframe=blue!75!black]
  % Upper half of box: my "title" area
  \textcolor{blue}{\textbf{Naming Conventions}}
  % Lower half of the box: the content
  \tcblower
  The bit on the right end of any binary number is the \ac{LSB} because it has the least weight (the ones place) while the bit on the left end is the \ac{MSB} because it has the greatest weight. Also, groups of bits are normally referred to as \emph{words}, so engineers would speak of $ 16 $-bit or $ 32 $-bit words. As exceptions, an eight-bit group is commonly called a \emph{byte} and a four-bit group is called a \emph{nibble} (occasionally spelled \emph{nybble}).
\end{tcolorbox}
% End Sidebar Box

\subsection{Binary to Octal}
\label{MF:sub:binary_to_octal}
The octal numeration system serves as a ``shorthand'' method of denoting a large binary number. Technicians find it easier to discuss a number like $ 57_8 $ rather than $ 101111_2 $.\marginpar{``Five Seven Octal'' is not pronounced ``Fifty Seven'' since ``fifty'' is a decimal number.}

Because octal is a base eight system, and eight is $ 2^3 $, binary numbers can be converted to octal by creating groups of three and then simplifying each group. As an example, convert $ 101111_2 $ to octal: 

% Verbatim uses spaces, not tabs, for alignment. It renders in fixed-width font.
\begin{verbatim}
     101 111
      5   7 
\end{verbatim}

Thus, $ 101111_2 $ is equal to $ 57_8 $.

If a binary integer cannot be grouped into an even grouping of three, it is padded on the left with zeros. For example, to convert $ 11011101_2 $ to octal, the most significant bit must be padded with a zero:

\begin{verbatim}
     011 011 101
      3   3   5 
\end{verbatim}

Thus, $ 11011101_2 $ is equal to $ 335_8 $. 

A binary fraction may need to be padded on the right with zeros in order to create even groups of three before it is converted into octal. For example, convert $ 0.1101101_2 $ to octal: 

\begin{verbatim}
     0 . 110 110 100
     0 .  6   6   4 
\end{verbatim}

Thus, $ 0.1101101_2 $ is equal to $ .664_8 $. 

A binary mixed number may need to be padded on both the left and right with zeros in order to create even groups of three before it can be converted into octal. For example, convert $ 10101.00101_2 $ to octal: 

\begin{verbatim}
     010 101 . 001 010
      2   5  .  1   2 
\end{verbatim}

Thus, $ 10101.00101_2 $ is equal to $ 25.12_8 $. 

Table \ref{MF:tab:Bin_Oct_conversion_example} lists additional examples of binary/octal conversion:

\begin{table}[H]
  \sffamily
  \newcommand{\head}[1]{\textcolor{white}{\textbf{#1}}}    
  \begin{center}
    \rowcolors{2}{gray!10}{white} % Color every other line a light gray
    \begin{tabular}{ S S } 
      \hline
      \rowcolor{black!75}
      \head{Binary} & \head{Octal} \\ 
      \hline
      100101.011 & 45.3 \\ 
      1100010.1101 &  142.64 \\ 
      100101011.1101001 & 453.644 \\ 
      1110010011101.00011010 & 16235.064 \\ 
      110011010100111.011101 & 63247.35 \\
      \hline 
    \end{tabular} 
  \end{center}
  \caption{Binary-Octal Conversion Examples}
  \label{MF:tab:Bin_Oct_conversion_example}
\end{table}

While it is easy to convert between binary and octal, the octal system is not frequently used in electronics since computers store and transmit binary numbers in words of 16, 32, or 64 bits, which are multiples of four rather than three. 

\subsection{Binary to Hexadecimal}
\label{MF:sub:binary_to_hexadecimal}
The hexadecimal numeration system serves as a ``shorthand'' method of denoting a large binary number. Technicians find it easier to discuss a number like $ 2F_{16} $ rather than $ 101111_2 $. Because hexadecimal is a base $ 16 $ system, and $ 16 $ is $ 2^4 $; binary numbers can be converted to hexadecimal by creating groups of four and then simplifying each group. As an example, convert $ 10010111_2 $ to hexadecimal: 

\begin{verbatim}
     1001 0111
      9    7 
\end{verbatim}

Thus, $ 10010111_2 $ is equal to $ 97_{16} $.\marginpar{``Nine Seven Hexadecimal,'' or, commonly, ``Nine Seven Hex,'' is not pronounced ``Ninety Seven'' since ``ninety'' is a decimal number.}

A binary integer may need to be padded on the left with zeros in order to create even groups of four before it can be converted into hexadecimal. For example, convert $ 1001010110_2 $ to hexadecimal: 

\begin{verbatim}
     0010 0101 0110
      2    5    6 
\end{verbatim}

Thus, $ 1001010110_2 $ is equal to $ 256_{16} $. 

A binary fraction may need to be padded on the right with zeros in order to create even groups of four before it can be converted into hexadecimal. For example, convert $ 0.1001010110_2 $ to hexadecimal: 

\begin{verbatim}
     0 . 1001 0101 1000
       .  9    5    8 
\end{verbatim}

Thus, $ 0.1001010110_2 $ is equal to $ 0.958_{16} $. 

A binary mixed number may need to be padded on both the left and right with zeros in order to create even groups of four before it can be converted into hexadecimal. For example, convert $ 11101.10101_2 $ to hexadecimal: 

\begin{verbatim}
     0001 1101 . 1010 1000
      1    D   .  A    8 
\end{verbatim}

Thus, $ 11101.10101_2 $ is equal to $ 1D.A8_{16} $. 

Table \ref{MF:tab:bin_hex_conversion_example} lists additional examples of binary/hexadecimal conversion:

\begin{table}[H]
  \sisetup{input-digits={0123456789ABCDEF}}
  \sffamily
  \newcommand{\head}[1]{\textcolor{white}{\textbf{#1}}}    
  \begin{center}
    \rowcolors{2}{gray!10}{white} % Color every other line a light gray
    \begin{tabular}{ S S } 
      \hline
      \rowcolor{black!75}
      \head{Binary} & \head{Hexadecimal} \\
      \hline
      100101.011 & 25.6 \\ 
      1100010.1101 & 62.D \\ 
      100101011.1101001 & 12B.D2 \\ 
      1110010011101.00011010 & 1C9D.1A \\ 
      110011010100111.011101 & 66A7.74 \\
      \hline
    \end{tabular} 
  \end{center}
  \caption{Binary-Hexadecimal Conversion Examples}
  \label{MF:tab:bin_hex_conversion_example}  
\end{table}

\subsection{Octal to Decimal}
\label{MF:sub:octal_to_decimal}
The simplest way to convert an octal number to decimal is to write the octal number in expanded positional notation, calculate the values for each of the sets of parenthesis, and then add all of the values. For example, to convert $ 245_8 $ to decimal:

\begin{align}
  245_8 &= (2X8^2)+(4X8^1)+(5X8^0) \\
  \nonumber
  &= (2X64)+(4X8)+(5X1) \\
  \nonumber
  &= (128)+(32)+(5) \\
  \nonumber
  &= 165_{10}
\end{align}

If the octal number has a fractional component, then that part would be converted using negative powers of eight. As an example, convert $ 25.71_8 $ to decimal: 

\begin{align}
  25.71_8 &= (2X8^1)+(5X8^0)+(7X8^{-1})+(1X8^{-2}) \\
  \nonumber
  &= (2X8)+(5X1)+(7X0.125)+(1X0.015625) \\
  \nonumber
  &= (16)+(5)+(0.875)+(0.015625) \\
  \nonumber
  &= 21.890625_{10}
\end{align}

Other examples are:

\begin{align}
  42.6_8 &= (4X8^1)+(2X8^0)+(6X8^{-1}) \\
  \nonumber
  &= (4X8)+(2X1)+(6X0.125) \\
  \nonumber
  &= (32)+(2)+(0.75) \\
  \nonumber
  &= 34.75_{10}
\end{align}

\begin{align}
  32.54_8 &= (3X8^1)+(2X8^0)+(5X8^{-1})+(4X8^{-2}) \\
  \nonumber
  &= (3X8)+(2X1)+(5X0.125)+(4X0.015625) \\
  \nonumber
  &= (24)+(2)+(0.625)+(0.0625) \\
  \nonumber
  &= 26.6875_{10}
\end{align}

\begin{align}
  436.27_8 &= (4X8^2)+(3X8^1)+(6X8^0)+(2X8^{-1})+(7X8^{-2}) \\
  \nonumber
  &= (4X64)+(3X8)+(6X1)+(2X0.125)+(7X0.015625) \\
  \nonumber
  &= (256)+(24)+(6)+(0.25)+(0.109375) \\
  \nonumber
  &= 286.359375_{10}
\end{align}

\subsection{Hexadecimal to Decimal} 
\label{MF:sub:hexadecimal_to_decimal}
The simplest way to convert a hexadecimal number to decimal is to write the hexadecimal number in expanded positional notation, calculate the values for each of the sets of parenthesis, and then add all of the values. For example, to convert $ 2A6_{16} $ to decimal:

\begin{align}
  2A6_{16} &= (2X16^2)+(AX16^1)+(6X16^0) \\
  \nonumber
  &= (2X256)+(10X16)+(6X1) \\
  \nonumber
  &= (512)+(160)+(6) \\
  \nonumber
  &= 678_{10}
\end{align}


If the hexadecimal number has a fractional component, then that part would be converted using negative powers of $ 16 $. As an example, convert $ 1B.36_{16} $ to decimal:

\begin{align}
  1B.36_{16} &= (1X16^1)+(11X16^0)+(3X16^{-1})+(6X16^{-2}) \\
  \nonumber
  &= (16)+(11)+(3X\frac{1}{16})+(6X\frac{1}{256}) \\
  \nonumber
  &= 16+11+0.1875+0.0234375 \\
  \nonumber
  &= 27.2109375_{10}
\end{align}

Other examples are:

\begin{align}
  A32.1C_{16} &= (AX16^2)+(3X16^1)+(2X16^0)+(1X16^{-1})+(CX16^{-2}) \\
  \nonumber
  &= (10X256)+(3X16)+(2X1)+(1X\frac{1}{16})+(12X\frac{1}{256}) \\
  \nonumber
  &= 2560+48+2+0.0625+0.046875 \\
  \nonumber
  &= 6300.109375_{10}
\end{align}

\begin{align}
  439.A_{16} &= (4X16^2)+(3X16^1)+(9X16^0)+(AX16^{-1}) \\
  \nonumber
  &= (4X256)+(3X16)+(9X1)+(10X\frac{1}{16}) \\
  \nonumber
  &= 1024+48+9+0.625 \\
  \nonumber
  &= 1081.625_{10}
\end{align}

\subsection{Decimal to Binary} 
\label{MF:sub:decimal_to_binary}
\subsubsection{Integers} 
\label{MF:subsub:decimal_to_binary_integers}

\marginpar{Note: Converting decimal fractions is a bit different and is covered on page \pageref{MF:subsub:decimal_to_binary_fractions}.} Converting decimal integers to binary (indeed, any other radix) involves repeated cycles of division. In the first cycle of division, the original decimal integer is divided by the base of the target numeration system (binary=$ 2 $, octal=$ 8 $, hex=$ 16 $), and then the whole-number portion of the quotient is divided by the base value again. This process continues until the quotient is less than one. Finally, the binary, octal, or hexadecimal digits are determined by the ``remainders'' left over at each division step. \marginpar{After a decimal number is converted to binary it can be easily converted to either octal or hexadecimal.}

Table \ref{MF:tab:dec_to_bin_integer} shows how to convert $ 87_{10} $ to binary by repeatedly dividing $ 87 $ by $ 2 $ (the radix for binary) until reaching zero. The number in column one is divided by two and that quotient is placed on the next row in column one with the remainder in column two. For example, when $ 87 $ is divided by $ 2 $, the quotient is $ 43 $ with a remainder of one. This division process is continued until the quotient is less than one. When the division process is completed, the binary number is found by using the remainders, \emph{reading from the bottom to top}. Thus $ 87_{10} $ is $ 1010111_2 $.

\begin{table}[H]
  \sffamily
  \newcommand{\head}[1]{\textcolor{white}{\textbf{#1}}}    
  \begin{center}
    \rowcolors{2}{gray!10}{white} % Color every other line a light gray
    \begin{tabular}{ c c } 
      \hline
      \rowcolor{black!75}
      \head{Integer} & \head{Remainder} \\
      87 &  \\
      43 & 1 \\
      21 & 1 \\
      10 & 1 \\
      5 & 0 \\
      2 & 1 \\
      1 & 0 \\
      0 & 1 \\ \hline
    \end{tabular}
  \end{center}
  \caption{Decimal to Binary}
  \label{MF:tab:dec_to_bin_integer}
\end{table}

This repeat-division technique will also work for numeration systems other than binary. To convert a decimal integer to octal, for example, divide each line by $ 8 $; but follow the process as described above. As an example, Table \ref{MF:tab:dec_to_oct} shows how to convert $ 87_{10} $ to $ 127_8 $.

\begin{table}[H]
  \sffamily
  \newcommand{\head}[1]{\textcolor{white}{\textbf{#1}}}    
  \begin{center}
    \rowcolors{2}{gray!10}{white} % Color every other line a light gray
    \begin{tabular}{ c c } 
      \hline
      \rowcolor{black!75}
      \head{Integer} & \head{Remainder} \\
      87 &  \\
      10 & 7 \\
      1 & 2 \\
      0 & 1 \\ \hline
    \end{tabular}
  \end{center}
  \caption{Decimal to Octal}
  \label{MF:tab:dec_to_oct}
\end{table}

The same process can be used to convert a decimal integer to hexadecimal; except, of course, the divisor would be $ 16 $. Also, some of the remainders could be greater than $ 10 $, so these are written as letters. For example, to convert $ 678_{10} $ to $ 2A6_{16} $ use the process illustrated in Table \ref{MF:tab:dec_to_hex}.

\begin{table}[H]
  \sffamily
  \newcommand{\head}[1]{\textcolor{white}{\textbf{#1}}}    
  \begin{center}
    \rowcolors{2}{gray!10}{white} % Color every other line a light gray
    \begin{tabular}{ c c } 
      \hline
      \rowcolor{black!75}
      \head{Integer} & \head{Remainder} \\
      678 &  \\
      42 & 6 \\
      2 & A \\
      0 & 2 \\ \hline
    \end{tabular}
  \end{center}
  \caption{Decimal to Hexadecimal}
  \label{MF:tab:dec_to_hex}
\end{table}

\subsubsection{Fractions}
\label{MF:subsub:decimal_to_binary_fractions}
Converting decimal fractions to binary is a repeating operation similar to converting decimal integers, but each step repeats multiplication rather than division. To convert $ 0.8215_{10} $ to binary, repeatedly multiply the fractional part of the number by two until the fractional part is zero (or whatever degree of precision is desired). As an example, in Table \ref{MF:tab:dec_to_bin_fraction}, the number in column two, $ 8215 $, is multiplied by two and the integer part of the product is placed in column one on the next row while the fractional part in column two. Keep in mind that the ``Remainder'' is a decimal fraction with an assumed leading decimal point. That process continues until the fractional part reaches zero.

\begin{table}[H]
  \sffamily
  \newcommand{\head}[1]{\textcolor{white}{\textbf{#1}}}    
  \begin{center}
    \rowcolors{2}{gray!10}{white} % Color every other line a light gray
    \begin{tabular}{ c c } 
      \hline
      \rowcolor{black!75}
      \head{Integer} & \head{Remainder} \\
        & 8125 \\
      1 & 625 \\
      1 & 25 \\
      0 & 5 \\
      1 & 0 \\ \hline
    \end{tabular}
  \end{center}
  \caption{Decimal to Binary Fraction}
  \label{MF:tab:dec_to_bin_fraction}
\end{table}

When the multiplication process is completed, the binary number is found by using the integer parts and \emph{reading from the top to the bottom}. Thus $ 0.8125_{10} $ is $ 0.1101_2 $.

As another example, Table \ref{MF:tab:dec_to_bin_fraction_example_2} converts $ 0.78125_{10} $ to $ 0.11001_2 $. The solution was carried out to full precision (that is, the last multiplication yielded a fractional part of zero).

\begin{table}[H]
  \sffamily
  \newcommand{\head}[1]{\textcolor{white}{\textbf{#1}}}    
  \begin{center}
    \rowcolors{2}{gray!10}{white} % Color every other line a light gray
    \begin{tabular}[htb]{ c c } 
      \hline
      \rowcolor{black!75}
      \head{Integer} & \head{Remainder} \\
        & 78125 \\
      1 & 5625 \\
      1 & 125 \\
      0 & 25 \\
      0 & 5 \\
      1 & 0 \\ \hline
    \end{tabular}
  \end{center}
  \caption{Decimal to Binary Fraction Example}
  \label{MF:tab:dec_to_bin_fraction_example_2}
\end{table}

Often, a decimal fraction will create a huge binary fraction. In that case, continue the multiplication until the desired number of binary places are achieved. As an example, in Table \ref{MF:tab:dec_to_bin_long_fraction}, the fraction $ 0.1437_{10} $ was converted to binary, but the process stopped after $ 10 $ bits.

\begin{table}[H]
  \sffamily
  \newcommand{\head}[1]{\textcolor{white}{\textbf{#1}}}    
  \begin{center}
    \rowcolors{2}{gray!10}{white} % Color every other line a light gray
    \begin{tabular}{ c c } 
      \hline
      \rowcolor{black!75}
      \head{Integer} & \head{Remainder} \\
      & 1437 \\
      0 & 2874 \\
      0 & 5748 \\
      1 & 1496 \\
      0 & 2992 \\
      0 & 5984 \\
      1 & 1968 \\
      0 & 3936 \\
      0 & 7872 \\
      1 & 5744 \\
      1 & 1488 \\ \hline
    \end{tabular}
  \end{center}
  \caption{Decimal to Long Binary Fraction}
  \label{MF:tab:dec_to_bin_long_fraction}
\end{table}

Thus, $ 0.1437_{10} = 0.0010010011_2 $ (with $ 10 $ bits of precision). \marginpar{To calculate this to full precision requires $ 6617 $ bits; thus, it is normally wise to specify the desired precision.}

Converting decimal fractions to any other base would involve the same process, but the base is used as a multiplier. Thus, to convert a decimal fraction to hexadecimal multiply each line by $ 16 $ rather than $ 2 $. 

 \subsubsection{Mixed Numbers}
 \label{MF:subsub:decimal_to_binary_mixednumbers}
 To convert a mixed decimal number (one that contains both an integer and fraction part) to binary, treat each component as a separate problem and then combine the result. As an example, Table \ref{MF:tab:dec_to_bin_mixed_integer_part} and Table \ref{MF:tab:dec_to_bin_mixed_fraction_part} show how to convert $ 375.125_{10} $ to $ 1 0111 0111.001_2 $.
 
 \begin{table}[H]
  \sffamily
  \newcommand{\head}[1]{\textcolor{white}{\textbf{#1}}}    
   \begin{center}
    \rowcolors{2}{gray!10}{white} % Color every other line a light gray
     \begin{tabular}{ c c } 
       \hline
      \rowcolor{black!75}
      \head{Integer} & \head{Remainder} \\
       375 &  \\
       187 & 1 \\
       93 & 1 \\
       46 & 1 \\
       23 & 0 \\
       11 & 1 \\
       5 & 1 \\
       2 & 1 \\
       1 & 0 \\
       0 & 1 \\ \hline
     \end{tabular}
   \end{center}
   \caption{Decimal to Binary Mixed Integer}
   \label{MF:tab:dec_to_bin_mixed_integer_part}
 \end{table} 
 
 \begin{table}[H]
  \sffamily
  \newcommand{\head}[1]{\textcolor{white}{\textbf{#1}}}    
   \begin{center}
    \rowcolors{2}{gray!10}{white} % Color every other line a light gray
     \begin{tabular}{ c c } 
       \hline
      \rowcolor{black!75}
      \head{Integer} & \head{Remainder} \\
       & 125 \\
       0 & 25 \\
       0 & 5 \\
       1 & 0 \\ \hline
     \end{tabular}
   \end{center}
    \caption{Decimal to Binary Mixed Fraction}
    \label{MF:tab:dec_to_bin_mixed_fraction_part}
    \end{table} 
 
 A similar process could be used to convert decimal numbers into octal or hexadecimal, but those radix numbers would be used instead of two.
 
 \subsection{Calculators}
 \label{MF:sub:calculators}
 For the most part, converting numbers between the various ``computer'' bases (binary, octal, or hexadecimal) is done with a calculator. Using a calculator is quick and error-free. However, for the sake of applying digital logic to a mathematical problem, it is essential to understand the theory behind converting bases. It will not be possible to construct a digital circuit where one step is ``calculate the next answer on a hand-held calculator.'' Conversion circuits (like all circuits) need to be designed with simple gate logic, and an understanding of the theory behind the conversion process is important for that type of problem. 
 
 % Begin Sidebar Box
 \begin{tcolorbox}[colback=blue!5!white,colframe=blue!75!black]
   % Upper half of box: my "title" area
   \textcolor{blue}{\textbf{Online Conversion Tool}}
   % Lower half of the box: the content
   \tcblower
   Excel will convert between decimal/binary/octal/hexadecimal integers (including negative integers), but cannot handle fractions; however, the following website has a conversion tool that can convert between common bases, both integer and fraction: \url{http://baseconvert.com/}. An added benefit for this site is conversion with twos complement, which is how negative binary numbers are represented and is covered on page \pageref{MO:subsub:signed_complement}.
 \end{tcolorbox}
 % End Sidebar Box
 
 \subsection{Practice Problems}
 \label{MF:sub:practice_problems_convert_dec_bin_oct_hex}
 Table \ref{MF:tab:practice_problems_convert_dec_bin_oct_hex} lists several numbers in decimal, binary, octal, and hexadecimal form. To practice converting between numbers, select a number on any row and then covert it to the other bases.
 
 \begin{footnotesize}
 \begin{table}[H]
   \sisetup{input-digits={0123456789ABCDEF}}
  \sffamily
  \newcommand{\head}[1]{\textcolor{white}{\textbf{#1}}}    
   \begin{center}
     \rowcolors{2}{gray!10}{white} % Color every other line a light gray    
     \begin{tabular}{ S S S S } 
       \hline
       \rowcolor{black!75}
       \head{Decimal} & \head{Binary} & \head{Octal} & \head{Hexadecimal} \\
       \hline
       13.        & 1101.         & 15.    & D.    \\
       1872.      & 11101010000.  & 3520.  & 750.  \\
       0.0625     & 0.0001        & 0.04   & 0.1   \\
       0.45703125 & 0.01110101    & 0.352  & 0.75  \\
       43.125     & 101011.001    & 53.1   & 2B.2  \\
       108.71875  & 1101100.10111 & 154.56 & 6C.B8 \\
       \hline
     \end{tabular}
   \end{center}
   \caption{Practice Problems}
  \label{MF:tab:practice_problems_convert_dec_bin_oct_hex}
 \end{table} 
 \end{footnotesize}
 
\section{Floating Point Numbers}
\label{MF:sec:floating_point_numbers}

Numbers can take two forms: Fixed Point and Floating Point. A fixed point number generally has no fractional component and is used for integer operations (though it is possible to design a system with a fixed fractional width). On the other hand, a floating point number has a fractional component with a variable number of places.

Before considering how floating point numbers are stored in memory and manipulated, it is important to recall that any number can be represented using \emph{scientific notation}. Thus, $ 123.45_{10} $ can be represented as $ 1.2345 X 10^2 $ and $ 0.0012345_{10} $ can be represented as $ 1.2345 X 10^{-3} $. Numbers in scientific notation with one place to the left of the radix point, as illustrated in the previous sentence, are considered \emph{normalized}. While most people are familiar with normalized decimal numbers, the same process can be used for any other base, including binary. Thus, $ 1101.101_2 $ can be written as $ 1.101101 X 2^3 $. Notice that for normalized binary numbers the radix is two rather than ten since binary is a radix two system, also there is only one bit to the left of the radix point.

\marginpar{\ac{IEEE} Standard 754 defines Floating Point Numbers} By definition, floating point numbers are stored and manipulated in a computer using a 32-bit word (64 bits for ``double precision'' numbers). For this discussion, imagine that the number $ 10010011.0010_2 $ is to be stored as a floating point number. That number would first be normalized to $ 1.00100110010 X 2^7 $ and then placed into the floating point format:

\begin{verbatim}
     x    xxxxxxxx xxxxxxxxxxxxxxxxxxxxxxx
     sign exponent mantissa
\end{verbatim}

\begin{itemize}
  \item \textsc{Sign}: The sign field is a single bit that is zero for a positive number and one for a negative number. Since the example number is positive the sign bit is zero.

  \item \textsc{Exponent}: This is an eight-bit field containing the radix’s exponent, or $ 7 $ in the example. However, the field must be able to contain both positive and negative exponents, so it is offset by $ 127 $. The exponent of the example, $ 7 $, is stored as $ 7 + 127 $, or $ 134 $; therefore, the exponent field contains $ 10000110 $. 

  \item \textsc{Mantissa} (sometimes called \emph{significand}): This 23-bit field contains the number that is being stored, or $ 100100110010 $ in the example. While it is tempting to just place that entire number in the mantissa field, it is possible to squeeze one more bit of precision from this number with a simple adjustment. A normalized binary number will always have a one to the left of the radix point since in scientific notation a significant bit must appear in that position and one is the only possible significant bit in a binary number. Since the first bit of the stored number is assumed to be one it is dropped. Thus, the mantissa for the example number is $ 00100110010000000000000 $.

\end{itemize}

Here is the example floating point number (the spaces have been added for clarity):

\begin{verbatim}
     10010011.0010 = 0 10000110 00100110010000000000000
\end{verbatim}

A few floating point special cases have been defined:

\begin{itemize}
  \item \textsc{zero}: The exponent and mantissa are both zero and it does not matter whether the sign bit is one or zero. 
  \item \textsc{infinity}: The exponent is all ones and the mantissa is all zeros. The sign bit is used to represent either positive or negative infinity. 
  \item \textsc{\ac{NaN}}: The exponent is all ones and the mantissa has at least one one (it does not matter how many or where). \ac{NaN} is returned as the result of an illegal operation, like an attempted division by zero.
\end{itemize}

Two specific problems may show up with floating point calculations: 

\begin{itemize}
  \item \textsc{Overflow}. If the result of a floating point operation creates a positive number that is greater than $ (2-2^{-23})X 2^{127} $ it is a positive overflow or a negative number less than $ -(2-2^{-23})X 2^{127} $ it is a negative overflow. These types of numbers cannot be contained in a 32-bit floating point number; however, the designer could opt to increase the circuit to 64-bit numbers (called ``double precision'' floating point) in order to work with these large numbers.
  \item \textsc{Underflow}. If the result of a floating point operation creates a positive number that is less than $ 2^{-149} $ it is a positive underflow or a negative number greater than $ -2^{-149} $ it is a negative underflow. These numbers are vanishingly small and are sometimes simply rounded to zero. However, in certain applications, such as multiplication problems, even a tiny fraction is important and there are ways to use ``denormalized numbers'' that will sacrifice precision in order to permit smaller numbers. Of course, the circuit designer can always opt to use 64-bit (``double precision'') floating point numbers which would permit negative exponents about twice as large as 32-bit numbers.

\end{itemize}
Table \ref{MF:tab:floating_point_example} contains a few example floating point numbers.

\begin{table}[H]
  \small
  \sffamily
  {\footnotesize }
  \newcommand{\head}[1]{\textcolor{white}{\textbf{#1}}}    
  \begin{center}
    \rowcolors{2}{gray!10}{white} % Color every other line a light gray
    \begin{tabular}{ S S l l } 
      \hline
      \rowcolor{black!75}
      \head{{\footnotesize Decimal}} & \head{{\footnotesize Binary}} & \head{{\footnotesize Normalized}} & \head{{\footnotesize Floating Point}} \\
      \hline
      {\footnotesize 34.5}    & {\footnotesize 100010.1} & {\footnotesize $ 1.000101 X 2^5 $} &
      {\footnotesize 0 10000100 00010100000000000000000} \\
      {\footnotesize 324.75}  & {\footnotesize 101000100.11} & {\footnotesize $ 1.0100010011 X 2^8 $} &
      {\footnotesize 0 10000111 01000100010000111101100} \\
      {\footnotesize -147.25} & {\footnotesize 10010011.01} & {\footnotesize $ 1.001001101 X 2^7 $} &
      {\footnotesize 1 10000110 00100110100000000000000} \\
      {\footnotesize 0.0625}  & {\footnotesize 0.0001} & {\footnotesize $ 1.0 X 2^{-4} $} &
      {\footnotesize 0 01111011 00000000000000000000000} \\ \hline
    \end{tabular}
  \end{center}
  \caption{Floating Point Examples}
  \label{MF:tab:floating_point_example}
\end{table} 

\chapter{Binary Mathematics Operations}\label{ch03}
\section{Binary Addition}

Adding binary numbers is a simple task similar to the longhand addition of decimal numbers. As with decimal numbers, the bits are added one column at a time, from right to left. Unlike decimal addition, there is little to memorize in the way of an ``Addition Table,'' as seen in Table \ref{MO:tab:binary_addition_table}

\begin{table}[H]
  \sffamily
  \newcommand{\head}[1]{\textcolor{white}{\textbf{#1}}}    
  \begin{center}
    \rowcolors{2}{gray!10}{white} % Color every other line a light gray
    \begin{tabular}{ c c c | c c } \hline
      \multicolumn{3}{c|}{\textbf{Inputs}} & \multicolumn{2}{c}{\textbf{Outputs}} \\
      \hline
      \rowcolor{black!75}
      \head{Carry In} & \head{Augend} & \head{Addend} & \head{Sum} & \head{Carry Out} \\
      \hline
      0        & 0      & 0      & 0   & 0 \\
      0        & 0      & 1      & 1   & 0 \\
      0        & 1      & 0      & 1   & 0 \\
      0        & 1      & 1      & 0   & 1 \\
      1        & 0      & 0      & 1   & 0 \\
      1        & 0      & 1      & 0   & 1 \\
      1        & 1      & 0      & 0   & 1 \\
      1        & 1      & 1      & 1   & 1 \\ \hline
    \end{tabular}
  \end{center}
  \caption{Addition Table}
  \label{MO:tab:binary_addition_table}
\end{table} 

Just as with decimal addition, two binary integers are added one column at a time, starting from the \ac{LSB} (the right-most bit in the integer): 

\begin{binDisp}[commandchars=~\[\]]
      1001101
     +~underline[0010010]
      1011111
\end{binDisp}

When the sum in one column includes a carry out, it is added to the next column to the left (again, like decimal addition). Consider the following examples:

\begin{binDisp}[commandchars=~\[\]]
       11  1  <--Carry Bits
      1001001
     +~underline[0011001]
      1100010
\end{binDisp}

\begin{binDisp}[commandchars=~\[\], samepage=true]
         11   <--Carry Bits
      1000111
     +~underline[0010110]
      1011101
\end{binDisp}

The ``ripple-carry'' process is simple for humans to understand, but it causes a significant problem for designers of digital circuits. Consequently, ways were developed to carry a bit to the left in an electronic adder circuit and that is covered in Section \ref{CL:sec:adders_and_subtractors}, page \pageref{CL:sec:adders_and_subtractors}.

Binary numbers that include a fractional component are added just like binary integers; however, the radix points must align so the augend and addend may need to be padded with zeroes on either the left or the right. Here is an example: 

\begin{binDisp}[commandchars=~\[\]]
       111 1    <--Carry Bits
      1010.0100
     +~underline[0011.1101]
      1110.0001
\end{binDisp}

\subsection{Overflow Error}
\label{MO:sub:overflow_error}

One problem circuit designers must consider is a carry out bit in the \ac{MSB} (left-most bit) in the answer. Consider the following:

\begin{binDisp}[commandchars=~\[\]]
      11 11    <--Carry Bits
      10101110
     +~underline[11101101]
     110011011
\end{binDisp}

This example illustrates a significant problem for circuit designers. Suppose the above calculation was done with a circuit that could only accommodate eight data bits. The augend and addend are both eight bits wide, so they are fine; however, the sum is nine bits wide due to the carry out in the \ac{MSB}. In an eight-bit circuit (that is, a circuit where the devices and data lines can only accommodate eight bits of data), the carry out bit would be dropped since there is not enough room to accommodate it.

The result of a dropped bit cannot be ignored. The example problem above, when calculated in decimal, is $ 174_{10} + 237_{10} = 411_{10} $. If, though, the \ac{MSB} carry out is dropped, then the answer becomes $ 155_{10} $, which is, of course, incorrect. This type of error is called an \emph{Overflow Error}, and a circuit designer must find a way to correct overflow. One typical solution is to simply alert the user that there was an overflow error. For example, on a handheld calculator, the display may change to something like \emph{-E-} if there is an error of any sort, including overflow. 

\subsection{Sample Binary Addition Problems}
\label{MO:sub:sample_binary_addition_problems}
The following table lists several binary addition problems that can be used for practice.

\begin{table}[H]
  \sffamily
  \newcommand{\head}[1]{\textcolor{white}{\textbf{#1}}}    
  \begin{center}
    \rowcolors{2}{gray!10}{white} % Color every other line a light gray
    \begin{tabular}{ S S S }
      \hline
      \rowcolor{black!75}
      {\head{Augend}} & {\head{Addend}} & {\head{Sum}}     \\
      \hline
      10110.   & 11101.   & 110011.   \\ 
      111010.  & 110011.  & 1101101.  \\
      1011.    & 111000.  & 1000011.  \\ 
      1101001. & 11010.   & 10000011. \\ 
      1010.111 & 1100.001 & 10111.000 \\ 
      101.01   & 1001.001 & 1110.011  \\ 
      \hline
    \end{tabular}
  \end{center}
  \caption{Binary Addition Problems}
  \label{MO:tab:binary_addition_problems}
\end{table} 

\section{Binary Subtraction}
\label{MO:sec:binary_subtraction}
\subsection{Simple Manual Subtraction}
\label{MO:sub:simple_manual_subtraction}

Subtracting binary numbers is similar to subtracting decimal numbers and uses the same process children learn in primary school. The minuend and subtrahend are aligned on the radix point, and then columns are subtracted one at a time, starting with the least significant place and moving to the left. If the subtrahend is larger than the minuend for any one column, an amount is ``borrowed'' from the column to the immediate left. Binary numbers are subtracted in the same way, but it is important to keep in mind that binary numbers have only two possible values: zero and one. Consider the following problem: 

\begin{binDisp}[commandchars=~\[\]]
      10.1
     -~underline[01.0]
      01.1
\end{binDisp}

In this problem, the \ac{LSB} column is $ 1 - 0 $, and that equals one. The middle column, though, is $ 0 - 1 $, and one cannot be subtracted from zero. Therefore, one is borrowed from the most significant bit, so the problem in middle column becomes $ 10 - 1 $. (Note: do not think of this as ``ten minus one'' - remember that this is binary so this problem is ``one-zero minus one,'' or two minus one in decimal) The middle column is $ 10 - 1 = 1 $, and the \ac{MSB} column then becomes $ 0 - 0 = 0 $. 

The radix point must be kept in alignment throughout the problem, so if one of the two operands has too few places it is padded on the left or right (or both) to make both operands the same length. As an example, subtract: $ 101101.01 - 1110.1 $: 

\begin{binDisp}[commandchars=~\[\], samepage=true]
      101101.01
     -~underline[001110.10]
       11110.11
\end{binDisp}

There is no difference between decimal and binary as far as the subtraction process is concerned. In each of the problems in this section the minuend is greater than the subtrahend, leading to a positive difference; however, if the minuend is less than the subtrahend, the result is a negative number and negative numbers are developed in the next section of this chapter. 

Table \ref{MO:tab:binary_subtraction_problems} includes some subtraction problems for practice: 

\begin{table}[H]
  \sffamily
  \newcommand{\head}[1]{\textcolor{white}{\textbf{#1}}}    
  \begin{center}
    \rowcolors{2}{gray!10}{white} % Color every other line a light gray
    \begin{tabular}{ S S S }
      \hline
      \rowcolor{black!75}
      {\head{Minuend}} & {\head{Subtrahend}} & {\head{Difference}}     \\
      \hline
      1001011.      & 0111010.    & 10001.   \\ 
      100010.       & 010010.     & 10000.  \\
      101110110.    & 11001010.   & 10101100.  \\ 
      1110101.      & 111010.     & 111011. \\ 
      11011010.1101 & 101101.1    & 10101101.0101 \\ 
      10101101.1    & 1101101.101 & 111111.111  \\ 
      \hline
    \end{tabular}
  \end{center}
  \caption{Binary Subtraction Problems}
  \label{MO:tab:binary_subtraction_problems}
\end{table} 

\subsection{Representing Negative Binary Numbers Using Sign-and-Magnitude}
\label{MO:sub:representing_negative_sign_magnitude}

\marginpar{Sign-and-magnitude was used in early computers since it mimics real number arithmetic, but has been replaced by more efficient negative number systems in modern computers.} Binary numbers, like decimal numbers, can be both positive and negative. While there are several methods of representing negative binary numbers; one of the most intuitive is using \emph{sign-and-magnitude}, which is essentially the same as placing a ``–'' in front of a decimal number. With the sign-and-magnitude system, the circuit designer simply designates the \ac{MSB} as the \emph{sign bit} and all others as the magnitude of the number. When the sign bit is one the number is negative, and when it is zero the number is positive. Thus, $ -5_{10} $ would be written as $ 1101_2 $. 

Unfortunately, despite the simplicity of the sign-and-magnitude approach, it is not very practical for binary arithmetic, especially when done by a computer. For instance, negative five ($ 1101_2 $) cannot be added to any other binary number using standard addition technique since the sign bit would interfere. As a general rule, errors can easily occur when bits are used for any purpose other than standard place-weighted values; for example, $ 1101_2 $ could be misinterpreted as the number $ 13_{10} $ when, in fact, it is meant to represent $ -5 $. To keep things straight, the circuit designer must first decide how many bits are going to be used to represent the largest numbers in the circuit, add one more bit for the sign, and then be sure to never exceed that bit field length in arithmetic operations. For the above example, three data bits plus a sign bit would limit arithmetic operations to numbers from negative seven ($ 1111_2 $) to positive seven ($ 0111_2 $), and no more. 

This system also has the quaint property of having two values for zero. If using three magnitude bits, these two numbers are both zero: $ 0000_2 $ (positive zero) and $ 1000_2 $ (negative zero). 

\subsection{Representing Negative Binary Numbers Using Signed Complements }
\label{MO:sub:representing_negative_sign_complement}

\subsubsection{About Complementation}
\label{MO:subsub:about_complementation}

Before discussing negative binary numbers, it is important to understand the concept of complementation. To start, recall that the \emph{radix} (or base) of any number system is the number of ciphers available for counting; the decimal (or base-ten) number system has ten ciphers ($ 0, 1, 2, 3, 4, 5, 6, 7, 8, 9 $) while the binary (or base-two) number system has two ciphers ($ 0, 1 $). By definition, a number plus its complement equals the radix (this is frequently called the \emph{radix complement}). For example, in the decimal system four is the radix complement of six since $ 4 + 6 = 10 $. Another type of complement is the \emph{diminished radix complement}, which is the complement of the radix minus one. For example, in the decimal system six is the diminished radix complement of three since $ 6 + 3 = 9 $ and nine is equal to the radix minus one.

\paragraph{Decimal.} In the decimal system the radix complement is usually called the \emph{tens} complement since the radix of the decimal system is ten. Thus, the tens complement of eight is two since $ 8 + 2 = 10 $. The diminished radix complement is called the \emph{nines} complement in the decimal system. As an example, the nines complement of decimal eight is one since $ 8 + 1 = 9 $ and nine is the diminished radix of the decimal system. 

To find the nines complement for a number larger than one place, the nines complement must be found for each place in the number. For example, to find the nines complement for $ 538_{10} $, find the nines complement for each of those three digits, or $ 461 $. The easiest way to find the tens complement for a large decimal number is to first find the nines complement and then add one. For example, the tens complement of $ 283 $ is $ 717 $, which is calculated by finding the nines complement, $ 716 $, and then adding one.

\paragraph{Binary.} Since the radix for a binary number is $ 10_2 $, (be careful! this is not ten, it is one-zero in binary) the diminished radix is $ 1_2 $. The diminished radix complement is normally called the \emph{ones complement} and is obtained by reversing (or ``flipping'') each bit in a binary number; so the ones complement of $ 100101_2 $ is $ 011010_2 $. 

The radix complement (or \emph{twos complement}) of a binary number is found by first calculating the ones complement and then adding one to that number. The ones complement of $ 101101_2 $ is $ 010010_2 $, so the twos complement is $ 010010_2 + 1_2 = 010011_2 $.  

\subsubsection{Signed Complements} 
\label{MO:subsub:signed_complement}
In circuits that use binary mathematics, a circuit designer can opt to use ones complement for negative numbers and designate the most significant bit as the sign bit; and, if so, the other bits are the magnitude of the number. This is similar to the \emph{sign-and-magnitude} system discussed on page \pageref{MO:sub:representing_negative_sign_magnitude}. By definition, when using ones complement negative numbers, if the most significant bit is zero, then the number is positive and the magnitude of the number is determined by the remaining bits; but if the most significant bit is one, then the number is negative and the magnitude of the number is determined by calculating the ones complement of the number. Thus: $ 0111_2 = +7_{10} $, and $ 1000_2 = -7_{10} $ (the ones complement for $ 1000_2 $ is $ 0111_2 $). Table \ref{MO:tab:ones_complement} may help to clarify this concept: 

\begin{table}[H]
  \sffamily
  \newcommand{\head}[1]{\textcolor{white}{\textbf{#1}}}    
  \begin{center}
    \rowcolors{2}{gray!10}{white} % Color every other line a light gray
    \begin{tabular}{ S c c } 
      \hline
      \rowcolor{black!75}
      {\head{Decimal}} & {\head{Positive}} & {\head{Negative}} \\
      \hline
      0 & 0000 & 1111 \\
      1 & 0001 & 1110 \\
      2 & 0010 & 1101 \\
      3 & 0011 & 1100 \\
      4 & 0100 & 1011 \\
      5 & 0101 & 1010 \\
      6 & 0110 & 1001 \\
      7 & 0111 & 1000 \\
      \hline
    \end{tabular}
  \end{center}
  \caption{Ones Complement}
  \label{MO:tab:ones_complement}
\end{table} 

In a four-bit binary number, any decimal number from $ -7 $ to $ +7 $ can be represented; but, notice that, like the sign-and-magnitude system, there are two values for zero, one positive and one negative. This requires extra circuitry to test for both values of zero after subtraction operations.

To simplify circuit design, a designer can opt to use twos complement negative numbers and designate the most significant bit as the sign bit so the other bits are the number's magnitude. To use twos complement numbers, if the most significant bit is zero, then the number is positive and the magnitude of the number is determined by the remaining bits; but if the most significant bit is one, then the number is negative and the magnitude of the number is determined by taking the twos complement of the number (that is, the ones complement plus one). Thus: $ 0111 = 7 $, and $ 1001 = -7 $ (the ones complement of $ 1001 $ is $ 0110 $, and $ 0110 + 1 = 0111 $). Table \ref{MO:tab:twos_complement} may help to clarify this concept:

\begin{table}[H]
  \sffamily
  \newcommand{\head}[1]{\textcolor{white}{\textbf{#1}}}    
  \begin{center}
    \rowcolors{2}{gray!10}{white} % Color every other line a light gray
    \begin{tabular}{ S c c } 
      \hline
      \rowcolor{black!75}
      {\head{Decimal}} & {\head{Positive}} & {\head{Negative}} \\
      \hline
      0 & 0000 & 10000 \\
      1 & 0001 & 1111 \\
      2 & 0010 & 1110 \\
      3 & 0011 & 1101 \\
      4 & 0100 & 1100 \\
      5 & 0101 & 1011 \\
      6 & 0110 & 1010 \\
      7 & 0111 & 1001 \\
      8 & N/A & 1000 \\
      \hline
    \end{tabular}
  \end{center}
  \caption{Twos Complement}
  \label{MO:tab:twos_complement}
\end{table} 

The twos complement removes that quirk of having two values for zero. Table \ref{MO:tab:twos_complement} shows that zero is either $ 0000 $ or $ 10000 $; but since this is a four-bit number the initial one is discarded, leaving $ 0000 $ for zero whether the number is positive or negative. Also, $ 0000 $ is considered a positive number since the sign bit is zero. Finally, notice that $ 1000 $ is $ -8 $ (ones complement of $ 1000 $ is $ 0111 $, and $ 0111 + 1 = 1000 $). \marginpar{Programmers reading this book may have wondered why the maximum/minimum values for various types of variables is asymmetrical.}This means that binary number systems that use a twos complement method of designating negative numbers will be asymmetrical; running, for example, from $ -8 $ to $ +7 $. A twos complement system still has the same number of positive and negative numbers, but zero is considered positive, not neutral.

\marginpar{All modern computer systems use radix (or twos) complements to represent negative numbers.} One other quirk about the twos complement system is that the decimal value of the binary number can be quickly calculated by assuming the sign bit has a negative place value and all other places are added to it. For example, in the negative number $ 1010_2 $, if the sign bit is assumed to be worth $ -8 $ and the other places are added to that, the result is $ -8+2 $, or $ -6 $; and $ -6 $ is the value of $ 1010_2 $ in a twos complement system.

\subsubsection{About Calculating the Twos Complement}
\label{MO:subsub:about_calculating_twos_complement}

In the above section, the twos (or radix) complement is calculated by finding the ones complement of a number and then adding one. For machines, this is the most efficient method of calculating the twos complement; but there is a method that is much easier for humans to use to find the twos complement of a number. Start with the \ac{LSB} (the right-most bit) and then read the number from right to left. Look for the first one and then invert every bit to the left of that one. As an example, the twos complement for $ 1010\underline{10}_2 $ is formed by starting with the least significant bit (the zero on the right), and working to the left, looking for the first one, which is in the second place from the right. Then, every bit to the left of that one is inverted, ending with: $ 0101\underline{10}_2 $ (the two \acp{LSB} are underlined to show that they are the same in both the original and twos complement number).

Table \ref{MO:tab:example_twos_comp} displays a few examples:

\begin{table}[H]
  \sisetup{parse-numbers = false}
  \sffamily
  \newcommand{\head}[1]{\textcolor{white}{\textbf{#1}}}    
  \begin{center}
    \rowcolors{2}{gray!10}{white} % Color every other line a light gray
    \begin{tabular}{ r l } 
      \hline
      \rowcolor{black!75}
      {\head{Number}} & {\head{Twos Complement}} \\
      \hline
      0110100   & 1001100   \\
      11010     & 00110     \\
      001010    & 110110    \\
      1001011   & 0110101   \\
      111010111 & 000101001 \\
      \hline
    \end{tabular}
  \end{center}
  \caption{Example Twos Complement}
  \label{MO:tab:example_twos_comp}
\end{table} 

\subsection{Subtracting Using the Diminished Radix Complement }
\label{MO:sub:subtracting_using_diminished_radix}

When thinking about subtraction, it is helpful to remember that $ A - B $ is the same as $ A + (-B) $. Computers can find the complement of a particular number and add it to another number much faster and easier than attempting to create separate subtraction circuits. Therefore, subtraction is normally carried out by adding the complement of the subtrahend to the minuend.

\subsubsection{Decimal}
\label{MO:subsub:decimal_subtraction_with_diminished_radix}

\marginpar{This method is commonly used by stage performers who can subtract large numbers in their heads. While it seems somewhat convoluted, it is fairly easy to master.} It is possible to subtract two decimal numbers by adding the nines complement, as in the following example:

\begin{binDisp}[commandchars=~\[\]]
      735
     -~underline[142]
\end{binDisp}

Calculate the nines complement of the subtrahend: $ 857 $ (that is $ 9-1 $, $ 9-4 $, and $ 9-2 $). Then, add that nines complement to the minuend: 

\begin{binDisp}[commandchars=~\[\]]
      735
     +~underline[857]
     1592
\end{binDisp}

The initial one in the sum (the thousands place) is dropped so the number of places in the answer is the same as for the two addends, leaving $ 592 $. Because the diminished radix used to create the subtrahend is one less than the radix, one must be added to the answer; giving $ 593 $, which is the correct answer for $ 735-142 $.

\subsubsection{Binary}
\label{MO:subsub:binary_subtraction_with_diminished_radix}

The diminished radix complement (or ones complement) of a binary number is found by simply ``flipping'' each bit. Thus, the ones complement of $ 11010 $ is $ 00101 $. Just as in decimal, a binary number can be subtracted from another by adding the diminished radix complement of the subtrahend to the minuend, and then adding one to the sum. Here is an example: 

\begin{binDisp}[commandchars=~\[\]]
      101001
     -~underline[011011]
\end{binDisp}

Add the ones complement of the subtrahend:

\begin{binDisp}[commandchars=~\[\]]
      101001
     +~underline[100100]
     1001101
\end{binDisp}

The most significant bit is discarded so the solution has the same number of bits as for the two addends. This leaves $ 001101_2 $ and adding one to that number (because the diminished radix is one less than the radix) leaves $ 1110_2 $. In decimal, the problem is $ 41-27=14 $.

Often, diminished radix subtraction circuits are created such that they use \emph{end around} carry bits. In this case, the most significant bit is carried around and added to the final sum. If that bit is one, then that increases the final answer by one, and the answer is a positive number. If, though, the most significant bit is zero, then there is no end around carry so the answer is negative and must be complemented to find the true value. Either way, the correct answer is found. 

Here is an example:

\begin{binDisp}[commandchars=~\[\]]
      0110  (6)
     -~underline[0010  (2)]
\end{binDisp}

Solution:

\begin{binDisp}[commandchars=~\[\]]
      0110  (6)
     +~underline[1101  (-2 in ones complement)]
     10011
         1  (End-around carry the MSB)
     =0100  (4)
\end{binDisp}

Answer: 4 (since there was an end-around carry the solution is a positive number). Here is a second example:

\begin{binDisp}[commandchars=~\[\], samepage=true]
      0010  (2)
     -~underline[0110  (6)]
\end{binDisp}

Solution:

\begin{binDisp}[commandchars=~\[\], samepage=true]
      0010  (2)
     +~underline[1001  (-6 in ones complement)]
      1011  (No end-around carry, so ones complement)
     =0100  (-4: no end-around carry so negative answer)
\end{binDisp}

Because the diminished radix (or ones) complement of a binary number includes that awkward problem of having two representations for zero, this form of subtraction is not used in digital circuits; instead, the radix (or twos) complement is used (this process is discussed next). It is worth noting that subtracting by adding the diminished radix of the subtrahend and then adding one is awkward for humans, but complementing and adding is a snap for digital circuits. In fact, many early mechanical calculators used a system of adding complements rather than having to turn gears backwards for subtraction. 

\subsection{Subtracting Using the Radix Complement}
\label{MO:sub:subtracting_using_radix_complement}

\subsubsection{Decimal}
\label{MO:subsub:decimal_subtraction_with_radix_complement}

The radix (or tens) complement of a decimal number is the nines complement plus one. Thus, the tens complement of $ 7 $ is $ 3 $; or ($ (9-7)+1 $) and the tens complement of $ 248 $ is $ 752 $ (find the nines complement of each place and then add one to the complete number: $ 751 + 1 $). It is possible to subtract two decimal numbers using the tens complement, as in the following example: 

\begin{binDisp}[commandchars=~\[\]]
     735
    -~underline[142]
\end{binDisp}

Calculate the tens complement of the subtrahend, $ 142 $, by finding the nines complement for each digit and then adding one to the complete number: $ 858 $ (that is $ 9-1 $, $ 9-4 $, and $ 9-2+1 $). Then, add that tens complement number to the original minuend: 

\begin{binDisp}[commandchars=~\[\]]
     735
    +~underline[858]
    1593
\end{binDisp}

The initial one in the answer (the thousands place) is dropped so the answer has the same number of decimal places as the addends, leaving $ 593 $, which is the correct answer for $ 735-142 $.

\subsubsection{Binary}
\label{MO:subsub:binary_subtraction_with_radix_complement}

To find the radix (or twos) complement of a binary number, each bit in the number is ``flipped'' (making the ones complement) and then one is added to the result. Thus, the twos complement of $ 11010_2 $ is $ 00110_2 $ (or $ (00101_2)+1_2 $). Just as in decimal, a binary number can be subtracted from another by adding the radix complement of the subtrahend to the minuend. Here's an example: 

\begin{binDisp}[commandchars=~\[\]]
     101001
    -~underline[011011]
\end{binDisp}

Add the twos complement of the subtrahend:

\begin{binDisp}[commandchars=~\[\]]
     101001
    +~underline[011011]
    1001110
\end{binDisp}

The most significant bit is discarded so the solution has the same number of bits as for the two addends. This leaves $ 001110_2 $ (or $ 14_{10} $). Converting all of this to decimal, the original problem is $ 41-27=14 $.

Here are two worked out examples:

Calculate $ 0110_2 - 0010_2 $ (or $ 6_{10} - 2_{10} $):

\begin{binDisp}[commandchars=~\[\]]
     0110  (6)
    -~underline[0010  (2)]
\end{binDisp}

Solution:

\begin{binDisp}[commandchars=~\[\]]
     0110  (6)
    +~underline[1110  (-2 in twos complement)]
    10100  (Discard the MSB, the answer is 4)
\end{binDisp}

Calculate $ 0010_2 - 0110_2 $ (or $ 2_{10} - 6_{10} $)

\begin{binDisp}[commandchars=~\[\]]
     0010  (2)
    -~underline[0110  (6)]
\end{binDisp}

Solution:

\begin{binDisp}[commandchars=~\[\]]
     0010  (2)
    +~underline[1010  (-6 in twos complement)]
     1100
     0100  (Twos complement of the sum, -4)
\end{binDisp}

\subsection{Overflow}
\label{MO:sub:overflow}

One caveat with signed binary numbers is that of overflow, where the answer to an addition or subtraction problem exceeds the magnitude which can be represented with the allotted number of bits. Remember that the sign bit is defined as the most significant bit in the number. For example, with a six-bit number, five bits are used for magnitude, so there is a range from $ 00000_2 $ to $ 11111_2 $, or $ 0_{10} $ to $ 31_{10} $. If a sign bit is included, and using twos complement, numbers as high as $ 011111_2 $ ($ +31_{10} $) or as low as $ 100000_2 $ ($ -32_{10} $) are possible. However, an addition problem with two signed six-bit numbers that results in a sum greater than $ +31_{10} $ or less than $ -32_{10} $ will yield an incorrect answer. As an example, add $ 17_{10} $ and $ 19_{10} $ with signed six-bit numbers: 

\begin{binDisp}[commandchars=~\[\]]
     010001  (17)
    +~underline[010011  (19)]
     100100
\end{binDisp}

The answer ($ 100100_2 $), interpreted with the most significant bit as a sign, is equal to $ -28_{10} $, not $ +36_{10} $ as expected. Obviously, this is not correct. The problem lies in the restrictions of a six-bit number field. Since the true sum ($ 36 $) exceeds the allowable limit for our designated bit field (five magnitude bits, or $ +31 $), it produces what is called an overflow error. Simply put, six places is not large enough to represent the correct sum if the \ac{MSB} is being used as a sign bit, so whatever sum is obtained will be incorrect. A similar error will occur if two negative numbers are added together to produce a sum that is too small for a six-bit binary field. As an example, add $ -17_{10} $ and $ -19_{10} $: 

\begin{binDisp}[commandchars=~\[\]]
     -17 = 101111
     -19 = 101101

     101111  (-17)
    +~underline[101101  (-19)]
    1011100
\end{binDisp}

The solution as shown: $ 011100_2 $ = $ +28_{10} $. (Remember that the most significant bit is dropped in order for the answer to have the same number of places as the two addends.) The calculated (incorrect) answer for this addition problem is $ 28 $ because true sum of $ -17 + -19 $ was too small to be properly represented with a five bit magnitude field.

Here is the same overflow problem again, but expanding the bit field to six magnitude bits plus a seventh sign bit. In the following example, both $ 17 + 19 $ and $ (-17) + (-19) $ are calculated to show that both can be solved using a seven-bit field rather than six-bits. 

Add 17 + 19:

\begin{binDisp}[commandchars=~\[\], samepage=true]
     0010001  (17)
    +~underline[0010011  (19)]
     0100100  (36)
\end{binDisp}

Add (-17) + (-19):

\begin{binDisp}[commandchars=~\[\], samepage=true]
     -17 = 1101111
     -19 = 1101101

     1101111  (-17)
    +~underline[1101101  (-19)]
    11011100  (-36)
\end{binDisp}

The correct answer is only found by using bit fields sufficiently large to handle the magnitude and sign bits in the sum.

\subsubsection{Error Detection}
\label{MO:subsub:error_detection}

Overflow errors in the above problems were detected by checking the problem in decimal form and then comparing the results with the binary answers calculated. For example, when adding $ +17 $ and $ +19 $, the answer was supposed to be $ +36 $, so when the binary sum was $ -28 $, something had to be wrong. Although this is a valid way of detecting overflow errors, it is not very efficient, especially for computers. After all, the whole idea is to reliably add binary numbers together and not have to double-check the result by adding the same numbers together in decimal form. This is especially true when building logic circuits to add binary quantities: the circuit must detect an overflow error without the supervision of a human who already knows the correct answer.

The simplest way to detect overflow errors is to check the sign of the sum and compare it to the signs of the addends. Obviously, two positive numbers added together will give a positive sum and two negative numbers added together will give a negative sum. With an overflow error, however, the sign of the sum is always opposite that of the two addends: $ (+17) + (+19) = -28 $ and $ (-17) + (-19) = +28 $. By checking the sign bits an overflow error can be detected. 

It is not possible to generate an overflow error when the two addends have opposite signs. The reason for this is apparent when the nature of overflow is considered. Overflow occurs when the magnitude of a number exceeds the range allowed by the size of the bit field. If a positive number is added to a negative number then the sum will always be closer to zero than either of the two added numbers; its magnitude must be less than the magnitude of either original number, so overflow is impossible.

\section{Binary Multiplication}
\label{MO:sec:binary_multiplication}

\subsection{Multiplying Unsigned Numbers}
\label{MO:sub:multiplying_unsigned_numbers}

Multiplying binary numbers is very similar to multiplying decimal numbers. There are only four entries in the Binary Multiplication Table:

\begin{table}[H]
  \sffamily
  \newcommand{\head}[1]{\textcolor{white}{\textbf{#1}}}    
  \begin{center}
    \rowcolors{1}{gray!10}{white} % Color every other line a light gray
    \begin{tabular}{ c } 
      $ 0X0=0 $   \\
      $ 0X1=0 $   \\
      $ 1X0=0 $   \\
      $ 1X1=1 $
    \end{tabular}
  \end{center}
  \caption{Binary Multiplication Table}
  \label{MO:tab:binary_multiplication}
\end{table} 

To multiply two binary numbers, work through the multiplier one number at a time (right-to-left) and if that number is one, then shift left and copy the multiplicand as a partial product; if that number is zero, then shift left but do not copy the multiplicand (zeros can be used as placeholders if desired). When the multiplying is completed add all partial products. This sounds much more complicated than it actually is in practice and is the same process that is used to multiply two decimal numbers. Here is an example problem.

\begin{binDisp}[commandchars=~\[\]]
        1011  (11)
      X ~underline[1101  (13)]
        1011
       0000
      1011
     ~underline[1011          ]
    10001111  (143)
\end{binDisp}

\subsection{Multiplying Signed Numbers}
\label{MO:sub:multiplying_signed_numbers}

The simplest method used to multiply two numbers where one or both are negative is to use the same technique that is used for decimal numbers: multiply the two numbers and then determine the sign from the signs of the original numbers: if those signs are the same then the result is positive, if they are different then the result is negative. Multiplication by zero is a special case where the result is always zero.

The multiplication method discussed above works fine for paper-and-pencil; but is not appropriate for designing binary circuits. Unfortunately, the mathematics for binary multiplication using an algorithm that can become an electronic circuit is beyond the scope of this book. Fortunately, though, \acp{IC} already exist that carry out multiplication of both signed and floating-point numbers, so a circuit designer can use a pre-designed circuit and not worry about the complexity of the multiplication process.

\section{Binary Division}
\label{MO:sec:binary_division}

Binary division is accomplished by repeated subtraction and a right shift function; the reverse of multiplication. The actual process is rather convoluted and complex and is not covered in this book. Fortunately, though, \acp{IC} already exist that carry out division of both signed and floating-point numbers, so a circuit designer can use a pre-designed circuit and not worry about the complexity of the division process.

\section{Bitwise Operations}
\label{MO:sec:bitwise_operations}

It is sometimes desirable to find the value of a given bit in a byte. For example, if the \ac{LSB} is zero then the number is even, but if it is one then the number is odd. To determine the ``evenness'' of a number, a bitwise mask is multiplied with the original number. As an example, imagine that it is desired to know if $ 1001010_2 $ is even, then:

\begin{binDisp}[commandchars=~\[\]]
         1001010  <- Original Number
    BitX ~underline[0000001]  <- "Evenness" Mask
         0000000
\end{binDisp}

The bits are multiplied one position at a time, from left-to-right. Any time a zero appears in the mask that bit position in the product will be zero since any number multiplied by zero yields zero. When a one appears in the mask, then the bit in the original number will be copied to the solution. In the given example, the zero in the least significant bit of the top number is multiplied with one and the result is zero. If that \ac{LSB} in the top number had been one then the \ac{LSB} in the result would have also been one. Therefore, an ``even'' original number would yield a result of all zeros while an odd number would yield a one.

\section{Codes}
\label{MO:sec:codes}

\subsection{Introduction}
\label{MO:sub:codes_introduction}

Codes are nothing more than using one system of symbols to represent another system of symbols or information. Humans have used codes to encrypt secret information from ancient times. However, digital logic codes have nothing to do with secrets; rather, they are only concerned with the efficient storage, retrieval, and use of information. 

\subsubsection{Morse Code}
\label{MO:subsub:morse_code}

As an example of a familiar code, Morse code changes letters to electric pulses that can be easily transmitted over a radio or telegraph wire. Samuel Morse's code uses a series of dots and dashes to represent letters so an operator at one end of the wire can use electromagnetic pulses to send a message to some receiver at a distant end. Most people are familiar with at least one phrase in Morse code: \emph{SOS}. Here is a short sentence in Morse: 

\begin{binDisp}[commandchars=~\[\]]
  -.-. --- -.. . ...   .- .-. .   ..-. ..- -.
   c    o   d  e  s    a   r  e    f    u  n
\end{binDisp}

\subsubsection{Braille Alphabet}
\label{MO:subsub:braille_alphabet}

As one other example of a commonly-used code, in $ 1834 $ Louis Braille, at the age of $ 15 $, created a code of raised dots that enable blind people to read books. For those interested in this code, the Braille alphabet can be found at \url{http://braillebug.afb.org/braille_print.asp}. 

\subsection{Computer Codes}
\label{MO:sub:computer_codes}

The fact is, computers can only work with binary numbers; that is how information is stored in memory, how it is processed by the \ac{CPU}, how it is transmitted over a network, and how it is manipulated in any of a hundred different ways. It all boils down to binary numbers. However, humans generally want a computer to work with words (such as email or a word processor), ciphers (such as a spreadsheet), or graphics (such as photos). All of that information must be encoded into binary numbers for the computer and then decoded back into understandable information for humans. Thus, binary numbers stored in a computer are often codes used to represent letters, programming steps, or other non-numeric information. 

\subsubsection{ASCII}
\label{MO:subsub:ascii}

Computers must be able to store and process letters, like those on this page. At first, it would seem easiest to create a code by simply making \lstinline[columns=fixed]|A=1|, \lstinline[columns=fixed]|B=2|, and so forth. While this simple code does not work for a number of reasons, the idea is on the right track and the code that is actually used for letters is similar to this simple example. 

In the early $ 1960 $s, computer scientists came up with a code they named \ac{ASCII} and this is still among the most common ways to encode letters and other symbols for a computer. If the computer program knows that a particular spot in memory contains binary numbers that are actually ASCII-coded letters, it is a fairly easy job to convert those codes to letters for a screen display. For simplicity, \ac{ASCII} is usually represented by hexadecimal numbers rather than binary. For example, the word \emph{Hello} in \ac{ASCII} is: $ 048 \; 065 \; 06C \; 06C \; 06F $.

\ac{ASCII} code also has a predictable relationship between letters. For example, capital letters are exactly $ 20_{16} $ higher in ASCII than their lower-case version. Thus, to change a letter from lower-case to upper-case, a programmer can add $ 20_{16} $ to the \ac{ASCII} code for the lower-case letter. This can be done in a single processing step by using what is known as a \emph{bit-wise \textsf{AND}} on the bit representing $ 20_{16} $ in the \ac{ASCII} code's binary number.

An \ac{ASCII} chart using hexadecimal values is presented in Table \ref{MO:tab:ascii_table}. The most significant digit is read across the top row and the least significant digit is read down the left column. For example, the letter \emph{A} is $ 41_{16} $ and the number \emph{6} is $ 36_{16} $.

\begin{table}[H]
  \sisetup{parse-numbers = false}
  \sffamily
  \newcommand{\head}[1]{\textcolor{white}{\textbf{#1}}}    
  \begin{center}
    \rowcolors{2}{gray!10}{white} % Color every other line a light gray
    \begin{tabular}{ c c c c c c c c c } 
      \hline
      \rowcolor{black!75}
      & {\head{0}} & {\head{1}} & {\head{2}} & {\head{3}}
      & {\head{4}} & {\head{5}} & {\head{6}} & {\head{7}} \\
      \hline  
      \cellcolor{black!75}\head{0} & {NUL} & {DLE} & {} & {0} & {@} & 
      {P} & {'} & {p} 
      \\
      \cellcolor{black!75}\head{1} & {SOH} & {DC1} & {!} & {1} & {A} & 
      {Q} & {a} & {q} 
      \\
      \cellcolor{black!75}\head{2} & {STX} & {DC2} & {''} & {2} & {B} & 
      {R} & {b} & {r} 
      \\
      \cellcolor{black!75}\head{3} & {ETX} & {DC3} & {\#} & {3} & {C} & 
      {S} & {c} & {s} 
      \\
      \cellcolor{black!75}\head{4} & {EOT} & {DC4} & {\$} & {4} & {D} & 
      {T} & {d} & {t} 
      \\
      \cellcolor{black!75}\head{5} & {ENQ} & {NAK} & {\%} & {5} & {E} & 
      {U} & {e} & {u} 
      \\
      \cellcolor{black!75}\head{6} & {ACK} & {SYN} & {\&} & {6} & {F} & 
      {V} & {f} & {v} 
      \\
      \cellcolor{black!75}\head{7} & {BEL} & {ETB} & {'} & {7} & {G} & 
      {W} & {g} & {w} 
      \\
      \cellcolor{black!75}\head{8} & {BS} & {CAN} & {(} & {8} & {H} & 
      {X} & {h} & {x}
      \\
      \cellcolor{black!75}\head{9} & {HT} & {EM} & {)} & {9} & {I} & 
      {Y} & {i} & {y} 
      \\
      \cellcolor{black!75}\head{A} & {LF} & {SUB} & {*} & {:} & {J} & 
      {Z} & {j} & {z} 
      \\
      \cellcolor{black!75}\head{B} & {VT} & {ESC} & {+} & {;} & {K} & 
      $ [ $ & {k} & \{ 
      \\
      \cellcolor{black!75}\head{C} & {FF} & {FS} & {,} & $ < $ & {L} & 
      \textbackslash & {l} & {$ \arrowvert $} 
      \\
      \cellcolor{black!75}\head{D} & {CR} & {GS} & {-} & {=} & {M} & 
      $ ] $ & {m} & {\}} 
      \\
      \cellcolor{black!75}\head{E} & {SO} & {RS} & {.} & $ > $ & {N} & 
      $ \wedge $ & {n} & $ \sim $ 
      \\
      \cellcolor{black!75}\head{F} & {SI} & {US} & {/} & {?} & {O} & 
      {\_} & {o} & {DEL}       
      \\
      \hline
    \end{tabular}
  \end{center}
  \caption{ASCII Table}
  \label{MO:tab:ascii_table}
\end{table}

\marginpar{Teletype operators from decades past tell stories of sending 25 or more $ 07_{16} $ codes (ring the bell) to a receiving terminal just to irritate another operator in the middle of the night.} \ac{ASCII} $ 20_{16} $ is a space character used to separate words in a message and the first two columns of ASCII codes (where the high-order nibble are zero and one) were codes essential for teletype machines, which were common from the $ 1920 $s until the $ 1970 $s. The meanings of a few of those special codes are given in Table \ref{MO:tab:ascii_symbols}.

\begin{table}[H]
  \sisetup{parse-numbers = false}
  \sffamily
  \newcommand{\head}[1]{\textcolor{white}{\textbf{#1}}}    
  \begin{center}
    \rowcolors{1}{gray!10}{white} % Color every other line a light gray
    \begin{tabular}{ c l } 
      \hline \hline % The gray in the first row overwrites the first hline - so do it twice.
      {NUL} & {All Zeros (a ``null'' byte)} \\
      {SOH} & {Start of Header} \\
      {STX} & {Start of Text} \\
      {ETX} & {End of Text} \\
      {EOT} & {End of Transmission} \\
      {ENQ} & {Enquire (is the remote station on?)} \\
      {ACK} & {Acknowledge (the station is on)} \\
      {BEL} & {Ring the terminal bell (get the operator's attention)} \\
      \hline  
    \end{tabular}
  \end{center}
  \caption{ASCII Symbols}
  \label{MO:tab:ascii_symbols}
\end{table}

Table \ref{MO:tab:ascii_practice} contains a few phrases in both plain text and ASCII for practice.

\begin{table}[H]
  \sisetup{parse-numbers = false}
  \sffamily
  \newcommand{\head}[1]{\textcolor{white}{\textbf{#1}}}    
  \begin{center}
    \rowcolors{2}{gray!10}{white} % Color every other line a light gray
    \begin{tabular}{ c l } 
      \hline
      \rowcolor{black!75}
      {\head{Plain Text}} & {\head{ASCII}} \\
      \hline    
      {codes are fun} & {63 6f 64 65 73 20 61 72 65 20 66 75 6e} \\
      {This is ASCII} & {54 68 69 73 20 69 73 20 41 53 43 49 49} \\
      {365.25 days} & {33 36 35 2e 32 35 20 64 61 79 73} \\
      {It's a gr8 day!} & {49 74 27 73 20 61 20 67 72 38 20 64 61 79 21} \\
      \hline  
    \end{tabular}
  \end{center}
  \caption{ASCII Practice}
  \label{MO:tab:ascii_practice}
\end{table}
% Pull Quote - Marginal Note - Sidebar

While the ASCII code is the most commonly used text representation, it is certainly not the only way to encode words. Another popular code is \ac{EBCDIC} (pronounced like ``Eb See Deck''), which was invented by IBM in $ 1963 $ and has been used in most of their computers ever since.

Since the early 2000's, computer programs have begun to use Unicode character sets, which are similar to ASCII but multiple bytes are combined to expand the number of characters available for non-English languages like Cyrillic.

\subsubsection{Binary Coded Decimal (BCD)}
\label{MO:subsub:binary_coded_decimal}

It is often desirable to have numbers coded in such a way that they can be easily translated back and forth between decimal (which is easy for humans to manipulate) and binary (which is easy for computers to manipulate). \ac{BCD} is the code used to represent decimal numbers in binary systems. \ac{BCD} is useful when working with decimal input (keypads or transducers) and output (displays) devices.

There are, in general, two types of \ac{BCD} systems: non-weighted and weighted. Non-weighted codes are special codes devised for a single purpose where there is no implied relationship between one value and the next. As an example, $ 1001 $ could mean one and $ 1100 $ could mean two in some device. The circuit designer would create whatever code meaning is desired for the application. 

Weighted \ac{BCD} is a more generalized system where each bit position is assigned a ``weight,'' or value. These types of \ac{BCD} systems are far more common than non-weighted and are found in all sorts of applications. Weighted \ac{BCD} codes can be converted to decimal by adding the place value for each position in exactly the same way that Expanded Positional Notation is used for to covert between decimal and binary numbers. As an example, the weights for the Natural \ac{BCD} system are $ 8-4-2-1 $. (These are the same weights used for binary numbers; thus the name ``natural'' for this system.) The code $ 1001_{BCD} $ is converted to decimal like this: 

\begin{align}
  1001_{BCD} &= (1X8)+(0X4)+(0X2)+(1X1) \\
  \nonumber
  &= (8)+(0)+(0)+(1) \\
  \nonumber
  &= 9_{10}
\end{align}

Because there are ten decimal ciphers ($ 0, 1, 2, 3, 4, 5, 6, 7, 8, 9 $), it requires four bits to represent all decimal digits; so most \ac{BCD} code systems are four bits wide. In practice, only a few different weighted \ac{BCD} code systems are commonly used and the most common are shown in Table \ref{MO:tab:bcd_systems}. 

\begin{table}[H]
  \sffamily
  \newcommand{\head}[1]{\textcolor{white}{\textbf{#1}}}    
  \begin{center}
    \rowcolors{2}{gray!10}{white} % Color every other line a light gray
    \begin{tabular}{ c c c c c } 
      \hline
      \rowcolor{black!75}
      {\head{Decimal}} & {\head{8421 (Natural)}} & \head{2421} 
      & \head{Ex3} & \head{5421} \\
      \hline    
      0 & 0000 & 0000 & 0011 & 0000 \\
      1 & 0001 & 0001 & 0100 & 0001 \\
      2 & 0010 & 0010 & 0101 & 0010 \\
      3 & 0011 & 0011 & 0110 & 0011 \\
      4 & 0100 & 0100 & 0111 & 0100 \\
      5 & 0101 & 1011 & 1000 & 1000 \\
      6 & 0110 & 1100 & 1001 & 1001 \\
      7 & 0111 & 1101 & 1010 & 1010 \\
      8 & 1000 & 1110 & 1011 & 1011 \\
      9 & 1001 & 1111 & 1100 & 1100 \\
      \hline  
    \end{tabular}
  \end{center}
  \caption{BCD Systems}
  \label{MO:tab:bcd_systems}
\end{table}

\marginpar{Remember that BCD is a code system, not a number system; so the meaning of each combination of four-bit codes is up to the designer and will not necessarily follow any sort of binary numbering sequence.}The name of each type of \ac{BCD} code indicates the various place values. Thus, the $ 2421 $ \ac{BCD} system gives the most significant bit of the number a value of two, not eight as in the natural code. The \emph{Ex3} code (for ``Excess 3'') is the same as the natural code, but each value is increased by three (that is, three is added to the natural code). 

In each of the \ac{BCD} code systems in Table \ref{MO:tab:bcd_systems} there are six unused four-bit combinations; for example, in the \emph{Natural} system the unused codes are: $ 1010, 1011, 1100, 1101, 1110, $ and $ 1111 $. Thus, any circuit designed to use \ac{BCD} must include some sort of check to ensure that if unused binary values are accidentally input into a circuit it does not create an undefined outcome.

Normally, two \ac{BCD} codes, each of which are four bits wide, are packed into an eight-bit byte in order to reduce wasted computer memory. Thus, the packed \ac{BCD} $ 0111 0010 $ contains two BCD numbers: $ 72 $. In fact, a single 32-bit word, which is common in many computers, can contain $ 8 $ \ac{BCD} codes. It is a trivial matter for software to either pack or unpack \ac{BCD} codes from a longer word.

It is natural to wonder why there are so many different ways to code decimal numbers. Each of the \ac{BCD} systems shown in Table \ref{MO:tab:bcd_systems} has certain strengths and weaknesses and a circuit designer would choose a specific system based upon those characteristics.

\paragraph{Converting between BCD and Other Systems.} One thing that makes \ac{BCD} so useful is the ease of converting from \ac{BCD} to decimal. Each decimal digit is converted into a four-bit \ac{BCD} code, one at a time. Here is $ 37_{10} $ in Natural \ac{BCD}:

\begin{binDisp}
     0011 0111
       3    7
\end{binDisp}

It is, generally, very easy to convert Natural \ac{BCD} to decimal since the \ac{BCD} codes are the same as binary numbers. Other \ac{BCD} systems use different place values, and those require more thought to convert (though the process is the same). The place values for \ac{BCD} systems other than Natural are indicated in the name of the system; so, for example, the $ 5421 $ system would interpret the number $ 1001_{BCD5421} $ as:

\begin{align}
  1001_{BCD5421} &= (1X5)+(0X4)+(0X2)+(1X1) \\
  \nonumber
  &= (5)+(0)+(0)+(1) \\
  \nonumber
  &= 6_{10}
\end{align}

Converting from decimal to \ac{BCD} is also a rather simple process. Each decimal digit is converted to a four-bit \ac{BCD} equivalent. In the case of Natural \ac{BCD} the four-bit code is the binary equivalent to the decimal number, other weighted \ac{BCD} codes would be converted with a similar process.

\begin{binDisp}
      2    4    5
    0010 0100 0101
\end{binDisp}

%TODO If codes are separated from this chapter then the Double-Dabble system could be explained in some detail.

To convert binary to \ac{BCD} is no trivial exercise and is best done with an automated process. The normal method used is called the \emph{Shift Left and Add Three} algorithm (or, frequently, \emph{Double-Dabble}). The process involves a number of steps where the binary number is shifted left and occasionally three is added to the resulting shift. Wikipedia (\url{https://en.wikipedia.org/wiki/Double_dabble}) has a good explanation of this process, along with some examples.

Converting \ac{BCD} to any other system (like hexadecimal) is most easily done by first converting to binary and then to whatever base is desired. Unfortunately, converting \ac{BCD} to binary is not as simple as concatenating two \ac{BCD} numbers; for example, $ 0100 0001 $ is 41 in \ac{BCD}, but those two \ac{BCD} numbers concatenated, $ 01000001 $, is $ 65 $ in binary. One way to approach this type of problem is to use the reverse of the \emph{Double-Dabble} process: \emph{Shift Right and Subtract Three}. As in converting binary to \ac{BCD}, this is most easily handled by an automated process.

\paragraph{Self-Complementing.} The Excess-3 code (called \emph{Ex3} in the table) is self-complementing; that is, the nines complement of any decimal number is found by complementing each bit in the Ex3 code. As an example, find the nines complement for $ 127_{10} $:

\begin{table}[H]
  \sisetup{parse-numbers = false}
  \sffamily
  \newcommand{\head}[1]{\textcolor{white}{\textbf{#1}}}    
  \begin{center}
    %\rowcolors{2}{gray!10}{white} % Color every other line a light gray
    \begin{tabular}{ c | c l } 
      \hline
      1 & $ 127_{10} $ & {Original Number} \\
      2 & $ 0100 \; 0101 \; 1010_{Ex3} $ & {Convert 127 to Excess 3} \\
      3 & $ 1011 \; 1010 \; 0101_{Ex3} $ & {Ones Complement} \\
      4 & $ 872_{10} $ & {Convert to Decimal} \\
      \hline  
    \end{tabular}
  \end{center}
  \caption{Nines Complement for 127}
  \label{MO:tab:nines_complement}
\end{table}

Thus, $ 872_{10} $, is the nines complement of $ 127_{10} $. It is a powerful feature to be able to find the nines complement of a decimal number by simply complementing each bit of its Ex3 \ac{BCD} representation. 

\paragraph{Reflexive.} Some \ac{BCD} codes exhibit a reflexive property where each of the upper five codes are complementary reflections of the lower five codes. For example, $ 0111_{Ex3} $ (4) and $ 1000_{Ex3} $ (5) are complements, $ 0110_{Ex3} $ (3) and $ 1001_{Ex3} $ (6) are complements, and so forth. The reflexive property for the $ 5421 $ code is different. Notice that the codes for zero through four are the same as those for five through nine, except for the \ac{MSB} (zero for the lower codes, one for the upper codes). Thus, $ 0000_{5421} $ (zero) is the same as $ 1000_{5421} $ (five) except for the first bit, $ 0001_{5421} $ (one) is the same as $ 1001_{5421} $ (six) except for the first bit, and so forth. Studying Table \ref{MO:tab:bcd_systems} should reveal the various reflexive patterns found in these codes.

\paragraph{Practice.} Table \ref{MO:tab:bcd_practice} shows several decimal numbers in various \ac{BCD} systems which can be used for practice in converting between these number systems.

\begin{table}[H]
  \sffamily
  \newcommand{\head}[1]{\textcolor{white}{\textbf{#1}}}    
  \begin{center}
    \rowcolors{2}{gray!10}{white} % Color every other line a light gray
    {\small         
      \begin{tabular}{ c c c c c } 
        \hline
        \rowcolor{black!75}
        {\head{Dec}} & {\head{8421}} & \head{2421} 
        & \head{Ex3} & \head{5421} \\
        \hline    
        57  & 0101 1110      & 10111 1101     
            & 1000 1010      & 1000 1010 \\
        79  & 0111 1001      & 1101 1111      
            & 1010 1100      & 1010 1100 \\
        
        28  & 0010 1000      & 0010 1110      
            & 0101 1011      & 0010 1011 \\
        421 & 0100 0010 0001 & 0100 0010 0001 
            & 0111 0101 0100 & 0100 0010 0001 \\
        
        903 & 1001 0000 0011 & 1111 0000 0011 
            & 1100 0011 0110 & 1100 0000 0011 \\
        \hline  
      \end{tabular}
    }  % End small font size
  \end{center}
  \caption{BCD Practice}
  \label{MO:tab:bcd_practice}
\end{table}

\paragraph{Adding BCD Numbers.} \ac{BCD} numbers can be added in either of two ways. Probably the simplest is to convert the \ac{BCD} numbers to binary, add them as binary numbers, and then convert the sum back to \ac{BCD}. However, it is possible to add two \ac{BCD} numbers without converting. When two \ac{BCD} numbers are added such that the result is less than ten, then the addition is the same as for binary numbers:

\begin{binDisp}[commandchars=~\[\]]
     0101  (5)
    +~underline[0010  (2)]
     0111  (7)
\end{binDisp}

However, four-bit binary numbers greater than $ 1001_2 $ (that is: $ 9_{10} $) are invalid \ac{BCD} codes, so adding two \ac{BCD} numbers where the result is greater than nine requires a bit more effort:

\begin{binDisp}[commandchars=~\[\]]
     0111  (7)
    +~underline[0101  (5)]
     1100  (12 -- not valid in BCD)
\end{binDisp}

When the sum is greater than nine, then six must be added to that result since there are six invalid binary codes in \ac{BCD}.

\begin{binDisp}[commandchars=~\[\]]
     1100  (12 -- from previous addition)
    +~underline[0110  (6)]
   1 0010  (12 in BCD)
\end{binDisp}

When adding two-digit \ac{BCD} numbers, start with the \ac{LSN}, the right-most nibble, then add the nibbles with carry bits from the right. Here are some examples to help clarify this concept:

\begin{binDisp}[commandchars=~\[\], samepage=true]
     0101 0010  (52)
    +~underline[0011 0110  (36)]
     1000 1000  (88 in BCD)
\end{binDisp}

\begin{binDisp}[commandchars=~\[\], samepage=true]
     0101 0010  (52)
    +~underline[0101 0110  (56)]
     1010 1000  (1010, MSN, invalid BCD code)
    +~underline[0110 0000  (Add 6 to invalid code)]
   1 0000 1000  (108 in BCD)
\end{binDisp}

\begin{binDisp}[commandchars=~\[\], samepage=true]
     0101 0101  (55)
    +~underline[0101 0110  (56)]
     1010 1011  (both nibbles invalid BCD code)
    +~underline[0000 0110  (Add 6 to LSN)]
     1011 0001  (1 carried over to MSN)
    +~underline[0110 0000  (Add 6 to MSN)]
   1 0001 0001  (111 in BSD)
\end{binDisp}

\paragraph{Negative Numbers.} \ac{BCD} codes do not have any way to store negative numbers, so a sign nibble must be used. One approach to this problem is to use a sign-and-magnitude value where a sign nibble is prefixed onto the \ac{BCD} value. By convention, a sign nibble of $ 0000 $ makes the \ac{BCD} number positive while $ 1001 $ makes it negative. Thus, the \ac{BCD} number $ 0000 0010 0111 $ is $ 27 $, but $ 1001 0010 0111 $ is $ -27 $. 

A more mathematically rigorous, and useful, method of indicating negative \ac{BCD} numbers is use the tens complement of the \ac{BCD} number since \ac{BCD} is a code for decimal numbers, exactly like the twos complement is used for binary numbers. It may be useful to review Section \ref{MO:subsub:about_complementation} on page \pageref{MO:subsub:about_complementation} for information about the tens complement. In the \emph{Natural BCD} system the tens complement is found by adding one to the nines complement, which is found by subtracting each digit of the original \ac{BCD} number from nine. Here are some examples to clarify this concept:

\begin{binDisp}[commandchars=~\[\]]
    0111 (7 in BCD)
    0010 (2, the 9's complement of 7 since 9-7=2)
    0011 (3, the 10's complement of 7, or 2+1)
\end{binDisp}

\begin{binDisp}[commandchars=~\[\]]
    0010 0100 (24 in BCD)
    0111 0101 (75, the 9's complement of 24)
    0111 0110 (76, the 10's complement of 24)
\end{binDisp}

Also, some \ac{BCD} code systems are designed to easily create the tens complement of a number. For example, in the $ 2421 $ \ac{BCD} system the tens complement is found by nothing more than inverting the \ac{MSB}. Thus, three is the tens complement of seven and in the $ 2421 $ \ac{BCD} system $ 0011_{BCD2421} $ is the tens complement of $ 1101_{BCD2421} $, a difference of only the \ac{MSB}. Therefore, designers creating circuits that must work with negative \ac{BCD} numbers may opt to use the $ 2421 $ \ac{BCD} system.

\paragraph{Subtracting BCD Numbers.} A \ac{BCD} number can be subtracted from another by changing it to a negative number and adding. Just like in decimal, $ 5 - 2 $ is the same as $ 5 + (-2) $. Either a nines or tens complement can be used to change a \ac{BCD} number to its negative, but for this book, the tens complement will be used. If there is a carry-out bit then it can be ignored and the result is positive, but if there is no carry-out bit then answer is negative so the magnitude must be found by finding the tens complement of the calculated sum. Compare this process with subtracting regular binary numbers. Here are a few examples:

\begin{minipage}{\linewidth} % This keeps the block on the same page
\begin{binDisp}[commandchars=~\[\]]

    7 - 3 = 4

          0111  (7 in BCD)
         +~underline[0111  (add the 10's complement of 3)]
          1110  (invalid BCD code)
         +~underline[0110  (add 6 to invalid BCD code)]
        1 0100  (4 - drop the carry bit)

\end{binDisp}
\end{minipage}

\begin{minipage}{\linewidth} % This keeps the block on the same page
\begin{binDisp}[commandchars=~\[\]]

    7 - 9 = -2

          0111  (7 in BCD)
         +~underline[0001  (10's complement of 9)]
          1000  (valid BCD code)
          0010  (10's complement)

\end{binDisp}
\end{minipage}

\begin{minipage}{\linewidth} % This keeps the block on the same page
\begin{binDisp}[commandchars=~\[\]]

    32 - 15 = 17

          0011 0010  (32 in BCD)
         +~underline[1000 0101  (10's complement of 15)]
          1011 0111  (MSB is invalid BCD code)
         +~underline[0110 0000  (add 6 to MSB)]
        1 0001 0111  (17, drop the carry bit)

\end{binDisp}
\end{minipage}

\begin{minipage}{\linewidth} % This keeps the block on the same page
\begin{binDisp}[commandchars=~\[\]]

    427 - 640 = -213

          0100 0010 0111  (427 in BCD)
         +~underline[0011 0110 0000  (10's complement of 640)]
          0111 1000 0111  (no invalid BCD code)
          0010 0001 0011  (10's complement)

\end{binDisp}
\end{minipage}

\begin{minipage}{\linewidth} % This keeps the block on the same page

\begin{binDisp}[commandchars=~\[\]]

    369 - 532 = -163

          0011 0110 1001  (369 in BCD)
         +~underline[0100 0110 1000  (10's complement of 532)]
          0111 1100 0001  (two invalid BCD codes)
         +~underline[0000 0110 0110  (add 6 to invalid codes)]
          1000 0011 0111  (sum)
          0001 0110 0011  (10's complement)

\end{binDisp}
\end{minipage}

Here are some notes on the last example: adding the \ac{LSN} yields $ 1001+1000=10001 $. The initial one is ignored, but this is an invalid \ac{BCD} code so this byte needs to be corrected by adding six to it. Then the result of that addition includes an understood carry into the next nibble after the correction is applied. In the same way, the middle nibble was corrected: $ 1100+0110=10011 $ but the initial one in this answer is carried to the \ac{MSN} and added there.

\subsubsection{Gray Code}
\label{MO:subsub:gray_code}

% Draw a Gray Code Wheel
% Cite: http://tex.stackexchange.com/questions/56176/handling-of-wrapfig-pictures-in-latex
\begin{wrapfigure}{r}{0.3\textwidth}
  \caption{Optical Disc}
  \label{MO:fig:gray_code_disc}
  \centering

  \xdef\IntRad{2}
  \xdef\Rad{.5}

  \newcommand{\Sector}[2][]{%
    \draw[#1] (22.5:#2) arc (22.5:0:#2)
    --(#2+\Rad,0) arc (0:22.5:#2+\Rad)
    -- cycle ;
  }

\begin{tikzpicture}[scale=0.40]

  \foreach \Loop [count=\j from 0] in {%
    {white,,,white,white,,,white,white,,,white,white,,,white},
    {,,white,white,white,white,,,,,white,white,white,white,,},
    {white,white,white,white,white,white,white,white,,,,,,,,},
    {white,white,white,white,,,,,,,,,white,white,white,white}}
  {\foreach \col [count=\i from 0] in \Loop {%
      \begin{scope}[rotate={22.5*\i}]
      \Sector[fill=\col]{\IntRad+\j*\Rad} ;   
      \end{scope}
    }
  }

  \begin{scope}[rotate=11.75]
  \draw[fill=white] (\IntRad-.2,-.5*\Rad) rectangle (\IntRad+.2+4*\Rad,.5*\Rad) ;

  \foreach \col [count=\i from 0, evaluate=\i as \j using 0.5+\i]
  in {white,black,white,white} {%
    \draw[fill=\col] (\IntRad+\j*\Rad,0) circle (.25*\Rad) ;
  }
  \end{scope}
\end{tikzpicture}
\end{wrapfigure}

It is often desirable to use a wheel to encode digital input for a circuit. As an example, consider the tuning knob on a radio. The knob is attached to a shaft that has a small, clear disk which is etched with a code, similar to Figure \ref{MO:fig:gray_code_disc}. As the disk turns, the etched patterns pass or block a laser beam from reaching an optical sensor, and that pass/block pattern is encoded into binary input. 

One of the most challenging aspects of using a mechanical device to encode binary is ensuring that the input is stable. As the wheel turns past the light beam, if two of the etched areas change at the same time (thus, changing two bits at once), it is certain that the input will fluctuate between those two values for a tiny, but significant, period of time. For example, imagine that the encoded circuit changes from $ 1111 $ to $ 0000 $ at one time. Since it is impossible to create a mechanical wheel precise enough to change those bits at exactly the same moment in time (remember that the light sensors will ``see'' an input several million times a second), as the bits change from $ 1111 $ to $ 0000 $ they may also change to $ 1000 $ or $ 0100 $ or any of dozens of other possible combinations for a few microseconds. The entire change may form a pattern like $ 1111-0110-0010-0000 $ and that instability would be enough to create havoc in a digital circuit. 

The solution to the stability problem is to etch the disk with a code designed in such a way that only one bit changes at a time. The code used for that task is the Gray code. Additionally, a Gray code is cyclic, so when it reaches its maximum value it can cycle back to its minimum value by changing only a single bit. In Figure \ref{MO:fig:gray_code_disc}, each of the concentric rings encodes one bit in a four-bit number. Imagine that the disk is rotating past the fixed laser beam reader \textemdash the black areas (``blocked light beam'') change only one bit at a time, which is characteristic of a Gray code pattern.

It is fairly easy to create a Gray code from scratch. Start by writing two bits, a zero and one: 

\begin{binDisp}[commandchars=~\[\]]
     0
     1
\end{binDisp}

Then, reflect those bits by writing them in reverse order underneath the original bits:

\begin{binDisp}[commandchars=~\[\]]
     0
     1
   -----
     1
     0
\end{binDisp}

Next, prefix the top half of the group with a zero and the bottom half with a one to get a two-bit Gray code. 

\begin{binDisp}
     00
     01
     11
     10  (2-bit Gray code)
\end{binDisp}

Now, reflect all four values of the two-bit Gray code.

\begin{binDisp}
     00
     01
     11
     10
   ------
     10
     11
     01
     00
\end{binDisp}

Next, prefix the top half of the group with a zero and the bottom half with a one to get a three-bit Gray code.

\begin{binDisp}
     000
     001
     011
     010
     110
     111
     101
     100  (3-bit Gray code)
\end{binDisp}

Now, reflect all eight values of a three-bit Gray code.

\begin{binDisp}
     000
     001
     011
     010
     110
     111
     101
     100
   -------
     100
     101
     111
     110
     010
     011
     001
     000
\end{binDisp}

Finally, prefix the top half of the group with a zero and the bottom half with a one to get a four-bit Gray code. 

\begin{binDisp}
     0000
     0001
     0011
     0010
     0110
     0111
     0101
     0100
     1100
     1101
     1111
     1110
     1010
     1011
     1001
     1000  (four-bit Gray code)
\end{binDisp}

The process of reflecting and prefixing can continue indefinitely to create a Gray code of any desired bit length. Of course, Gray code tables are also available in many different bit lengths. Table \ref{MO:tab:gray_codes} contains a two-bit, three-bit, and four-bit Gray code:

\begin{table}[H]
  \sffamily
  \newcommand{\head}[1]{\textcolor{white}{\textbf{#1}}}    
  \begin{center}
    \rowcolors{2}{gray!10}{white} % Color every other line a light gray
    \begin{tabular}{ c c c } 
      \hline
      \rowcolor{black!75}
      {\head{2-Bit Code}} & {\head{3-Bit Code}} & \head{4-Bit Code} \\ 
      \hline    
      00  & 000 & 0000 \\     
      01  & 001 & 0001 \\     
      11  & 011 & 0011 \\     
      10  & 010 & 0010 \\     
      & 110 & 0110 \\     
      & 111 & 0111 \\     
      & 101 & 0101 \\     
      & 100 & 0100 \\     
      &     & 1100 \\     
      &     & 1101 \\     
      &     & 1111 \\     
      &     & 1110 \\     
      &     & 1010 \\     
      &     & 1011 \\     
      &     & 1001 \\     
      &     & 1000 \\     
      \hline  
    \end{tabular}
  \end{center}
  \caption{Gray Codes}
  \label{MO:tab:gray_codes}
\end{table}


\chapter{Boolean Functions}\label{ch04}
\section{Introduction to Boolean Functions}

Before starting a study of Boolean functions, it is important to keep in mind that this mathematical system concerns electronic components that are capable of only two states: \emph{True} and \emph{False} (sometimes called \emph{High-Low} or $ 1 $ - $ 0 $). Boolean functions are based on evaluating a series of \emph{\emph{True}-\emph{False}} statements to determine the output of a circuit.

For example, a Boolean expression could be created that would describe ``If the floor is dirty OR company is coming THEN I will mop.'' (The things that I do for visiting company!) These types of logic statements are often represented symbolically using the symbols $ 1 $ and $ 0 $, where $ 1 $ stands for \emph{True} and $ 0 $ stands for \emph{False}. So, let ``Floor is dirty'' equal $ 1 $ and ``Floor is not dirty'' equal $ 0 $. Also, let ``Company is coming'' equal $ 1 $ and ``Company is not coming'' equal $ 0 $. Then, ``Floor is dirty OR company is coming'' can be symbolically represented by $ 1 $ \textsf{OR} $ 1 $. Within the discipline of Boolean algebra, common mathematical symbols are used to represent Boolean expressions; for example, Boolean \textsf{OR} is frequently represented by a mathematics plus sign, as shown below. 

% Using + to represent OR
\begin{align}
  \label{04eq01}
  0 + 0 &= 0 \\
  \nonumber
  0 + 1 &= 1 \\
  \nonumber
  1 + 0 &= 1 \\
  \nonumber
  1 + 1 &= 1
\end{align}

These look like addition problems, but they are not (as evidenced by the last line). It is essential to keep in mind that these are merely symbolic representations of \emph{True}-\emph{False} statements. The first three lines make perfect sense and look like elementary addition. The last line, though, violates the principles of addition for real numbers; but it is a perfectly valid Boolean expression. Remember, in the world of Boolean algebra, there are only two possible values for any quantity: $ 1 $ or $ 0 $; and that last line is actually saying \emph{True \textsf{OR} True is True}. To use the dirty floor example from above, ``The floor is dirty'' (\emph{True}) OR ``Company is coming'' (\emph{True}) SO ``I will mop the floor'' (\emph{True}) is symbolized by: $ 1 + 1 = 1 $. This could be expressed as \emph{T \textsf{OR} T SO T}; but the convention is to use common mathematical symbols; thus: $ 1 + 1 = 1 $. 

Moreover, it does not matter how many or few terms are \textsf{OR} ed together; if just one is \emph{True}, then the output is \emph{True}, as illustrated below:

\begin{align}
  \label{04eq02}
  1 + 1 + 1 &= 1 \\
  \nonumber
  1 + 0 + 0 + 0 + 0 &= 1 \\
  \nonumber
  1 + 1 + 1 + 1 + 1 + 1 &= 1
\end{align}

Next, consider a very simple electronic sensor in an automobile: IF the headlights are on AND the driver's door is open THEN a buzzer will sound. In the same way that the plus sign is used to mathematically represent \textsf{OR} , a times sign is used to represent \textsf{AND} . Therefore, using common mathematical symbols, this automobile alarm circuit would be represented by $ 1 X 1 = 1 $. The following list shows all possible states of the headlights and door:

% Using X to represent AND
\begin{align}
  \label{04eq03}
  0 X 0 &= 0 \\
  \nonumber
  0 X 1 &= 0 \\
  \nonumber
  1 X 0 &= 0 \\
  \nonumber
  1 X 1 &= 1
\end{align}

The first row above shows ``\emph{False} (the lights are not on) AND \emph{False} (the door is not open) results in \emph{False} (the alarm does not sound)''. For \textsf{AND}  logic, the only time the output is \emph{True} is when all inputs are also \emph{True}; therefore: $ 1 X 1 X 1 X 1 X 0 = 0 $. In this way, Boolean \textsf{AND}  behaves somewhat like algebraic multiplication.

Within Boolean algebra's simple \emph{True}-\emph{False} system, there is no equivalent for subtraction or division, so those mathematical symbols are not used. Like real-number algebra, Boolean algebra uses alphabetical letters to denote variables; however, Boolean variables are always CAPITAL letters, never lower-case, and normally only a single letter. Thus, a Boolean equation would look something like this:

\begin{align}
  \label{04eq04}
  A + B &= Y 
\end{align}

As Boolean expressions are realized (that is, turned into a real, or physical, circuit), the various operators become \emph{gates}. For example, the above equation would be realized using an \textsf{OR}  gate with two inputs (labeled $ A $ and $ B $) and one output (labeled $ Y $). 

Boolean algebra includes three primary and four secondary logic operations (plus a Buffer, which has no logic value), six univariate, and six multivariate properties. All of these will be explored in this chapter.

\section{Primary Logic Operations}\label{0401}
\subsection{AND}

An \textsf{AND}  gate is a Boolean operation that will output a logical one, or \emph{True}, only if all of the inputs are \emph{True}. As an example, consider this statement: ``If I have ten bucks AND there is a good movie at the cinema, then I will go see the movie.'' In this statement, ``If I have ten bucks'' is one variable and ``there is a good movie at the cinema'' is another variable. If both of these inputs are \emph{True}, then the output variable (``I will go see the movie'') will also be \emph{True}. However, if either of the two inputs is \emph{False}, then the output will also be \emph{False} (or, ``I will not go see the movie''). Of course, if I want popcorn, I would need another ten spot, but that is not germane to this example. When written in an equation, the Boolean \textsf{AND}  term is represented a number of different ways. One method is to use the logic \textsf{AND} symbol as found in Equation \ref{BF:eq:and_symbol_logical}.

\begin{align}
  \label{04eq05}
  A \wedge B &= Y 
\end{align}

One other method is to use the same symbols that are used for multiplication in traditional algebra; that is, by writing the variables next to each other, with parenthesis, or, sometimes, with an asterisk between them, as in Equation \ref{04eq06}.

% Using X to represent AND
\begin{align}
  \label{04eq06}
  AB &= Y \\
  \nonumber
  (A)(B) &= Y \\
  \nonumber
  A * B &= Y
\end{align}

\marginpar{The multiplication symbols $ X $ and $\bullet$ (dot) are not commonly used in digital logic equations.} Logic \textsf{AND} is normally represented in equations by using an algebra multiplication symbol since it is easy to type; however, if there is any chance for ambiguity, then the Logic \textsf{AND}  symbol ($ \wedge $) can be used to differentiate between multiplication and a logic \textsf{AND}  function.

Following is the truth table for the \textsf{AND}  operator.

%******************************************************
% AND Truth Table
%******************************************************
\begin{table}[H]
  \sffamily
  \newcommand{\head}[1]{\textcolor{white}{\textbf{#1}}}    
  \begin{center}
    \rowcolors{2}{gray!10}{white} % Color every other line a light gray
    \begin{tabular}{ccc} 
      \rowcolor{black!75}
      \multicolumn{2}{c}{\head{Inputs}} & \head{Output} \\
      A & B & Y \\
      \hline
      0 & 0 & 0 \\
      0 & 1 & 0 \\
      1 & 0 & 0 \\
      1 & 1 & 1 
    \end{tabular}
  \end{center}
  \caption{Truth Table for AND}
  \label{04tt01}
\end{table}

A truth table is used to record all possible inputs and the output for each combination of inputs. For example, in the first line of Table \ref{04tt01}, if input $ A $ is $ 0 $ and input $ B $ is $ 0 $ then the output, $ Y $, will be $ 0 $. All possible input combinations are normally formed in a truth table by counting in binary starting with all variables having a $ 0 $ value to all variables having a $ 1 $ value. Thus, in Table \ref{04tt01}, the inputs are $ 00 $, $ 01$, $ 10$, $ 11 $. Notice that the output for the \textsf{AND}  operator is \emph{False} (that is, $ 0 $) until the last row, when both inputs are \emph{True}. Therefore, it could be said that just one \emph{False} input would inactivate a physical \textsf{AND}  gate. For that reason, an \textsf{AND}  operation is sometimes called an \emph{inhibitor}.

% Begin Sidebar Box
\begin{tcolorbox}[colback=blue!5!white,colframe=blue!75!black]
  % Upper half of box: my "title" area
  \textcolor{blue}{\textbf{AND Gate Switches}}
  % Lower half of the box: the content
  \tcblower
  Because a single \emph{False} input can turn an \textsf{AND} gate off, these types of gates are frequently used as a switch in a logic circuit. As a simple example, imagine an assembly line where there are four different safety sensors of some sort. The sensor outputs could be routed to a single four-input \textsf{AND}  gate and then as long as all sensors are \emph{True} the assembly line motor will run. If, however, any one of those sensors goes \emph{False} due to some unsafe condition, then the \textsf{AND}  gate would also output a \emph{False} and cause the motor to stop.
\end{tcolorbox}
% End Sidebar Box

Logic gates are realized (or created) in electronic circuits by using transistors, resistors, and other components. These components are normally packaged into a single \ac{IC} ``chip,'' so the logic circuit designer does not need to know all of the details of the electronics in order to use the gate. In logic diagrams, an \textsf{AND}  gate is represented by a shape that looks like a capital \textsf{D}. In Figure \ref{fig:04_01}, the input variables $ A $ and $ B $ are wired to an \textsf{AND}  gate and the output from that gate goes to $ Y $.

\begin{figure}[H]
	\centering
	\includegraphics[width=\maxwidth{.95\linewidth}]{gfx/04_01}
	\caption{AND Gate}
	\label{fig:04_01}
\end{figure}

Notice that each input and output is named in order to make it easier to describe the circuit algebraically. In reality, \textsf{AND}  gates are not packaged or sold one at a time; rather, several would be placed on a single \ac{IC}, like the \emph{7408 Quad AND Gate}. The designer would design the circuit card to use whichever of the four gates are needed while leaving the unused gates unconnected. 

There are two common sets of symbols used to represent the various elements in logic diagrams, and whichever is used is of little consequence since the logic is the same. Shaped symbols, as used in Figure \ref{fig:04_01}, are more common in the United States; but the \ac{IEEE} has its own symbols which are sometimes used, especially in Europe. Figure \ref{fig:04_02} illustrates the circuit in Figure \ref{fig:04_01} using \ac{IEEE} symbols:

% Pull Quote - Marginal Note - Sidebar
\marginpar{In this book, only shaped symbols will be used.}

\begin{figure}[H]
	\centering
	\includegraphics[width=\maxwidth{.95\linewidth}]{gfx/04_02}
	\caption{AND Gate Using IEEE Symbols}
	\label{fig:04_02}
\end{figure}

As an example of an \textsf{AND}  gate at work, consider an elevator: if the door is closed (logic $ 1 $) \textsf{AND} someone in the elevator car presses a floor button (logic $ 1 $), THEN the elevator will move (logic $ 1 $). If both sensors (door and button) are input to an \textsf{AND}  gate, then the elevator motor will only operate if the door is closed \textsf{AND}  someone presses a floor button. 

\subsection{OR}
\label{BF:subsec:or}

An \textsf{OR}  gate is a Boolean operation that will output a logical one, or \emph{True}, if any or all of the inputs are \emph{True}. As an example, consider this statement: ``If my dog needs a bath OR I am going swimming, then I will put on a bathing suit.'' In this statement, ``if my dog needs a bath'' is one input variable and ``I am going swimming'' is another input variable. If either of these is \emph{True}, then the output variable, ``I will put on a bathing suit,'' will also be \emph{True}. However, if both of the inputs are \emph{False}, then the output will also be \emph{False} (or, ``I will not put on a bathing suit''). If you think it odd that I would wear a bathing suit to bathe my dog then you have obviously never met my dog. 

When written in an equation, the Boolean \textsf{OR}  term is represented a number of different ways. One method is to use the logic \textsf{OR}  symbol, as found in Equation \ref{BF:eq:or_vee}.

\begin{align}
  \label{BF:eq:or_vee}
  A \vee B &= Y 
\end{align}

A more common method is to use the \emph{plus} sign that is used for addition in traditional algebra, as in Equation \ref{BF:eq:or_plus}.

\begin{align}
  \label{BF:eq:or_plus}
  A + B &= Y 
\end{align}

For simplicity, the mathematical \emph{plus} symbol is normally used to indicate \textsf{OR}  in printed material since it is easy to enter with a keyboard; however, if there is any chance for ambiguity, then the logic \textsf{OR}  symbol ($ \vee $) is used to differentiate between addition and logic \textsf{OR}.

Table \ref{BF:tab:truth_table_for_or} is the truth table for an \textsf{OR}  operation.

%******************************************************
% OR Truth Table
%******************************************************
\begin{table}[H]
  \sffamily
  \newcommand{\head}[1]{\textcolor{white}{\textbf{#1}}}    
  \begin{center}
    \rowcolors{2}{gray!10}{white} % Color every other line a light gray
    \begin{tabular}{ccc} 
      \rowcolor{black!75}
      \multicolumn{2}{c}{\head{Inputs}} & \head{Output} \\
      A & B & Y \\
      \hline
      0 & 0 & 0 \\
      0 & 1 & 1 \\
      1 & 0 & 1 \\
      1 & 1 & 1 
    \end{tabular}
  \end{center}
  \caption{Truth Table for OR}
  \label{BF:tab:truth_table_for_or}
\end{table}

Notice for the \textsf{OR} truth table that the output is \emph{True} ($ 1 $) whenever at least one input is \emph{True}. Therefore, it could be said that one \emph{True} input would activate an \textsf{OR} Gate. In the following diagram, the input variables $ A $ and $ B $ are wired to an \textsf{OR}  gate, and the output from that gate goes to $ Y $. 

\begin{figure}[H]
	\centering
	\includegraphics[width=\maxwidth{.95\linewidth}]{gfx/04_03}
	\caption{OR Gate}
	\label{fig:04_03}
\end{figure}

As an example of an \textsf{OR} gate at work, consider a traffic signal. Suppose an intersection is set up such that the light for the main road is normally green; however, if a car pulls up to the intersection from the crossroad, or if a pedestrian presses the ``cross'' button, then the main light is changed to red to stop traffic. This could be done with a simple \textsf{OR} gate. An automobile sensor on the crossroad would be one input and the pedestrian ``cross'' button would be the other input; the output of the \textsf{OR}  gate connecting these two inputs would change the light to red when either input is activated.

\subsection{NOT}
\label{BF:subsec:not}

\textsf{NOT} (or \emph{inverter}) is a Boolean operation that inverts the input. That is, if the input is \emph{True} then the output will be \emph{False} or if the input is \emph{False} then the output will be \emph{True}. When written in an equation, the Boolean \textsf{NOT}  operator is represented in many ways, though two are most popular. The older method is to overline (that is, a line above) a term, or group of terms, that are to be inverted, as in Equation \ref{BF:eq:not_bar}.\marginpar{Equation \ref{BF:eq:not_bar} is read \emph{A OR B NOT = Q} (notice that when spoken, the word \emph{not} follows the term that is inverted).}

\begin{align}
  \label{BF:eq:not_bar}
  A + \overline{B} &= Y 
\end{align}

Another method of indicating \textsf{NOT}  is to use the algebra \emph{prime} indicator, an apostrophe, as in Equation \ref{BF:eq:not_apostrophe}.

\begin{align}
  \label{BF:eq:not_apostrophe}
  A + B' &= Y 
\end{align}

The reason that \textsf{NOT} is most commonly indicated with an apostrophe is because that is easier to enter on a computer keyboard. There are many other ways authors use to represent \textsf{NOT}  in a formula, but none are considered standardized. For example, some authors use an exclamation point: $ A+!B=Q $, others use a broken line: $ A+ \neg B=Q $, others use a backslash: $ A+ \backslash B=Q $, and still others use a tilde: $ A+ \sim B =Q $. However, only the apostrophe and overline are consistently used to indicate \textsf{NOT}. Table \ref{BF:tab:truth_table_for_not} is the truth table for \textsf{NOT}:

%******************************************************
% NOT Truth Table
%******************************************************
\begin{table}[H]
  \sffamily
  \newcommand{\head}[1]{\textcolor{white}{\textbf{#1}}}    
  \begin{center}
    \rowcolors{2}{gray!10}{white} % Color every other line a light gray
    \begin{tabular}{ccc} 
      \rowcolor{black!75}
      \head{Input} & \head{Output} \\
      0 & 1 \\
      1 & 0 \\
    \end{tabular}
  \end{center}
  \caption{Truth Table for NOT}
  \label{BF:tab:truth_table_for_not}
\end{table}

In a logic diagram, \textsf{NOT}  is represented by a small triangle with a ``bubble'' on the output. In Figure \ref{fig:04_04}, the input variable $ A $ is inverted by a \textsf{NOT}  gate and then sent to output $ Y $. 

\begin{figure}[H]
	\centering
	\includegraphics[width=\maxwidth{.95\linewidth}]{gfx/04_04}
	\caption{NOT Gate}
	\label{fig:04_04}
\end{figure}


%*********************************************************************
% NAND Gate
%*********************************************************************
\section{Secondary Logic Functions}
\label{BF:sec:secondary_logic_functions}
\subsection{NAND}
\label{BF:subsec:nand}

\textsf{NAND} is a Boolean operation that outputs the opposite of \textsf{AND}, that is, \textsf{NOT AND}; thus, it will output a logic \emph{False} only if all of the inputs are \emph{True}. The \textsf{NAND} operation is not often used in Boolean equations, but when necessary it is represented by a vertical line. Equation \ref{BF:eq:nand_symbol} shows a \textsf{NAND}  operation.

\begin{align}
  \label{BF:eq:nand_symbol}
  A | B &= Y 
\end{align}

Table \ref{BF:tab:truth_table_for_nand_gate} is the Truth Table for a \textsf{NAND}  gate.

%******************************************************
% NAND Truth Table
%******************************************************
\begin{table}[H]
  \sffamily
  \newcommand{\head}[1]{\textcolor{white}{\textbf{#1}}}    
  \begin{center}
    \rowcolors{2}{gray!10}{white} % Color every other line a light gray
    \begin{tabular}{ccc} 
      \rowcolor{black!75}
      \multicolumn{2}{c}{\head{Inputs}} & \head{Output} \\
      A & B & Y \\
      \hline
      0 & 0 & 1 \\
      0 & 1 & 1 \\
      1 & 0 & 1 \\
      1 & 1 & 0 
    \end{tabular}
  \end{center}
  \caption{Truth Table for \textsf{NAND}  Gate}
  \label{BF:tab:truth_table_for_nand_gate}
\end{table}

In Figure \ref{fig:04_05}, the input variables $ A $ and $ B $ are wired to a NAND gate, and the output from that gate goes to $ Y $.

\begin{figure}[H]
	\centering
	\includegraphics[width=\maxwidth{.95\linewidth}]{gfx/04_05}
	\caption{NAND Gate}
	\label{fig:04_05}
\end{figure}


\marginpar{Inverting bubbles are never found by themselves on a wire; they are always associated with either the inputs or output of a logic gate. To invert a signal on a wire, a \textsf{NOT} gate is used.}The logic diagram symbol for a \textsf{NAND}  gate looks like an \textsf{AND} gate, but with a small bubble on the output port. A bubble in a logic diagram always represents some sort of signal inversion, and it can appear at the inputs or outputs of nearly any logic gate. For example, the bubble on a \textsf{NAND}  gate could be interpreted as ``take whatever the output would be generated by an \textsf{AND} gate\textemdash then invert it.'' 

%*************************************************************
% NOR Gate
%*************************************************************
\subsection{NOR}
\label{BF:subsec:nor}

\textsf{NOR}  is a Boolean operation that is the opposite of OR, that is, \textsf{NOT OR}; thus, it will output a logic \emph{True} only if all of the inputs are \emph{False}. The \textsf{NOR} operation is not often used in Boolean equations, but when necessary it is represented by a downward-pointing arrow. Equation \ref{BF:eq:nor_symbol} shows a \textsf{NOR} operation.

\begin{align}
  \label{BF:eq:nor_symbol}
  A \downarrow B &= Y 
\end{align}

Table \ref{BF:tab:truth_table_for_nor} is the truth table for \textsf{NOR} . 

%******************************************************
% NOR Truth Table
%******************************************************
\begin{table}[H]
  \sffamily
  \newcommand{\head}[1]{\textcolor{white}{\textbf{#1}}}    
  \begin{center}
    \rowcolors{2}{gray!10}{white} % Color every other line a light gray
    \begin{tabular}{ccc} 
      \rowcolor{black!75}
      \multicolumn{2}{c}{\head{Inputs}} & \head{Output} \\
      A & B & Y \\
      \hline
      0 & 0 & 1 \\
      0 & 1 & 0 \\
      1 & 0 & 0 \\
      1 & 1 & 0 
    \end{tabular}
  \end{center}
  \caption{Truth Table for NOR}
  \label{BF:tab:truth_table_for_nor}
\end{table}

In Figure \ref{fig:04_06}, the input variables $ A $ and $ B $ are wired to a \textsf{NOR}  gate, and the output from that gate goes to $ Y $. 

\begin{figure}[H]
	\centering
	\includegraphics[width=\maxwidth{.95\linewidth}]{gfx/04_06}
	\caption{NOR Gate}
	\label{fig:04_06}
\end{figure}


\subsection{XOR}
\label{BF:subsec:xor}

\textsf{XOR} (\emph{Exclusive OR}) is a Boolean operation that outputs a logical one, or \emph{True}, only if the two inputs are different. This is useful for circuits that compare inputs; if they are different then the output is \emph{True}, otherwise it is \emph{False}. Because of this, an \textsf{XOR}  gate is sometimes referred to as a \emph{Difference Gate}. The \textsf{XOR} operation is not often used in Boolean equations, but when necessary it is represented by a plus sign (like the OR function) inside a circle. Equation \ref{BF:eq:xor_symbol} shows an \textsf{XOR}  operation.

% XOR in equation
\begin{align}
  \label{BF:eq:xor_symbol}
  A \oplus B &= Y
\end{align}

Table \ref{BF:tab:truth_table_for_xor} is the truth table for an \textsf{XOR}  gate.

%******************************************************
% XOR Truth Table
%******************************************************
\begin{table}[H]
  \sffamily
  \newcommand{\head}[1]{\textcolor{white}{\textbf{#1}}}    
  \begin{center}
    \rowcolors{2}{gray!10}{white} % Color every other line a light gray
    \begin{tabular}{ccc} 
      \rowcolor{black!75}
      \multicolumn{2}{c}{\head{Inputs}} & \head{Output} \\
      A & B & Y \\
      \hline
      0 & 0 & 0 \\
      0 & 1 & 1 \\
      1 & 0 & 1 \\
      1 & 1 & 0 
    \end{tabular}
  \end{center}
  \caption{Truth Table for XOR}
  \label{BF:tab:truth_table_for_xor}
\end{table}

In Figure \ref{fig:04_07}, the input variables $ A $ and $ B $ are wired to an \textsf{XOR}  gate, and the output from that gate goes to $ Y $. 

\begin{figure}[H]
	\centering
	\includegraphics[width=\maxwidth{.95\linewidth}]{gfx/04_07}
	\caption{XOR Gate}
	\label{fig:04_07}
\end{figure}


There is some debate about the proper behavior of an \textsf{XOR}  gate that has more than two inputs. Some experts believe that an \textsf{XOR} gate should output a \emph{True} if one, and only one, input is \emph{True} regardless of the number of inputs. This would seem to be in keeping with the rules of digital logic developed by George Boole and other early logisticians and is the strict definition of \textsf{XOR} promulgated by the \ac{IEEE}. This is also the behavior of the \textsf{XOR} gate found in \textit{Logisim-evolution}, the digital logic simulator used in the lab manual accompanying this text. Others believe, though, that an \textsf{XOR} gate should output a \emph{True} if an odd number of inputs is \emph{True}. in \Le this type of behavior is found in a device called a ``parity gate'' and is covered in more detail elsewhere in this book.

%**********************************************************
% XNOR Gate
%**********************************************************
\subsection{XNOR}
\label{BF:subsec:xnor}

\textsf{XNOR} is a Boolean operation that will output a logical one, or \emph{True}, only if the two inputs are the same; thus, an \textsf{XNOR} gate is often referred to as an \emph{Equivalence Gate}. The \textsf{XNOR} operation is not often used in Boolean equations, but when necessary it is represented by a dot inside a circle. Equation \ref{BF:eq:xnor_symbol} shows an \textsf{XNOR}  operation. 

% XNOR in equation
\begin{align}
  \label{BF:eq:xnor_symbol}
  A \odot B &= Y
\end{align}

Table \ref{BF:tab:truth_table_for_xnor} is the truth table for \textsf{XNOR} . 

%******************************************************
% XNOR Truth Table
%******************************************************
\begin{table}[H]
  \sffamily
  \newcommand{\head}[1]{\textcolor{white}{\textbf{#1}}}    
  \begin{center}
    \rowcolors{2}{gray!10}{white} % Color every other line a light gray
    \begin{tabular}{ccc} 
      \rowcolor{black!75}
      \multicolumn{2}{c}{\head{Inputs}} & \head{Output} \\
      A & B & Y \\
      \hline
      0 & 0 & 1 \\
      0 & 1 & 0 \\
      1 & 0 & 0 \\
      1 & 1 & 1 
    \end{tabular}
  \end{center}
  \caption{Truth Table for XNOR}
  \label{BF:tab:truth_table_for_xnor}
\end{table}

In Figure \ref{fig:04_08}, the input variables $ A $ and $ B $ are wired to an \textsf{XNOR}  gate, and the output from that gate goes to $ Y $. 

\begin{figure}[H]
	\centering
	\includegraphics[width=\maxwidth{.95\linewidth}]{gfx/04_08}
	\caption{XNOR Gate}
	\label{fig:04_08}
\end{figure}


\subsection{Buffer}
\label{BF:subsec:buffer}

A buffer (sometimes called \emph{Transfer}) is a Boolean operation that transfers the input to the output without change. If the input is \emph{True}, then the output will be \emph{True} and if the input is \emph{False}, then the output will be \emph{False}. It may seem to be an odd function since this operation does not change anything, but it has an important use in a circuit. As logic circuits become more complex, the signal from input to output may become weak and no longer able to drive (or activate) additional gates. A buffer is used to boost (and stabilize) a logic level so it is more dependable. Another important function for a buffer is to clean up an input signal. As an example, when an electronic circuit interacts with the physical world (such as a user pushing a button), there is often a very brief period when the signal from that physical device waivers between high and low unpredictably. A buffer can smooth out that signal so it is a constant high or low without voltage spikes in between. 

Table \ref{BF:tab:truth_table_for_a_buffer} is the truth table for buffer.

%******************************************************
% Buffer Truth Table
%******************************************************
\begin{table}[H]
  \sffamily
  \newcommand{\head}[1]{\textcolor{white}{\textbf{#1}}}    
  \begin{center}
    \rowcolors{2}{gray!10}{white} % Color every other line a light gray
    \begin{tabular}{ccc} 
      \rowcolor{black!75}
      \head{Input} & \head{Output} \\
      0 & 0 \\
      1 & 1 
    \end{tabular}
  \end{center}
  \caption{Truth Table for a Buffer}
  \label{BF:tab:truth_table_for_a_buffer}
\end{table}

Buffers are rarely used in schematic diagrams since they do not actually change a signal; however, Figure \ref{fig:04_09}, illustrates a buffer. 

\begin{figure}[H]
	\centering
	\includegraphics[width=\maxwidth{.95\linewidth}]{gfx/04_09}
	\caption{Buffer}
	\label{fig:04_09}
\end{figure}

\section{Univariate Boolean Algebra Properties}
\label{BF:sec:univariate_boolean_algebra_properties}
\subsection{Introduction}
\label{BF:subsec:introduction_to_univariate}

Boolean Algebra, like real number algebra, includes a number of properties. This unit introduces the univariate Boolean properties, or those properties that involve only one input variable. These properties permit Boolean expressions to be simplified, and circuit designers are interested in simplifying circuits to reduce construction expense, power consumption, heat loss (wasted energy), and troubleshooting time.  

\subsection{Identity}
\label{BF:subsec:identity}

In mathematics, an identity is an equality where the left and right members are the same regardless of the values of the variables present. As an example, Equation \ref{BF:eq:identity_example} is an identity since the two members are identical regardless of the value of $ \alpha $:

\begin{align}
  \label{BF:eq:identity_example}
  \frac{\alpha}{2} &= 0.5\alpha
\end{align}

An \emph{Identity Element} is a special member of a set such that when that element is used in a binary operation the other element in that operation is not changed. This is sometimes called the \emph{Neutral Element} since it has no effect on binary operations. As an example, in Equation \ref{BF:eq:identity_element_of_addition} the two members of the equation are always identical. Therefore, zero is the identity element for addition since anything added to zero remains unchanged.

\begin{align}
  \label{BF:eq:identity_element_of_addition}
  a + 0 &= a 
\end{align}

In a logic circuit, combining any logic input with a logic zero through an \textsf{OR}  gate yields the original input. Logic zero, then, is considered the \textsf{OR} identity element because it causes the input of the gate to be copied to the output unchanged. Because \textsf{OR}  is represented by a plus sign when written in a Boolean equation, and the identity element for \textsf{OR}  is zero, Equation \ref{BF:eq:or_identity} is \emph{True}.

\begin{align}
  \label{BF:eq:or_identity}
  A + 0 &= A 
\end{align}

The bottom input to the \textsf{OR} gate in \ref{fig:04_10} is a constant logic zero, or \emph{False}. The output for this circuit, $ Y $, will be the same as input $ A $; therefore, the identity element for \textsf{OR}  is zero.  

\begin{figure}[H]
	\centering
	\includegraphics[width=\maxwidth{.95\linewidth}]{gfx/04_10}
	\caption{OR Identity Element}
	\label{fig:04_10}
\end{figure}

In the same way, combining any logic input with a logic one through an \textsf{AND}  gate yields the original input. Logic one, then, is considered the \textsf{AND}  identity element because it causes the input of the gate to be copied to the output unchanged. Because \textsf{AND}  is represented by a multiplication sign when written in a Boolean equation, and the identity element for \textsf{AND} is one, Equation \ref{BF:eq:and_identity} is \emph{True}.

\begin{align}
  \label{BF:eq:and_identity}
  A * 1 &= A 
\end{align}

The bottom input to the \textsf{AND} gate in \ref{fig:04_11} is a constant logic one, or \emph{True}. The output for this circuit, $ Y $, will be the same as input $ A $; therefore, the identity element for \textsf{AND}  is one.  

\begin{figure}[H]
	\centering
	\includegraphics[width=\maxwidth{.95\linewidth}]{gfx/04_11}
	\caption{AND Identity Element}
	\label{fig:04_11}
\end{figure}

\subsection{Idempotence}
\label{BF:subsec:idempotence}

If the two inputs of either an \textsf{OR} or \textsf{AND} gate are tied together, then the same signal will be applied to both inputs. This results in the output of either of those gates being the same as the input; and this is called the idempotence property. An electronic gate wired in this manner performs the same function as a buffer. 

% Pull Quote - Marginal Note - Sidebar
\marginpar{Remember that in Boolean expressions a plus sign represents an \textsf{OR}  gate, not mathematical addition.}

\begin{align}
  \label{BF:eq:idempotence_for_or}
  A + A &= A 
\end{align}

\begin{figure}[H]
	\centering
	\includegraphics[width=\maxwidth{.95\linewidth}]{gfx/04_12}
	\caption{Idempotence Property for OR Gate}
	\label{fig:04_12}
\end{figure}

Figure \ref{fig:04_12} illustrates the idempotence property for an \textsf{AND} gate.

% Pull Quote - Marginal Note - Sidebar
\marginpar{Remember that in a Boolean expression a multiplication sign represents an AND gate, not mathematical multiplying.}

\begin{align}
  \label{BF:eq:idempotence_for_and}
  A * A &= A 
\end{align}

\begin{figure}[H]
	\centering
	\includegraphics[width=\maxwidth{.95\linewidth}]{gfx/04_13}
	\caption{Idempotence Property for AND Gate}
	\label{fig:04_13}
\end{figure}

\subsection{Annihilator}
\label{BF:subsec:annihilator}

Combining any data and a logic one through an \textsf{OR} gate yields a constant output of one. This property is called the annihilator since the \textsf{OR} gate outputs a constant one; in other words, whatever other data were input are lost. Because \textsf{OR} is represented by a plus sign when written in a Boolean equation, and the annihilator for \textsf{OR} is one, the following is true:

\begin{align}
  \label{BF:eq:annihilator_or}
  A + 1 &= 1 
\end{align}

\begin{figure}[H]
	\centering
	\includegraphics[width=\maxwidth{.95\linewidth}]{gfx/04_14}
	\caption{Annihilator For OR Gate}
	\label{fig:04_14}
\end{figure}

The bottom input for the \textsf{OR}  gate in Figure \ref{fig:04_14} is a constant logic one, or \emph{True}. The output for this circuit will be \emph{True} (or $ 1 $) no matter whether input $ A $ is \emph{True} or \emph{False} ($ 1 $ or $ 0 $).

Combining any data and a logic zero with an \textsf{AND} gate yields a constant output of zero. This property is called the annihilator since the \textsf{AND}  gate outputs a constant zero; in other words, whatever logic data were input are lost. Because \textsf{AND}  is represented by a multiplication sign when written in a Boolean equation, and the annihilator for \textsf{AND} is zero, the following is true:

\begin{align}
  \label{BF:eq:annihilator_and}
  A * 0 &= 0 
\end{align}

\begin{figure}[H]
	\centering
	\includegraphics[width=\maxwidth{.95\linewidth}]{gfx/04_15}
	\caption{Annihilator For AND Gate}
	\label{fig:04_15}
\end{figure}

The bottom input for the \textsf{AND} gate in Figure \ref{fig:04_15} is a constant logic zero, or \emph{False}. The output for this circuit will be \emph{False} (or $ 0 $) no matter whether input $ A $ is \emph{True} or \emph{False} ($ 1 $ or $ 0 $).

\subsection{Complement}
\label{BF:subsec:complement}

In Boolean logic there are only two possible values for variables: $ 0 $ and $ 1 $. Since either a variable or its complement must be one, and since combining any data with one through an \textsf{OR} gate yields one (see the Annihilator in Equation \ref{BF:eq:annihilator_or}), then the following is true:

\begin{align}
  \label{BF:eq:complement_or}
  A + A' &= 1 
\end{align}

\begin{figure}[H]
	\centering
	\includegraphics[width=\maxwidth{.95\linewidth}]{gfx/04_16}
	\caption{OR Complement}
	\label{fig:04_16}
\end{figure}

In Figure \ref{fig:04_16}, the output ($ Y $) will always equal one, regardless of the value of input $ A $. This leads to the general property that when a variable and its complement are combined through an \textsf{OR} gate the output will always be one. 

In the same way, since either a variable or its complement must be zero, and since combining any data with zero through an \textsf{AND} gate yields zero (see the Annihilator in Equation \ref{BF:eq:annihilator_and}), then the following is true: 

\begin{align}
  \label{BF:eq:complement_and}
  A * A' &= 0
\end{align}

\begin{figure}[H]
	\centering
	\includegraphics[width=\maxwidth{.95\linewidth}]{gfx/04_17}
	\caption{AND Complement}
	\label{fig:04_17}
\end{figure}

\subsection{Involution}
\label{BF:subsec:involution}

\marginpar{The Involution Property is sometimes called the ``Double Complement'' Property.}

Another law having to do with complementation is that of Involution. Complementing a Boolean variable two times (or any even number of times) results in the original Boolean value. 

\begin{align}
  \label{BF:eq:involuton}
  (A')' &= A
\end{align}

\begin{figure}[H]
	\centering
	\includegraphics[width=\maxwidth{.95\linewidth}]{gfx/04_18}
	\caption{Involution Property}
	\label{fig:04_18}
\end{figure}

In the circuit illustrated in Figure \ref{fig:04_18}, the output ($ Y $) will always be the same as the input ($ A $). 

% Begin Sidebar Box
\begin{tcolorbox}[colback=blue!5!white,colframe=blue!75!black]
  % Upper half of box: my "title" area
  \textcolor{blue}{\textbf{Propagation Delay}}
  % Lower half of the box: the content
  \tcblower
  It takes the two \textsf{NOT} gates a short period of time to pass a signal from input to output, which is known as ``propagation delay.'' A designer occasionally needs to build an intentional signal delay into a circuit for some reason and two (or any even number of) consecutive \textsf{NOT} gates would be one option.
\end{tcolorbox}
% End Sidebar Box

\section{Multivariate Boolean Algebra Properties}
\label{BF:sec:multivariate_boolean_algebra_properties}
\subsection{Introduction}
\label{BF:subsec:introduction_to_multivariate}

Boolean Algebra, like real number algebra, includes a number of properties. This unit introduces the multivariate Boolean properties, or those properties that involve more than one input variable. These properties permit Boolean expressions to be simplified, and circuit designers are interested in simplifying circuits to reduce construction expense, power consumption, heat loss (wasted energy), and troubleshooting time.  

\subsection{Commutative}
\label{BF:subsec:commutative_property}

\marginpar{The examples here show only two variables, but this property is true for any number of variables.}

In essence, the commutative property indicates that the order of the input variables can be reversed in either \textsf{OR} or \textsf{AND} gates without changing the truth of the expression. Equation \ref{BF:eq:commutative} expresses this property algebraically.

\begin{align}
  \label{BF:eq:commutative}
  A + B &= B + A \\
  \nonumber
  A * B &= B * A
\end{align}

\begin{figure}[H]
	\centering
	\includegraphics[width=\maxwidth{.95\linewidth}]{gfx/04_19}
	\caption{Commutative Property for OR}
	\label{fig:04_19}
\end{figure}

\marginpar{\textsf{XOR} and \textsf{XNOR} are also commutative; but for only two variables, not three or more.}

\begin{figure}[H]
	\centering
	\includegraphics[width=\maxwidth{.95\linewidth}]{gfx/04_20}
	\caption{Commutative Property for AND}
	\label{fig:04_20}
\end{figure}

In Figures \ref{fig:04_19} and \ref{fig:04_20} the inputs are reversed for the two gates, but the outputs are the same. For example, $ A $ is entering the top input for the upper \textsf{OR} gate, but the bottom input for the lower gate; however, $ Y1 $ is always equal to $ Y2 $.

\subsection{Associative}
\label{BF:subsec:associative_property}

\marginpar{The examples here show only three variables, but this property is true for any number of variables.}

This property indicates that groups of variables in an \textsf{OR} or \textsf{AND} gate can be associated in various ways without altering the truth of the equations. Equation \ref{BF:eq:associative} expresses this property algebraically: 

\begin{align}
  \label{BF:eq:associative}
  ( A + B ) + C &= A + ( B + C ) \\
  \nonumber
  ( A * B ) * C &= A * ( B * C )
\end{align}

\marginpar{\textsf{XOR} and \textsf{XNOR} are also associative; but for only two variables, not three or more.}

In the circuits in Figure \ref{fig:04_21} and \ref{fig:04_22} , notice that $ A $ and $ B $ are associated together in the first gate, and then $ C $ is associated with the output of that gate. Then, in the lower half of the circuit, $ B $ and $ C $ are associated together in the first gate, and then $ A $ is associated with the output of that gate. Since $ Y1 $ is always equal to $ Y2 $ for any combination of inputs, it does not matter which of the two variables are associated together in a group of gates.  

\begin{figure}[H]
	\centering
	\includegraphics[width=\maxwidth{.95\linewidth}]{gfx/04_21}
	\caption{Associative Property for OR}
	\label{fig:04_21}
\end{figure}

\begin{figure}[H]
	\centering
	\includegraphics[width=\maxwidth{.95\linewidth}]{gfx/04_22}
	\caption{Associative Property for AND}
	\label{fig:04_22}
\end{figure}

\subsection{Distributive}
\label{BF:subsec:distributive_property}

The distributive property of real number algebra permits certain variables to be ``distributed'' to other variables. This operation is frequently used to create groups of variables that can be simplified; thus, simplifying the entire expression. Boolean algebra also includes a distributive property, and that can be used to combine \textsf{OR}  or \textsf{AND}  gates in various ways that make it easier to simplify the circuit. Equation \ref{BF:eq:distributive} expresses this property algebraically:

\begin{align}
  \label{BF:eq:distributive}
  A( B + C ) &= AB + AC \\
  \nonumber
  A + (BC) &= (A + B) (A + C)
\end{align}

In the circuits illustrated in Figures \ref{fig:04_23} and \ref{fig:04_24}, notice that input $ A $ in the top half of the circuit is distributed to inputs $ B $ and $ C $ in the bottom half. However, output $ Y1 $ is always equal to output $ Y2 $ regardless of how the inputs are set. These two circuits illustrate Distributive of \textsf{AND} over \textsf{OR}  and Distributive of \textsf{OR} over \textsf{AND} .
 
\begin{figure}[H]
	\centering
	\includegraphics[width=\maxwidth{.95\linewidth}]{gfx/04_23}
	\caption{Distributive Property for AND over OR}
	\label{fig:04_23}
\end{figure}

\begin{figure}[H]
	\centering
	\includegraphics[width=\maxwidth{.95\linewidth}]{gfx/04_24}
	\caption{Distributive Property for OR over AND}
	\label{fig:04_24}
\end{figure}

\subsection{Absorption}
\label{BF:subsec:absorption_property}

The absorption property is used to remove logic gates from a circuit if those gates have no effect on the output. In essence, a gate is ``absorbed'' if it is not needed. There are two different absorption properties:

\begin{align}
  \label{BF:eq:absorption}
  A + (AB) &= A \\
  \nonumber
  A(A + B) &= A
\end{align}

The best way to think about why these properties are true is to imagine a circuit that contains them. The first circuit below illustrates the top equation. 

\begin{figure}[H]
	\centering
	\includegraphics[width=\maxwidth{.95\linewidth}]{gfx/04_25}
	\caption{Absorption Property (Version 1)}
	\label{fig:04_25}
\end{figure}

Table \ref{BF:tab:truth_table_for_absorption_property_version_1} is the truth table for the circuit in Figure \ref{fig:04_25}.

%******************************************************
% Absorption (Version 1) Truth Table
%******************************************************
\begin{table}[H]
  \sffamily
  \newcommand{\head}[1]{\textcolor{white}{\textbf{#1}}}    
  \begin{center}
    \rowcolors{2}{gray!10}{white} % Color every other line a light gray
    \begin{tabular}{ccc} 
      \rowcolor{black!75}
      \multicolumn{2}{c}{\head{Inputs}} & \head{Output} \\
      A & B & Y \\
      \hline
      0 & 0 & 0 \\
      0 & 1 & 0 \\
      1 & 0 & 1 \\
      1 & 1 & 1 
    \end{tabular}
  \end{center}
  \caption{Truth Table for Absorption Property}
  \label{BF:tab:truth_table_for_absorption_property_version_1}
\end{table}

Notice that the output, $ Y $, is always the same as input $ A $. This means that input $ B $ has no bearing on the output of the circuit; therefore, the circuit could be replaced by a piece of wire from input $ A $ to output $ Y $. Another way to state that is to say that input $ B $ is absorbed by the circuit.

The circuit illustrated in Figure \ref{fig:04_26} is the second version of the Absorption Property. Like the first Absorption Property circuit, a truth table would demonstrate that input $ B $ is absorbed by the circuit. 

\begin{figure}[H]
	\centering
	\includegraphics[width=\maxwidth{.95\linewidth}]{gfx/04_26}
	\caption{Absorption Property (Version 2)}
	\label{fig:04_26}
\end{figure}

\subsection{Adjacency}
\label{BF:subsec:adjacency_property}

The adjacency property simplifies a circuit by removing unnecessary gates.

\begin{align}
  \label{BF:eq:adjacency}
  AB + AB' &= A
\end{align}

This property can be proven by simple algebraic manipulation:

\begin{align}
  \label{BF:eq:adjacency_solved}
  AB + AB' && \text{Original Expression} \\
  \nonumber
  A ( B + B') && \text{Distributive Property} \\
  \nonumber
  A1 && \text{Complement Property} \\
  \nonumber
  A && \text{Identity Element}
\end{align}

The circuit in Figure \ref{fig:04_27} illustrates the adjacency property. If this circuit were constructed it would be seen that the output, $ Y $, is always the same as input $ A $; therefore, this entire circuit could be replaced by a single wire from input $ A $ to output $ Y $.

\begin{figure}[H]
	\centering
	\includegraphics[width=\maxwidth{.95\linewidth}]{gfx/04_27}
	\caption{Adjacency Property}
	\label{fig:04_27}
\end{figure}

\section{DeMorgan's Theorem}
\label{BF:sec:demorgans_theorem}
\subsection{Introduction}
\label{BF:subsec:introduction_demorgans}

A mathematician named Augustus DeMorgan developed a pair of important theorems regarding the complementation of groups in Boolean algebra. DeMorgan found that an \textsf{OR} gate with all inputs inverted (a Negative-\textsf{OR} gate) behaves the same as a \textsf{NAND}  gate with non-inverted inputs; and an \textsf{AND} gate with all inputs inverted (a Negative-\textsf{AND} gate) behaves the same as a \textsf{NOR} gate with non-inverted inputs. DeMorgan's theorem states that inverting the output of any gate is the same as using the opposite type of gate with inverted inputs. Figure \ref{fig:04_28} illustrates this in circuit terms: the \textsf{NAND} gate with normal inputs and the \textsf{OR} gate with inverted inputs are functionally equivalent; that is, $ Y1 $ will always equal $ Y2 $, regardless of the values of input $ A $ or $ B $.  

\begin{figure}[H]
	\centering
	\includegraphics[width=\maxwidth{.95\linewidth}]{gfx/04_28}
	\caption{DeMorgan's Theorem Defined}
	\label{fig:04_28}
\end{figure}

The NOT function is commonly represented in an equation as an apostrophe because it is easy to enter with a keyboard, like: $ (AB)' $ for \emph{A AND B NOT}. However, it is easiest to work with DeMorgan's theorem if \textsf{NOT} is represented by an overline rather than an apostrophe, so it would be written as $ \overline{AB} $ rather than $ (AB)' $. Remember that an overline is a grouping symbol (like parenthesis) and it means that everything under that bar would first be combined (using an \textsf{AND} or \textsf{OR} gate) and then the output of the combination would be complemented.

\subsection{Applying DeMorgan's Theorem}
\label{BF:subsec:applying_demorgans_theorem}

Applying DeMorgan's theorem to a Boolean expression may be thought of in terms of \emph{breaking the bar}. When applying DeMorgan's theorem to a Boolean expression:

\begin{enumerate}
\item A complement bar is broken over a group of variables.
\item The operation (\textsf{AND}  or \textsf{OR} ) directly underneath the broken bar changes.
\item Pieces of the broken bar remain over the individual variables. 
\end{enumerate}

To illustrate:

\begin{align}
  \label{BF:eq:demorgan_nand}
  \overline{A*B} \leftrightarrow \overline{A}+\overline{B}
\end{align}

\begin{align}
  \label{BF:eq:demorgan_nor}
  \overline{A+B} \leftrightarrow \overline{A}*\overline{B}
\end{align}

Equation \ref{BF:eq:demorgan_nand} shows how a two-input \textsf{NAND} gate is ``broken'' to form an \textsf{OR} gate with two inverted inputs and equation \ref{BF:eq:demorgan_nor} shows how a two-input \textsf{NOR} gate is ``broken'' to form an \textsf{AND} gate with two complemented inputs.

\subsection{Simple Example}
\label{BF:subsec:simple_example_demorgan}

When multiple ``layers'' of bars exist in an expression, only one bar is broken at a time, and the longest, or uppermost, bar is broken first. As an example, consider the circuit in Figure \ref{fig:04_29}: 

\begin{figure}[H]
	\centering
	\includegraphics[width=\maxwidth{.95\linewidth}]{gfx/04_29}
	\caption{DeMorgan's Theorem Example 1}
	\label{fig:04_29}
\end{figure}

By writing the output at each gate (as illustrated in Figure \ref{fig:04_29}), it is easy to determine the Boolean expression for the circuit. Note: all circuit diagrams in this book are generated with \Le and the text tool in that software does not permit drawing overbars. Therefore, the circuit diagram will use the apostrophe method of indicating NOT but overbars will be used in the text.

\begin{align}
  \label{BF:eq:demorgan_simple_example}
  \overline{A+\overline{BC}}
\end{align}

To simplify the circuit, break the bar covering the entire expression (the ``longest bar''), and then simplify the resulting expression.

\begin{align}
  \label{BF:eq:demorgan_simple_solved}
  \overline{A+\overline{BC}} && \text{Original Expression} \\
  \nonumber
  \overline{A}\,\overline{\overline{BC}} && \text{''Break'' the longer bar} \\
  \nonumber
  \overline{A}BC && \text{Involution Property}
\end{align}

As a result, the original circuit is reduced to a three-input \textsf{AND} gate with one inverted input.

\subsection{Incorrect Application of DeMorgan's Theorem}
\label{BF:subsec:incorrect_application_of_demorgans_theorem}

More than one bar is never broken in a single step, as illustrated in Equation \ref{BF:eq:demorgan_incorrect_solution}: 

\begin{align}
  \label{BF:eq:demorgan_incorrect_solution}
  \overline{A+\overline{BC}} && \text{Original Expression} \\
  \nonumber
  \overline{A\overline{B}}+\overline{\overline{C}} && \text{Improperly Breaking Two Bars} \\
  \nonumber
  \overline{A}B+C && \text{Incorrect Solution}
\end{align}

Thus, as tempting as it may be to take a shortcut and break more than one bar at a time, it often leads to an incorrect result. Also, while it is possible to properly reduce an expression by breaking the short bar first; more steps are usually required and that process is not recommended.  

\subsection{About Grouping}
\label{BF:subsec:demorgans_about_grouping}

An important, but easily neglected, aspect of DeMorgan's theorem concerns grouping. Since a bar functions as a grouping symbol, the variables formerly grouped by a broken bar must remain grouped or else proper precedence (order of operation) will be lost. Therefore, after simplifying a large grouping of variables, it is a good practice to place them in parentheses in order to keep the order of operation the same.  

Consider the circuit in Figure \ref{fig:04_30}.

\begin{figure}[H]
	\centering
	\includegraphics[width=\maxwidth{.95\linewidth}]{gfx/04_30}
	\caption{DeMorgan's Theorem Example 2}
	\label{fig:04_30}
\end{figure}

As always, the first step in simplifying this circuit is to generate the Boolean expression for the circuit, which is done by writing the sub-expression at the output of each gate. That results in Expression \ref{BF:eq:demorgan_grouping_solution}, which is then simplified. 

\begin{align}
  \label{BF:eq:demorgan_grouping_solution}
  \overline{ \overline{A+BC} + \overline{A \overline{B}}} && \text{Original Expression} \\
  \nonumber
  ( \overline{\overline{A+BC}} ) ( \overline{\overline{A\overline{B}}} ) && \text{Breaking the Longest Bar} \\
  \nonumber
  (A+BC)(A\overline{B}) && \text{Involution} \\
  \nonumber
  (AA\overline{B}) (BCA\overline{B}) && \text{Distribute $ A\overline{B} $ to $ (A+BC) $} \\
  \nonumber
  (A\overline{B})+(BCA\overline{B}) && \text{Idempotence: $ AA=A $} \\
  \nonumber
  (A\overline{B})+(0CA)) && \text{Complement: $ B\overline{B}=0 $} \\
  \nonumber
  (A\overline{B})+0 && \text{Annihilator: $ 0CA=0 $} \\
  \nonumber
  A\overline{B} && \text{Identity: $ A+0=A $}
\end{align}

The equivalent gate circuit for this much-simplified expression is as follows:

\begin{figure}[H]
	\centering
	\includegraphics[width=\maxwidth{.95\linewidth}]{gfx/04_31}
	\caption{DeMorgan's Theorem Example 2 Simplified}
	\label{fig:04_31}
\end{figure}

\subsection{Summary}
\label{BF:subsec:demorgans_summary}

Here are the important points to remember about DeMorgan's Theorem: 

\begin{itemize}
  \item It describes the equivalence between gates with inverted inputs and gates with inverted outputs. 
  \item When "breaking" a complementation (or \textsf{NOT}) bar in a Boolean expression, the operation directly underneath the break (\textsf{AND} or \textsf{OR}) reverses and the broken bar pieces remain over the respective terms. 
  \item It is normally easiest to approach a problem by breaking the longest (uppermost) bar before breaking any bars under it. 
  \item Two complementation bars are never broken in one step. 
  \item Complementation bars function as grouping symbols. Therefore, when a bar is broken, the terms underneath it must remain grouped. Parentheses may be placed around these grouped terms as a help to avoid changing precedence.
\end{itemize}

\subsection{Example Problems}
\label{BF:subsec:demorgans_example_problems}

The following examples use DeMorgan's Theorem to simplify a Boolean expression.

%******************************************************
% DeMorgan's Problems
%******************************************************
\begin{table}[H]
  \newcommand{\head}[1]{\textcolor{white}{\textbf{#1}}}    
  \begin{center}
    \rowcolors{2}{gray!10}{white} % Color every other line a light gray
    \begin{tabular}{c|cc} 
      \rowcolor{black!75}
      & \head{Original Expression} & \head{Simplified} \\
      1 & $ (\overline{A+B})(\overline{ABC})(\overline{\overline{A}C}) $ 
        & $ \overline{A}\,\overline{B}\,\overline{C} $ \\
      2 & $ \overline{(AB+\overline{B}C)+(B\overline{C}+\overline{A}B)} $ 
        & $ \overline{B}\,\overline{C} $ \\
      3 & $ (AB+\overline{B}C)(AC+\overline{A}\,\overline{C}) $ 
        & $ \overline{A}+\overline{C} $ 
    \end{tabular}
  \end{center}
  \label{BF:tab:demorgans_example_problems}
  \sffamily
\end{table}

\section{Boolean Functions}
\label{BF:sec:boolean_functions}

Consider Figure \ref{fig:04_32}, which is a generic circuit with two inputs and one output.

\begin{figure}[H]
	\centering
	\includegraphics[width=\maxwidth{.95\linewidth}]{gfx/04_32}
	\caption{Generic Function}
	\label{fig:04_32}
\end{figure}

Without knowing anything about what is in the unlabeled box at the center of the circuit, there are a number of possible truth tables which could describe the circuit's output. Two possibilities are shown in Truth Table \ref{BF:tab:truth_table_for_generic_circuit_one} and Truth Table \ref{BF:tab:truth_table_for_generic_circuit_two}.

%******************************************************
% Generic Truth Table 1
%******************************************************
\begin{table}[H]
  \sffamily
  \newcommand{\head}[1]{\textcolor{white}{\textbf{#1}}}    
  \begin{center}
    \rowcolors{2}{gray!10}{white} % Color every other line a light gray
    \begin{tabular}{ccc} 
      \rowcolor{black!75}
      \multicolumn{2}{c}{\head{Inputs}} & \head{Output} \\
      A & B & Y \\
      \hline
      0 & 0 & 0 \\
      0 & 1 & 1 \\
      1 & 0 & 0 \\
      1 & 1 & 1 
    \end{tabular}
  \end{center}
  \caption{Truth Table for Generic Circuit One}
  \label{BF:tab:truth_table_for_generic_circuit_one}
\end{table}

%******************************************************
% Generic Truth Table 2
%******************************************************
\begin{table}[H]
  \sffamily
  \newcommand{\head}[1]{\textcolor{white}{\textbf{#1}}}    
  \begin{center}
    \rowcolors{2}{gray!10}{white} % Color every other line a light gray
    \begin{tabular}{ccc} 
      \rowcolor{black!75}
      \multicolumn{2}{c}{\head{Inputs}} & \head{Output} \\
      A & B & Y \\
      \hline
      0 & 0 & 0 \\
      0 & 1 & 1 \\
      1 & 0 & 1 \\
      1 & 1 & 0 
    \end{tabular}
  \end{center}
  \caption{Truth Table for Generic Circuit Two}
  \label{BF:tab:truth_table_for_generic_circuit_two}
\end{table}

In fact, there are $ 16 $ possible truth tables for this circuit. Each of those truth tables reflect a single potential function of the circuit by setting various combinations of input/output. Therefore, any two-input, one-output circuit has $ 16 $ possible functions. It is easiest to visualize all $ 16 $ combinations of inputs/outputs by using an odd-looking truth table. Consider only one of those $ 16 $ functions, the one for the generic circuit described by Truth Table \ref{BF:tab:truth_table_for_generic_circuit_two}. That function is also found in the Boolean Functions Table \ref{BF:tab:boolean_functions} and one row from that table is reproduced in Table \ref{BF:tab:boolean_function_six}.

\begin{table}[H]
  \sffamily
  \newcommand{\head}[1]{\textcolor{white}{\textbf{#1}}}    
  \begin{center}
    %\rowcolors{2}{gray!10}{white} % Color every other line a light gray
    \begin{tabular}{llllll}
      \textbf{A} & 0 & \cellcolor{gray!10}0 & 1 & 1 &  \\ 
      \textbf{B} & 0 & \cellcolor{gray!10}1 & 0 & 1 &  \\ \hline
      $ F_{6} $ & 0 & \cellcolor{gray!10}1 & 1 & 0 
      & Exclusive Or (XOR): $ A \oplus B $ \\ 
    \end{tabular} 
  \end{center}
  \caption{Boolean Function Six}
  \label{BF:tab:boolean_function_six}
\end{table}

The line shown in Table \ref{BF:tab:boolean_function_six} is for \emph{Function 6}, or $ F_6 $ (note that the pattern of the outputs is $ 0110 $, which is binary $ 6 $). Inputs $ A $ and $ B $ are listed at the top of the table. For example, the highlighted column of the table shows that when $ A $ is zero and $ B $ is one the output is one. Therefore, on the line that defines $ F_6 $, the output is \emph{True} when $ [ (A=0 $ \textsf{AND} $ B=1) $ \textsf{OR} $ (A=1 $ \textsf{AND} $ B=0) ] $ This is an \textsf{XOR} function, and the last column of the table verbally describes that function.   

Table \ref{BF:tab:boolean_functions} is the complete Boolean Function table.

\begin{table}[H]
  \sffamily
  \newcommand{\head}[1]{\textcolor{white}{\textbf{#1}}}    
  \begin{center}
    \rowcolors{2}{gray!10}{white} % Color every other line a light gray
    \begin{tabular}{llllll}
    \textbf{A} & 0 & 0 & 1 & 1 &  \\ 
    \textbf{B} & 0 & 1 & 0 & 1 &  \\ \hline
    $ F_{0} $ & 0 & 0 & 0 & 0 
      & Zero or Clear. Always zero (Annihilation) \\ 
    $ F_{1} $ & 0 & 0 & 0 & 1 
      & Logical AND: $ A * B $  \\ 
    $ F_{2} $ & 0 & 0 & 1 & 0 
      & Inhibition: $ AB' $ or $ A>B $ \\ 
    $ F_{3} $ & 0 & 0 & 1 & 1 
      & Transfer A to Output, Ignore B \\ 
    $ F_{4} $ & 0 & 1 & 0 & 0 
      & Inhibition: $ A'B $ or $ B>A $ \\ 
    $ F_{5} $ & 0 & 1 & 0 & 1 
      & Transfer B to Output, Ignore A \\ 
    $ F_{6} $ & 0 & 1 & 1 & 0 
      & Difference, XOR: $ A \oplus B $ \\ 
    $ F_{7} $ & 0 & 1 & 1 & 1 
      & Logical OR: $ A + B $ \\ 
    $ F_{8} $ & 1 & 0 & 0 & 0 
      & Logical NOR: $ (A + B)' $ \\ 
    $ F_{9} $ & 1 & 0 & 0 & 1 
      & Equivalence, XNOR: $ (A = B)' $ \\ 
    $ F_{10} $ & 1 & 0 & 1 & 0 
      & Not B and ignore A, B Complement \\ 
    $ F_{11} $ & 1 & 0 & 1 & 1 
      & Implication, $ A + B' $, $ B >= A $ \\ 
    $ F_{12} $ & 1 & 1 & 0 & 0 
      & Not A and ignore B, A Complement \\ 
    $ F_{13} $ & 1 & 1 & 0 & 1 
      & Implication, $ A' + B $, $ A >= B $ \\ 
    $ F_{14} $ & 1 & 1 & 1 & 0 
      & Logical NAND: $ (A*B)' $ \\ 
    $ F_{15} $ & 1 & 1 & 1 & 1 
      & One or Set. Always one (Identity) \\ 
    \end{tabular} 
  \end{center}
  \caption{Boolean Functions}
  \label{BF:tab:boolean_functions}
\end{table}

\section{Functional Completeness}
\label{BF:sec:functional_completeness}

A set of Boolean operations is said to be \emph{functionally complete} if every possible Boolean function can be derived from that set. The Primary Logic Operations (page \pageref{BF:sec:primary_logic_operations}) are functionally complete since the Secondary Logic Functions (page \pageref{BF:sec:secondary_logic_functions}) can be derived from them. As an example, Equation \ref{BF:eq:xor_from_and} and Figure \ref{fig:04_33} shows how an \textsf{XOR} function can be derived from only \textsf{AND}, \textsf{OR}, and \textsf{NOT} gates.

% XOR in equation
\begin{align}
  \label{BF:eq:xor_from_and}
  ( (A * B)' * ( A + B) )  &= A \oplus B
\end{align}

\begin{figure}[H]
	\centering
	\includegraphics[width=\maxwidth{.95\linewidth}]{gfx/04_33}
	\caption{XOR Derived From AND/OR/NOT}
	\label{fig:04_33}
\end{figure}

While the Primary Operations are functionally complete, it is possible to define other functionally complete sets of operations. For example, using DeMorgan's Theorem, the set of \{\textsf{AND}, \textsf{NOT}\} is also functionally complete since the \textsf{OR}  operation can be defined as $ (A'B')' $. In fact, both \{\textsf{NAND}\} and \{NOR\} operations are functionally complete by themselves. As an example, the \textsf{NOT}  operation can be derived using only \textsf{NAND} gates: $ ( A | A ) $. Because all Boolean functions can be derived from either \textsf{NAND} or \textsf{NOR} operations, these are sometimes considered \emph{universal} operations and it is a common challenge for students to create some complex Boolean function using only one of these two types of operations.
\chapter{Boolean Expressions}\label{ch05}

\begin{tcolorbox}[colback=blue!5!white,colframe=blue!75!black]
	% Upper half of box: my "title" area
	\textcolor{blue}{\textbf{What to Expect}}
	% Lower half of the box: the content
	\tcblower
	A Boolean expression uses the various Boolean functions to create a mathematical model of a digital logic circuit. That model can then be used to build a circuit in a simulator like \textit{Logisim-Evolution}. The following topics are included in this chapter.
	
	\begin{itemize}
		\item Creating a Boolean expression from a description
		\item Analyzing a circuit's properties to develop a minterm or maxterm expression
		\item Determining if a Boolean expression is in canonical form
		\item Converting a Boolean expression with missing terms to its canonical equivalent
		\item Simplifying a complex Boolean expression using algebraic methods
	\end{itemize}
	
\end{tcolorbox}

\section{Introduction}

Electronic circuits that do not require any memory devices (like flip-flops or registers) are created using what is called ``Combinational Logic.'' These systems can be quite complex, but all outputs are determined solely by input signals that are processed through a series of logic gates. Combinational circuits can be reduced to a Boolean Algebra expression, though it may be quite complex; and that expression can be simplified using methods developed in this chapter and Chapter \ref{ch06}, \nameref{ch06}, page \pageref{ch06}, and Chapter \ref{ch07}, \nameref{ch07}, page \pageref{ch07}. Combinational circuits are covered in Chapter \ref{ch08}, \nameref{ch08}, page \pageref{ch08}.

In contrast, electronic circuits that require memory devices (like flip-flops or registers) use what is called ``Sequential Logic.'' Those circuits often include feedback loops, so the final output is determined by input signals plus the feedback loops that are processed through a series of logic gates. This makes sequential logic circuits much more complex than combinational and the simplification of those circuits is covered in Chapter \ref{ch09}, \nameref{ch09}, page \pageref{ch09}. 

Finally, most complex circuits include both combinational and sequential sub-circuits. In that case, the various sub-circuits would be independently simplified using appropriate methods. Several examples of these types of circuits are analyzed in Chapter \ref{ch10}, \nameref{ch10}, page \pageref{ch10}.

\section{Creating Boolean Expressions}

A circuit designer is often only given a written (or oral) description of a circuit and then asked to build that device. Too often, the designer may receive notes scribbled on the back of a dinner napkin, along with some verbal description of the desired output, and be expected to build a circuit to accomplish that task. Regardless of the form of the request, the process that the designer follows, in general, is: 

\begin{enumerate}
  \item \textsc{Write The Problem Statement}. The problem to be solved is written in a clear, concise statement. The better this statement is written the easier each of the following steps will be, so time spent polishing the problem statement is worthwhile. 
  
  \item \textsc{Construct A Truth Table}. Once the problem is clearly defined, the circuit designer constructs a truth table where all inputs/outputs are included. It is essential that all possible input combinations that lead to a \emph{True} output are identified. 
  
  \item \textsc{Write A Boolean Expression}. When the truth table is completed, it is easy to create a Boolean expression from that table as covered in Example \ref{05:subsec:example}. 
  
  \item \textsc{Simplify the Boolean Expression}. The expression should be simplified as much as possible, and that process is covered in this chapter, Chapter \ref{ch06}, \nameref{ch06}, page \pageref{ch06}, and Chapter \ref{ch07}, \nameref{ch07}, page \pageref{ch07}. 
  
  \item \textsc{Draw The Logic Diagram}. The logic diagram for a circuit is constructed from the simplified Boolean expression. 
  
  \item \textsc{Build The Circuit}. If desired, a physical circuit can be built using the logic diagram. 
\end{enumerate}

\subsection{Example}
\label{05:subsec:example}

A machine is to be programmed to help pack shipping boxes for the ABC Novelty Company. They are running a promotion so if a customer purchases any two of the following items, but not all three, a free poster will be added to the purchase: joy buzzer, fake blood, itching powder. Design the logic needed to add the poster to appropriate orders. 

\begin{enumerate}
  \item \textsc{Problem Statement}. The problem is already fairly well stated. A circuit is needed that will activate the ``drop poster'' machine when any two of three inputs (joy buzzer, fake blood, itching powder), but not all three, are \emph{True}. 

  \item \textsc{Truth Table}. Let $ J $ be the Joy Buzzer, $ B $ be the Fake Blood, and $ P $ be the Itching Powder; and let the truth table inputs be \emph{True} (or $ 1 $) when any of those items are present in the shipping box. Let the output $ D $ be for ``Drop Poster'' and when it is \emph{True} (or $ 1 $) then a poster will be dropped into the shipping box. The Truth table is illustrated in Table  \ref{05:tab:truth_table_for_example}.
  
  \begin{table}[H]
    \sffamily
    \newcommand{\head}[1]{\textcolor{white}{\textbf{#1}}}    
    \begin{center}
      \rowcolors{2}{gray!10}{white} % Color every other line a light gray
      \begin{tabular}{cccc} 
        \rowcolor{black!75}
        \multicolumn{3}{c}{\head{Inputs}} & \head{Output} \\
        J & B & P & D \\
        \hline
        0 & 0 & 0 & 0 \\
        0 & 0 & 1 & 0 \\
        0 & 1 & 0 & 0 \\
        0 & 1 & 1 & 1 \\
        1 & 0 & 0 & 0 \\
        1 & 0 & 1 & 1 \\
        1 & 1 & 0 & 1 \\
        1 & 1 & 1 & 0
      \end{tabular}
    \end{center}
    \caption{Truth Table for Example}
    \label{05:tab:truth_table_for_example}
  \end{table}
  
  \item \textsc{Write Boolean Expression}. According to the Truth Table, the poster will be dropped into the shipping box in only three cases (when output $ D $ is \emph{True}). Equation \ref{05:eq:example} was generated from the truth table. 
  
  \begin{align}
  \label{05:eq:example}
  BP + JP + JB &= D 
  \end{align}
  
  \item \textsc{Simplify Boolean Expression}. The Boolean expression for this problem is already as simple as possible so no further simplification is needed. 
  
  \item \textsc{Draw Logic Diagram}. Figure \ref{fig:05_01} was drawn from the switching equation.
  
	\begin{figure}[H]
		\centering
		\includegraphics[width=\maxwidth{.95\linewidth}]{gfx/05_01}
		\caption{Logic Diagram From Switching Equation}
		\label{fig:05_01}
	\end{figure}
  
  \item \textsc{Build The Circuit}. This circuit could be built (or ``realized'') with three \textsf{AND}  gates and one 3-input \textsf{OR}  gate. 
  
\end{enumerate}

%***************************************************************************
% Section: Minterms and Maxterms
%***************************************************************************
\section{Minterms and Maxterms}
\label{05:sec:minterms_and_maxterms}

\subsection{Introduction}
\label{05:subsec:introduction_to_minterms_and_maxterms}

The solution to a Boolean equation is normally expressed in one of two formats: \gls{sop} or \gls{pos}.

\subsection{Sum Of Products (SOP) Defined}
\label{05:subsec:sum_of_products_sop_defined}

Equation \ref{05:eq:sop_example} is an example of a \gls{sop} expression. Notice that the expression describes four inputs ($ A $, $ B $, $ C $, $ D $) that are combined through two \textsf{AND} gates and then the output of those \textsf{AND} gates are combined through an \textsf{OR} gate. 

\begin{align}
  \label{05:eq:sop_example}
  (A'BC'D)+(AB'CD) &= Y
\end{align}

Each of the two terms in this expression is a \emph{minterm}. Minterms can be identified in a Boolean expression as a group of inputs joined by an \textsf{AND} gate and then two or more minterms are combined with an \textsf{OR} gate. \marginpar{Notice the inverting bubble on three of the AND gate inputs.}The circuit illustrated in Figure \ref{fig:05_02} would realize Equation \ref{05:eq:sop_example}.

\begin{figure}[H]
	\centering
	\includegraphics[width=\maxwidth{.95\linewidth}]{gfx/05_02}
	\caption{Logic Diagram For SOP Example}
	\label{fig:05_02}
\end{figure}

\subsection{Product of Sums (POS) Defined}
\label{05:subsec:product_of_sums_pos_defined}

Equation \ref{05:eq:pos_example} is an example of a \gls{pos} expression. Notice that the expression describes four inputs ($ A $, $ B $, $ C $, $ D $) that are combined through two \textsf{OR} gates and then the output of those \textsf{OR} gates are combined through an \textsf{AND} gate. 

\begin{align}
  \label{05:eq:pos_example}
  (A'+B+C+'D)(A+B'+C+D) &= Y
\end{align}

Each term in this expression is called a \emph{maxterm}. Maxterms can be identified in a Boolean expression as a group of inputs joined by an \textsf{OR} gate; and then two or more maxterms are combined with an \textsf{AND} gate. \marginpar{Notice the inverting bubble on three of the OR gate inputs.}The circuit illustrated in Figure \ref{fig:05_03} would realize Equation \ref{05:eq:pos_example}. 

\begin{figure}[H]
	\centering
	\includegraphics[width=\maxwidth{.95\linewidth}]{gfx/05_03}
	\caption{Logic Diagram For POS Example}
	\label{fig:05_03}
\end{figure}

\subsection{About Minterms}
\label{05:subsec:about_minterms}

A term that contains all of the input variables in one row of a truth table joined with an \textsf{AND} gate is called a minterm. Consider truth table \ref{05:tab:truth_table_for_first_minterm_example} which is for a circuit with three inputs ( $ A $,  $ B $, and  $ C $) and one output ( $ Q $).

\begin{table}[H]
  \sffamily
  \newcommand{\head}[1]{\textcolor{white}{\textbf{#1}}}    
  \begin{center}
    \rowcolors{2}{gray!10}{white} % Color every other line a light gray
    \begin{tabular}{ccc|cc} 
      \rowcolor{black!75}
      \multicolumn{3}{c}{\head{Inputs}} & \multicolumn{2}{c}{\head{Outputs}} \\
      A & B & C & Q & m\\
      \hline
      0 & 0 & 0 & 0 & 0 \\
      0 & 0 & 1 & 0 & 1 \\
      0 & 1 & 0 & 0 & 2 \\
      0 & 1 & 1 & 1 & 3 \\
      1 & 0 & 0 & 0 & 4 \\
      1 & 0 & 1 & 1 & 5 \\
      1 & 1 & 0 & 0 & 6 \\
      1 & 1 & 1 & 0 & 7 
    \end{tabular}
  \end{center}
  \caption{Truth Table for First Minterm Example}
  \label{05:tab:truth_table_for_first_minterm_example}
\end{table}

The circuit that is represented by this truth table would output a \emph{True} in only two cases, when the inputs are $ A'BC $ or $ AB'C $. Equation \ref{05:eq:first_minterm_example} describes this circuit.

\begin{align}
  \label{05:eq:first_minterm_example}
  (A'BC)+(AB'C) &= Q
\end{align}

The terms $ A'BC $ and $ AB'C $ are called \emph{minterms} and they contain every combination of input variables that outputs a \emph{True} when the three inputs are joined by an \textsf{AND} gate. Minterms are most often used to describe circuits that have fewer \emph{True} outputs than \emph{False} (that is, there are fewer $ 1 $'s than $ 0 $'s in the output column). In the example above, there are only two \emph{True} outputs with six \emph{False} outputs, so minterms describe the circuit most efficiently. 

Minterms are frequently abbreviated with a lower-case  $ m $ along with a subscript that indicates the decimal value of the variables. For example, $ A'BC $, the first of the \emph{True} outputs in the truth table above, has a binary value of $ 011 $, which is a decimal value of $ 3 $; thus, the minterm is $ m_3 $. The other minterm in this equation is $ m_5 $ since its binary value is $ 101 $, which equals decimal $ 5 $. It is possible to verbally describe the entire circuit as: $ m_3 $ \textsf{OR} $ m_5 $. For convenience, each of the minterm numbers are indicated in the last column of the truth table.

As another example, consider truth table \ref{05:tab:truth_table_for_second_minterm_example}.

\begin{table}[H]
  \sffamily
  \newcommand{\head}[1]{\textcolor{white}{\textbf{#1}}}    
  \begin{center}
    \rowcolors{2}{gray!10}{white} % Color every other line a light gray
    \begin{tabular}{cccc|cc} 
      \rowcolor{black!75}
      \multicolumn{4}{c}{\head{Inputs}} & \multicolumn{2}{c}{\head{Outputs}} \\
      A & B & C & D & Q & m \\
      \hline
      0 & 0 & 0 & 0 & 1 & 0 \\
      0 & 0 & 0 & 1 & 0 & 1 \\
      0 & 0 & 1 & 0 & 0 & 2 \\
      0 & 0 & 1 & 1 & 0 & 3 \\
      0 & 1 & 0 & 0 & 0 & 4 \\
      0 & 1 & 0 & 1 & 1 & 5 \\
      0 & 1 & 1 & 0 & 1 & 6 \\
      0 & 1 & 1 & 1 & 0 & 7 \\
      1 & 0 & 0 & 0 & 0 & 8 \\
      1 & 0 & 0 & 1 & 0 & 9 \\
      1 & 0 & 1 & 0 & 0 & 10 \\
      1 & 0 & 1 & 1 & 0 & 11 \\
      1 & 1 & 0 & 0 & 0 & 12 \\
      1 & 1 & 0 & 1 & 0 & 13 \\
      1 & 1 & 1 & 0 & 1 & 14 \\
      1 & 1 & 1 & 1 & 0 & 15 
    \end{tabular}
  \end{center}
  \caption{Truth Table for Second Minterm Example}
  \label{05:tab:truth_table_for_second_minterm_example}
\end{table}

Equation \ref{05:eq:second_minterm_example} describes this circuit because these are the rows where the output is \emph{True}.

\begin{align}
  \label{05:eq:second_minterm_example}
  (A'B'C'D')+(A'BC'D)+(A'BCD')+(ABCD') &= Q
\end{align}

These would be minterms $ m_0 $, $ m_5 $, $ m_6 $, and $ m_{14} $. Equation \ref{05:eq:sigma_notation_second_minterm_example} shows a commonly used, more compact way to express this result.

\begin{align}
  \label{05:eq:sigma_notation_second_minterm_example}
  \int(A,B,C,D) &= \sum(0,5,6,14)
\end{align}

Equation \ref{05:eq:sigma_notation_second_minterm_example} would read: ``For the function of inputs  $ A $,  $ B $,  $ C $, and  $ D $, the output is \emph{True} for minterms $ 0 $, $ 5 $, $ 6 $, and $ 14 $ when they are combined with an \textsf{OR}  gate.'' This format is called \emph{Sigma Notation}, and it is easy to derive the full Boolean equation from it by remembering that $ m_0 $ is $ 0000 $, or $ A'B'C'D' $; $ m_5 $ is $ 0101 $, or $ A'BC'D $; $ m_6 $ is $ 0110 $, or $ A'BCD' $; and $ m_{14} $ is $ 1110 $, or $ ABCD' $. Therefore, the Boolean equation can be quickly created from the Sigma Notation. 

A logic equation that is created using minterms is often called the \emph{Sum of Products} (or \emph{SOP}) since each term is composed of inputs \textsf{AND}ed together (``products'') and the terms are then joined by \textsf{OR} gates (``sums'').

\subsection{About Maxterms}
\label{05:subsec:about_maxterms}

A term that contains all of the input variables joined with an \textsf{OR} gate (``added together'') for a \emph{False} output is called a \emph{maxterm}. Consider Truth Table \ref{05:tab:truth_table_for_first_maxterm_example}, which has three inputs ( $ A $,  $ B $, and  $ C $) and one output ( $ Q $): 

\begin{table}[H]
  \sffamily
  \newcommand{\head}[1]{\textcolor{white}{\textbf{#1}}}    
  \begin{center}
    \rowcolors{2}{gray!10}{white} % Color every other line a light gray
    \begin{tabular}{ccc|cc} 
      \rowcolor{black!75}
      \multicolumn{3}{c}{\head{Inputs}} & \multicolumn{2}{c}{\head{Outputs}} \\
      A & B & C & Q & M \\
      \hline
      0 & 0 & 0 & 1 & 0 \\
      0 & 0 & 1 & 1 & 1 \\
      0 & 1 & 0 & 0 & 2 \\
      0 & 1 & 1 & 1 & 3 \\
      1 & 0 & 0 & 1 & 4 \\
      1 & 0 & 1 & 1 & 5 \\
      1 & 1 & 0 & 0 & 6 \\
      1 & 1 & 1 & 1 & 7 
    \end{tabular}
  \end{center}
  \caption{Truth Table for First Maxterm Example}
  \label{05:tab:truth_table_for_first_maxterm_example}
\end{table}

The circuit that is represented by this truth table would output a \emph{False} in only two cases. Since there are fewer \emph{False} outputs than \emph{True}, it is easier to create a Boolean equation that would generate the \emph{False} outputs. Because the equation describes the \emph{False} outputs, each term is built by \emph{complementing} the inputs for each of the \emph{False} output lines. After the output groups are defined, they are joined with an \textsf{AND} gate. Equation \ref{05:eq:first_maxterm_example} is the Boolean equation for Truth Table \ref{05:tab:truth_table_for_first_maxterm_example}.

\begin{align}
  \label{05:eq:first_maxterm_example}
  (A+B'+C)(A'+B'+C) &= Q
\end{align}

The terms $ A+B'+C $ and $ A'+B'+C $ are called $ maxterms $, and they contain the complement of the input variables for each of the \emph{False} output lines. Maxterms are most often used to describe circuits that have fewer \emph{False} outputs than \emph{True}. In Truth Table \ref{05:tab:truth_table_for_first_maxterm_example}, there are only two \emph{False} outputs with six \emph{True} outputs, so maxterms describe the circuit most efficiently.

Maxterms are frequently abbreviated with an upper-case  $ M $ along with a subscript that indicates the decimal value of the complements of the variables. For example, the complement of $ A+B'+C $, the first of the \emph{False} outputs in Truth Table \ref{05:tab:truth_table_for_first_maxterm_example}, is $ 010 $, which is a decimal value of 2; thus, the maxterm would be $ M_2 $. This can be confusing, but remember that the \emph{complements} of the inputs are used to form the expression. Thus, $ A+B'+C $ is $ 010 $, not $ 101 $. The other maxterm in this equation is $ M_6 $ since the binary value of its complement is $ 110 $, which equals decimal 6. It is possible to describe the entire circuit as a the product of two groups of maxterms: $ M_2 $ and $ M_6 $.

As another example, consider Truth Table \ref{05:tab:truth_table_for_second_maxterm_example}.

\begin{table}[H]
  \sffamily
  \newcommand{\head}[1]{\textcolor{white}{\textbf{#1}}}    
  \begin{center}
    \rowcolors{2}{gray!10}{white} % Color every other line a light gray
    \begin{tabular}{cccc|cc} 
      \rowcolor{black!75}
      \multicolumn{4}{c}{\head{Inputs}} & \multicolumn{2}{c}{\head{Outputs}} \\
      A & B & C & D & Q & M \\
      \hline
      0 & 0 & 0 & 0 & 1 & 0 \\
      0 & 0 & 0 & 1 & 1 & 1 \\
      0 & 0 & 1 & 0 & 1 & 2 \\
      0 & 0 & 1 & 1 & 0 & 3 \\
      0 & 1 & 0 & 0 & 1 & 4 \\
      0 & 1 & 0 & 1 & 1 & 5 \\
      0 & 1 & 1 & 0 & 1 & 6 \\
      0 & 1 & 1 & 1 & 1 & 7 \\
      1 & 0 & 0 & 0 & 1 & 8 \\
      1 & 0 & 0 & 1 & 0 & 9 \\
      1 & 0 & 1 & 0 & 1 & 10 \\
      1 & 0 & 1 & 1 & 1 & 11 \\
      1 & 1 & 0 & 0 & 0 & 12 \\
      1 & 1 & 0 & 1 & 1 & 13 \\
      1 & 1 & 1 & 0 & 1 & 14 \\
      1 & 1 & 1 & 1 & 1 & 15 
    \end{tabular}
  \end{center}
  \caption{Truth Table for Second Maxterm Example}
  \label{05:tab:truth_table_for_second_maxterm_example}
\end{table}

Equation \ref{05:eq:second_maxterm_example} describes this circuit.

\begin{align}
  \label{05:eq:second_maxterm_example}
  (A+B+C'+D')(A'+B+C+D')(A'+B'+C+D) &= Q
\end{align}

These would be maxterms $ M_3 $, $ M_9 $, and $ M_{12} $. Equation \ref{05:eq:pi_notation_second_maxterm_example} shows a commonly used, more compact way to express this result.

\begin{align}
  \label{05:eq:pi_notation_second_maxterm_example}
  \int(A,B,C,D) &= \prod(3,9,12)
\end{align}

Equation \ref{05:eq:pi_notation_second_maxterm_example} would read: ``For the function of  $ A $,  $ B $,  $ C $,  $ D $, the output is \emph{False} for maxterms $ 3 $, $ 9 $, and $ 12 $ when they are combined with an \textsf{AND}  gate.'' This format is called \emph{Pi Notation}, and it is easy to derive the Boolean equation from it. Remember that $ M_3 $ is $ 0011 $, or $ A+B+C'+D' $ (the complement of the inputs), $ M_9 $ is $ 1001 $, or $ A'+B+C+D' $, and $ M_{12} $ is $ 1100 $, or $ A'+B'+C+D $. The original equation can be quickly created from the Pi Notation. 

A logic equation that is created using maxterms is often called the \emph{Product of Sums} (or \emph{POS}) since each term is composed of inputs \textsf{OR} ed together (``sums'') and the terms are then joined by \textsf{AND}  gates (``products'').

\subsection{Minterm and Maxterm Relationships}
\label{05:subsec:minterm_and_maxterm_relationships}

The minterms and maxterms of a circuit have three interesting relationships: equivalence, duality, and inverse. To define and understand these terms, consider Truth Table \ref{05:tab:minterm_and_maxterm_relationships_truth_table} for some unspecified ``black box'' circuit: 

\begin{table}[H]
  \sffamily
  \newcommand{\head}[1]{\textcolor{white}{\textbf{#1}}}    
  \begin{center}
    \rowcolors{2}{gray!10}{white} % Color every other line a light gray
    \begin{tabular}{ccc|cc|cc} 
      \rowcolor{black!75}
      \multicolumn{3}{c}{\head{Inputs}} & \multicolumn{2}{c}{\head{Outputs}} & \multicolumn{2}{c}{\head{Terms}} \\
      A & B & C & Q & Q' & minterm & Maxterm \\
      \hline
      0 & 0 & 0 & 0 & 1 & $ A'B'C' $ ($ m_0 $) & $ A+B+C $    ($ M_0 $) \\
      0 & 0 & 1 & 1 & 0 & $ A'B'C $  ($ m_1 $) & $ A+B+C' $   ($ M_1 $) \\
      0 & 1 & 0 & 0 & 1 & $ A'BC' $  ($ m_2 $) & $ A+B'+C $   ($ M_2 $) \\
      0 & 1 & 1 & 0 & 1 & $ A'BC $   ($ m_3 $) & $ A+B'+C' $  ($ M_3 $) \\
      1 & 0 & 0 & 0 & 1 & $ AB'C' $  ($ m_4 $) & $ A'+B+C $   ($ M_4 $) \\
      1 & 0 & 1 & 1 & 0 & $ AB'C $   ($ m_5 $) & $ A'+B+C' $  ($ M_5 $) \\
      1 & 1 & 0 & 0 & 1 & $ ABC' $   ($ m_6 $) & $ A'+B'+C $  ($ M_6 $) \\
      1 & 1 & 1 & 1 & 0 & $ ABC $    ($ m_7 $) & $ A'+B'+C' $ ($ M_7 $) 
    \end{tabular}
  \end{center}
  \caption{Minterm and Maxterm Relationships}
  \label{05:tab:minterm_and_maxterm_relationships_truth_table}
\end{table}

\subsubsection{Equivalence}
\label{05:subsubsec:equivalence}

The minterms and the maxterms for a given circuit are considered equivalent ways to describe that circuit. For example, the circuit described by Truth Table \ref{05:tab:minterm_and_maxterm_relationships_truth_table} could be defined using minterms (Equation \ref{05:eq:sigma_notation_minterm_and_maxterm_equivalence}).

\begin{align}
  \label{05:eq:sigma_notation_minterm_and_maxterm_equivalence}
  \int(A,B,C) &= \sum(1,5,7)
\end{align}

However, that same circuit could also be defined using maxterms (Equation \ref{05:eq:pi_notation_minterm_and_maxterm_equivalence}).

\begin{align}
  \label{05:eq:pi_notation_minterm_and_maxterm_equivalence}
  \int(A,B,C) &= \prod(0,2,3,4,6)
\end{align}

These two functions describe the same circuit and are, consequently, equivalent. The \emph{Sigma Function} includes the terms $ 1 $, $ 5 $, and $ 7 $ while the \emph{Pi Function} includes all other terms in the truth table ($ 0 $, $ 2 $, $ 3 $, $ 4 $, and $ 6 $). To put it a slightly different way, the \emph{Sigma Function} describes the truth table rows where Q = $ 1 $ (minterms) while the \emph{Pi Function} describes the rows in the same truth table where Q =  $ 0 $ (maxterms). Therefore, Equation \ref{05:eq:minterm_and_maxterm_equivalence} can be derived.

\begin{align}
  \label{05:eq:minterm_and_maxterm_equivalence}
  \sum(1,5,7) &\equiv \prod(0,2,3,4,6)
\end{align}

\subsubsection{Duality}
\label{05:subsubsec:duality}

Each row in Truth Table \ref{05:tab:minterm_and_maxterm_relationships_truth_table} describes two terms that are considered duals. For example, minterm $ m_5 $ ($ AB'C $) and maxterm $ M_5 $ ($ A'+B+C' $) are duals. Terms that are duals are complements of each other ($ Q $ vs. $ Q' $) and the input variables are also complements of each other; moreover, the inputs for the three minterms are combined with an \textsf{AND}  while the maxterms are combined with an \textsf{OR}. The output of the circuit described by Truth Table \ref{05:tab:minterm_and_maxterm_relationships_truth_table} could be defined using minterms (Equation \ref{05:eq:q_output_dual}).

\begin{align}
  \label{05:eq:q_output_dual}
  Q &= \sum(1,5,7)
\end{align}

The dual of the circuit would be defined by using the maxterms for the same output rows. However, those rows are the \emph{complement} of the circuit (Equation \ref{05:eq:q_not_output_dual}).

\begin{align}
  \label{05:eq:q_not_output_dual}
  Q' &= \prod(1,5,7)
\end{align}

This leads to the conclusion that the complement of a \textit{Sigma Function} is the \textit{Pi Function} with the same inputs, as in Equation \ref{05:eq:dual} (the overline was used to emphasize the the fact that the \textit{PI Function} is complemented).

\begin{align}
  \label{05:eq:dual}
  \sum(1,5,7) &= \overline{\prod(1,5,7)}
\end{align}


\subsubsection{Inverse}
\label{05:subsubsec:inverse}

The complement of a function yields the opposite output. For example the following functions are inverses because one defines $ Q $ while the other defines $ Q' $ using only minterms of the same circuit (or truth table).

\begin{align}
  \label{05:eq:q_output_inverse}
  Q &= \sum(1,5,7)
\end{align}

\begin{align}
  \label{05:eq:q_not_output_inverse}
  Q' &= \sum(0,2,3,4,6)
\end{align}

\subsubsection{Summary}
\label{05:subsubsec:summary_minterm_and_maxterm_relationships}

These three relationships are summarized in the following table. Imagine a circuit with two or more inputs and an output of  $ Q $. Table \ref{05:tab:minterm_and_maxterm_relationships} summarizes the various relationships in the Truth Table for that circuit.

\begin{table}[H]
  \sffamily
  \begin{center}
    \begin{tabular}{|l|l|} 
      \hline
      Minterms where Q is 1 & Minterms where Q' is 1 \\
      \hline
      Maxterms where Q is 1 & Maxterms where Q' is 1 \\
      \hline
    \end{tabular}
  \end{center}
  \caption{Minterm-Maxterm Relationships}
  \label{05:tab:minterm_and_maxterm_relationships}
\end{table}

The adjacent items in a single column are equivalent (that is,  $ Q $ Minterms are equivalent to  $ Q $ Maxterms), items that are diagonal are duals ( $ Q $ Minterms and  $ Q' $ Maxterms are duals), and items that are adjacent in a single row are inverses ( $ Q $ Minterms and  $ Q' $ Minterms are inverses). 

\subsection{Sum of Products Example}
\label{05:subsubsec:sum_of_products_example}

\subsubsection{Given}
\label{05:subsubsec:given_sop_example}

A ``vote-counter'' machine is designed to turn on a light if any two or more of three inputs are \emph{True}. Create a circuit to realize this machine. 

\subsubsection{Truth Table}
\label{05:subsubsec:truth_table_sop_example}

When realizing a circuit from a verbal description, the best place to start is constructing a truth table. This will make the Boolean expression easy to write and then make the circuit easy to realize. For the ``vote-counter'' problem, start by defining variables for the truth table: Inputs  $ A $,  $ B $,  $ C $ and Output  $ Q $. 

Next, construct the truth table by identifying columns for each of the three input variables and then indicate the output for every possible input condition (Table \ref{05:tab:truth_table_for_sop_example}).

\begin{table}[H]
  \sffamily
  \newcommand{\head}[1]{\textcolor{white}{\textbf{#1}}}    
  \begin{center}
    \rowcolors{2}{gray!10}{white} % Color every other line a light gray
    \begin{tabular}{ccc|cc} 
      \rowcolor{black!75}
      \multicolumn{3}{c}{\head{Inputs}} & \multicolumn{2}{c}{\head{Outputs}} \\
      A & B & C & Q & m\\
      \hline
      0 & 0 & 0 & 0 & 0 \\
      0 & 0 & 1 & 0 & 1 \\
      0 & 1 & 0 & 0 & 2 \\
      0 & 1 & 1 & 1 & 3 \\
      1 & 0 & 0 & 0 & 4 \\
      1 & 0 & 1 & 1 & 5 \\
      1 & 1 & 0 & 1 & 6 \\
      1 & 1 & 1 & 1 & 7 
    \end{tabular}
  \end{center}
  \caption{Truth Table for SOP Example}
  \label{05:tab:truth_table_for_sop_example}
\end{table}

\subsubsection{Boolean Equation}
\label{05:subsubsec:boolean_equation_sop_example}

In a truth table, if there are fewer \emph{True} outputs than \emph{False}, then it is easiest to construct a Sum-of-Products equation. In that case, a Boolean expression can be derived by creating the minterms for all \emph{True} outputs and combining those minterms with \textsf{OR}  gates. Equation \ref{05:eq:sigma_notation_sop_example} is the \emph{Sigma Function} of this circuit.

\begin{align}
  \label{05:eq:sigma_notation_sop_example}
  \int(A,B,C) &= \sum(3,5,6,7)
\end{align}

% Pull Quote - Marginal Note - Sidebar
\marginpar{It may be possible to simplify the Boolean expression, but that process is covered elsewhere in this book.}

At this point, a circuit could be created with four 3-input \textsf{AND}  gates combined into one 4-input \textsf{OR}  gate.

\subsection{Product of Sums Example}
\label{05:subsubsec:product_of_sums_example}

\subsubsection{Given}
\label{05:subsubsec:given_pos_example}

A local supply company is designing a machine to sort packages for shipping. All packages go to the post office \emph{except} packages going to the local ZIP code containing chemicals (they are shipped by courier) and packages going to a distant ZIP code containing only perishables (they are shipped via air freight).

\subsubsection{Truth Table}
\label{05:subsubsec:truth_table_pos_example}

When realizing a circuit from a verbal description, the best place to start is constructing a truth table. This will make the Boolean expression easy to write and then make the circuit easy to realize. For the sorting machine problem, start by defining variables for the truth table: 

\begin{itemize}
  \item Ship via post office (the Output):  $ O $ is \emph{True} if ship by Post Office
  \item Zip Code:  $ Z $ is \emph{True} if the zip code is local
  \item Chemicals:  $ C $ is \emph{True} if the package contains chemicals
  \item Perishable:  $ P $ is \emph{True} if the package contains perishables
\end{itemize}

Next, construct the truth table by identifying columns for each of the three input variables and then indicate the output for every possible input condition (Table \ref{05:tab:truth_table_for_sop_example}).

\begin{table}[H]
  \sffamily
  \newcommand{\head}[1]{\textcolor{white}{\textbf{#1}}}    
  \begin{center}
    \rowcolors{2}{gray!10}{white} % Color every other line a light gray
    \begin{tabular}{ccc|cc} 
      \rowcolor{black!75}
      \multicolumn{3}{c}{\head{Inputs}} & \multicolumn{2}{c}{\head{Outputs}} \\
      Z & C & P & O & M \\
      \hline
      0 & 0 & 0 & 1 & 0 \\
      0 & 0 & 1 & 0 & 1 \\
      0 & 1 & 0 & 1 & 2 \\
      0 & 1 & 1 & 0 & 3 \\
      1 & 0 & 0 & 1 & 4 \\
      1 & 0 & 1 & 1 & 5 \\
      1 & 1 & 0 & 0 & 6 \\
      1 & 1 & 1 & 0 & 7 
    \end{tabular}
  \end{center}
  \caption{Truth Table for POS Example}
  \label{05:tab:truth_table_for_pos_example}
\end{table}

\subsubsection{Boolean Equation}
\label{05:subsubsec:boolean_equation_pos_example}

In the truth table, if there are fewer \emph{False} outputs than \emph{True} then it is easiest to construct a Products-of-Sums equation. In that case, a Boolean expression can be derived by creating the maxterms for all \emph{False} outputs and combining the complement those maxterms with \textsf{AND}  gates. Equation \ref{05:eq:pi_notation_pos_example} is the \emph{Pi Expression} of this circuit.

\begin{align}
  \label{05:eq:pi_notation_pos_example}
  \int(Z,C,P) &= \prod(1,3,6,7)
\end{align}

% Pull Quote - Marginal Note - Sidebar
\marginpar{It may be possible to simplify the Boolean expression, but that process is covered elsewhere in this book.}

At this point, a circuit could be created with four 3-input \textsf{OR}  gates combined into one 4-input \textsf{AND}  gate.

\subsection{Summary}
\label{05:subsubsec:summary_minterms_and_maxterms}

\gls{sop} Boolean expressions may be generated from truth tables quite easily, by determining which rows of the table have an output of \emph{True}, writing one minterm for each of those rows, and then summing all of the minterms. The resulting expression will lend itself well to implementation as a set of \textsf{AND} gates (products) feeding into a single \textsf{OR} gate (sum). 

\gls{pos} Boolean expressions may be generated from truth tables quite easily, by determining which rows of the table have an output of \emph{False}, writing one maxterm for each of those rows, and then multiplying all of the maxterms. The resulting expression will lend itself well to implementation as a set of \textsf{OR} gates (sums) feeding into a single \textsf{AND} gate (product). 

%***************************************************************************
% Section: Canonical Form
%***************************************************************************
\section{Canonical Form}
\label{05:sec:canonical_form}

\subsection{Introduction}
\label{05:subsec:introduction_to_canonical_form}

The word ``canonical'' simply means ``standard'' and it is used throughout mathematics and science to denote some standard form for equations. In digital electronics, Boolean equations are considered to be in canonical form when each of the terms in the equation includes all of the possible inputs and those terms appear in the same order as in the truth table. Using the canonical form is important when simplifying a Boolean equation. For example, imagine the solution to a given problem generated table \ref{05:tab:canonical_example_truth_table}.

\begin{table}[H]
  \sffamily
  \newcommand{\head}[1]{\textcolor{white}{\textbf{#1}}}    
  \begin{center}
    \rowcolors{2}{gray!10}{white} % Color every other line a light gray
    \begin{tabular}{ccc|cc} 
      \rowcolor{black!75}
      \multicolumn{3}{c}{\head{Inputs}} & \multicolumn{2}{c}{\head{Outputs}} \\
      A & B & C & Q & m \\
      \hline
      0 & 0 & 0 & 0 & 0 \\
      0 & 0 & 1 & 1 & 1 \\
      0 & 1 & 0 & 0 & 2 \\
      0 & 1 & 1 & 1 & 3 \\
      1 & 0 & 0 & 0 & 4 \\
      1 & 0 & 1 & 0 & 5 \\
      1 & 1 & 0 & 0 & 6 \\
      1 & 1 & 1 & 1 & 7 
    \end{tabular}
  \end{center}
  \caption{Canonical Example Truth Table}
  \label{05:tab:canonical_example_truth_table}
\end{table}

Minterm equation \ref{05:eq:canonical_example} is derived from the truth table and is presented in canonical form. Notice that each term includes all possible inputs ( $ A $,  $ B $, and  $ C $), and that the terms are in the same order as they appear in the truth table. 

\begin{align}
  \label{05:eq:canonical_example}
  (A'B'C)+(A'BC)+(ABC) &= Q
\end{align}

Frequently, though, a Boolean equation is expressed in standard form, which is not the same as canonical form. Standard form means that some of the terms have been simplified and not all of the inputs will appear in all of the terms. For example, consider Equation \ref{05:eq:standard_form_example}, which is the solution for a 4-input circuit.

\begin{align}
  \label{05:eq:standard_form_example}
  (A'C)+(B'CD) &= Q
\end{align}

This equation is in standard form so the first term, $ A'C $, does not include inputs  $ B $ or  $ D $ and the second term, $ B'CD $, does not include input  $ A $. However, all inputs must be present in every term for an equation to be in canonical form.

Building a truth table for an equation in standard form raises an important question. Consider the truth table for Equation \ref{05:eq:standard_form_example}.

\begin{table}[H]
  \sffamily
  \newcommand{\head}[1]{\textcolor{white}{\textbf{#1}}}    
  \begin{center}
    \rowcolors{2}{gray!10}{white} % Color every other line a light gray
    \begin{tabular}{cccc|cc} 
      \rowcolor{black!75}
      \multicolumn{4}{c}{\head{Inputs}} & \multicolumn{2}{c}{\head{Outputs}} \\
      A & B & C & D & Q & m \\
      \hline
      0 & 0 & 0 & 0 & 0 & 0 \\
      0 & 0 & 0 & 1 & 0 & 1 \\
      0 & 0 & 1 & 0 & 0 & 2 \\
      0 & 0 & 1 & 1 & ? & 3 \\
      0 & 1 & 0 & 0 & 0 & 4 \\
      0 & 1 & 0 & 1 & 0 & 5 \\
      0 & 1 & 1 & 0 & 0 & 6 \\
      0 & 1 & 1 & 1 & 0 & 7 \\
      1 & 0 & 0 & 0 & 0 & 8 \\
      1 & 0 & 0 & 1 & 0 & 9 \\
      1 & 0 & 1 & 0 & 0 & 10 \\
      1 & 0 & 1 & 1 & ? & 11 \\
      1 & 1 & 0 & 0 & 0 & 12 \\
      1 & 1 & 0 & 1 & 0 & 13 \\
      1 & 1 & 1 & 0 & 0 & 14 \\
      1 & 1 & 1 & 1 & 0 & 15 
    \end{tabular}
  \end{center}
  \caption{Truth Table for Standard Form Equation}
  \label{05:tab:truth_table_for_standard_form_equation}
\end{table}

In what row would $ B'CD $, the second term in Equation \ref{05:eq:standard_form_example}, be placed? $ B'CD $ is $ 011 $ (in binary), but since the $ A $ term is missing would it be a $ 0 $ or $ 1 $; in other words, would $ B'CD $ generate an output of $ 1 $ for row $ 0011\;(m_3)$ or $ 1011\;(m_{11})$? (The output for these two rows are marked with a question mark in Table \ref{05:tab:truth_table_for_standard_form_equation}.) In fact, the output for \emph{both} of these rows must be considered \emph{True} in order to ensure that all possible combinations of input are covered. Thus, the final equation for this circuit must include at least these two terms: $ (A'B'CD) + (AB'CD) $. In the same way, the term $ A'C $ means that the output is \emph{True} for $ m_2 $, $ m_3 $, $ m_6 $, and $ m_7 $ since input $ A'C $ is \emph{True} and any minterm that contains those two value is also considered \emph{True}. Thus, the final equation for this circuit must include at least these four terms: $ (A'B'CD') + (A'B'CD) + (A'BCD') + (A'BCD) $.

\subsection{Converting Terms Missing One Variable}
\label{05:subsec:converting_terms_missing_one_variable}

To change a standard Boolean expression that is missing one input term into a canonical Boolean expression, insert both \emph{True} and \emph{False} for the missing term into the original standard expression. As an example, consider the term $ B'CD $. Since term $ A $ is missing, both $ A $ and $ A' $ must be included in the converted canonical expression. Equation \ref{05:eq:canonical_expanding_3-variable_term} proves that $ B'CD $ can be expanded to include both possible values for $ A $ by using the Adjacency Property (page \pageref{BF:subsec:adjacency_property}).

\begin{align}
  \label{05:eq:canonical_expanding_3-variable_term}
  (B'CD) \rightarrow (AB'CD)+(A'B'CD)
\end{align}

A term that is missing one input variable will expand into two terms that include all variables. For example, in a system with four input variables (as above), any standard term with only three variables will expand to a canonical expression containing two groups of four variables. 

Expanding a standard term that is missing one variable can also be done with a truth table. To do that, fill in an output of $ 1 $ for every line where the \emph{True} inputs are found while ignoring all missing variables. As an example, consider Truth Table \ref{05:tab:truth_table_for_standard_form_equation} where the outputs for $ m_3 $ and $ m_11 $ are marked with a question mark. However, the output for both of these lines should be marked as \emph{True} because  $ B'CD $ is \emph{True} (input $ A $ is ignored). Then, those two minterms lead to the Boolean expression $ AB'CD + A'B'CD $.

\subsection{Converting Terms Missing Two Variables}
\label{05:subsec:converting_terms_missing_two_variables}

It is easiest to expand a standard expression that is missing two terms by first inserting one of the missing variables and then inserting the other missing variable in two distinct steps. The process for inserting a single missing variable is found in Section \ref{05:subsec:converting_terms_missing_one_variable}. Consider the term $ A'C $ in a four-variable system. It is missing both the $ B $ and $ D $ variables. To expand that term to its canonical form, start by inserting either of the two missing variables. For example, Equation \ref{05:eq:canonical_expanding_2-variable_term_step_one} illustrates entering $ B $ and $ B' $ into the expression.

\begin{align}
  \label{05:eq:canonical_expanding_2-variable_term_step_one}
  (A'C) &\rightarrow (A'BC)+(A'B'C)
\end{align}

Then, Equation \ref{05:eq:canonical_expanding_2-variable_term_step_two} illustrates inserting $ D $ and $ D' $ into the expression.

\begin{align}
  \label{05:eq:canonical_expanding_2-variable_term_step_two}
  (A'BC) &\rightarrow (A'BCD)+(A'BCD') \\
  \nonumber
  (A'B'C) &\rightarrow (A'B'CD)+(A'B'CD')
\end{align}

In the end, $ A'C $ expands to Equation \ref{05:eq:canonical_expanding_2-variable_term_step_three}:

\begin{align}
  \label{05:eq:canonical_expanding_2-variable_term_step_three}
  (A'C) \rightarrow &(A'BCD) + (A'BCD') \\
  \nonumber
  &+ (A'B'CD) + (A'B'CD')
\end{align}

Thus, in a four-variable system, any standard term with only two variables will expand to a canonical expression with four groups of four variables.

Expanding a standard term that is missing two variables can also be done with a truth table. To do that, fill in an output of $ 1 $ for every line where the \emph{True} inputs are found while ignoring all missing variables. As an example, consider a Table \ref{05:tab:truth_table_for_expression_with_missing_variables}, where $ A'C $ is marked as \emph{True}:

\begin{table}[H]
  \sffamily
  \newcommand{\head}[1]{\textcolor{white}{\textbf{#1}}}    
  \begin{center}
    \rowcolors{2}{gray!10}{white} % Color every other line a light gray
    \begin{tabular}{cccc|cc} 
      \rowcolor{black!75}
      \multicolumn{4}{c}{\head{Inputs}} & \multicolumn{2}{c}{\head{Outputs}} \\
      A & B & C & D & Q & m \\
      \hline
      0 & 0 & 0 & 0 & 0 & 0 \\
      0 & 0 & 0 & 1 & 0 & 1 \\
      0 & 0 & 1 & 0 & 1 & 2 \\
      0 & 0 & 1 & 1 & 1 & 3 \\
      0 & 1 & 0 & 0 & 0 & 4 \\
      0 & 1 & 0 & 1 & 0 & 5 \\
      0 & 1 & 1 & 0 & 1 & 6 \\
      0 & 1 & 1 & 1 & 1 & 7 \\
      1 & 0 & 0 & 0 & 0 & 8 \\
      1 & 0 & 0 & 1 & 0 & 9 \\
      1 & 0 & 1 & 0 & 0 & 10 \\
      1 & 0 & 1 & 1 & 0 & 11 \\
      1 & 1 & 0 & 0 & 0 & 12 \\
      1 & 1 & 0 & 1 & 0 & 13 \\
      1 & 1 & 1 & 0 & 0 & 14 \\
      1 & 1 & 1 & 1 & 0 & 15 
    \end{tabular}
  \end{center}
  \caption{Truth Table for Standard Form Equation}
  \label{05:tab:truth_table_for_expression_with_missing_variables}
\end{table}

Notice that outputs for $ m_2 $, $ m_3 $, $ m_6 $, and $ m_7 $ are \emph{True} because for each of those minterms $ A'C $ is \emph{True} (inputs $ B $ and $ D $ are ignored). Then, those four minterms lead to the Boolean expression $ A'B'CD'+A'B'CD+A'BCD'+A'BCD $.

\subsection{Summary}
\label{05:subsec:summary_of_canonical_forms}

This discussion started with Equation \ref{05:eq:standard_form_example_repeated}, which is in standard form.

\begin{align}
  \label{05:eq:standard_form_example_repeated}
  (A'C)+(B'CD) &= Q
\end{align}

After expanding both terms, Equation \ref{05:eq:canonical_form_summary_equation_not_simplified} is generated.

\begin{align}
  \label{05:eq:canonical_form_summary_equation_not_simplified}
  &(A'BCD)+(A'BCD')+(A'B'CD) \\
  \nonumber
  +&(A'B'CD')+(AB'CD)+(A'B'CD) = Q
\end{align}

Notice, though, that the term $ A'B'CD $ appears two times, so one of those can be eliminated by the Idempotence property (page \pageref{BF:subsec:idempotence}), leaving Equation \ref{05:eq:canonical_form_summary_equation_simplified}.

\begin{align}
  \label{05:eq:canonical_form_summary_equation_simplified}
  &(A'BCD)+(A'BCD') \\
  \nonumber
  +&(A'B'CD')+(AB'CD)+(A'B'CD) = Q
\end{align}

To put the equation in canonical form, which is important for simplification; all that remains is to rearrange the terms so they are in the same order as they would appear in a truth table, which results in Equation \ref{05:eq:canonical_form_summary_equation_final}.

\begin{align}
  \label{05:eq:canonical_form_summary_equation_final}
  &(A'B'CD')+(A'B'CD) \\
  \nonumber
  +&(A'BCD')+(A'BCD)+(AB'CD) = Q
\end{align}

\subsection{Practice Problems}
\label{05:subsec:practice_with_canonical_forms}

\marginpar{Parenthesis were not used in order to save space; however, the variable groups are evident.}

\begin{table}[H]
  \sffamily
  \begin{center}
    \begin{tabular}{c c p{6cm} }
      \multirow{2}{*}{\textbf{1}} 
        & Standard (A,B,C) & $ A'B+C+AB' $ \\
        & \cellcolor{gray!10} Cannonical 
        & \cellcolor{gray!10} $ A'B'C+A'BC'+A'BC+AB'C'+AB'C+ABC $ \\
      \hline
      \multirow{2}{*}{\textbf{2}} 
      & Standard (A,B,C,D) & $ A'BC+B'D $ \\
      & \cellcolor{gray!10} Cannonical 
      & \cellcolor{gray!10} $ A'B'C'D'+A'B'CD'+A'BCD'+A'BCD+AB'C'D'+AB'CD' $ \\
      \hline
      \multirow{2}{*}{\textbf{3}} 
      & Standard (A,B,C,D) & $ A'+D $ \\
      & \cellcolor{gray!10} Cannonical 
      & \cellcolor{gray!10} $ A'B'C'D'+A'B'C'D+A'B'CD'+A'B'CD+A'BC'D'       +A'BC'D+A'BCD'+A'BCD+AB'C'D+AB'CD+ABC'D+ABCD $ \\
      \hline
      \multirow{2}{*}{\textbf{4}} 
      & Standard (A,B,C) & $ A(B'+C) $ \\
      & \cellcolor{gray!10} Cannonical 
      & \cellcolor{gray!10} $ AB'C'+AB'C+ABC $ \\
    \end{tabular}
  \end{center}
  \caption{Canonical Form Practice Problems}
  \label{05:tab:canonical_form_practice_problems}
\end{table}

%***************************************************************************
% Section: Simplification Using Algebraic Methods
%***************************************************************************
\section{Simplification Using Algebraic Methods}
\label{05:sec:simplification_using_algebraic_methods}

\subsection{Introduction}
\label{05:subsec:introduction_to_algebraic_methods}

One method of simplifying a Boolean equation is to use common algebraic processes. It is possible to reduce an equation step-by-step using the various properties of Boolean algebra in the same way that real-number equations can be simplified. 

\subsection{Starting From a Circuit}
\label{05:subsec:starting_from_a_circuit}

Occasionally, the circuit designer is faced with an existing circuit and must attempt to simplify it. In that case, the first step is to find the Boolean equation for the circuit and then simplify that equation. 

\subsubsection{Generate a Boolean Equation}
\label{05:subsubsec:generate_a_boolean_equation}

In the circuit illustrated in Figure \ref{fig:05_04}, the  $ A $,  $ B $, and  $ C $ input signals are assumed to be provided from switches, sensors, or perhaps other sub-circuits. Where these signals originate is of no concern in the task of gate reduction.

\begin{figure}[H]
	\centering
	\includegraphics[width=\maxwidth{.95\linewidth}]{gfx/05_04}
	\caption{Example Circuit}
	\label{fig:05_04}
\end{figure}

To generate the Boolean equation for a circuit, write the output of each gate as determined by the input signals and type of gate, working from the inputs to the final output. Figure \ref{fig:05_05} illustrates the result of this process.

\begin{figure}[H]
	\centering
	\includegraphics[width=\maxwidth{.95\linewidth}]{gfx/05_05}
	\caption{Example Circuit With Gate Outputs}
	\label{fig:05_05}
\end{figure}

This process leads to Equation \ref{05:eq:equation_from_circuit}.

\begin{align}
  \label{05:eq:equation_from_circuit}
  AB+BC(B+C) &= Y
\end{align}

\subsection{Starting From a Boolean Equation}
\label{05:subsec:starting_from_a_boolean_equation}

If a logic circuit's function is expressed as a Boolean equation, then algebraic methods can be applied to reduce the number of logic gates, resulting in a circuit that performs the same function with fewer components. As an example, Equation \ref{05:soln:equation_to_simplify_one_solution} simplifies the circuit found in Figure \ref{05:fig:circuit_with_boolean_expression}. 

\begin{align}
\label{05:soln:equation_to_simplify_one_solution}
  AB+BC(B+C) && \text{Original Expression} \\
  \nonumber
  AB+BBC+BCC && \text{Distribute BC} \\
  \nonumber
  AB+BC+BC && \text{Idempotence: BB=B and CC=C} \\
  \nonumber
  AB+BC && \text{Idempotence: BC+BC=BC} \\
  \nonumber
  B(A+C) && \text{Factor} 
\end{align}

The final expression, $ B(A + C) $, requires only two gates, and is much simpler than the original, yet performs the same function. Such component reduction results in higher operating speed (less gate propagation delay), less power consumption, less cost to manufacture, and greater reliability. 

As a second example, consider Equation \ref{05:eq:equation_to_simplify_two}.

\begin{align}
  \label{05:eq:equation_to_simplify_two}
  A+AB &= Y
\end{align}

This is simplified below.

\begin{align}
\label{05:soln:equation_to_simplify_two}
  A+AB && \text{Original Expression} \\
  \nonumber
  A(1+B) && \text{Factor} \\
  \nonumber
  A(1) && \text{Annihilation (1+B=1)} \\
  \nonumber
  A && \text{Identity A1=A}
\end{align}

The original expression, $ A+AB $ has been reduced to  $ A $ so the original circuit could be replaced by a wire directly from input  $ A $ to output  $ Y $. Equation \ref{05:eq:equation_to_simplify_three} looks similar to Equation \ref{05:eq:equation_to_simplify_two}, but is quite different and requires a more clever simplification.

\begin{align}
  \label{05:eq:equation_to_simplify_three}
  A+A'B &= Y
\end{align}

This is simplified below.

\begin{align}
  \label{05:soln:equation_to_simplify_three}
  A+A'B && \text{Original Expression} \\
  \nonumber
  A+AB+A'B && \text{Expand A to A+AB (Absorption)} \\
  \nonumber
  A+B(A+A') && \text{Factor B out of the last two terms} \\
  \nonumber
  A+B(1) && \text{Complement Property} \\
  \nonumber
  A+B && \text{Identityt: B(1)=B} 
\end{align}

Note how the Absorption Property ($ A + AB = A $) is used to ``un-simplify'' the first  $ A $ term, changing  $ A $ into $ A + AB $. While this may seem like a backward step, it ultimately helped to reduce the expression to something simpler. Sometimes ``backward'' steps must be taken to achieve the most elegant solution. Knowing when to take such a step is part of the art of algebra. 

As another example, simplify this \gls{pos} expression equation: 

\begin{align}
  \label{05:eq:equation_to_simplify_four}
  (A+B)(A+C) &= Y
\end{align}

This is simplified below.

\begin{align}
  \label{05:soln:equation_to_simplify_four}
  (A+B)(A+C) && \text{Original Expression} \\
  \nonumber
  AA+AC+AB+BC && \text{Distribute A+B} \\
  \nonumber
  A+AC+AB+BC && \text{Idempotence: AA=A} \\
  \nonumber
  A+AB+BC && \text{Absorption: A+AC=A} \\
  \nonumber
  A+BC && \text{Absorption: A+AB=A}
\end{align}

In each of the examples in this section, a Boolean expression was simplified using algebraic methods, which led to a reduction in the number of gates needed and made the final circuit more economical to construct and reliable to operate. 

\subsection{Practice Problems}
\label{05:subsec:practice_problems_for_simplifying_with_algebra}

Table \ref{05:tab:simplifying_boolean_expressions} shows a Boolean expression on the left and its simplified version on the right. This is provided for practice in simplifying expressions using algebraic methods. 

\begin{table}[H]
  \sffamily
  \newcommand{\head}[1]{\textcolor{white}{\textbf{#1}}}    
  \begin{center}
    \rowcolors{2}{gray!10}{white} % Color every other line a light gray
    \begin{tabular}{cc}
      \rowcolor{black!75}
      \head{Original Expression} & \head{Simplified} \\
      $ A(A'+B) $ & $ AB $ \\
      $ A+A'B $ & $ A+B $ \\
      $ (A+B)(A+B') $ & $ A $ \\
      $ AB+A'C+BC $ & $ AB+A'C $ \\
    \end{tabular}
  \end{center}
  \caption{Simplifying Boolean Expressions}
  \label{05:tab:simplifying_boolean_expressions}
\end{table}
%***************************************************************************
% Chapter: Karnaugh maps
%***************************************************************************
\chapter{Karnaugh maps}\label{ch06}

\section{Introduction}

% Pull Quote - Marginal Note - Sidebar
\marginpar{Maurice Karnaugh developed this process at Bell Labs in 1953 while designing switching circuits for landline telephones.}

A Karnaugh map, like Boolean algebra, is a tool used to simplify a digital circuit. Keep in mind that ``simplify'' means reducing the number of gates and inputs for each gate and as components are eliminated, not only does the manufacturing cost go down, but the circuit becomes simpler, more stable, and more energy efficient. 

In general, using Boolean Algebra is the easiest way to simplify a circuit involving one to three input variables. For four input variables, Boolean algebra becomes tedious and Karnaugh maps are both faster and easier (and are less prone to error). However, Karnaugh maps become rather complex with five input variables and are generally too difficult to use above six variables (with more than four input variables, a Karnaugh map uses multiple dimensions that become very challenging to manipulate). For five or more input variables, circuit simplification should be done by Quine-McClusky methods  (page \pageref{ASM:sec:quine-mccluskey_simplification_method}) or the use of \gls{cat} (page \pageref{ASM:sec:automated_tools}).

In theory, any of the methods will work for any number of variables; however, as a practical matter, the guidelines presented in Table \ref{KM:tab:circuit_simplification_methods} work well. Normally, there is no need to resort to \gls{cat} to simplify a simple equation involving two or three variables since it is much quickly to use either Boolean Algebra or Karnaugh maps. However, for more complex input/output combinations, then \gls{cat} become essential to both speed the process and improve accuracy. 

\begin{table}[H]
  \sffamily
  \newcommand{\head}[1]{\textcolor{white}{\textbf{#1}}}    
  \begin{center}
    \rowcolors{2}{gray!10}{white} % Color every other line a light gray
    \begin{tabular}{ccccc} 
      \rowcolor{black!75}
      \head{Variables} & \head{Algebra} & \head{K-Map} & \head{Quine-McClusky} & \head{CAT} \\
      1-2 & X &   &   &   \\
      3   & X & X &   &   \\
      4   & X & X &   &   \\
      5-6 &   & X & X &   \\
      7-8 &   &   & X & X \\
      >8  &   &   &   & X
    \end{tabular}
  \end{center}
  \caption{Circuit Simplification Methods}
  \label{KM:tab:circuit_simplification_methods}
\end{table}

\section{Reading Karnaugh maps}
\label{KM:sec:reading_karnaugh_maps}

Following are four different ways to represent the same thing, a two-input digital logic function. 

\begin{figure}[H]
	\centering
	\includegraphics[width=\maxwidth{.95\linewidth}]{gfx/06_01}
	\caption{Simple Circuit For K-Map}
	\label{fig:06_01}
\end{figure}

\begin{align}
  \label{KM:eq:simple_circuit_for_karnaugh_map}
  AB+A'B &= Y
\end{align}

\begin{table}[H]
  \sffamily
  \newcommand{\head}[1]{\textcolor{white}{\textbf{#1}}}    
  \begin{center}
    \rowcolors{2}{gray!10}{white} % Color every other line a light gray
    \begin{tabular}{ccc} 
      \rowcolor{black!75}
      \multicolumn{2}{c}{\head{Inputs}} & \head{Output} \\
      A & B & Y \\
      \hline
      0 & 0 & 0 \\
      0 & 1 & 1 \\
      1 & 0 & 0 \\
      1 & 1 & 1 
    \end{tabular}
  \end{center}
  \caption{Truth Table for Simple Circuit}
  \label{KM:tab:truth_table_simple_circuit}
\end{table}

%****************************************************************
% Karnaugh map For Simple Circuit
%****************************************************************
\begin{figure}[H]
  \caption{Karnaugh map for Simple Circuit}
  \label{KM:tab:k-map_for_simple_circuit}
  \myfloatalign
  \begin{tikzpicture} [circuit logic US, scale=1.00]
  % make all path lines (the node shapes) a little thicker
  \tikzstyle{every path}=[line width=0.50mm]
  
  %********************************************************************
  % Adjust the settings below to display the 1's and rectangles
  %********************************************************************
  % Uncomment the appropriate lines below to insert ones where needed
    \node[] at (1.4,1.5) {\huge $ 0 $}; % 00
    \node[] at (1.4,0.5) {\huge $ 1 $}; % 01
    \node[] at (2.4,1.5) {\huge $ 0 $}; % 02
    \node[] at (2.4,0.5) {\huge $ 1 $}; % 03
  
  % The coords for each cell - this is used as the origin for the solution box
  \coordinate (cell00) at (1.0,3.0); \coordinate (cell01) at (1.0,2.0);
  \coordinate (cell02) at (1.0,0.0); \coordinate (cell03) at (1.0,1.0);
      
  %********************************************************************
  % Shouldn't need to adjust anything below this point - this is just
  % the grid and the minterms.
  %********************************************************************  
  % Text in top-Left cell
  \node[] at (0.35,2.22) { \footnotesize $ \mathsf{ B } $ }; % B
  \node[] at (0.70,2.75) { \footnotesize $ \mathsf{ A } $ }; % A
  
  % Populate the top row header
  % In the following, the foreach lists a location/text pair
  % The the draw line draws the text at each location
  \foreach \loc/\txt in {
    (1.5,2.5)/{0},(2.5,2.5)/{1}
  }
  \draw \loc node{\large $\txt$};
  
  % Populate the header in column one
  \foreach \loc/\txt in { 
    (0.5,1.5)/{0},(0.5,0.5)/{1}
  }
  \draw \loc node{\large $\txt$};
  
  % Populate the minterms
  \foreach \loc/\txt in { 
    (1.75,1.15)/{00} , (2.75,1.15)/{02} , (1.75,0.15)/{01} , (2.75,0.15)/{03} 
    }
  \draw \loc node{ \color{blue!90!black} \footnotesize { $\txt$ }};
  
  % Draw the lines
  \draw
  % Finish drawing the grid
  [step=1.0cm,black,thin] (0,0) grid (3.0,3.0) % The Grid
  (0.0,3.0) -- (1.0,2.0) % Diagonal in the top left cell
  (1.0,2.05) -- (3.0,2.05) % Double line under top header row
  (0.95,0.0) -- (0.95,2.0) % Double line on left of header column one
  ;    
  \end{tikzpicture}
\end{figure}

% Pull Quote - Marginal Note - Sidebar
\marginpar{Karnaugh maps frequently include the minterm numbers in each cell to aid in placing variables.}

First is a circuit diagram, followed by its Boolean equation, truth table, and, finally, Karnaugh map. Think of a Karnaugh map as simply a rearranged truth table; but simplifying a three or four input circuit using a Karnaugh map is much easier and more accurate than with either a truth table or Boolean equation. 

\section{Drawing Two-Variable Karnaugh maps}
\label{KM:sec:drawing_2_variable_karnaugh_maps}

Truth Table \ref{KM:tab:truth_table_with_greek_letters} and Karnaugh map \ref{KM:fig:k-map_with_greek_letters} illustrate the relationship between these two representations of the same circuit using Greek symbols. 

% Table
\begin{table}[H]
  \sffamily
  \newcommand{\head}[1]{\textcolor{white}{\textbf{#1}}}    
  \begin{center}
    \rowcolors{2}{gray!10}{white} % Color every other line a light gray
    \begin{tabular}{ccc} 
      \rowcolor{black!75}
      \multicolumn{2}{c}{\head{Inputs}} & \head{Output} \\
      A & B & Y \\
      \hline
      0 & 0 & $ \alpha $ \\
      0 & 1 & $ \beta $ \\
      1 & 0 & $ \gamma $ \\
      1 & 1 & $ \delta $ 
    \end{tabular}
  \end{center}
  \caption{Truth Table with Greek Letters}
  \label{KM:tab:truth_table_with_greek_letters}
\end{table}

%****************************************************************
% Karnaugh map For 2-Input Circuit With Greek Letters
%****************************************************************
\begin{figure}[H]
  \caption{Karnaugh map With Greek Letters}
  \label{KM:fig:k-map_with_greek_letters}
  \myfloatalign
  \begin{tikzpicture} [circuit logic US, scale=1.00]
  % make all path lines (the node shapes) a little thicker
  \tikzstyle{every path}=[line width=0.50mm]
  
  %********************************************************************
  % Adjust the settings below to display the 1's and rectangles
  %********************************************************************
  % Uncomment the appropriate lines below to insert ones where needed
  \node[] at (1.4,1.5) {\huge $ \alpha $}; % 00
  \node[] at (1.4,0.5) {\huge $ \beta $}; % 01
  \node[] at (2.4,1.5) {\huge $ \gamma $}; % 02
  \node[] at (2.4,0.5) {\huge $ \delta $}; % 03
  
  % The coords for each cell - this is used as the origin for the solution box
  \coordinate (cell00) at (1.0,1.0); \coordinate (cell01) at (1.0,0.0);
  \coordinate (cell02) at (2.0,1.0); \coordinate (cell03) at (2.0,0.0);
  
  %********************************************************************
  % Shouldn't need to adjust anything below this point - this is just
  % the grid and the minterms.
  %********************************************************************  
  % Text in top-Left cell
  \node[] at (0.35,2.22) { \footnotesize $ \mathsf{ B } $ }; % B
  \node[] at (0.70,2.75) { \footnotesize $ \mathsf{ A } $ }; % A
  
  % Populate the top row header
  % In the following, the foreach lists a location/text pair
  % The the draw line draws the text at each location
  \foreach \loc/\txt in {
    (1.5,2.5)/{0},(2.5,2.5)/{1}
  }
  \draw \loc node{\large $\txt$};
  
  % Populate the header in column one
  \foreach \loc/\txt in { 
    (0.5,1.5)/{0},(0.5,0.5)/{1}
  }
  \draw \loc node{\large $\txt$};
  
  % Populate the minterms
  \foreach \loc/\txt in { 
    (1.75,1.15)/{00} , (2.75,1.15)/{02} , (1.75,0.15)/{01} , (2.75,0.15)/{03} 
  }
  \draw \loc node{ \color{blue!90!black} \footnotesize { $\txt$ }};
  
  % Draw the lines
  \draw
  % Finish drawing the grid
  [step=1.0cm,black,thin] (0,0) grid (3.0,3.0) % The Grid
  (0.0,3.0) -- (1.0,2.0) % Diagonal in the top left cell
  (1.0,2.05) -- (3.0,2.05) % Double line under top header row
  (0.95,0.0) -- (0.95,2.0) % Double line on left of header column one
  ;    
  \end{tikzpicture}
\end{figure}

On the Karnaugh map, all of the possible values for input $ A $ are listed across the top of the map and values for input $ B $ are listed down the left side. Thus, to find the output for $ A=0 $; $ B=0 $, look for the cell where those two quantities intersect; which is output $ \alpha $ in the example. It should be clear how all four data squares on the Karnaugh map correspond to their equivalent rows (the minterms) in the Truth Table. 

Truth Table \ref{KM:tab:truth_table_for_2-input_circuit} and Karnaugh map \ref{KM:fig:k-map_for_2_input_circuit} illustrate another example of this relationship.

% Truth Table
\begin{table}[H]
  \sffamily
  \newcommand{\head}[1]{\textcolor{white}{\textbf{#1}}}    
  \begin{center}
    \rowcolors{2}{gray!10}{white} % Color every other line a light gray
    \begin{tabular}{ccc} 
      \rowcolor{black!75}
      \multicolumn{2}{c}{\head{Inputs}} & \head{Output} \\
      A & B & Y \\
      \hline
      0 & 0 & 1 \\
      0 & 1 & 1 \\
      1 & 0 & 0 \\
      1 & 1 & 1 
    \end{tabular}
  \end{center}
  \caption{Truth Table for Two-Input Circuit}
  \label{KM:tab:truth_table_for_2-input_circuit}
\end{table}

%****************************************************************
% Karnaugh map For 2-Input Circuit 
%****************************************************************
\begin{figure}[H]
  \caption{Karnaugh map For Two-Input Circuit}
  \label{KM:fig:k-map_for_2_input_circuit}
  \myfloatalign
  \begin{tikzpicture} [circuit logic US, scale=1.00]
  % make all path lines (the node shapes) a little thicker
  \tikzstyle{every path}=[line width=0.50mm]
  
  %********************************************************************
  % Adjust the settings below to display the 1's and rectangles
  %********************************************************************
  % Uncomment the appropriate lines below to insert ones where needed
  \node[] at (1.4,1.5) {\huge $ 1 $}; % 00
  \node[] at (1.4,0.5) {\huge $ 1 $}; % 01
%  \node[] at (2.4,1.5) {\huge $ 1 $}; % 02
  \node[] at (2.4,0.5) {\huge $ 1 $}; % 03
  
  % The coords for each cell - this is used as the origin for the solution box
  \coordinate (cell00) at (1.0,1.0); \coordinate (cell01) at (1.0,0.0);
  \coordinate (cell02) at (2.0,1.0); \coordinate (cell03) at (2.0,0.0);
  
  %********************************************************************
  % Shouldn't need to adjust anything below this point - this is just
  % the grid and the minterms.
  %********************************************************************  
  % Text in top-Left cell
  \node[] at (0.35,2.22) { \footnotesize $ \mathsf{ B } $ }; % B
  \node[] at (0.70,2.75) { \footnotesize $ \mathsf{ A } $ }; % A
  
  % Populate the top row header
  % In the following, the foreach lists a location/text pair
  % The the draw line draws the text at each location
  \foreach \loc/\txt in {
    (1.5,2.5)/{0},(2.5,2.5)/{1}
  }
  \draw \loc node{\large $\txt$};
  
  % Populate the header in column one
  \foreach \loc/\txt in { 
    (0.5,1.5)/{0},(0.5,0.5)/{1}
  }
  \draw \loc node{\large $\txt$};
  
  % Populate the minterms
  \foreach \loc/\txt in { 
    (1.75,1.15)/{00} , (2.75,1.15)/{02} , (1.75,0.15)/{01} , (2.75,0.15)/{03} 
  }
  \draw \loc node{ \color{blue!90!black} \footnotesize { $\txt$ }};
  
  % Draw the lines
  \draw
  % Finish drawing the grid
  [step=1.0cm,black,thin] (0,0) grid (3.0,3.0) % The Grid
  (0.0,3.0) -- (1.0,2.0) % Diagonal in the top left cell
  (1.0,2.05) -- (3.0,2.05) % Double line under top header row
  (0.95,0.0) -- (0.95,2.0) % Double line on left of header column one
  ;    
  \end{tikzpicture}
\end{figure}
\marginpar{Karnaugh maps usually do not include zeros to decrease the chance for error.}

\section{Drawing Three-Variable Karnaugh maps}
\label{KM:sec:drawing_3-variable_karnaugh_maps}

Consider Equation \ref{KM:eq:karnaugh_map_for_3_variables}.

\begin{align}
  \label{KM:eq:karnaugh_map_for_3_variables}
  ABC'+AB'C+A'B'C &= Y
\end{align}

Table \ref{KM:tab:truth_table_for_3-input_circuit} and Karnaugh map \ref{KM:fig:k-map_for_3-input_circuit} represent this equation.

% Truth Table
\begin{table}[H]
  \sffamily
  \newcommand{\head}[1]{\textcolor{white}{\textbf{#1}}}    
  \begin{center}
    \rowcolors{2}{gray!10}{white} % Color every other line a light gray
    \begin{tabular}{ccccc} 
      \rowcolor{black!75}
      \multicolumn{3}{c}{\head{Inputs}} & \multicolumn{2}{l}{\head{Output}} \\
      A & B & C & Y & minterm \\
      \hline
      0 & 0 & 0 & 0 & 0 \\
      0 & 0 & 1 & 1 & 1 \\
      0 & 1 & 0 & 0 & 2 \\
      0 & 1 & 1 & 0 & 3 \\ 
      1 & 0 & 0 & 0 & 4 \\
      1 & 0 & 1 & 1 & 5 \\
      1 & 1 & 0 & 1 & 6 \\
      1 & 1 & 1 & 0 & 7  
    \end{tabular}
  \end{center}
  \caption{Truth Table for Three-Input Circuit}
  \label{KM:tab:truth_table_for_3-input_circuit}
\end{table}

%****************************************************************
% Karnaugh map For 3-Input Circuit
%****************************************************************
\begin{figure}[H]
  \caption{Karnaugh map for Three-Input Circuit}
  \label{KM:fig:k-map_for_3-input_circuit}
  \myfloatalign
  \begin{tikzpicture} [circuit logic US, scale=1.00]
  % make all path lines (the node shapes) a little thicker
  \tikzstyle{every path}=[line width=0.50mm]
  
  %********************************************************************
  % Adjust the settings below to display the 1's and rectangles
  %********************************************************************
  % Uncomment the appropriate lines below to insert ones where needed
  %  \node[] at (1.4,1.5) {\huge $ 1 $}; % 00
    \node[] at (1.4,0.5) {\huge $ 1 $}; % 01
  %  \node[] at (2.4,1.5) {\huge $ 1 $}; % 02
  %  \node[] at (2.4,0.5) {\huge $ 1 $}; % 03
  %  \node[] at (4.4,1.5) {\huge $ 1 $}; % 04
    \node[] at (4.4,0.5) {\huge $ 1 $}; % 05
    \node[] at (3.4,1.5) {\huge $ 1 $}; % 06
  %  \node[] at (3.4,0.5) {\huge $ 1 $}; % 07
  
  % The coords for each cell - this is used as the origin for the solution box
  \coordinate (cell00) at (1.0,1.0); \coordinate (cell01) at (1.0,0.0);
  \coordinate (cell02) at (2.0,1.0); \coordinate (cell03) at (2.0,0.0);
  
  \coordinate (cell04) at (4.0,1.0); \coordinate (cell05) at (4.0,0.0);
  \coordinate (cell06) at (3.0,1.0); \coordinate (cell07) at (3.0,0.0);
    
  %********************************************************************
  % Shouldn't need to adjust anything below this point - this is just
  % the grid and the minterms.
  %********************************************************************  
  % Text in top-Left cell
  \node[] at (0.35,2.22) { \footnotesize $ \mathsf{ C } $ }; % C
  \node[] at (0.70,2.75) { \footnotesize $ \mathsf{ AB } $ }; % ab
  
  % Populate the top row header
  % In the following, the foreach lists a location/text pair
  % The the draw line draws the text at each location
  \foreach \loc/\txt in {
    (1.5,2.5)/{00},(2.5,2.5)/{01},(3.5,2.5)/{11},(4.5,2.5)/{10}
  }
  \draw \loc node{\large $\txt$};
  
  % Populate the header in column one
  \foreach \loc/\txt in { 
    (0.5,1.5)/{0},(0.5,0.5)/{1}
  }
  \draw \loc node{\large $\txt$};
  
  % Populate the minterms
  \foreach \loc/\txt in { 
    (1.75,1.15)/{00} , (2.75,1.15)/{02} , (3.75,1.15)/{06} , (4.75,1.15)/{04} ,
    (1.75,0.15)/{01} , (2.75,0.15)/{03} , (3.75,0.15)/{07} , (4.75,0.15)/{05} }
  \draw \loc node{ \color{blue!90!black} \footnotesize { $\txt$ }};
  
  % Draw the lines
  \draw
  % Finish drawing the grid
  [step=1.0cm,black,thin] (0,0) grid (5.0,3.0) % The Grid
  (0.0,3.0) -- (1.0,2.0) % Diagonal in the top left cell
  (1.0,2.05) -- (5.0,2.05) % Double line under top header row
  (0.95,0.0) -- (0.95,2.0) % Double line on left of header column one
  ;    
  \end{tikzpicture}
\end{figure}

In Karnaugh map \ref{KM:fig:k-map_for_3-input_circuit} all possible values for inputs $ A $ and $ B $ are listed across the top of the map while input $ C $ is listed down the left side. Therefore, minterm $ m_{05} $, in the lower right corner of the Karnaugh map, is for $ A=1 $; $ B=0 $; $ C=1 $, ($ AB'C $), one of the \emph{True} terms in the original equation. 

\subsection{The Gray Code}
\label{KM:subsec:the_gray_code_for_karnaugh_maps}

It should be noted that the values across the top of the Karnaugh map are not in binary order. Instead, those values are in ``Gray Code'' order. Gray code is essential for a Karnaugh map since the values for adjacent cells must change by only one bit. Constructing the Gray Code for three, four, and five variables is covered on page \pageref{MO:subsub:gray_code}; however, for the Karnaugh maps used in this chapter, it is enough to know the two-bit Gray code: $ 00 $, $ 01 $, $ 11 $, $ 10 $. 

\section{Drawing Four-Variable Karnaugh maps}
\label{KM:sec:drawing_4-variable_karnaugh_maps}

Consider Equation \ref{KM:eq:karnaugh_map_for_4_variables}.

\begin{align}
  \label{KM:eq:karnaugh_map_for_4_variables}
  ABCD'+AB'CD+A'B'CD &= Y
\end{align}

Table \ref{KM:tab:truth_table_for_4-input_circuit} and the Karnaugh map in Figure \ref{KM:fig:kmap_for_four_input_circuit} illustrate this equation.

% Truth Table
\begin{table}[H]
  \sffamily
  \newcommand{\head}[1]{\textcolor{white}{\textbf{#1}}}    
  \begin{center}
    \rowcolors{2}{gray!10}{white} % Color every other line a light gray
    \begin{tabular}{cccccc} 
      \rowcolor{black!75}
      \multicolumn{4}{c}{\head{Inputs}} & \multicolumn{2}{l}{\head{Output}} \\
      A & B & C & D & Y & minterm \\
      \hline
      0 & 0 & 0 & 0 & 0 & 0 \\
      0 & 0 & 0 & 1 & 0 & 1 \\
      0 & 0 & 1 & 0 & 0 & 2 \\
      0 & 0 & 1 & 1 & 1 & 3 \\ 
      0 & 1 & 0 & 0 & 0 & 4 \\
      0 & 1 & 0 & 1 & 0 & 5 \\
      0 & 1 & 1 & 0 & 0 & 6 \\
      1 & 1 & 1 & 1 & 0 & 7 \\
      1 & 0 & 0 & 0 & 0 & 8 \\
      1 & 0 & 0 & 1 & 0 & 9 \\
      1 & 0 & 1 & 0 & 0 & 10 \\
      1 & 0 & 1 & 1 & 1 & 11 \\ 
      1 & 1 & 0 & 0 & 0 & 12 \\
      1 & 1 & 0 & 1 & 0 & 13 \\
      1 & 1 & 1 & 0 & 1 & 14 \\
      1 & 1 & 1 & 1 & 0 & 15  
    \end{tabular}
  \end{center}
  \caption{Truth Table for Four-Input Circuit}
  \label{KM:tab:truth_table_for_4-input_circuit}
\end{table}

%****************************************************************
% Karnaugh map for 4-Input Circuit
%****************************************************************
\begin{figure}[H]
  \caption{K-Map For Four Input Circuit}
  \label{KM:fig:kmap_for_four_input_circuit}
  \myfloatalign
  \begin{tikzpicture} [circuit logic US, scale=1.00]
  % make all path lines (the node shapes) a little thicker
  \tikzstyle{every path}=[line width=0.50mm]
  
  %********************************************************************
  % Adjust the settings below to display the 1's and rectangles
  %********************************************************************
  % Uncomment the appropriate lines below to insert ones where needed
%  \node[] at (1.4,3.5) {\huge $ 1 $}; % 00
%  \node[] at (1.4,2.5) {\huge $ 1 $}; % 01
%  \node[] at (1.4,0.5) {\huge $ 1 $}; % 02
  \node[] at (1.4,1.5) {\huge $ 1 $}; % 03
%  \node[] at (2.4,3.5) {\huge $ 1 $}; % 04
%  \node[] at (2.4,2.5) {\huge $ 1 $}; % 05
%  \node[] at (2.4,0.5) {\huge $ 1 $}; % 06
%  \node[] at (2.4,1.5) {\huge $ 1 $}; % 07
%  \node[] at (4.4,3.5) {\huge $ 1 $}; % 08
%  \node[] at (4.4,2.5) {\huge $ 1 $}; % 09
%  \node[] at (4.4,0.5) {\huge $ 1 $}; % 10
  \node[] at (4.4,1.5) {\huge $ 1 $}; % 11
%  \node[] at (3.4,3.5) {\huge $ 1 $}; % 12
%  \node[] at (3.4,2.5) {\huge $ 1 $}; % 13
  \node[] at (3.4,0.5) {\huge $ 1 $}; % 14
%  \node[] at (3.4,1.5) {\huge $ 1 $}; % 15
  
  % The coords for each cell - this is used as the origin for the solution box
  \coordinate (cell00) at (1.0,3.0); \coordinate (cell01) at (1.0,2.0);
  \coordinate (cell02) at (1.0,0.0); \coordinate (cell03) at (1.0,1.0);

  \coordinate (cell04) at (2.0,3.0); \coordinate (cell05) at (2.0,2.0);
  \coordinate (cell06) at (2.0,0.0); \coordinate (cell07) at (2.0,1.0);

  \coordinate (cell12) at (3.0,3.0); \coordinate (cell13) at (3.0,2.0);
  \coordinate (cell14) at (3.0,0.0); \coordinate (cell15) at (3.0,1.0);

  \coordinate (cell08) at (4.0,3.0); \coordinate (cell09) at (4.0,2.0);
  \coordinate (cell10) at (4.0,0.0); \coordinate (cell11) at (4.0,1.0);
  
  % Horizontal Group
%  \node [draw,
%  color=red!70!black,
%  fill=red!20!white,
%  fill opacity=0.3,
%  minimum height=0.95cm,
%  minimum width=1.95cm, % Adjust to (number of cells) * 1 - 0.95
%  rounded corners,
%  anchor=south west] at (cell02) {}; % Enter left cell minterm
  
  % Vertical Group
%  \node [draw,
%  color=blue!70!black,
%  fill=blue!20!white,
%  fill opacity=0.3,
%  minimum height=1.95cm, % Adjust to (number of cells) * 1 - 0.95
%  minimum width=0.95cm,
%  rounded corners,
%  anchor=south west] at (cell07) {}; % Enter bottom cell minterm
  
  % Single Cell
%  \node [draw,
%  color=green!70!black,
%  fill=green!20!white,
%  fill opacity=0.3,
%  minimum height=0.95cm,
%  minimum width=0.95cm,
%  rounded corners,
%  anchor=south west] at (cell09) {}; % Enter cell minterm
  
  %********************************************************************
  % Shouldn't need to adjust anything below this point - this is just
  % the grid and the minterms.
  %********************************************************************  
  % Text in top-Left cell
  \node[] at (0.35,4.22) { \footnotesize $ \mathsf{ CD } $ }; % cd
  \node[] at (0.70,4.75) { \footnotesize $ \mathsf{ AB } $ }; % ab
  
  % Populate the top row header
  % In the following, the foreach lists a location/text pair
  % The the draw line draws the text at each location
  \foreach \loc/\txt in {
    (1.5,4.5)/{00},(2.5,4.5)/{01},(3.5,4.5)/{11},(4.5,4.5)/{10}
  }
  \draw \loc node{\large $\txt$};
  
  % Populate the header in column one
  \foreach \loc/\txt in { 
    (0.5,3.5)/{00},(0.5,2.5)/{01},(0.5,1.5)/{11},(0.5,0.5)/{10}
  }
  \draw \loc node{\large $\txt$};
  
  % Populate the minterms
  \foreach \loc/\txt in { 
    (1.75,3.15)/{00} , (2.75,3.15)/{04} , (3.75,3.15)/{12} , (4.75,3.15)/{08} ,
    (1.75,2.15)/{01} , (2.75,2.15)/{05} , (3.75,2.15)/{13} , (4.75,2.15)/{09} ,
    (1.75,1.15)/{03} , (2.75,1.15)/{07} , (3.75,1.15)/{15} , (4.75,1.15)/{11} ,
    (1.75,0.15)/{02} , (2.75,0.15)/{06} , (3.75,0.15)/{14} , (4.75,0.15)/{10} }
  \draw \loc node{ \color{blue!90!black} \footnotesize { $\txt$ }};
  
  % Draw the lines
  \draw
  % Finish drawing the grid
  [step=1.0cm,black,thin] (0,0) grid (5.0,5.0) % The Grid
  (0.0,5.0) -- (1.0,4.0) % Diagonal in the top left cell
  (1.0,4.05) -- (5.0,4.05) % Double line under top header row
  (0.95,0.0) -- (0.95,4.0) % Double line on left of header column one
  ;    
  \end{tikzpicture}
\end{figure}

This Karnaugh map is similar to those for two and three variables, but the top row is for the $ A $ and $ B $ inputs while the left column is for the $ C $ and $ D $ inputs. Notice that the values in both the top row and left column use Gray Code sequencing rather than binary counting.

It is easy to indicate the minterms that are \emph{True} on a Karnaugh map if Sigma Notation is available since the numbers following the Sigma sign are the minterms. As an example, Equation \ref{KM:eq:karnaugh_map_when_given_the_sigma_function} creates the Karnaugh map in Figure \ref{KM:fig:kmap_for_sigma_notation}. 

\begin{align}
  \label{KM:eq:karnaugh_map_when_given_the_sigma_function}
  \int{(A,B,C,D)} = \sum(0,1,2,4,5)
\end{align}

%****************************************************************
% Karnaugh map For a Sigma Function
%****************************************************************
\begin{figure}[H]
  \caption{K-Map For Sigma Notation}
  \label{KM:fig:kmap_for_sigma_notation}
  \myfloatalign
  \begin{tikzpicture} [circuit logic US, scale=1.00]
  % make all path lines (the node shapes) a little thicker
  \tikzstyle{every path}=[line width=0.50mm]
  
  %********************************************************************
  % Adjust the settings below to display the 1's and rectangles
  %********************************************************************
  % Uncomment the appropriate lines below to insert ones where needed
    \node[] at (1.4,3.5) {\huge $ 1 $}; % 00
    \node[] at (1.4,2.5) {\huge $ 1 $}; % 01
    \node[] at (1.4,0.5) {\huge $ 1 $}; % 02
  %  \node[] at (1.4,1.5) {\huge $ 1 $}; % 03
    \node[] at (2.4,3.5) {\huge $ 1 $}; % 04
    \node[] at (2.4,2.5) {\huge $ 1 $}; % 05
  %  \node[] at (2.4,0.5) {\huge $ 1 $}; % 06
  %  \node[] at (2.4,1.5) {\huge $ 1 $}; % 07
  %  \node[] at (4.4,3.5) {\huge $ 1 $}; % 08
  %  \node[] at (4.4,2.5) {\huge $ 1 $}; % 09
  %  \node[] at (4.4,0.5) {\huge $ 1 $}; % 10
  %  \node[] at (4.4,1.5) {\huge $ 1 $}; % 11
  %  \node[] at (3.4,3.5) {\huge $ 1 $}; % 12
  %  \node[] at (3.4,2.5) {\huge $ 1 $}; % 13
  %  \node[] at (3.4,0.5) {\huge $ 1 $}; % 14
  %  \node[] at (3.4,1.5) {\huge $ 1 $}; % 15
  
  % The coords for each cell - this is used as the origin for the solution box
  \coordinate (cell00) at (1.0,3.0); \coordinate (cell01) at (1.0,2.0);
  \coordinate (cell02) at (1.0,0.0); \coordinate (cell03) at (1.0,1.0);
  
  \coordinate (cell04) at (2.0,3.0); \coordinate (cell05) at (2.0,2.0);
  \coordinate (cell06) at (2.0,0.0); \coordinate (cell07) at (2.0,1.0);
  
  \coordinate (cell12) at (3.0,3.0); \coordinate (cell13) at (3.0,2.0);
  \coordinate (cell14) at (3.0,0.0); \coordinate (cell15) at (3.0,1.0);
  
  \coordinate (cell08) at (4.0,3.0); \coordinate (cell09) at (4.0,2.0);
  \coordinate (cell10) at (4.0,0.0); \coordinate (cell11) at (4.0,1.0);
  
  % Horizontal Group
  %  \node [draw,
  %  color=red!70!black,
  %  fill=red!20!white,
  %  fill opacity=0.3,
  %  minimum height=0.95cm,
  %  minimum width=1.95cm, % Adjust to (number of cells) * 1 - 0.95
  %  rounded corners,
  %  anchor=south west] at (cell02) {}; % Enter left cell minterm
  
  % Vertical Group
  %  \node [draw,
  %  color=blue!70!black,
  %  fill=blue!20!white,
  %  fill opacity=0.3,
  %  minimum height=1.95cm, % Adjust to (number of cells) * 1 - 0.95
  %  minimum width=0.95cm,
  %  rounded corners,
  %  anchor=south west] at (cell07) {}; % Enter bottom cell minterm
  
  % Single Cell
  %  \node [draw,
  %  color=green!70!black,
  %  fill=green!20!white,
  %  fill opacity=0.3,
  %  minimum height=0.95cm,
  %  minimum width=0.95cm,
  %  rounded corners,
  %  anchor=south west] at (cell09) {}; % Enter cell minterm
  
  %********************************************************************
  % Shouldn't need to adjust anything below this point - this is just
  % the grid and the minterms.
  %********************************************************************  
  % Text in top-Left cell
  \node[] at (0.35,4.22) { \footnotesize $ \mathsf{ CD } $ }; % cd
  \node[] at (0.70,4.75) { \footnotesize $ \mathsf{ AB } $ }; % ab
  
  % Populate the top row header
  % In the following, the foreach lists a location/text pair
  % The the draw line draws the text at each location
  \foreach \loc/\txt in {
    (1.5,4.5)/{00},(2.5,4.5)/{01},(3.5,4.5)/{11},(4.5,4.5)/{10}
  }
  \draw \loc node{\large $\txt$};
  
  % Populate the header in column one
  \foreach \loc/\txt in { 
    (0.5,3.5)/{00},(0.5,2.5)/{01},(0.5,1.5)/{11},(0.5,0.5)/{10}
  }
  \draw \loc node{\large $\txt$};
  
  % Populate the minterms
  \foreach \loc/\txt in { 
    (1.75,3.15)/{00} , (2.75,3.15)/{04} , (3.75,3.15)/{12} , (4.75,3.15)/{08} ,
    (1.75,2.15)/{01} , (2.75,2.15)/{05} , (3.75,2.15)/{13} , (4.75,2.15)/{09} ,
    (1.75,1.15)/{03} , (2.75,1.15)/{07} , (3.75,1.15)/{15} , (4.75,1.15)/{11} ,
    (1.75,0.15)/{02} , (2.75,0.15)/{06} , (3.75,0.15)/{14} , (4.75,0.15)/{10} }
  \draw \loc node{ \color{blue!90!black} \footnotesize { $\txt$ }};
  
  % Draw the lines
  \draw
  % Finish drawing the grid
  [step=1.0cm,black,thin] (0,0) grid (5.0,5.0) % The Grid
  (0.0,5.0) -- (1.0,4.0) % Diagonal in the top left cell
  (1.0,4.05) -- (5.0,4.05) % Double line under top header row
  (0.95,0.0) -- (0.95,4.0) % Double line on left of header column one
  ;    
  \end{tikzpicture}
\end{figure}

It is also possible to map values if the circuit is represented in Pi notation; but remember that maxterms indicate where zeros are placed on the Karnaugh map and the simplified circuit would actually be the inverse of the needed circuit. As an example, Equation \ref{KM:eq:karnaugh_map_when_given_the_pi_function} creates the Karnaugh map at Figure \ref{KM:fig:kmap_for_pi_notation}. 

\begin{align}
  \label{KM:eq:karnaugh_map_when_given_the_pi_function}
  \int{(A,B,C,D)} = \prod(8,9,12,13)
\end{align}

%****************************************************************
% Karnaugh map for a PI Function
%****************************************************************
\begin{figure}[H]
  \caption{K-Map For PI Notation}
  \label{KM:fig:kmap_for_pi_notation}
  \myfloatalign
  \begin{tikzpicture} [circuit logic US, scale=1.00]
  % make all path lines (the node shapes) a little thicker
  \tikzstyle{every path}=[line width=0.50mm]
  
  %********************************************************************
  % Adjust the settings below to display the 1's and rectangles
  %********************************************************************
  % Uncomment the appropriate lines below to insert ones where needed
  %  \node[] at (1.4,3.5) {\huge $ 0 $}; % 00
  %  \node[] at (1.4,2.5) {\huge $ 0 $}; % 01
  %  \node[] at (1.4,0.5) {\huge $ 0 $}; % 02
  %  \node[] at (1.4,1.5) {\huge $ 0 $}; % 03
  %  \node[] at (2.4,3.5) {\huge $ 0 $}; % 04
  %  \node[] at (2.4,2.5) {\huge $ 0 $}; % 05
  %  \node[] at (2.4,0.5) {\huge $ 0 $}; % 06
  %  \node[] at (2.4,1.5) {\huge $ 0 $}; % 07
    \node[] at (4.4,3.5) {\huge $ 0 $}; % 08
    \node[] at (4.4,2.5) {\huge $ 0 $}; % 09
  %  \node[] at (4.4,0.5) {\huge $ 0 $}; % 10
  %  \node[] at (4.4,1.5) {\huge $ 0 $}; % 11
    \node[] at (3.4,3.5) {\huge $ 0 $}; % 12
    \node[] at (3.4,2.5) {\huge $ 0 $}; % 13
  %  \node[] at (3.4,0.5) {\huge $ 0 $}; % 14
  %  \node[] at (3.4,1.5) {\huge $ 0 $}; % 15
  
  % The coords for each cell - this is used as the origin for the solution box
  \coordinate (cell00) at (1.0,3.0); \coordinate (cell01) at (1.0,2.0);
  \coordinate (cell02) at (1.0,0.0); \coordinate (cell03) at (1.0,1.0);
  
  \coordinate (cell04) at (2.0,3.0); \coordinate (cell05) at (2.0,2.0);
  \coordinate (cell06) at (2.0,0.0); \coordinate (cell07) at (2.0,1.0);
  
  \coordinate (cell12) at (3.0,3.0); \coordinate (cell13) at (3.0,2.0);
  \coordinate (cell14) at (3.0,0.0); \coordinate (cell15) at (3.0,1.0);
  
  \coordinate (cell08) at (4.0,3.0); \coordinate (cell09) at (4.0,2.0);
  \coordinate (cell10) at (4.0,0.0); \coordinate (cell11) at (4.0,1.0);
  
  % Horizontal Group
  %  \node [draw,
  %  color=red!70!black,
  %  fill=red!20!white,
  %  fill opacity=0.3,
  %  minimum height=0.95cm,
  %  minimum width=1.95cm, % Adjust to (number of cells) * 1 - 0.95
  %  rounded corners,
  %  anchor=south west] at (cell02) {}; % Enter left cell minterm
  
  % Vertical Group
  %  \node [draw,
  %  color=blue!70!black,
  %  fill=blue!20!white,
  %  fill opacity=0.3,
  %  minimum height=1.95cm, % Adjust to (number of cells) * 1 - 0.95
  %  minimum width=0.95cm,
  %  rounded corners,
  %  anchor=south west] at (cell07) {}; % Enter bottom cell minterm
  
  % Single Cell
  %  \node [draw,
  %  color=green!70!black,
  %  fill=green!20!white,
  %  fill opacity=0.3,
  %  minimum height=0.95cm,
  %  minimum width=0.95cm,
  %  rounded corners,
  %  anchor=south west] at (cell09) {}; % Enter cell minterm
  
  %********************************************************************
  % Shouldn't need to adjust anything below this point - this is just
  % the grid and the minterms.
  %********************************************************************  
  % Text in top-Left cell
  \node[] at (0.35,4.22) { \footnotesize $ \mathsf{ CD } $ }; % cd
  \node[] at (0.70,4.75) { \footnotesize $ \mathsf{ AB } $ }; % ab
  
  % Populate the top row header
  % In the following, the foreach lists a location/text pair
  % The the draw line draws the text at each location
  \foreach \loc/\txt in {
    (1.5,4.5)/{00},(2.5,4.5)/{01},(3.5,4.5)/{11},(4.5,4.5)/{10}
  }
  \draw \loc node{\large $\txt$};
  
  % Populate the header in column one
  \foreach \loc/\txt in { 
    (0.5,3.5)/{00},(0.5,2.5)/{01},(0.5,1.5)/{11},(0.5,0.5)/{10}
  }
  \draw \loc node{\large $\txt$};
  
  % Populate the minterms
  \foreach \loc/\txt in { 
    (1.75,3.15)/{00} , (2.75,3.15)/{04} , (3.75,3.15)/{12} , (4.75,3.15)/{08} ,
    (1.75,2.15)/{01} , (2.75,2.15)/{05} , (3.75,2.15)/{13} , (4.75,2.15)/{09} ,
    (1.75,1.15)/{03} , (2.75,1.15)/{07} , (3.75,1.15)/{15} , (4.75,1.15)/{11} ,
    (1.75,0.15)/{02} , (2.75,0.15)/{06} , (3.75,0.15)/{14} , (4.75,0.15)/{10} }
  \draw \loc node{ \color{blue!90!black} \footnotesize { $\txt$ }};
  
  % Draw the lines
  \draw
  % Finish drawing the grid
  [step=1.0cm,black,thin] (0,0) grid (5.0,5.0) % The Grid
  (0.0,5.0) -- (1.0,4.0) % Diagonal in the top left cell
  (1.0,4.05) -- (5.0,4.05) % Double line under top header row
  (0.95,0.0) -- (0.95,4.0) % Double line on left of header column one
  ;    
  \end{tikzpicture}
\end{figure}

To simplify this map, the designer could place ones in all of the empty cells and then simplify the ``ones'' circuit using the techniques explained below. Alternatively, the designer could also simplify the map by combining the zeros as if they were ones, and then finding the DeMorgan inverse (page  \pageref{BF:sec:demorgans_theorem}) of that simplification. As an example, the maxterm Karnaugh map above would simplify to $ \overline{AC'} $, and the DeMorgan equivalent for that is $ A'+C $, which is the minterm version of the simplified circuit.

\section{Simplifying Groups of Two}
\label{KM:sec:simplifying_groups_of_two}

To simplify a Boolean equation using a Karnaugh map, start by creating the Karnaugh map, indicating the input variable combinations that lead to a \emph{True} output for the circuit. Equations \ref{KM:eq:simplifying_4-input_equations_groups_of_two} and \ref{KM:eq:sigma_notation_solving_k-map_groups_of_two} are for the same circuit and the Karnaugh map in Figure \ref{KM:fig:k-map_for_groups_of_two_ex_1} was built from these equations. 

\begin{align}
  \label{KM:eq:simplifying_4-input_equations_groups_of_two}
  A'B'C'D'+A'BC'D'+A'BCD+AB'CD'+AB'CD
\end{align}

\begin{align}
  \label{KM:eq:sigma_notation_solving_k-map_groups_of_two}
  \int{(A,B,C,D)} = \sum(0,4,7,10,11)
\end{align}

%****************************************************************
% Karnaugh map For Groups of 2, Example 1
%****************************************************************
\begin{figure}[H]
  \caption{K-Map for Groups of Two: Ex 1}
  \label{KM:fig:k-map_for_groups_of_two_ex_1}
  \myfloatalign
  \begin{tikzpicture} [circuit logic US, scale=1.00]
  % make all path lines (the node shapes) a little thicker
  \tikzstyle{every path}=[line width=0.50mm]
  
  %********************************************************************
  % Adjust the settings below to display the 1's and rectangles
  %********************************************************************
  % Uncomment the appropriate lines below to insert ones where needed
    \node[] at (1.4,3.5) {\huge $ 1 $}; % 00
  %  \node[] at (1.4,2.5) {\huge $ 1 $}; % 01
  %  \node[] at (1.4,0.5) {\huge $ 1 $}; % 02
  %  \node[] at (1.4,1.5) {\huge $ 1 $}; % 03
    \node[] at (2.4,3.5) {\huge $ 1 $}; % 04
  %  \node[] at (2.4,2.5) {\huge $ 1 $}; % 05
  %  \node[] at (2.4,0.5) {\huge $ 1 $}; % 06
    \node[] at (2.4,1.5) {\huge $ 1 $}; % 07
  %  \node[] at (4.4,3.5) {\huge $ 1 $}; % 08
  %  \node[] at (4.4,2.5) {\huge $ 1 $}; % 09
    \node[] at (4.4,0.5) {\huge $ 1 $}; % 10
    \node[] at (4.4,1.5) {\huge $ 1 $}; % 11
  %  \node[] at (3.4,3.5) {\huge $ 1 $}; % 12
  %  \node[] at (3.4,2.5) {\huge $ 1 $}; % 13
  %  \node[] at (3.4,0.5) {\huge $ 1 $}; % 14
  %  \node[] at (3.4,1.5) {\huge $ 1 $}; % 15
  
  % The coords for each cell - this is used as the origin for the solution box
  \coordinate (cell00) at (1.0,3.0); \coordinate (cell01) at (1.0,2.0);
  \coordinate (cell02) at (1.0,0.0); \coordinate (cell03) at (1.0,1.0);
  
  \coordinate (cell04) at (2.0,3.0); \coordinate (cell05) at (2.0,2.0);
  \coordinate (cell06) at (2.0,0.0); \coordinate (cell07) at (2.0,1.0);
  
  \coordinate (cell12) at (3.0,3.0); \coordinate (cell13) at (3.0,2.0);
  \coordinate (cell14) at (3.0,0.0); \coordinate (cell15) at (3.0,1.0);
  
  \coordinate (cell08) at (4.0,3.0); \coordinate (cell09) at (4.0,2.0);
  \coordinate (cell10) at (4.0,0.0); \coordinate (cell11) at (4.0,1.0);
  
  % Horizontal Group
  %  \node [draw,
  %  color=red!70!black,
  %  fill=red!20!white,
  %  fill opacity=0.3,
  %  minimum height=0.95cm,
  %  minimum width=1.95cm, % Adjust to (number of cells) * 1 - 0.95
  %  rounded corners,
  %  anchor=south west] at (cell02) {}; % Enter left cell minterm
  
  % Vertical Group
  %  \node [draw,
  %  color=blue!70!black,
  %  fill=blue!20!white,
  %  fill opacity=0.3,
  %  minimum height=1.95cm, % Adjust to (number of cells) * 1 - 0.95
  %  minimum width=0.95cm,
  %  rounded corners,
  %  anchor=south west] at (cell07) {}; % Enter bottom cell minterm
  
  % Single Cell
  %  \node [draw,
  %  color=green!70!black,
  %  fill=green!20!white,
  %  fill opacity=0.3,
  %  minimum height=0.95cm,
  %  minimum width=0.95cm,
  %  rounded corners,
  %  anchor=south west] at (cell09) {}; % Enter cell minterm
  
  %********************************************************************
  % Shouldn't need to adjust anything below this point - this is just
  % the grid and the minterms.
  %********************************************************************  
  % Text in top-Left cell
  \node[] at (0.35,4.22) { \footnotesize $ \mathsf{ CD } $ }; % cd
  \node[] at (0.70,4.75) { \footnotesize $ \mathsf{ AB } $ }; % ab
  
  % Populate the top row header
  % In the following, the foreach lists a location/text pair
  % The the draw line draws the text at each location
  \foreach \loc/\txt in {
    (1.5,4.5)/{00},(2.5,4.5)/{01},(3.5,4.5)/{11},(4.5,4.5)/{10}
  }
  \draw \loc node{\large $\txt$};
  
  % Populate the header in column one
  \foreach \loc/\txt in { 
    (0.5,3.5)/{00},(0.5,2.5)/{01},(0.5,1.5)/{11},(0.5,0.5)/{10}
  }
  \draw \loc node{\large $\txt$};
  
  % Populate the minterms
  \foreach \loc/\txt in { 
    (1.75,3.15)/{00} , (2.75,3.15)/{04} , (3.75,3.15)/{12} , (4.75,3.15)/{08} ,
    (1.75,2.15)/{01} , (2.75,2.15)/{05} , (3.75,2.15)/{13} , (4.75,2.15)/{09} ,
    (1.75,1.15)/{03} , (2.75,1.15)/{07} , (3.75,1.15)/{15} , (4.75,1.15)/{11} ,
    (1.75,0.15)/{02} , (2.75,0.15)/{06} , (3.75,0.15)/{14} , (4.75,0.15)/{10} }
  \draw \loc node{ \color{blue!90!black} \footnotesize { $\txt$ }};
  
  % Draw the lines
  \draw
  % Finish drawing the grid
  [step=1.0cm,black,thin] (0,0) grid (5.0,5.0) % The Grid
  (0.0,5.0) -- (1.0,4.0) % Diagonal in the top left cell
  (1.0,4.05) -- (5.0,4.05) % Double line under top header row
  (0.95,0.0) -- (0.95,4.0) % Double line on left of header column one
  ;    
  \end{tikzpicture}
\end{figure}

Once the \emph{True} outputs are indicated on the Karnaugh map, mark any groups of ones that are adjacent to each other, either horizontally or vertically (but not diagonally). Also, mark any ones that are ``left over'' and are not adjacent to any other ones, as illustrated in the Karnaugh map in Figure \ref{KM:fig:kmap_solving_groups_of_two_ex_1}. 

%****************************************************************
% Karnaugh map For Groups of 2, Example 1, Simplified
%****************************************************************
\begin{figure}[H]
  \caption{K-Map for Groups of Two: Ex 1, Solved}
  \label{KM:fig:k-map_for_groups_of_two_ex_1_solved}
  \myfloatalign
  \begin{tikzpicture} [circuit logic US, scale=1.00]
  % make all path lines (the node shapes) a little thicker
  \tikzstyle{every path}=[line width=0.50mm]
  
  %********************************************************************
  % Adjust the settings below to display the 1's and rectangles
  %********************************************************************
  % Uncomment the appropriate lines below to insert ones where needed
  \node[] at (1.4,3.5) {\huge $ 1 $}; % 00
  %  \node[] at (1.4,2.5) {\huge $ 1 $}; % 01
  %  \node[] at (1.4,0.5) {\huge $ 1 $}; % 02
  %  \node[] at (1.4,1.5) {\huge $ 1 $}; % 03
  \node[] at (2.4,3.5) {\huge $ 1 $}; % 04
  %  \node[] at (2.4,2.5) {\huge $ 1 $}; % 05
  %  \node[] at (2.4,0.5) {\huge $ 1 $}; % 06
  \node[] at (2.4,1.5) {\huge $ 1 $}; % 07
  %  \node[] at (4.4,3.5) {\huge $ 1 $}; % 08
  %  \node[] at (4.4,2.5) {\huge $ 1 $}; % 09
  \node[] at (4.4,0.5) {\huge $ 1 $}; % 10
  \node[] at (4.4,1.5) {\huge $ 1 $}; % 11
  %  \node[] at (3.4,3.5) {\huge $ 1 $}; % 12
  %  \node[] at (3.4,2.5) {\huge $ 1 $}; % 13
  %  \node[] at (3.4,0.5) {\huge $ 1 $}; % 14
  %  \node[] at (3.4,1.5) {\huge $ 1 $}; % 15
  
  % The coords for each cell - this is used as the origin for the solution box
  \coordinate (cell00) at (1.0,3.0); \coordinate (cell01) at (1.0,2.0);
  \coordinate (cell02) at (1.0,0.0); \coordinate (cell03) at (1.0,1.0);
  
  \coordinate (cell04) at (2.0,3.0); \coordinate (cell05) at (2.0,2.0);
  \coordinate (cell06) at (2.0,0.0); \coordinate (cell07) at (2.0,1.0);
  
  \coordinate (cell12) at (3.0,3.0); \coordinate (cell13) at (3.0,2.0);
  \coordinate (cell14) at (3.0,0.0); \coordinate (cell15) at (3.0,1.0);
  
  \coordinate (cell08) at (4.0,3.0); \coordinate (cell09) at (4.0,2.0);
  \coordinate (cell10) at (4.0,0.0); \coordinate (cell11) at (4.0,1.0);
  
  % Horizontal Group
    \node [draw,
    color=red!70!black,
    fill=red!20!white,
    fill opacity=0.3,
    minimum height=0.95cm,
    minimum width=1.95cm, % Adjust to (number of cells) * 1 - 0.95
    rounded corners,
    anchor=south west] at (cell00) {}; % Enter left cell minterm
  
  % Vertical Group
    \node [draw,
    color=blue!70!black,
    fill=blue!20!white,
    fill opacity=0.3,
    minimum height=1.95cm, % Adjust to (number of cells) * 1 - 0.95
    minimum width=0.95cm,
    rounded corners,
    anchor=south west] at (cell10) {}; % Enter bottom cell minterm
  
  % Single Cell
    \node [draw,
    color=green!70!black,
    fill=green!20!white,
    fill opacity=0.3,
    minimum height=0.95cm,
    minimum width=0.95cm,
    rounded corners,
    anchor=south west] at (cell07) {}; % Enter cell minterm
  
  %********************************************************************
  % Shouldn't need to adjust anything below this point - this is just
  % the grid and the minterms.
  %********************************************************************  
  % Text in top-Left cell
  \node[] at (0.35,4.22) { \footnotesize $ \mathsf{ CD } $ }; % cd
  \node[] at (0.70,4.75) { \footnotesize $ \mathsf{ AB } $ }; % ab
  
  % Populate the top row header
  % In the following, the foreach lists a location/text pair
  % The the draw line draws the text at each location
  \foreach \loc/\txt in {
    (1.5,4.5)/{00},(2.5,4.5)/{01},(3.5,4.5)/{11},(4.5,4.5)/{10}
  }
  \draw \loc node{\large $\txt$};
  
  % Populate the header in column one
  \foreach \loc/\txt in { 
    (0.5,3.5)/{00},(0.5,2.5)/{01},(0.5,1.5)/{11},(0.5,0.5)/{10}
  }
  \draw \loc node{\large $\txt$};
  
  % Populate the minterms
  \foreach \loc/\txt in { 
    (1.75,3.15)/{00} , (2.75,3.15)/{04} , (3.75,3.15)/{12} , (4.75,3.15)/{08} ,
    (1.75,2.15)/{01} , (2.75,2.15)/{05} , (3.75,2.15)/{13} , (4.75,2.15)/{09} ,
    (1.75,1.15)/{03} , (2.75,1.15)/{07} , (3.75,1.15)/{15} , (4.75,1.15)/{11} ,
    (1.75,0.15)/{02} , (2.75,0.15)/{06} , (3.75,0.15)/{14} , (4.75,0.15)/{10} }
  \draw \loc node{ \color{blue!90!black} \footnotesize { $\txt$ }};
  
  % Draw the lines
  \draw
  % Finish drawing the grid
  [step=1.0cm,black,thin] (0,0) grid (5.0,5.0) % The Grid
  (0.0,5.0) -- (1.0,4.0) % Diagonal in the top left cell
  (1.0,4.05) -- (5.0,4.05) % Double line under top header row
  (0.95,0.0) -- (0.95,4.0) % Double line on left of header column one
  ;    
  \end{tikzpicture}
\end{figure}

Notice the group in the top-left corner (minterms $ 00 $ and $ 04 $). This group includes the following two input combinations: $ A'B'C'D' + A'BC'D' $. In this expression, the $ B $ and $ B' $ terms can be removed by the Complement Property (page \pageref{BF:subsec:complement}); so this group reduces to $ A'C'D' $. To simplify this expression by inspecting the Karnaugh map, notice that the variable $ A $ is zero for both of these minterms; therefore, $ A' $ must be part of the final expression. In the same way, variables $ C $ and $ D $ are zero for both terms; therefore, $ C'D' $ must be part of the final expression. Since variable $ B $ changes it can be ignored when forming the simplified expression.

The group in the lower-right corner (minterms $ 10 $ and $ 11 $) includes the following two input variable combinations: $ AB'CD + AB'CD' $. The $ D $ and $ D' $ terms can be removed by the Complement Property; so this group simplifies to $ AB'C $. Again, inspecting these two terms would reveal that the variables $ AB'C $ do not change between the two terms, so they must appear in the final expression. 

Minterm $ 07 $, the lone term indicated in column two, cannot be reduced since it is not adjacent to any other ones. Therefore, it must go into the simplified equation unchanged: $ A'BCD $. 

When finished, the original equation reduces to Equation \ref{KM:eq:sigma_notation_solving_k-map_groups_of_two_solution}.

\begin{align}
  \label{KM:eq:sigma_notation_solving_k-map_groups_of_two_solution}
  A'C'D'+AB'C+A'BCD
\end{align}

Using a Karnaugh map, the circuit was simplified from four four-input \textsf{AND} gates to two three-input \textsf{AND} gates and one four-input \textsf{AND} gate. 

The various ones on a Karnaugh map are called the \emph{Implicants} of the solution. These are the algebraic products that are necessary to ``imply'' (or bring about) the final simplification of the circuit. When an implicant cannot be grouped with any others, or when two or more implicants are grouped together, they are called \emph{Prime Implicants}. The three groups ($ A'C'D' $, $ AB'C $, and $ A'BCD $) found by analyzing the Karnaugh map above are the prime implicants for this equation. When prime implicants are a necessary part of the final simplified equation, and they are not subsumed by any other implicants, they are called \emph{Essential Prime Implicants}. For the simple example given above, all of the prime implicants are essential; however, more complex Karnaugh maps may have numerous prime implicants that are subsumed by other implicants; thus, are not essential. There are examples of these types of maps later in this chapter. 

A second example is illustrated in Equation \ref{KM:eq:simplifying_4-input_equations_groups_of_two_ex_2} and the Karnaugh map in Figure \ref{KM:fig:kmap_solving_groups_of_two_ex_2}.

\begin{align}
  \label{KM:eq:simplifying_4-input_equations_groups_of_two_ex_2}
  &A'B'C'D+A'B'CD+A'BCD'+ \\
  \nonumber
  &ABC'D+ABCD'+AB'C'D'=Y
\end{align}

\begin{align}
  \label{KM:eq:sigma_notation_solving_k-map_groups_of_two_ex_2}
  \int{(A,B,C,D)} = \sum(1,3,6,8,13,14)
\end{align}

%****************************************************************
% Karnaugh map For Groups of 2, Example 2
%****************************************************************
\begin{figure}[H]
  \caption{K-Map Solving Groups of Two: Example 2}
  \label{KM:fig:kmap_solving_groups_of_two_ex_2}
  \myfloatalign
  \begin{tikzpicture} [circuit logic US, scale=1.00]
  % make all path lines (the node shapes) a little thicker
  \tikzstyle{every path}=[line width=0.50mm]
  
  %********************************************************************
  % Adjust the settings below to display the 1's and rectangles
  %********************************************************************
  % Uncomment the appropriate lines below to insert ones where needed
  %  \node[] at (1.4,3.5) {\huge $ 1 $}; % 00
    \node[] at (1.4,2.5) {\huge $ 1 $}; % 01
  %  \node[] at (1.4,0.5) {\huge $ 1 $}; % 02
    \node[] at (1.4,1.5) {\huge $ 1 $}; % 03
  %  \node[] at (2.4,3.5) {\huge $ 1 $}; % 04
  %  \node[] at (2.4,2.5) {\huge $ 1 $}; % 05
    \node[] at (2.4,0.5) {\huge $ 1 $}; % 06
  %  \node[] at (2.4,1.5) {\huge $ 1 $}; % 07
    \node[] at (4.4,3.5) {\huge $ 1 $}; % 08
  %  \node[] at (4.4,2.5) {\huge $ 1 $}; % 09
  %  \node[] at (4.4,0.5) {\huge $ 1 $}; % 10
  %  \node[] at (4.4,1.5) {\huge $ 1 $}; % 11
  %  \node[] at (3.4,3.5) {\huge $ 1 $}; % 12
    \node[] at (3.4,2.5) {\huge $ 1 $}; % 13
    \node[] at (3.4,0.5) {\huge $ 1 $}; % 14
  %  \node[] at (3.4,1.5) {\huge $ 1 $}; % 15
  
  % The coords for each cell - this is used as the origin for the solution box
  \coordinate (cell00) at (1.0,3.0); \coordinate (cell01) at (1.0,2.0);
  \coordinate (cell02) at (1.0,0.0); \coordinate (cell03) at (1.0,1.0);
  
  \coordinate (cell04) at (2.0,3.0); \coordinate (cell05) at (2.0,2.0);
  \coordinate (cell06) at (2.0,0.0); \coordinate (cell07) at (2.0,1.0);
  
  \coordinate (cell12) at (3.0,3.0); \coordinate (cell13) at (3.0,2.0);
  \coordinate (cell14) at (3.0,0.0); \coordinate (cell15) at (3.0,1.0);
  
  \coordinate (cell08) at (4.0,3.0); \coordinate (cell09) at (4.0,2.0);
  \coordinate (cell10) at (4.0,0.0); \coordinate (cell11) at (4.0,1.0);
  
  % Horizontal Group
    \node [draw,
    color=red!70!black,
    fill=red!20!white,
    fill opacity=0.3,
    minimum height=0.95cm,
    minimum width=1.95cm, % Adjust to (number of cells) * 1 - 0.95
    rounded corners,
    anchor=south west] at (cell06) {}; % Enter left cell minterm
  
  % Vertical Group
    \node [draw,
    color=blue!70!black,
    fill=blue!20!white,
    fill opacity=0.3,
    minimum height=1.95cm, % Adjust to (number of cells) * 1 - 0.95
    minimum width=0.95cm,
  %  rounded corners,
    anchor=south west] at (cell03) {}; % Enter bottom cell minterm
  
  % Single Cell
    \node [draw,
    color=green!70!black,
    fill=green!20!white,
    fill opacity=0.3,
    minimum height=0.95cm,
    minimum width=0.95cm,
    rounded corners,
    anchor=south west] at (cell13) {}; % Enter cell minterm

  % Single Cell
    \node [draw,
    color=green!70!black,
    fill=green!20!white,
    fill opacity=0.3,
    minimum height=0.95cm,
    minimum width=0.95cm,
    rounded corners,
    anchor=south west] at (cell08) {}; % Enter cell minterm
  
  %********************************************************************
  % Shouldn't need to adjust anything below this point - this is just
  % the grid and the minterms.
  %********************************************************************  
  % Text in top-Left cell
  \node[] at (0.35,4.22) { \footnotesize $ \mathsf{ CD } $ }; % cd
  \node[] at (0.70,4.75) { \footnotesize $ \mathsf{ AB } $ }; % ab
  
  % Populate the top row header
  % In the following, the foreach lists a location/text pair
  % The the draw line draws the text at each location
  \foreach \loc/\txt in {
    (1.5,4.5)/{00},(2.5,4.5)/{01},(3.5,4.5)/{11},(4.5,4.5)/{10}
  }
  \draw \loc node{\large $\txt$};
  
  % Populate the header in column one
  \foreach \loc/\txt in { 
    (0.5,3.5)/{00},(0.5,2.5)/{01},(0.5,1.5)/{11},(0.5,0.5)/{10}
  }
  \draw \loc node{\large $\txt$};
  
  % Populate the minterms
  \foreach \loc/\txt in { 
    (1.75,3.15)/{00} , (2.75,3.15)/{04} , (3.75,3.15)/{12} , (4.75,3.15)/{08} ,
    (1.75,2.15)/{01} , (2.75,2.15)/{05} , (3.75,2.15)/{13} , (4.75,2.15)/{09} ,
    (1.75,1.15)/{03} , (2.75,1.15)/{07} , (3.75,1.15)/{15} , (4.75,1.15)/{11} ,
    (1.75,0.15)/{02} , (2.75,0.15)/{06} , (3.75,0.15)/{14} , (4.75,0.15)/{10} }
  \draw \loc node{ \color{blue!90!black} \footnotesize { $\txt$ }};
  
  % Draw the lines
  \draw
  % Finish drawing the grid
  [step=1.0cm,black,thin] (0,0) grid (5.0,5.0) % The Grid
  (0.0,5.0) -- (1.0,4.0) % Diagonal in the top left cell
  (1.0,4.05) -- (5.0,4.05) % Double line under top header row
  (0.95,0.0) -- (0.95,4.0) % Double line on left of header column one
  ;    
  \end{tikzpicture}
\end{figure}

All groups of adjacent ones have been marked, so this circuit can be simplified by looking for groups of two. Starting with minterms $ 01 $ and $ 03 $, $ A'B'C'D + A'B'CD $ simplifies to $ A'B'D $. Minterms $ 06 $ and $ 14 $ simplifies to $ BCD' $. The other two marked minterms are not adjacent to any others, so they cannot be simplified. Each of the marked terms are prime implicants; and since they are not subsumed by any other implicants, they are essential prime implicants. Equation \ref{KM:eq:simplifying_4-input_equations_groups_of_two_ex_2_solution} is the simplified solution.

\begin{align}
  \label{KM:eq:simplifying_4-input_equations_groups_of_two_ex_2_solution}
  &ABC'D+AB'C'D'+A'B'D+BCD' = Y
\end{align}

\section{Simplifying Larger Groups}
\label{KM:sec:simplifying_larger_groups}

When simplifying Karnaugh maps, it is most efficient to find groups of $ 16 $, $ 8 $, $ 4 $, and $ 2 $ adjacent ones, in that order. In general, the larger the group, the simpler the expression becomes; so one large group is preferable to two smaller groups. However, remember that any group can only use ones that are adjacent along a horizontal or vertical line, not diagonal. 

\subsection{Groups of 16}
\label{KM:subsec:groups_of_16}

Groups of $ 16 $ reduce to a constant output of one. This is because if a circuit is built such that every possible combination of four inputs yields a \emph{True} output, then the circuit is unnecessary and can be replaced by a wire. There is no example Karnaugh map posted here to illustrate a circuit like this because if every cell in a Karnaugh map contains a one, then the circuit is unnecessary. By the same token, any Karnaugh map that contains only zeros indicates that the circuit would never output a \emph{True} condition so the circuit is unnecessary.

\subsection{Groups of Eight}
\label{KM:subsec:groups_of_8}

Groups of eight simplifies to a single output variable. Consider the Karnaugh map in Figure \ref{KM:fig:kmap_solving_groups_of_8}. 

%****************************************************************
% Karnaugh map For Groups of 8
%****************************************************************
\begin{figure}[H]
  \caption{K-Map Solving Groups of 8}
  \label{KM:fig:kmap_solving_groups_of_8}
  \myfloatalign
  \begin{tikzpicture} [circuit logic US, scale=1.00]
  % make all path lines (the node shapes) a little thicker
  \tikzstyle{every path}=[line width=0.50mm]
  
  %********************************************************************
  % Adjust the settings below to display the 1's and rectangles
  %********************************************************************
  % Uncomment the appropriate lines below to insert ones where needed
  %  \node[] at (1.4,3.5) {\huge $ 1 $}; % 00
    \node[] at (1.4,2.5) {\huge $ 1 $}; % 01
  %  \node[] at (1.4,0.5) {\huge $ 1 $}; % 02
    \node[] at (1.4,1.5) {\huge $ 1 $}; % 03
  %  \node[] at (2.4,3.5) {\huge $ 1 $}; % 04
    \node[] at (2.4,2.5) {\huge $ 1 $}; % 05
  %  \node[] at (2.4,0.5) {\huge $ 1 $}; % 06
    \node[] at (2.4,1.5) {\huge $ 1 $}; % 07
  %  \node[] at (4.4,3.5) {\huge $ 1 $}; % 08
    \node[] at (4.4,2.5) {\huge $ 1 $}; % 09
  %  \node[] at (4.4,0.5) {\huge $ 1 $}; % 10
    \node[] at (4.4,1.5) {\huge $ 1 $}; % 11
  %  \node[] at (3.4,3.5) {\huge $ 1 $}; % 12
    \node[] at (3.4,2.5) {\huge $ 1 $}; % 13
  %  \node[] at (3.4,0.5) {\huge $ 1 $}; % 14
    \node[] at (3.4,1.5) {\huge $ 1 $}; % 15
  
  % The coords for each cell - this is used as the origin for the solution box
  \coordinate (cell00) at (1.0,3.0); \coordinate (cell01) at (1.0,2.0);
  \coordinate (cell02) at (1.0,0.0); \coordinate (cell03) at (1.0,1.0);
  
  \coordinate (cell04) at (2.0,3.0); \coordinate (cell05) at (2.0,2.0);
  \coordinate (cell06) at (2.0,0.0); \coordinate (cell07) at (2.0,1.0);
  
  \coordinate (cell12) at (3.0,3.0); \coordinate (cell13) at (3.0,2.0);
  \coordinate (cell14) at (3.0,0.0); \coordinate (cell15) at (3.0,1.0);
  
  \coordinate (cell08) at (4.0,3.0); \coordinate (cell09) at (4.0,2.0);
  \coordinate (cell10) at (4.0,0.0); \coordinate (cell11) at (4.0,1.0);
  
  % Horizontal Group
    \node [draw,
    color=red!70!black,
    fill=red!20!white,
    fill opacity=0.3,
    minimum height=1.95cm,
    minimum width=3.95cm, % Adjust to (number of cells) * 1 - 0.95
    rounded corners,
    anchor=south west] at (cell03) {}; % Enter left cell minterm
  
  % Vertical Group
  %  \node [draw,
  %  color=blue!70!black,
  %  fill=blue!20!white,
  %  fill opacity=0.3,
  %  minimum height=1.95cm, % Adjust to (number of cells) * 1 - 0.95
  %  minimum width=0.95cm,
  %  rounded corners,
  %  anchor=south west] at (cell07) {}; % Enter bottom cell minterm
  
  % Single Cell
  %  \node [draw,
  %  color=green!70!black,
  %  fill=green!20!white,
  %  fill opacity=0.3,
  %  minimum height=0.95cm,
  %  minimum width=0.95cm,
  %  rounded corners,
  %  anchor=south west] at (cell09) {}; % Enter cell minterm
  
  %********************************************************************
  % Shouldn't need to adjust anything below this point - this is just
  % the grid and the minterms.
  %********************************************************************  
  % Text in top-Left cell
  \node[] at (0.35,4.22) { \footnotesize $ \mathsf{ CD } $ }; % cd
  \node[] at (0.70,4.75) { \footnotesize $ \mathsf{ AB } $ }; % ab
  
  % Populate the top row header
  % In the following, the foreach lists a location/text pair
  % The the draw line draws the text at each location
  \foreach \loc/\txt in {
    (1.5,4.5)/{00},(2.5,4.5)/{01},(3.5,4.5)/{11},(4.5,4.5)/{10}
  }
  \draw \loc node{\large $\txt$};
  
  % Populate the header in column one
  \foreach \loc/\txt in { 
    (0.5,3.5)/{00},(0.5,2.5)/{01},(0.5,1.5)/{11},(0.5,0.5)/{10}
  }
  \draw \loc node{\large $\txt$};
  
  % Populate the minterms
  \foreach \loc/\txt in { 
    (1.75,3.15)/{00} , (2.75,3.15)/{04} , (3.75,3.15)/{12} , (4.75,3.15)/{08} ,
    (1.75,2.15)/{01} , (2.75,2.15)/{05} , (3.75,2.15)/{13} , (4.75,2.15)/{09} ,
    (1.75,1.15)/{03} , (2.75,1.15)/{07} , (3.75,1.15)/{15} , (4.75,1.15)/{11} ,
    (1.75,0.15)/{02} , (2.75,0.15)/{06} , (3.75,0.15)/{14} , (4.75,0.15)/{10} }
  \draw \loc node{ \color{blue!90!black} \footnotesize { $\txt$ }};
  
  % Draw the lines
  \draw
  % Finish drawing the grid
  [step=1.0cm,black,thin] (0,0) grid (5.0,5.0) % The Grid
  (0.0,5.0) -- (1.0,4.0) % Diagonal in the top left cell
  (1.0,4.05) -- (5.0,4.05) % Double line under top header row
  (0.95,0.0) -- (0.95,4.0) % Double line on left of header column one
  ;    
  \end{tikzpicture}
\end{figure}

The expression for row two is: $ A'B'C'D + A'BC'D + ABC'D + AB'C'D $. The term $ C'D $ is constant in this group, while $ A $ and $ B $ change, so this one line would simplify to $ C'D $. The expression for row three is: $ A'B'CD + A'BCD + ABCD + AB'CD $. The term $ CD $ is constant in this group, so this one line would simplify to $ CD $. Then, if the two rows are combined: $ C'D + CD $, $ C $ and $ C' $ are dropped by the complement property and the circuit simplifies to $ D $. To put this another way, since the only term in this group of eight that never changes is $ D $, then Equation \ref{KM:eq:simplifying_4-input_equations_groups_of_eight_solution} is the simplified solution.

\begin{align}
  \label{KM:eq:simplifying_4-input_equations_groups_of_eight_solution}
  D = Y
\end{align}

This Karnaugh map also provides a good example of prime implicants that are not essential. Consider row two of the map. Minterms $ 01 $ and $ 05 $ form a Prime Implicant for this circuit since it is a group of two; however, that group was subsumed by the group of eight that was formed with the next row. Since every cell in the group of two is also present in the group of eight, then the group of two is not essential to the final circuit simplification. While this may seem to be rather obvious, it is important to remember that frequently implicants are formed that are not essential and they can be ignored. This concept will come up again when using the Quine-McCluskey Simplification method on page \pageref{ASM:sec:quine-mccluskey_simplification_method}. 

\subsection{Groups of Four}
\label{KM:subsec:groups_of_4}

Groups of four can form as a single row, a single column, or a square. In any case, the four cells will simplify to a two-variable expression. Consider the Karnaugh map in Figure \ref{KM:fig:kmap_solving_groups_of_four_ex_1} 

%****************************************************************
% Karnaugh map For Groups of 4, Example 1
%****************************************************************
\begin{figure}[H]
  \caption{K-Map Solving Groups of Four, Example 1}
  \label{KM:fig:kmap_solving_groups_of_four_ex_1}
  \myfloatalign
  \begin{tikzpicture} [circuit logic US, scale=1.00]
  % make all path lines (the node shapes) a little thicker
  \tikzstyle{every path}=[line width=0.50mm]
  
  %********************************************************************
  % Adjust the settings below to display the 1's and rectangles
  %********************************************************************
  % Uncomment the appropriate lines below to insert ones where needed
  %  \node[] at (1.4,3.5) {\huge $ 1 $}; % 00
    \node[] at (1.4,2.5) {\huge $ 1 $}; % 01
  %  \node[] at (1.4,0.5) {\huge $ 1 $}; % 02
  %  \node[] at (1.4,1.5) {\huge $ 1 $}; % 03
  %  \node[] at (2.4,3.5) {\huge $ 1 $}; % 04
    \node[] at (2.4,2.5) {\huge $ 1 $}; % 05
  %  \node[] at (2.4,0.5) {\huge $ 1 $}; % 06
  %  \node[] at (2.4,1.5) {\huge $ 1 $}; % 07
  %  \node[] at (4.4,3.5) {\huge $ 1 $}; % 08
    \node[] at (4.4,2.5) {\huge $ 1 $}; % 09
  %  \node[] at (4.4,0.5) {\huge $ 1 $}; % 10
  %  \node[] at (4.4,1.5) {\huge $ 1 $}; % 11
  %  \node[] at (3.4,3.5) {\huge $ 1 $}; % 12
    \node[] at (3.4,2.5) {\huge $ 1 $}; % 13
  %  \node[] at (3.4,0.5) {\huge $ 1 $}; % 14
  %  \node[] at (3.4,1.5) {\huge $ 1 $}; % 15
  
  % The coords for each cell - this is used as the origin for the solution box
  \coordinate (cell00) at (1.0,3.0); \coordinate (cell01) at (1.0,2.0);
  \coordinate (cell02) at (1.0,0.0); \coordinate (cell03) at (1.0,1.0);
  
  \coordinate (cell04) at (2.0,3.0); \coordinate (cell05) at (2.0,2.0);
  \coordinate (cell06) at (2.0,0.0); \coordinate (cell07) at (2.0,1.0);
  
  \coordinate (cell12) at (3.0,3.0); \coordinate (cell13) at (3.0,2.0);
  \coordinate (cell14) at (3.0,0.0); \coordinate (cell15) at (3.0,1.0);
  
  \coordinate (cell08) at (4.0,3.0); \coordinate (cell09) at (4.0,2.0);
  \coordinate (cell10) at (4.0,0.0); \coordinate (cell11) at (4.0,1.0);
  
  % Horizontal Group
    \node [draw,
    color=red!70!black,
    fill=red!20!white,
    fill opacity=0.3,
    minimum height=0.95cm,
    minimum width=3.95cm, % Adjust to (number of cells) * 1 - 0.95
    rounded corners,
    anchor=south west] at (cell01) {}; % Enter left cell minterm
  
  % Vertical Group
  %  \node [draw,
  %  color=blue!70!black,
  %  fill=blue!20!white,
  %  fill opacity=0.3,
  %  minimum height=1.95cm, % Adjust to (number of cells) * 1 - 0.95
  %  minimum width=0.95cm,
  %  rounded corners,
  %  anchor=south west] at (cell07) {}; % Enter bottom cell minterm
  
  % Single Cell
  %  \node [draw,
  %  color=green!70!black,
  %  fill=green!20!white,
  %  fill opacity=0.3,
  %  minimum height=0.95cm,
  %  minimum width=0.95cm,
  %  rounded corners,
  %  anchor=south west] at (cell09) {}; % Enter cell minterm
  
  %********************************************************************
  % Shouldn't need to adjust anything below this point - this is just
  % the grid and the minterms.
  %********************************************************************  
  % Text in top-Left cell
  \node[] at (0.35,4.22) { \footnotesize $ \mathsf{ CD } $ }; % cd
  \node[] at (0.70,4.75) { \footnotesize $ \mathsf{ AB } $ }; % ab
  
  % Populate the top row header
  % In the following, the foreach lists a location/text pair
  % The the draw line draws the text at each location
  \foreach \loc/\txt in {
    (1.5,4.5)/{00},(2.5,4.5)/{01},(3.5,4.5)/{11},(4.5,4.5)/{10}
  }
  \draw \loc node{\large $\txt$};
  
  % Populate the header in column one
  \foreach \loc/\txt in { 
    (0.5,3.5)/{00},(0.5,2.5)/{01},(0.5,1.5)/{11},(0.5,0.5)/{10}
  }
  \draw \loc node{\large $\txt$};
  
  % Populate the minterms
  \foreach \loc/\txt in { 
    (1.75,3.15)/{00} , (2.75,3.15)/{04} , (3.75,3.15)/{12} , (4.75,3.15)/{08} ,
    (1.75,2.15)/{01} , (2.75,2.15)/{05} , (3.75,2.15)/{13} , (4.75,2.15)/{09} ,
    (1.75,1.15)/{03} , (2.75,1.15)/{07} , (3.75,1.15)/{15} , (4.75,1.15)/{11} ,
    (1.75,0.15)/{02} , (2.75,0.15)/{06} , (3.75,0.15)/{14} , (4.75,0.15)/{10} }
  \draw \loc node{ \color{blue!90!black} \footnotesize { $\txt$ }};
  
  % Draw the lines
  \draw
  % Finish drawing the grid
  [step=1.0cm,black,thin] (0,0) grid (5.0,5.0) % The Grid
  (0.0,5.0) -- (1.0,4.0) % Diagonal in the top left cell
  (1.0,4.05) -- (5.0,4.05) % Double line under top header row
  (0.95,0.0) -- (0.95,4.0) % Double line on left of header column one
  ;    
  \end{tikzpicture}
\end{figure}

Since the $ A $ and $ B $ variables can be removed due to the Complement Property, Equation \ref{KM:eq:solving_groups_of_four_ex_1} shows the simplified solution. 

\begin{align}
  \label{KM:eq:solving_groups_of_four_ex_1}
  C'D = Y
\end{align}

The Karnaugh map in Figure \ref{KM:fig:kmap_solving_groups_of_four_ex_2} is a second example.

%****************************************************************
% Karnaugh map For Groups of 4, Example 2
%****************************************************************
\begin{figure}[H]
  \caption{K-Map Solving Groups of Four, Example 2}
  \label{KM:fig:kmap_solving_groups_of_four_ex_2}
  \myfloatalign
  \begin{tikzpicture} [circuit logic US, scale=1.00]
  % make all path lines (the node shapes) a little thicker
  \tikzstyle{every path}=[line width=0.50mm]
  
  %********************************************************************
  % Adjust the settings below to display the 1's and rectangles
  %********************************************************************
  % Uncomment the appropriate lines below to insert ones where needed
  %  \node[] at (1.4,3.5) {\huge $ 1 $}; % 00
  %  \node[] at (1.4,2.5) {\huge $ 1 $}; % 01
  %  \node[] at (1.4,0.5) {\huge $ 1 $}; % 02
  %  \node[] at (1.4,1.5) {\huge $ 1 $}; % 03
  %  \node[] at (2.4,3.5) {\huge $ 1 $}; % 04
  %  \node[] at (2.4,2.5) {\huge $ 1 $}; % 05
  %  \node[] at (2.4,0.5) {\huge $ 1 $}; % 06
  %  \node[] at (2.4,1.5) {\huge $ 1 $}; % 07
  %  \node[] at (4.4,3.5) {\huge $ 1 $}; % 08
  %  \node[] at (4.4,2.5) {\huge $ 1 $}; % 09
  %  \node[] at (4.4,0.5) {\huge $ 1 $}; % 10
  %  \node[] at (4.4,1.5) {\huge $ 1 $}; % 11
    \node[] at (3.4,3.5) {\huge $ 1 $}; % 12
    \node[] at (3.4,2.5) {\huge $ 1 $}; % 13
    \node[] at (3.4,0.5) {\huge $ 1 $}; % 14
    \node[] at (3.4,1.5) {\huge $ 1 $}; % 15
  
  % The coords for each cell - this is used as the origin for the solution box
  \coordinate (cell00) at (1.0,3.0); \coordinate (cell01) at (1.0,2.0);
  \coordinate (cell02) at (1.0,0.0); \coordinate (cell03) at (1.0,1.0);
  
  \coordinate (cell04) at (2.0,3.0); \coordinate (cell05) at (2.0,2.0);
  \coordinate (cell06) at (2.0,0.0); \coordinate (cell07) at (2.0,1.0);
  
  \coordinate (cell12) at (3.0,3.0); \coordinate (cell13) at (3.0,2.0);
  \coordinate (cell14) at (3.0,0.0); \coordinate (cell15) at (3.0,1.0);
  
  \coordinate (cell08) at (4.0,3.0); \coordinate (cell09) at (4.0,2.0);
  \coordinate (cell10) at (4.0,0.0); \coordinate (cell11) at (4.0,1.0);
  
  % Horizontal Group
  %  \node [draw,
  %  color=red!70!black,
  %  fill=red!20!white,
  %  fill opacity=0.3,
  %  minimum height=0.95cm,
  %  minimum width=1.95cm, % Adjust to (number of cells) * 1 - 0.95
  %  rounded corners,
  %  anchor=south west] at (cell02) {}; % Enter left cell minterm
  
  % Vertical Group
    \node [draw,
    color=blue!70!black,
    fill=blue!20!white,
    fill opacity=0.3,
    minimum height=3.95cm, % Adjust to (number of cells) * 1 - 0.95
    minimum width=0.95cm,
    rounded corners,
    anchor=south west] at (cell14) {}; % Enter bottom cell minterm
  
  % Single Cell
  %  \node [draw,
  %  color=green!70!black,
  %  fill=green!20!white,
  %  fill opacity=0.3,
  %  minimum height=0.95cm,
  %  minimum width=0.95cm,
  %  rounded corners,
  %  anchor=south west] at (cell09) {}; % Enter cell minterm
  
  %********************************************************************
  % Shouldn't need to adjust anything below this point - this is just
  % the grid and the minterms.
  %********************************************************************  
  % Text in top-Left cell
  \node[] at (0.35,4.22) { \footnotesize $ \mathsf{ CD } $ }; % cd
  \node[] at (0.70,4.75) { \footnotesize $ \mathsf{ AB } $ }; % ab
  
  % Populate the top row header
  % In the following, the foreach lists a location/text pair
  % The the draw line draws the text at each location
  \foreach \loc/\txt in {
    (1.5,4.5)/{00},(2.5,4.5)/{01},(3.5,4.5)/{11},(4.5,4.5)/{10}
  }
  \draw \loc node{\large $\txt$};
  
  % Populate the header in column one
  \foreach \loc/\txt in { 
    (0.5,3.5)/{00},(0.5,2.5)/{01},(0.5,1.5)/{11},(0.5,0.5)/{10}
  }
  \draw \loc node{\large $\txt$};
  
  % Populate the minterms
  \foreach \loc/\txt in { 
    (1.75,3.15)/{00} , (2.75,3.15)/{04} , (3.75,3.15)/{12} , (4.75,3.15)/{08} ,
    (1.75,2.15)/{01} , (2.75,2.15)/{05} , (3.75,2.15)/{13} , (4.75,2.15)/{09} ,
    (1.75,1.15)/{03} , (2.75,1.15)/{07} , (3.75,1.15)/{15} , (4.75,1.15)/{11} ,
    (1.75,0.15)/{02} , (2.75,0.15)/{06} , (3.75,0.15)/{14} , (4.75,0.15)/{10} }
  \draw \loc node{ \color{blue!90!black} \footnotesize { $\txt$ }};
  
  % Draw the lines
  \draw
  % Finish drawing the grid
  [step=1.0cm,black,thin] (0,0) grid (5.0,5.0) % The Grid
  (0.0,5.0) -- (1.0,4.0) % Diagonal in the top left cell
  (1.0,4.05) -- (5.0,4.05) % Double line under top header row
  (0.95,0.0) -- (0.95,4.0) % Double line on left of header column one
  ;    
  \end{tikzpicture}
\end{figure}

Since the $ C $ and $ D $ variables can be removed due to the Complement Property, Equation \ref{KM:eq:simplifying_4-input_equations_groups_of_four_ex_2_solution} shows the simplified circuit. 

\begin{align}
  \label{KM:eq:simplifying_4-input_equations_groups_of_four_ex_2_solution}
  AB = Y
\end{align}

The Karnaugh map in Figure \ref{KM:fig:kmap_solving_groups_of_four_ex_3} is an example of a group of four that forms a square.

%****************************************************************
% Karnaugh map For Groups of 4, Example 3
%****************************************************************
\begin{figure}[H]
  \caption{K-Map Solving Groups of Four, Example 3}
  \label{KM:fig:kmap_solving_groups_of_four_ex_3}
  \myfloatalign
  \begin{tikzpicture} [circuit logic US, scale=1.00]
  % make all path lines (the node shapes) a little thicker
  \tikzstyle{every path}=[line width=0.50mm]
  
  %********************************************************************
  % Adjust the settings below to display the 1's and rectangles
  %********************************************************************
  % Uncomment the appropriate lines below to insert ones where needed
  %  \node[] at (1.4,3.5) {\huge $ 1 $}; % 00
    \node[] at (1.4,2.5) {\huge $ 1 $}; % 01
  %  \node[] at (1.4,0.5) {\huge $ 1 $}; % 02
    \node[] at (1.4,1.5) {\huge $ 1 $}; % 03
  %  \node[] at (2.4,3.5) {\huge $ 1 $}; % 04
    \node[] at (2.4,2.5) {\huge $ 1 $}; % 05
  %  \node[] at (2.4,0.5) {\huge $ 1 $}; % 06
    \node[] at (2.4,1.5) {\huge $ 1 $}; % 07
  %  \node[] at (4.4,3.5) {\huge $ 1 $}; % 08
  %  \node[] at (4.4,2.5) {\huge $ 1 $}; % 09
  %  \node[] at (4.4,0.5) {\huge $ 1 $}; % 10
  %  \node[] at (4.4,1.5) {\huge $ 1 $}; % 11
  %  \node[] at (3.4,3.5) {\huge $ 1 $}; % 12
  %  \node[] at (3.4,2.5) {\huge $ 1 $}; % 13
  %  \node[] at (3.4,0.5) {\huge $ 1 $}; % 14
  %  \node[] at (3.4,1.5) {\huge $ 1 $}; % 15
  
  % The coords for each cell - this is used as the origin for the solution box
  \coordinate (cell00) at (1.0,3.0); \coordinate (cell01) at (1.0,2.0);
  \coordinate (cell02) at (1.0,0.0); \coordinate (cell03) at (1.0,1.0);
  
  \coordinate (cell04) at (2.0,3.0); \coordinate (cell05) at (2.0,2.0);
  \coordinate (cell06) at (2.0,0.0); \coordinate (cell07) at (2.0,1.0);
  
  \coordinate (cell12) at (3.0,3.0); \coordinate (cell13) at (3.0,2.0);
  \coordinate (cell14) at (3.0,0.0); \coordinate (cell15) at (3.0,1.0);
  
  \coordinate (cell08) at (4.0,3.0); \coordinate (cell09) at (4.0,2.0);
  \coordinate (cell10) at (4.0,0.0); \coordinate (cell11) at (4.0,1.0);
  
  % Horizontal Group
  %  \node [draw,
  %  color=red!70!black,
  %  fill=red!20!white,
  %  fill opacity=0.3,
  %  minimum height=0.95cm,
  %  minimum width=1.95cm, % Adjust to (number of cells) * 1 - 0.95
  %  rounded corners,
  %  anchor=south west] at (cell02) {}; % Enter left cell minterm
  
  % Vertical Group
  %  \node [draw,
  %  color=blue!70!black,
  %  fill=blue!20!white,
  %  fill opacity=0.3,
  %  minimum height=1.95cm, % Adjust to (number of cells) * 1 - 0.95
  %  minimum width=0.95cm,
  %  rounded corners,
  %  anchor=south west] at (cell07) {}; % Enter bottom cell minterm
  
  % Single Cell
    \node [draw,
    color=green!70!black,
    fill=green!20!white,
    fill opacity=0.3,
    minimum height=1.95cm,
    minimum width=1.95cm,
    rounded corners,
    anchor=south west] at (cell03) {}; % Enter cell minterm
  
  %********************************************************************
  % Shouldn't need to adjust anything below this point - this is just
  % the grid and the minterms.
  %********************************************************************  
  % Text in top-Left cell
  \node[] at (0.35,4.22) { \footnotesize $ \mathsf{ CD } $ }; % cd
  \node[] at (0.70,4.75) { \footnotesize $ \mathsf{ AB } $ }; % ab
  
  % Populate the top row header
  % In the following, the foreach lists a location/text pair
  % The the draw line draws the text at each location
  \foreach \loc/\txt in {
    (1.5,4.5)/{00},(2.5,4.5)/{01},(3.5,4.5)/{11},(4.5,4.5)/{10}
  }
  \draw \loc node{\large $\txt$};
  
  % Populate the header in column one
  \foreach \loc/\txt in { 
    (0.5,3.5)/{00},(0.5,2.5)/{01},(0.5,1.5)/{11},(0.5,0.5)/{10}
  }
  \draw \loc node{\large $\txt$};
  
  % Populate the minterms
  \foreach \loc/\txt in { 
    (1.75,3.15)/{00} , (2.75,3.15)/{04} , (3.75,3.15)/{12} , (4.75,3.15)/{08} ,
    (1.75,2.15)/{01} , (2.75,2.15)/{05} , (3.75,2.15)/{13} , (4.75,2.15)/{09} ,
    (1.75,1.15)/{03} , (2.75,1.15)/{07} , (3.75,1.15)/{15} , (4.75,1.15)/{11} ,
    (1.75,0.15)/{02} , (2.75,0.15)/{06} , (3.75,0.15)/{14} , (4.75,0.15)/{10} }
  \draw \loc node{ \color{blue!90!black} \footnotesize { $\txt$ }};
  
  % Draw the lines
  \draw
  % Finish drawing the grid
  [step=1.0cm,black,thin] (0,0) grid (5.0,5.0) % The Grid
  (0.0,5.0) -- (1.0,4.0) % Diagonal in the top left cell
  (1.0,4.05) -- (5.0,4.05) % Double line under top header row
  (0.95,0.0) -- (0.95,4.0) % Double line on left of header column one
  ;    
  \end{tikzpicture}
\end{figure}

Since the $ B $ and $ C $ variables can be removed due to the Complement Property, Equation \ref{KM:eq:simplifying_4-input_equations_groups_of_four_ex_3_solution} shows the simplified circuit. 

\begin{align}
  \label{KM:eq:simplifying_4-input_equations_groups_of_four_ex_3_solution}
  A'D = Y
\end{align}

\subsection{Groups of Two}
\label{KM:subsec:groups_of_2}

Groups of two will simplify to a three-variable expression. The Karnaugh map in Figure \ref{KM:fig:kmap_solving_groups_of_two_ex_1} is one example of a group of two. 

%****************************************************************
% Karnaugh map For Groups of 2, Example 1
%****************************************************************
\begin{figure}[H]
  \caption{K-Map Solving Groups of Two, Example 1}
  \label{KM:fig:kmap_solving_groups_of_two_ex_1}
  \myfloatalign
  \begin{tikzpicture} [circuit logic US, scale=1.00]
  % make all path lines (the node shapes) a little thicker
  \tikzstyle{every path}=[line width=0.50mm]
  
  %********************************************************************
  % Adjust the settings below to display the 1's and rectangles
  %********************************************************************
  % Uncomment the appropriate lines below to insert ones where needed
  %  \node[] at (1.4,3.5) {\huge $ 1 $}; % 00
  %  \node[] at (1.4,2.5) {\huge $ 1 $}; % 01
  %  \node[] at (1.4,0.5) {\huge $ 1 $}; % 02
  %  \node[] at (1.4,1.5) {\huge $ 1 $}; % 03
  %  \node[] at (2.4,3.5) {\huge $ 1 $}; % 04
  %  \node[] at (2.4,2.5) {\huge $ 1 $}; % 05
  %  \node[] at (2.4,0.5) {\huge $ 1 $}; % 06
  %  \node[] at (2.4,1.5) {\huge $ 1 $}; % 07
  %  \node[] at (4.4,3.5) {\huge $ 1 $}; % 08
    \node[] at (4.4,2.5) {\huge $ 1 $}; % 09
  %  \node[] at (4.4,0.5) {\huge $ 1 $}; % 10
    \node[] at (4.4,1.5) {\huge $ 1 $}; % 11
  %  \node[] at (3.4,3.5) {\huge $ 1 $}; % 12
  %  \node[] at (3.4,2.5) {\huge $ 1 $}; % 13
  %  \node[] at (3.4,0.5) {\huge $ 1 $}; % 14
  %  \node[] at (3.4,1.5) {\huge $ 1 $}; % 15
  
  % The coords for each cell - this is used as the origin for the solution box
  \coordinate (cell00) at (1.0,3.0); \coordinate (cell01) at (1.0,2.0);
  \coordinate (cell02) at (1.0,0.0); \coordinate (cell03) at (1.0,1.0);
  
  \coordinate (cell04) at (2.0,3.0); \coordinate (cell05) at (2.0,2.0);
  \coordinate (cell06) at (2.0,0.0); \coordinate (cell07) at (2.0,1.0);
  
  \coordinate (cell12) at (3.0,3.0); \coordinate (cell13) at (3.0,2.0);
  \coordinate (cell14) at (3.0,0.0); \coordinate (cell15) at (3.0,1.0);
  
  \coordinate (cell08) at (4.0,3.0); \coordinate (cell09) at (4.0,2.0);
  \coordinate (cell10) at (4.0,0.0); \coordinate (cell11) at (4.0,1.0);
  
  % Horizontal Group
  %  \node [draw,
  %  color=red!70!black,
  %  fill=red!20!white,
  %  fill opacity=0.3,
  %  minimum height=0.95cm,
  %  minimum width=1.95cm, % Adjust to (number of cells) * 1 - 0.95
  %  rounded corners,
  %  anchor=south west] at (cell02) {}; % Enter left cell minterm
  
  % Vertical Group
    \node [draw,
    color=blue!70!black,
    fill=blue!20!white,
    fill opacity=0.3,
    minimum height=1.95cm, % Adjust to (number of cells) * 1 - 0.95
    minimum width=0.95cm,
    rounded corners,
    anchor=south west] at (cell11) {}; % Enter bottom cell minterm
  
  % Single Cell
  %  \node [draw,
  %  color=green!70!black,
  %  fill=green!20!white,
  %  fill opacity=0.3,
  %  minimum height=0.95cm,
  %  minimum width=0.95cm,
  %  rounded corners,
  %  anchor=south west] at (cell09) {}; % Enter cell minterm
  
  %********************************************************************
  % Shouldn't need to adjust anything below this point - this is just
  % the grid and the minterms.
  %********************************************************************  
  % Text in top-Left cell
  \node[] at (0.35,4.22) { \footnotesize $ \mathsf{ CD } $ }; % cd
  \node[] at (0.70,4.75) { \footnotesize $ \mathsf{ AB } $ }; % ab
  
  % Populate the top row header
  % In the following, the foreach lists a location/text pair
  % The the draw line draws the text at each location
  \foreach \loc/\txt in {
    (1.5,4.5)/{00},(2.5,4.5)/{01},(3.5,4.5)/{11},(4.5,4.5)/{10}
  }
  \draw \loc node{\large $\txt$};
  
  % Populate the header in column one
  \foreach \loc/\txt in { 
    (0.5,3.5)/{00},(0.5,2.5)/{01},(0.5,1.5)/{11},(0.5,0.5)/{10}
  }
  \draw \loc node{\large $\txt$};
  
  % Populate the minterms
  \foreach \loc/\txt in { 
    (1.75,3.15)/{00} , (2.75,3.15)/{04} , (3.75,3.15)/{12} , (4.75,3.15)/{08} ,
    (1.75,2.15)/{01} , (2.75,2.15)/{05} , (3.75,2.15)/{13} , (4.75,2.15)/{09} ,
    (1.75,1.15)/{03} , (2.75,1.15)/{07} , (3.75,1.15)/{15} , (4.75,1.15)/{11} ,
    (1.75,0.15)/{02} , (2.75,0.15)/{06} , (3.75,0.15)/{14} , (4.75,0.15)/{10} }
  \draw \loc node{ \color{blue!90!black} \footnotesize { $\txt$ }};
  
  % Draw the lines
  \draw
  % Finish drawing the grid
  [step=1.0cm,black,thin] (0,0) grid (5.0,5.0) % The Grid
  (0.0,5.0) -- (1.0,4.0) % Diagonal in the top left cell
  (1.0,4.05) -- (5.0,4.05) % Double line under top header row
  (0.95,0.0) -- (0.95,4.0) % Double line on left of header column one
  ;    
  \end{tikzpicture}
\end{figure}

Since $ C $ is the only variable that can be removed due to the Complement Property, the above circuit simplifies to Equation \ref{KM:eq:simplifying_4-input_equations_groups_of_two_ex_1_solution}.

\begin{align}
  \label{KM:eq:simplifying_4-input_equations_groups_of_two_ex_1_solution}
  AB'D = Y
\end{align}

As a second example, consider the Karnaugh map in Figure \ref{KM:fig:kmap_solving_groups_of_two_ex_2}.

%****************************************************************
% Karnaugh map For Groups of 2, Example 2
%****************************************************************
\begin{figure}[H]
  \caption{K-Map Solving Groups of Two, Example 2}
  \label{KM:fig:kmap_solving_groups_of_two_ex_2}
  \myfloatalign
  \begin{tikzpicture} [circuit logic US, scale=1.00]
  % make all path lines (the node shapes) a little thicker
  \tikzstyle{every path}=[line width=0.50mm]
  
  %********************************************************************
  % Adjust the settings below to display the 1's and rectangles
  %********************************************************************
  % Uncomment the appropriate lines below to insert ones where needed
  %  \node[] at (1.4,3.5) {\huge $ 1 $}; % 00
  %  \node[] at (1.4,2.5) {\huge $ 1 $}; % 01
  %  \node[] at (1.4,0.5) {\huge $ 1 $}; % 02
  %  \node[] at (1.4,1.5) {\huge $ 1 $}; % 03
  %  \node[] at (2.4,3.5) {\huge $ 1 $}; % 04
  %  \node[] at (2.4,2.5) {\huge $ 1 $}; % 05
  %  \node[] at (2.4,0.5) {\huge $ 1 $}; % 06
    \node[] at (2.4,1.5) {\huge $ 1 $}; % 07
  %  \node[] at (4.4,3.5) {\huge $ 1 $}; % 08
  %  \node[] at (4.4,2.5) {\huge $ 1 $}; % 09
  %  \node[] at (4.4,0.5) {\huge $ 1 $}; % 10
  %  \node[] at (4.4,1.5) {\huge $ 1 $}; % 11
  %  \node[] at (3.4,3.5) {\huge $ 1 $}; % 12
  %  \node[] at (3.4,2.5) {\huge $ 1 $}; % 13
  %  \node[] at (3.4,0.5) {\huge $ 1 $}; % 14
    \node[] at (3.4,1.5) {\huge $ 1 $}; % 15
  
  % The coords for each cell - this is used as the origin for the solution box
  \coordinate (cell00) at (1.0,3.0); \coordinate (cell01) at (1.0,2.0);
  \coordinate (cell02) at (1.0,0.0); \coordinate (cell03) at (1.0,1.0);
  
  \coordinate (cell04) at (2.0,3.0); \coordinate (cell05) at (2.0,2.0);
  \coordinate (cell06) at (2.0,0.0); \coordinate (cell07) at (2.0,1.0);
  
  \coordinate (cell12) at (3.0,3.0); \coordinate (cell13) at (3.0,2.0);
  \coordinate (cell14) at (3.0,0.0); \coordinate (cell15) at (3.0,1.0);
  
  \coordinate (cell08) at (4.0,3.0); \coordinate (cell09) at (4.0,2.0);
  \coordinate (cell10) at (4.0,0.0); \coordinate (cell11) at (4.0,1.0);
  
  % Horizontal Group
    \node [draw,
    color=red!70!black,
    fill=red!20!white,
    fill opacity=0.3,
    minimum height=0.95cm,
    minimum width=1.95cm, % Adjust to (number of cells) * 1 - 0.95
    rounded corners,
    anchor=south west] at (cell07) {}; % Enter left cell minterm
  
  % Vertical Group
  %  \node [draw,
  %  color=blue!70!black,
  %  fill=blue!20!white,
  %  fill opacity=0.3,
  %  minimum height=1.95cm, % Adjust to (number of cells) * 1 - 0.95
  %  minimum width=0.95cm,
  %  rounded corners,
  %  anchor=south west] at (cell07) {}; % Enter bottom cell minterm
  
  % Single Cell
  %  \node [draw,
  %  color=green!70!black,
  %  fill=green!20!white,
  %  fill opacity=0.3,
  %  minimum height=0.95cm,
  %  minimum width=0.95cm,
  %  rounded corners,
  %  anchor=south west] at (cell09) {}; % Enter cell minterm
  
  %********************************************************************
  % Shouldn't need to adjust anything below this point - this is just
  % the grid and the minterms.
  %********************************************************************  
  % Text in top-Left cell
  \node[] at (0.35,4.22) { \footnotesize $ \mathsf{ CD } $ }; % cd
  \node[] at (0.70,4.75) { \footnotesize $ \mathsf{ AB } $ }; % ab
  
  % Populate the top row header
  % In the following, the foreach lists a location/text pair
  % The the draw line draws the text at each location
  \foreach \loc/\txt in {
    (1.5,4.5)/{00},(2.5,4.5)/{01},(3.5,4.5)/{11},(4.5,4.5)/{10}
  }
  \draw \loc node{\large $\txt$};
  
  % Populate the header in column one
  \foreach \loc/\txt in { 
    (0.5,3.5)/{00},(0.5,2.5)/{01},(0.5,1.5)/{11},(0.5,0.5)/{10}
  }
  \draw \loc node{\large $\txt$};
  
  % Populate the minterms
  \foreach \loc/\txt in { 
    (1.75,3.15)/{00} , (2.75,3.15)/{04} , (3.75,3.15)/{12} , (4.75,3.15)/{08} ,
    (1.75,2.15)/{01} , (2.75,2.15)/{05} , (3.75,2.15)/{13} , (4.75,2.15)/{09} ,
    (1.75,1.15)/{03} , (2.75,1.15)/{07} , (3.75,1.15)/{15} , (4.75,1.15)/{11} ,
    (1.75,0.15)/{02} , (2.75,0.15)/{06} , (3.75,0.15)/{14} , (4.75,0.15)/{10} }
  \draw \loc node{ \color{blue!90!black} \footnotesize { $\txt$ }};
  
  % Draw the lines
  \draw
  % Finish drawing the grid
  [step=1.0cm,black,thin] (0,0) grid (5.0,5.0) % The Grid
  (0.0,5.0) -- (1.0,4.0) % Diagonal in the top left cell
  (1.0,4.05) -- (5.0,4.05) % Double line under top header row
  (0.95,0.0) -- (0.95,4.0) % Double line on left of header column one
  ;    
  \end{tikzpicture}
\end{figure}

Equation \ref{KM:eq:simplifying_4-input_equations_groups_of_two_ex_2_solution} is the simplified equation for the Karnaugh map in \ref{KM:fig:kmap_solving_groups_of_two_ex_2}. 

\begin{align}
  \label{KM:eq:simplifying_4-input_equations_groups_of_two_ex_2_solution}
  BCD = Y
\end{align}

\section{Overlapping Groups}
\label{KM:sec:overlapping_groups}

Frequently, groups overlap to create numerous patterns on the Karnaugh map. Consider the following two examples. 

%****************************************************************
% Karnaugh map For Overlapping Groups, Example 1
%****************************************************************
\begin{figure}[H]
  \caption{K-Map Overlapping Groups, Example 1}
  \label{KM:fig:kmap_overlapping_groups_ex_1}
  \myfloatalign
  \begin{tikzpicture} [circuit logic US, scale=1.00]
  % make all path lines (the node shapes) a little thicker
  \tikzstyle{every path}=[line width=0.50mm]
  
  %********************************************************************
  % Adjust the settings below to display the 1's and rectangles
  %********************************************************************
  % Uncomment the appropriate lines below to insert ones where needed
  %  \node[] at (1.4,3.5) {\huge $ 1 $}; % 00
  %  \node[] at (1.4,2.5) {\huge $ 1 $}; % 01
  %  \node[] at (1.4,0.5) {\huge $ 1 $}; % 02
  %  \node[] at (1.4,1.5) {\huge $ 1 $}; % 03
  %  \node[] at (2.4,3.5) {\huge $ 1 $}; % 04
    \node[] at (2.4,2.5) {\huge $ 1 $}; % 05
  %  \node[] at (2.4,0.5) {\huge $ 1 $}; % 06
    \node[] at (2.4,1.5) {\huge $ 1 $}; % 07
  %  \node[] at (4.4,3.5) {\huge $ 1 $}; % 08
  %  \node[] at (4.4,2.5) {\huge $ 1 $}; % 09
  %  \node[] at (4.4,0.5) {\huge $ 1 $}; % 10
  %  \node[] at (4.4,1.5) {\huge $ 1 $}; % 11
  %  \node[] at (3.4,3.5) {\huge $ 1 $}; % 12
  %  \node[] at (3.4,2.5) {\huge $ 1 $}; % 13
  %  \node[] at (3.4,0.5) {\huge $ 1 $}; % 14
    \node[] at (3.4,1.5) {\huge $ 1 $}; % 15
  
  % The coords for each cell - this is used as the origin for the solution box
  \coordinate (cell00) at (1.0,3.0); \coordinate (cell01) at (1.0,2.0);
  \coordinate (cell02) at (1.0,0.0); \coordinate (cell03) at (1.0,1.0);
  
  \coordinate (cell04) at (2.0,3.0); \coordinate (cell05) at (2.0,2.0);
  \coordinate (cell06) at (2.0,0.0); \coordinate (cell07) at (2.0,1.0);
  
  \coordinate (cell12) at (3.0,3.0); \coordinate (cell13) at (3.0,2.0);
  \coordinate (cell14) at (3.0,0.0); \coordinate (cell15) at (3.0,1.0);
  
  \coordinate (cell08) at (4.0,3.0); \coordinate (cell09) at (4.0,2.0);
  \coordinate (cell10) at (4.0,0.0); \coordinate (cell11) at (4.0,1.0);
  
  % Horizontal Group
    \node [draw,
    color=red!70!black,
    fill=red!20!white,
    fill opacity=0.3,
    minimum height=0.95cm,
    minimum width=1.95cm, % Adjust to (number of cells) * 1 - 0.95
    rounded corners,
    anchor=south west] at (cell07) {}; % Enter left cell minterm
  
  % Vertical Group
    \node [draw,
    color=blue!70!black,
    fill=blue!20!white,
    fill opacity=0.3,
    minimum height=1.95cm, % Adjust to (number of cells) * 1 - 0.95
    minimum width=0.95cm,
    rounded corners,
    anchor=south west] at (cell07) {}; % Enter bottom cell minterm
  
  % Single Cell
  %  \node [draw,
  %  color=green!70!black,
  %  fill=green!20!white,
  %  fill opacity=0.3,
  %  minimum height=0.95cm,
  %  minimum width=0.95cm,
  %  rounded corners,
  %  anchor=south west] at (cell09) {}; % Enter cell minterm
  
  %********************************************************************
  % Shouldn't need to adjust anything below this point - this is just
  % the grid and the minterms.
  %********************************************************************  
  % Text in top-Left cell
  \node[] at (0.35,4.22) { \footnotesize $ \mathsf{ CD } $ }; % cd
  \node[] at (0.70,4.75) { \footnotesize $ \mathsf{ AB } $ }; % ab
  
  % Populate the top row header
  % In the following, the foreach lists a location/text pair
  % The the draw line draws the text at each location
  \foreach \loc/\txt in {
    (1.5,4.5)/{00},(2.5,4.5)/{01},(3.5,4.5)/{11},(4.5,4.5)/{10}
  }
  \draw \loc node{\large $\txt$};
  
  % Populate the header in column one
  \foreach \loc/\txt in { 
    (0.5,3.5)/{00},(0.5,2.5)/{01},(0.5,1.5)/{11},(0.5,0.5)/{10}
  }
  \draw \loc node{\large $\txt$};
  
  % Populate the minterms
  \foreach \loc/\txt in { 
    (1.75,3.15)/{00} , (2.75,3.15)/{04} , (3.75,3.15)/{12} , (4.75,3.15)/{08} ,
    (1.75,2.15)/{01} , (2.75,2.15)/{05} , (3.75,2.15)/{13} , (4.75,2.15)/{09} ,
    (1.75,1.15)/{03} , (2.75,1.15)/{07} , (3.75,1.15)/{15} , (4.75,1.15)/{11} ,
    (1.75,0.15)/{02} , (2.75,0.15)/{06} , (3.75,0.15)/{14} , (4.75,0.15)/{10} }
  \draw \loc node{ \color{blue!90!black} \footnotesize { $\txt$ }};
  
  % Draw the lines
  \draw
  % Finish drawing the grid
  [step=1.0cm,black,thin] (0,0) grid (5.0,5.0) % The Grid
  (0.0,5.0) -- (1.0,4.0) % Diagonal in the top left cell
  (1.0,4.05) -- (5.0,4.05) % Double line under top header row
  (0.95,0.0) -- (0.95,4.0) % Double line on left of header column one
  ;    
  \end{tikzpicture}
\end{figure}

The one in cell $ A'BCD $ (minterm $ 07 $) can be grouped with either the horizontal or vertical group (or both). This creates the following three potential simplified circuits:

\begin{itemize}
  \item Group minterms $ 05 $-$ 07 $ with a separate minterm ($ 15 $): $ Q = A'BD + ABCD $ 
  \item Group minterms $ 07 $-$ 15 $ with a separate minterm ($ 05 $): $ Q = BCD + A'BC'D $ 
  \item Two groups of two minterms ($ 05 $-$ 07 $ and $ 07 $-$ 15 $): $ Q = A'BD + BCD $ 
\end{itemize} 

In general, it would be considered simpler to have two three-input \textsf{AND} gates rather than one three-input \textsf{AND} gate and one four-input \textsf{AND} gate, so the last grouping option would be chosen. The designer always chooses whatever grouping yields the smallest number of gates and the smallest number of inputs per gate. The equation for the simplified circuit is: 

\begin{align}
  \label{KM:eq:overlapping_groups_ex_1}
  A'BD+BCD=Y
\end{align}

Karnaugh map \ref{KM:fig:kmap_overlapping_groups_ex_2} is a more complex example.

%****************************************************************
% Karnaugh map For Overlapping Groups, Example 2
%****************************************************************
\begin{figure}[H]
  \caption{K-Map Overlapping Groups, Example 2}
  \label{KM:fig:kmap_overlapping_groups_ex_2}
  \myfloatalign
  \begin{tikzpicture} [circuit logic US, scale=1.00]
  % make all path lines (the node shapes) a little thicker
  \tikzstyle{every path}=[line width=0.50mm]
  
  %********************************************************************
  % Adjust the settings below to display the 1's and rectangles
  %********************************************************************
  % Uncomment the appropriate lines below to insert ones where needed
    \node[] at (1.4,3.5) {\huge $ 1 $}; % 00
    \node[] at (1.4,2.5) {\huge $ 1 $}; % 01
    \node[] at (1.4,0.5) {\huge $ 1 $}; % 02
  %  \node[] at (1.4,1.5) {\huge $ 1 $}; % 03
  %  \node[] at (2.4,3.5) {\huge $ 1 $}; % 04
  %  \node[] at (2.4,2.5) {\huge $ 1 $}; % 05
  %  \node[] at (2.4,0.5) {\huge $ 1 $}; % 06
    \node[] at (2.4,1.5) {\huge $ 1 $}; % 07
    \node[] at (4.4,3.5) {\huge $ 1 $}; % 08
    \node[] at (4.4,2.5) {\huge $ 1 $}; % 09
    \node[] at (4.4,0.5) {\huge $ 1 $}; % 10
    \node[] at (4.4,1.5) {\huge $ 1 $}; % 11
    \node[] at (3.4,3.5) {\huge $ 1 $}; % 12
    \node[] at (3.4,2.5) {\huge $ 1 $}; % 13
    \node[] at (3.4,0.5) {\huge $ 1 $}; % 14
    \node[] at (3.4,1.5) {\huge $ 1 $}; % 15
  
  % The coords for each cell - this is used as the origin for the solution box
  \coordinate (cell00) at (1.0,3.0); \coordinate (cell01) at (1.0,2.0);
  \coordinate (cell02) at (1.0,0.0); \coordinate (cell03) at (1.0,1.0);
  
  \coordinate (cell04) at (2.0,3.0); \coordinate (cell05) at (2.0,2.0);
  \coordinate (cell06) at (2.0,0.0); \coordinate (cell07) at (2.0,1.0);
  
  \coordinate (cell12) at (3.0,3.0); \coordinate (cell13) at (3.0,2.0);
  \coordinate (cell14) at (3.0,0.0); \coordinate (cell15) at (3.0,1.0);
  
  \coordinate (cell08) at (4.0,3.0); \coordinate (cell09) at (4.0,2.0);
  \coordinate (cell10) at (4.0,0.0); \coordinate (cell11) at (4.0,1.0);
  
  % Horizontal Group
    \node [draw,
    color=red!70!black,
    fill=red!20!white,
    fill opacity=0.3,
    minimum height=0.95cm,
    minimum width=1.95cm, % Adjust to (number of cells) * 1 - 0.95
    rounded corners,
    anchor=south west] at (cell07) {}; % Enter left cell minterm
  
  % Vertical Group
    \node [draw,
    color=blue!70!black,
    fill=blue!20!white,
    fill opacity=0.3,
    minimum height=1.95cm, % Adjust to (number of cells) * 1 - 0.95
    minimum width=0.95cm,
    rounded corners,
    anchor=south west] at (cell01) {}; % Enter bottom cell minterm
  
  % Single Cell
    \node [draw,
    color=green!70!black,
    fill=green!20!white,
    fill opacity=0.3,
    minimum height=0.95cm,
    minimum width=0.95cm,
    rounded corners,
    anchor=south west] at (cell02) {}; % Enter cell minterm

  % Group of 8
    \node [draw,
    color=yellow!70!black,
    fill=yellow!20!white,
    fill opacity=0.3,
    minimum height=3.95cm,
    minimum width=1.95cm,
    rounded corners,
    anchor=south west] at (cell14) {}; % Enter cell minterm
  
  %********************************************************************
  % Shouldn't need to adjust anything below this point - this is just
  % the grid and the minterms.
  %********************************************************************  
  % Text in top-Left cell
  \node[] at (0.35,4.22) { \footnotesize $ \mathsf{ CD } $ }; % cd
  \node[] at (0.70,4.75) { \footnotesize $ \mathsf{ AB } $ }; % ab
  
  % Populate the top row header
  % In the following, the foreach lists a location/text pair
  % The the draw line draws the text at each location
  \foreach \loc/\txt in {
    (1.5,4.5)/{00},(2.5,4.5)/{01},(3.5,4.5)/{11},(4.5,4.5)/{10}
  }
  \draw \loc node{\large $\txt$};
  
  % Populate the header in column one
  \foreach \loc/\txt in { 
    (0.5,3.5)/{00},(0.5,2.5)/{01},(0.5,1.5)/{11},(0.5,0.5)/{10}
  }
  \draw \loc node{\large $\txt$};
  
  % Populate the minterms
  \foreach \loc/\txt in { 
    (1.75,3.15)/{00} , (2.75,3.15)/{04} , (3.75,3.15)/{12} , (4.75,3.15)/{08} ,
    (1.75,2.15)/{01} , (2.75,2.15)/{05} , (3.75,2.15)/{13} , (4.75,2.15)/{09} ,
    (1.75,1.15)/{03} , (2.75,1.15)/{07} , (3.75,1.15)/{15} , (4.75,1.15)/{11} ,
    (1.75,0.15)/{02} , (2.75,0.15)/{06} , (3.75,0.15)/{14} , (4.75,0.15)/{10} }
  \draw \loc node{ \color{blue!90!black} \footnotesize { $\txt$ }};
  
  % Draw the lines
  \draw
  % Finish drawing the grid
  [step=1.0cm,black,thin] (0,0) grid (5.0,5.0) % The Grid
  (0.0,5.0) -- (1.0,4.0) % Diagonal in the top left cell
  (1.0,4.05) -- (5.0,4.05) % Double line under top header row
  (0.95,0.0) -- (0.95,4.0) % Double line on left of header column one
  ;    
  \end{tikzpicture}
\end{figure}

The circuit represented by this Karnaugh map \ref{KM:fig:kmap_overlapping_groups_ex_2} would simplify to:

\begin{align}
  \label{KM:eq:overlapping_groups_ex_2}
  A+A'B'C'+BCD+A'B'CD' = Y
\end{align}

\section{Wrapping Groups}
\label{KM:sec:wrapping_groups}

A Karnaugh map also ``wraps'' around the edges (top/bottom and left/right), so groups can be formed around the borders. It is almost like the Karnaugh map is on some sort of weird sphere where every edge touches the edge across from it (but not diagonal corners). The Karnaugh map in Figure \ref{KM:fig:kmap_wrapping_groups_ex_1} is an example. 

%****************************************************************
% Karnaugh map For Wrapping Groups, Example 1
%****************************************************************
\begin{figure}[H]
  \caption{K-Map Wrapping Groups Example 1}
  \label{KM:fig:kmap_wrapping_groups_ex_1}
  \myfloatalign
  \begin{tikzpicture} [circuit logic US, scale=1.00]
  % make all path lines (the node shapes) a little thicker
  \tikzstyle{every path}=[line width=0.50mm]
  
  %********************************************************************
  % Adjust the settings below to display the 1's and rectangles
  %********************************************************************
  % Uncomment the appropriate lines below to insert ones where needed
  %  \node[] at (1.4,3.5) {\huge $ 1 $}; % 00
    \node[] at (1.4,2.5) {\huge $ 1 $}; % 01
  %  \node[] at (1.4,0.5) {\huge $ 1 $}; % 02
  %  \node[] at (1.4,1.5) {\huge $ 1 $}; % 03
  %  \node[] at (2.4,3.5) {\huge $ 1 $}; % 04
  %  \node[] at (2.4,2.5) {\huge $ 1 $}; % 05
  %  \node[] at (2.4,0.5) {\huge $ 1 $}; % 06
  %  \node[] at (2.4,1.5) {\huge $ 1 $}; % 07
  %  \node[] at (4.4,3.5) {\huge $ 1 $}; % 08
    \node[] at (4.4,2.5) {\huge $ 1 $}; % 09
  %  \node[] at (4.4,0.5) {\huge $ 1 $}; % 10
  %  \node[] at (4.4,1.5) {\huge $ 1 $}; % 11
  %  \node[] at (3.4,3.5) {\huge $ 1 $}; % 12
  %  \node[] at (3.4,2.5) {\huge $ 1 $}; % 13
  %  \node[] at (3.4,0.5) {\huge $ 1 $}; % 14
  %  \node[] at (3.4,1.5) {\huge $ 1 $}; % 15
  
  % The coords for each cell - this is used as the origin for the solution box
  \coordinate (cell00) at (1.0,3.0); \coordinate (cell01) at (1.0,2.0);
  \coordinate (cell02) at (1.0,0.0); \coordinate (cell03) at (1.0,1.0);
  
  \coordinate (cell04) at (2.0,3.0); \coordinate (cell05) at (2.0,2.0);
  \coordinate (cell06) at (2.0,0.0); \coordinate (cell07) at (2.0,1.0);
  
  \coordinate (cell12) at (3.0,3.0); \coordinate (cell13) at (3.0,2.0);
  \coordinate (cell14) at (3.0,0.0); \coordinate (cell15) at (3.0,1.0);
  
  \coordinate (cell08) at (4.0,3.0); \coordinate (cell09) at (4.0,2.0);
  \coordinate (cell10) at (4.0,0.0); \coordinate (cell11) at (4.0,1.0);
  
  % Horizontal Group
  %  \node [draw,
  %  color=red!70!black,
  %  fill=red!20!white,
  %  fill opacity=0.3,
  %  minimum height=0.95cm,
  %  minimum width=1.95cm, % Adjust to (number of cells) * 1 - 0.95
  %  rounded corners,
  %  anchor=south west] at (cell02) {}; % Enter left cell minterm
  
  % Vertical Group
  %  \node [draw,
  %  color=blue!70!black,
  %  fill=blue!20!white,
  %  fill opacity=0.3,
  %  minimum height=1.95cm, % Adjust to (number of cells) * 1 - 0.95
  %  minimum width=0.95cm,
  %  rounded corners,
  %  anchor=south west] at (cell07) {}; % Enter bottom cell minterm
  
  % Single Cell
  %  \node [draw,
  %  color=green!70!black,
  %  fill=green!20!white,
  %  fill opacity=0.3,
  %  minimum height=0.95cm,
  %  minimum width=0.95cm,
  %  rounded corners,
  %  anchor=south west] at (cell09) {}; % Enter cell minterm
  
  %********************************************************************
  % Shouldn't need to adjust anything below this point - this is just
  % the grid and the minterms.
  %********************************************************************  
  % Text in top-Left cell
  \node[] at (0.35,4.22) { \footnotesize $ \mathsf{ CD } $ }; % cd
  \node[] at (0.70,4.75) { \footnotesize $ \mathsf{ AB } $ }; % ab
  
  % Populate the top row header
  % In the following, the foreach lists a location/text pair
  % The the draw line draws the text at each location
  \foreach \loc/\txt in {
    (1.5,4.5)/{00},(2.5,4.5)/{01},(3.5,4.5)/{11},(4.5,4.5)/{10}
  }
  \draw \loc node{\large $\txt$};
  
  % Populate the header in column one
  \foreach \loc/\txt in { 
    (0.5,3.5)/{00},(0.5,2.5)/{01},(0.5,1.5)/{11},(0.5,0.5)/{10}
  }
  \draw \loc node{\large $\txt$};
  
  % Populate the minterms
  \foreach \loc/\txt in { 
    (1.75,3.15)/{00} , (2.75,3.15)/{04} , (3.75,3.15)/{12} , (4.75,3.15)/{08} ,
    (1.75,2.15)/{01} , (2.75,2.15)/{05} , (3.75,2.15)/{13} , (4.75,2.15)/{09} ,
    (1.75,1.15)/{03} , (2.75,1.15)/{07} , (3.75,1.15)/{15} , (4.75,1.15)/{11} ,
    (1.75,0.15)/{02} , (2.75,0.15)/{06} , (3.75,0.15)/{14} , (4.75,0.15)/{10} }
  \draw \loc node{ \color{blue!90!black} \footnotesize { $\txt$ }};
  
  % Draw the lines
  \draw
  % Finish drawing the grid
  [step=1.0cm,black,thin] (0,0) grid (5.0,5.0) % The Grid
  (0.0,5.0) -- (1.0,4.0) % Diagonal in the top left cell
  (1.0,4.05) -- (5.0,4.05) % Double line under top header row
  (0.95,0.0) -- (0.95,4.0) % Double line on left of header column one
  ;    
  \end{tikzpicture}
\end{figure}

The two ones on this map can be grouped ``around the edge'' to form Equation \ref{KM:eq:k-map_wrapping_groups_ex_1}.

\begin{align}
  \label{KM:eq:k-map_wrapping_groups_ex_1}
  B'C'D &= Y
\end{align}

The Karnaugh map in Figure \ref{KM:fig:kmap_wrapping_groups_ex_2} is another example of wrapping.

%****************************************************************
% Karnaugh map For Wrapping Groups, Example 2
%****************************************************************
\begin{figure}[H]
  \caption{K-Map Wrapping Groups Example 2}
  \label{KM:fig:kmap_wrapping_groups_ex_2}
  \myfloatalign
  \begin{tikzpicture} [circuit logic US, scale=1.00]
  % make all path lines (the node shapes) a little thicker
  \tikzstyle{every path}=[line width=0.50mm]
  
  %********************************************************************
  % Adjust the settings below to display the 1's and rectangles
  %********************************************************************
  % Uncomment the appropriate lines below to insert ones where needed
    \node[] at (1.4,3.5) {\huge $ 1 $}; % 00
  %  \node[] at (1.4,2.5) {\huge $ 1 $}; % 01
    \node[] at (1.4,0.5) {\huge $ 1 $}; % 02
  %  \node[] at (1.4,1.5) {\huge $ 1 $}; % 03
  %  \node[] at (2.4,3.5) {\huge $ 1 $}; % 04
  %  \node[] at (2.4,2.5) {\huge $ 1 $}; % 05
  %  \node[] at (2.4,0.5) {\huge $ 1 $}; % 06
  %  \node[] at (2.4,1.5) {\huge $ 1 $}; % 07
    \node[] at (4.4,3.5) {\huge $ 1 $}; % 08
  %  \node[] at (4.4,2.5) {\huge $ 1 $}; % 09
    \node[] at (4.4,0.5) {\huge $ 1 $}; % 10
  %  \node[] at (4.4,1.5) {\huge $ 1 $}; % 11
  %  \node[] at (3.4,3.5) {\huge $ 1 $}; % 12
  %  \node[] at (3.4,2.5) {\huge $ 1 $}; % 13
  %  \node[] at (3.4,0.5) {\huge $ 1 $}; % 14
  %  \node[] at (3.4,1.5) {\huge $ 1 $}; % 15
  
  % The coords for each cell - this is used as the origin for the solution box
  \coordinate (cell00) at (1.0,3.0); \coordinate (cell01) at (1.0,2.0);
  \coordinate (cell02) at (1.0,0.0); \coordinate (cell03) at (1.0,1.0);
  
  \coordinate (cell04) at (2.0,3.0); \coordinate (cell05) at (2.0,2.0);
  \coordinate (cell06) at (2.0,0.0); \coordinate (cell07) at (2.0,1.0);
  
  \coordinate (cell12) at (3.0,3.0); \coordinate (cell13) at (3.0,2.0);
  \coordinate (cell14) at (3.0,0.0); \coordinate (cell15) at (3.0,1.0);
  
  \coordinate (cell08) at (4.0,3.0); \coordinate (cell09) at (4.0,2.0);
  \coordinate (cell10) at (4.0,0.0); \coordinate (cell11) at (4.0,1.0);
  
  % Horizontal Group
  %  \node [draw,
  %  color=red!70!black,
  %  fill=red!20!white,
  %  fill opacity=0.3,
  %  minimum height=0.95cm,
  %  minimum width=1.95cm, % Adjust to (number of cells) * 1 - 0.95
  %  rounded corners,
  %  anchor=south west] at (cell02) {}; % Enter left cell minterm
  
  % Vertical Group
  %  \node [draw,
  %  color=blue!70!black,
  %  fill=blue!20!white,
  %  fill opacity=0.3,
  %  minimum height=1.95cm, % Adjust to (number of cells) * 1 - 0.95
  %  minimum width=0.95cm,
  %  rounded corners,
  %  anchor=south west] at (cell07) {}; % Enter bottom cell minterm
  
  % Single Cell
  %  \node [draw,
  %  color=green!70!black,
  %  fill=green!20!white,
  %  fill opacity=0.3,
  %  minimum height=0.95cm,
  %  minimum width=0.95cm,
  %  rounded corners,
  %  anchor=south west] at (cell09) {}; % Enter cell minterm
  
  %********************************************************************
  % Shouldn't need to adjust anything below this point - this is just
  % the grid and the minterms.
  %********************************************************************  
  % Text in top-Left cell
  \node[] at (0.35,4.22) { \footnotesize $ \mathsf{ CD } $ }; % cd
  \node[] at (0.70,4.75) { \footnotesize $ \mathsf{ AB } $ }; % ab
  
  % Populate the top row header
  % In the following, the foreach lists a location/text pair
  % The the draw line draws the text at each location
  \foreach \loc/\txt in {
    (1.5,4.5)/{00},(2.5,4.5)/{01},(3.5,4.5)/{11},(4.5,4.5)/{10}
  }
  \draw \loc node{\large $\txt$};
  
  % Populate the header in column one
  \foreach \loc/\txt in { 
    (0.5,3.5)/{00},(0.5,2.5)/{01},(0.5,1.5)/{11},(0.5,0.5)/{10}
  }
  \draw \loc node{\large $\txt$};
  
  % Populate the minterms
  \foreach \loc/\txt in { 
    (1.75,3.15)/{00} , (2.75,3.15)/{04} , (3.75,3.15)/{12} , (4.75,3.15)/{08} ,
    (1.75,2.15)/{01} , (2.75,2.15)/{05} , (3.75,2.15)/{13} , (4.75,2.15)/{09} ,
    (1.75,1.15)/{03} , (2.75,1.15)/{07} , (3.75,1.15)/{15} , (4.75,1.15)/{11} ,
    (1.75,0.15)/{02} , (2.75,0.15)/{06} , (3.75,0.15)/{14} , (4.75,0.15)/{10} }
  \draw \loc node{ \color{blue!90!black} \footnotesize { $\txt$ }};
  
  % Draw the lines
  \draw
  % Finish drawing the grid
  [step=1.0cm,black,thin] (0,0) grid (5.0,5.0) % The Grid
  (0.0,5.0) -- (1.0,4.0) % Diagonal in the top left cell
  (1.0,4.05) -- (5.0,4.05) % Double line under top header row
  (0.95,0.0) -- (0.95,4.0) % Double line on left of header column one
  ;    
  \end{tikzpicture}
\end{figure}

The ones on the above map can be formed into a group of four and simplify into Equation \ref{KM:eq:k-map_wrapping_groups_ex_2}. 

\begin{align}
  \label{KM:eq:k-map_wrapping_groups_ex_2}
  B'D' &= Y
\end{align}

\section{Karnaugh maps for Five-Variable Inputs}
\label{KM:sec:karnaugh_maps_for_5-Variable_inputs}

It is possible to create a Karnaugh map for circuits with five input variables; however, the map must be simplified as a three-dimensional object so it is more complex than the maps described above. As an example, imagine a circuit that is defined by Equation \ref{KM:eq:k-map_five_variables}.

\begin{align}
  \label{KM:eq:k-map_five_variables}
  \int(A,B,C,D,E) &= \sum(0,9,13,16,25,29)
\end{align}

The Karnaugh map in Figure \ref{KM:fig:kmap_for_five_variables_ex_1} would be used to simplify that circuit.

%****************************************************************
% Karnaugh map, Five Variables, Ex 1
%****************************************************************
\begin{figure}[H]
  \caption{K-Map for Five Variables, Example 1}
  \label{KM:fig:kmap_for_five_variables_ex_1}
  \myfloatalign
  \begin{tikzpicture} [circuit logic US, scale=1.00]
  % make all path lines (the node shapes) a little thicker
  \tikzstyle{every path}=[line width=0.50mm]
  
  %********************************************************************
  % Adjust the settings below to display the 1's and rectangles
  %********************************************************************
  % Uncomment the appropriate lines below to insert ones where needed
    \node[] at (1.4,3.5) {\huge $ 1 $}; % 00
%    \node[] at (2.4,3.5) {\huge $ 1 $}; % 04
%    \node[] at (3.4,3.5) {\huge $ 1 $}; % 12
%    \node[] at (4.4,3.5) {\huge $ 1 $}; % 08

    \node[] at (5.4,3.5) {\huge $ 1 $}; % 16
%    \node[] at (6.4,3.5) {\huge $ 1 $}; % 20
%    \node[] at (7.4,3.5) {\huge $ 1 $}; % 28
%    \node[] at (8.4,3.5) {\huge $ 1 $}; % 24

%    \node[] at (1.4,2.5) {\huge $ 1 $}; % 01
%    \node[] at (2.4,2.5) {\huge $ 1 $}; % 05
    \node[] at (3.4,2.5) {\huge $ 1 $}; % 13
    \node[] at (4.4,2.5) {\huge $ 1 $}; % 09
    
%    \node[] at (5.4,2.5) {\huge $ 1 $}; % 17
%    \node[] at (6.4,2.5) {\huge $ 1 $}; % 21
    \node[] at (7.4,2.5) {\huge $ 1 $}; % 29
    \node[] at (8.4,2.5) {\huge $ 1 $}; % 25

%    \node[] at (1.4,1.5) {\huge $ 1 $}; % 03
%    \node[] at (2.4,1.5) {\huge $ 1 $}; % 07
%    \node[] at (3.4,1.5) {\huge $ 1 $}; % 15
%    \node[] at (4.4,1.5) {\huge $ 1 $}; % 11
    
%    \node[] at (5.4,1.5) {\huge $ 1 $}; % 19
%    \node[] at (6.4,1.5) {\huge $ 1 $}; % 23
%    \node[] at (7.4,1.5) {\huge $ 1 $}; % 31
%    \node[] at (8.4,1.5) {\huge $ 1 $}; % 27
    
%    \node[] at (1.4,0.5) {\huge $ 1 $}; % 02
%    \node[] at (2.4,0.5) {\huge $ 1 $}; % 06
%    \node[] at (3.4,0.5) {\huge $ 1 $}; % 14
%    \node[] at (4.4,0.5) {\huge $ 1 $}; % 10
    
%    \node[] at (5.4,0.5) {\huge $ 1 $}; % 18
%    \node[] at (6.4,0.5) {\huge $ 1 $}; % 22
%    \node[] at (7.4,0.5) {\huge $ 1 $}; % 30
%    \node[] at (8.4,0.5) {\huge $ 1 $}; % 26
  
  % The coords for each cell - this is used as the origin for the solution box
    \coordinate (cell00) at (1.0,3.0); \coordinate (cell04) at (2.0,3.0);
    \coordinate (cell12) at (3.0,3.0); \coordinate (cell08) at (4.0,3.0);
    \coordinate (cell16) at (5.0,3.0); \coordinate (cell20) at (6.0,3.0);
    \coordinate (cell28) at (7.0,3.0); \coordinate (cell24) at (8.0,3.0);

    \coordinate (cell01) at (1.0,2.0); \coordinate (cell05) at (2.0,2.0);
    \coordinate (cell13) at (3.0,2.0); \coordinate (cell09) at (4.0,2.0);
    \coordinate (cell17) at (5.0,2.0); \coordinate (cell21) at (6.0,2.0);
    \coordinate (cell29) at (7.0,2.0); \coordinate (cell25) at (8.0,2.0);
  
    \coordinate (cell03) at (1.0,1.0); \coordinate (cell07) at (2.0,1.0);
    \coordinate (cell15) at (3.0,1.0); \coordinate (cell11) at (4.0,1.0);
    \coordinate (cell19) at (5.0,1.0); \coordinate (cell23) at (6.0,1.0);
    \coordinate (cell31) at (7.0,1.0); \coordinate (cell27) at (8.0,1.0);

    \coordinate (cell02) at (1.0,0.0); \coordinate (cell06) at (2.0,0.0);
    \coordinate (cell14) at (3.0,0.0); \coordinate (cell10) at (4.0,0.0);
    \coordinate (cell18) at (5.0,0.0); \coordinate (cell22) at (6.0,0.0);
    \coordinate (cell30) at (7.0,0.0); \coordinate (cell26) at (8.0,0.0);
    
  % Single Cell
    \node [draw,
    color=green!70!black,
    fill=green!20!white,
    fill opacity=0.3,
    minimum height=0.95cm,
    minimum width=0.95cm,
    rounded corners,
    anchor=south west] at (cell16) {}; % Enter cell minterm

    % Single Cell
    \node [draw,
    color=green!70!black,
    fill=green!20!white,
    fill opacity=0.3,
    minimum height=0.95cm,
    minimum width=0.95cm,
    rounded corners,
    anchor=south west] at (cell00) {}; % Enter cell minterm

  % Horizontal Group
    \node [draw,
    color=red!70!black,
    fill=red!20!white,
    fill opacity=0.3,
    minimum height=0.95cm,
    minimum width=1.95cm, % Adjust to (number of cells) * 1.0 - 0.05
    rounded corners,
    anchor=south west] at (cell29) {}; % Enter left cell minterm
  
  % Horizontal Group
    \node [draw,
    color=red!70!black,
    fill=red!20!white,
    fill opacity=0.3,
    minimum height=0.95cm,
    minimum width=1.95cm, % Adjust to (number of cells) * 1.0 - 0.05
    rounded corners,
    anchor=south west] at (cell13) {}; % Enter left cell minterm

  % Vertical Group
  %  \node [draw,
  %  color=blue!70!black,
  %  fill=blue!20!white,
  %  fill opacity=0.3,
  %  minimum height=1.95cm, % Adjust to (number of cells) * 1.0 - 0.05
  %  minimum width=0.95cm, % Adjust to (number of cells) * 1.0 - 0.05
  %  rounded corners,
  %  anchor=south west] at (cell15) {}; % Enter bottom cell minterm  

  %********************************************************************
  % Shouldn't need to adjust anything below this point - this is just
  % the grid and the minterms.
  %********************************************************************  
  % Text in top-Left cell
  \node[] at (0.25,4.25) { \footnotesize $ \mathsf{ DE } $ }; % de
  \node[] at (0.65,4.75) { \footnotesize $ \mathsf{ ABC } $ }; % abc
  
  % Populate the top row header
  % In the following, the foreach lists a location/text pair
  % The the draw line draws the text at each location
  \foreach \loc/\txt in {
  (1.5,4.5)/{000} , (2.5,4.5)/{001} , (3.5,4.5)/{011} , (4.5,4.5)/{010} ,
  (5.5,4.5)/{100} , (6.5,4.5)/{101} , (7.5,4.5)/{111} , (8.5,4.5)/{110}
  }
  \draw \loc node{\large $\txt$};
  
  % Populate the header in column one
  \foreach \loc/\txt in {
  (0.5,3.5)/{00} , (0.5,2.5)/{01} , (0.5,1.5)/{11} , (0.5,0.5)/{10} }
  \draw \loc node{\large $\txt$};
  
  % Populate the minterms
  \foreach \loc/\txt in { 
    (1.70,3.15)/{00} , (2.70,3.15)/{04} , (3.70,3.15)/{12} , (4.70,3.15)/{08} ,
    (1.70,2.15)/{01} , (2.70,2.15)/{05} , (3.70,2.15)/{13} , (4.70,2.15)/{09} ,
    (1.70,1.15)/{03} , (2.70,1.15)/{07} , (3.70,1.15)/{15} , (4.70,1.15)/{11} ,
    (1.70,0.15)/{02} , (2.70,0.15)/{06} , (3.70,0.15)/{14} , (4.70,0.15)/{10} ,
    (5.70,3.15)/{16} , (6.70,3.15)/{20} , (7.70,3.15)/{28} , (8.70,3.15)/{24} ,
    (5.70,2.15)/{17} , (6.70,2.15)/{21} , (7.70,2.15)/{29} , (8.70,2.15)/{25} ,
    (5.70,1.15)/{19} , (6.70,1.15)/{23} , (7.70,1.15)/{31} , (8.70,1.15)/{27} ,
    (5.70,0.15)/{18} , (6.70,0.15)/{22} , (7.70,0.15)/{30} , (8.70,0.15)/{26} }
  \draw \loc node{ \color{blue!90!black} \footnotesize{ $\txt$ }};
  
  % Draw the lines
  \draw
  % Finish drawing the grid
  [step=1.0cm,black,thin] (0,0) grid (9.0,5.0) % The Grid
  (0.0,5.0) -- (1.0,4.0) % Diagonal in the top left cell
  (1.0,4.05) -- (9.0,4.05) % Double line under top header row
  (0.95,0.0) -- (0.95,4.0) % Double line on left of header column one
  ;
  \draw
  [ultra thick] (4.97,0.0) -- (4.97,5.0) % Thick line separating the two grids
  (5.02,0.0) -- (5.02,5.0) % Thick line separating the two grids
  ;    
  \end{tikzpicture}
\end{figure}

This map lists variables $ A $, $ B $, and $ C $ across the top row with $ D $ and $ E $ down the left column. Variables $ B $ and $ C $ are in Gray Code order, while variable $ A $ is zero on the left side of the map and one on the right. To simplify the circuit, the map must be imagined to be a three-dimensional item; so the map would be cut along the heavy line between minterms $ 08 $ and $ 16 $ (where variables $ ABC $ change from $ 010 $ to $ 100 $) and then the right half would slide under the left half in such a way that minterm $ 16 $ ends up directly under minterm $ 00 $. 

By arranging the map in three dimensions there is only one bit different between minterms $ 00 $ and $ 16 $: bit $ A $ changes from zero to one while the bits $ B $ and $ C $ remain at zero. Those two cells then form a group of two and would be $ A'B'C'D'E'+AB'C'D'E' $. Since variable $ A $ and $ A' $ are both present in this expression, and none of the other variables change, it can be simplified to: $ B'C'D'E' $. This process is exactly the same as for a two-dimension Karnaugh map, except that adjacent cells may include those above or below each other (though diagonals are still not simplified).

Next, consider the group formed by minterms $ 09 $, $ 13 $, $ 25 $, and $ 29 $. The expression for that group is $ (A'BC'D'E+A'BCD'E+ABC'D'E+ABCD'E) $. The variables $ A $ and $ C $ can be removed from the simplified expression by the Complement Property, leaving: $ BD'E $ for this group.

Equation \ref{KM:eq:k-map_five_variables_ex_1_simplified} is the final simplified expression for this circuit.

\begin{align}
  \label{KM:eq:k-map_five_variables_ex_1_simplified}
  (B'C'D'E')+(BD'E) &= Y
\end{align}

The Karnaugh map in Figure \ref{KM:fig:kmap_solving_for_five_variables_ex_2} is a more complex example.

%****************************************************************
% Karnaugh map, Five Variables, Ex 2
%****************************************************************
\begin{figure}[H]
  \caption{K-Map Solving for Five Variables, Example 2}
  \label{KM:fig:kmap_solving_for_five_variables_ex_2}
  \myfloatalign
  \begin{tikzpicture} [circuit logic US, scale=1.00]
  % make all path lines (the node shapes) a little thicker
  \tikzstyle{every path}=[line width=0.50mm]
  
  %********************************************************************
  % Adjust the settings below to display the 1's and rectangles
  %********************************************************************
  % Uncomment the appropriate lines below to insert ones where needed
    \node[] at (1.4,3.5) {\huge $ 1 $}; % 00
  %  \node[] at (2.4,3.5) {\huge $ 1 $}; % 04
  %  \node[] at (3.4,3.5) {\huge $ 1 $}; % 12
  %  \node[] at (4.4,3.5) {\huge $ 1 $}; % 08
  
  %  \node[] at (5.4,3.5) {\huge $ 1 $}; % 16
  %  \node[] at (6.4,3.5) {\huge $ 1 $}; % 20
  %  \node[] at (7.4,3.5) {\huge $ 1 $}; % 28
    \node[] at (8.4,3.5) {\huge $ 1 $}; % 24
  
  %  \node[] at (1.4,2.5) {\huge $ 1 $}; % 01
    \node[] at (2.4,2.5) {\huge $ 1 $}; % 05
    \node[] at (3.4,2.5) {\huge $ 1 $}; % 13
  %  \node[] at (4.4,2.5) {\huge $ 1 $}; % 09
  
  %  \node[] at (5.4,2.5) {\huge $ 1 $}; % 17
  %  \node[] at (6.4,2.5) {\huge $ 1 $}; % 21
    \node[] at (7.4,2.5) {\huge $ 1 $}; % 29
  %  \node[] at (8.4,2.5) {\huge $ 1 $}; % 25
  
  %  \node[] at (1.4,1.5) {\huge $ 1 $}; % 03
  %  \node[] at (2.4,1.5) {\huge $ 1 $}; % 07
    \node[] at (3.4,1.5) {\huge $ 1 $}; % 15
  %  \node[] at (4.4,1.5) {\huge $ 1 $}; % 11
  
  %  \node[] at (5.4,1.5) {\huge $ 1 $}; % 19
  %  \node[] at (6.4,1.5) {\huge $ 1 $}; % 23
    \node[] at (7.4,1.5) {\huge $ 1 $}; % 31
    \node[] at (8.4,1.5) {\huge $ 1 $}; % 27
  
    \node[] at (1.4,0.5) {\huge $ 1 $}; % 02
  %  \node[] at (2.4,0.5) {\huge $ 1 $}; % 06
  %  \node[] at (3.4,0.5) {\huge $ 1 $}; % 14
  %  \node[] at (4.4,0.5) {\huge $ 1 $}; % 10
  
  %  \node[] at (5.4,0.5) {\huge $ 1 $}; % 18
  %  \node[] at (6.4,0.5) {\huge $ 1 $}; % 22
  %  \node[] at (7.4,0.5) {\huge $ 1 $}; % 30
  %  \node[] at (8.4,0.5) {\huge $ 1 $}; % 26
  
  % The coords for each cell - this is used as the origin for the solution box
  \coordinate (cell00) at (1.0,3.0); \coordinate (cell04) at (2.0,3.0);
  \coordinate (cell12) at (3.0,3.0); \coordinate (cell08) at (4.0,3.0);
  \coordinate (cell16) at (5.0,3.0); \coordinate (cell20) at (6.0,3.0);
  \coordinate (cell28) at (7.0,3.0); \coordinate (cell24) at (8.0,3.0);
  
  \coordinate (cell01) at (1.0,2.0); \coordinate (cell05) at (2.0,2.0);
  \coordinate (cell13) at (3.0,2.0); \coordinate (cell09) at (4.0,2.0);
  \coordinate (cell17) at (5.0,2.0); \coordinate (cell21) at (6.0,2.0);
  \coordinate (cell29) at (7.0,2.0); \coordinate (cell25) at (8.0,2.0);
  
  \coordinate (cell03) at (1.0,1.0); \coordinate (cell07) at (2.0,1.0);
  \coordinate (cell15) at (3.0,1.0); \coordinate (cell11) at (4.0,1.0);
  \coordinate (cell19) at (5.0,1.0); \coordinate (cell23) at (6.0,1.0);
  \coordinate (cell31) at (7.0,1.0); \coordinate (cell27) at (8.0,1.0);
  
  \coordinate (cell02) at (1.0,0.0); \coordinate (cell06) at (2.0,0.0);
  \coordinate (cell14) at (3.0,0.0); \coordinate (cell10) at (4.0,0.0);
  \coordinate (cell18) at (5.0,0.0); \coordinate (cell22) at (6.0,0.0);
  \coordinate (cell30) at (7.0,0.0); \coordinate (cell26) at (8.0,0.0);
  
  % Single Cell
  %    \node [draw,
  %    color=green!70!black,
  %    fill=green!20!white,
  %    fill opacity=0.3,
  %    minimum height=0.95cm,
  %    minimum width=0.95cm,
  %    rounded corners,
  %    anchor=south west] at (cell26) {}; % Enter cell minterm
  
  % Horizontal Group
  %    \node [draw,
  %    color=red!70!black,
  %    fill=red!20!white,
  %    fill opacity=0.3,
  %    minimum height=0.95cm,
  %    minimum width=1.95cm, % Adjust to (number of cells) * 1.0 - 0.05
  %    rounded corners,
  %    anchor=south west] at (cell12) {}; % Enter left cell minterm
  
  % Vertical Group
  %  \node [draw,
  %  color=blue!70!black,
  %  fill=blue!20!white,
  %  fill opacity=0.3,
  %  minimum height=1.95cm, % Adjust to (number of cells) * 1.0 - 0.05
  %  minimum width=0.95cm, % Adjust to (number of cells) * 1.0 - 0.05
  %  rounded corners,
  %  anchor=south west] at (cell15) {}; % Enter bottom cell minterm  
  
  %********************************************************************
  % Shouldn't need to adjust anything below this point - this is just
  % the grid and the minterms.
  %********************************************************************  
  % Text in top-Left cell
  \node[] at (0.25,4.25) { \footnotesize $ \mathsf{ DE } $ }; % de
  \node[] at (0.65,4.75) { \footnotesize $ \mathsf{ ABC } $ }; % abC
  
  % Populate the top row header
  % In the following, the foreach lists a location/text pair
  % The the draw line draws the text at each location
  \foreach \loc/\txt in {
    (1.5,4.5)/{000} , (2.5,4.5)/{001} , (3.5,4.5)/{011} , (4.5,4.5)/{010} ,
    (5.5,4.5)/{100} , (6.5,4.5)/{101} , (7.5,4.5)/{111} , (8.5,4.5)/{110}
  }
  \draw \loc node{\large $\txt$};
  
  % Populate the header in column one
  \foreach \loc/\txt in {
    (0.5,3.5)/{00} , (0.5,2.5)/{01} , (0.5,1.5)/{11} , (0.5,0.5)/{10} }
  \draw \loc node{\large $\txt$};
  
  % Populate the minterms
  \foreach \loc/\txt in { 
    (1.70,3.15)/{00} , (2.70,3.15)/{04} , (3.70,3.15)/{12} , (4.70,3.15)/{08} ,
    (1.70,2.15)/{01} , (2.70,2.15)/{05} , (3.70,2.15)/{13} , (4.70,2.15)/{09} ,
    (1.70,1.15)/{03} , (2.70,1.15)/{07} , (3.70,1.15)/{15} , (4.70,1.15)/{11} ,
    (1.70,0.15)/{02} , (2.70,0.15)/{06} , (3.70,0.15)/{14} , (4.70,0.15)/{10} ,
    (5.70,3.15)/{16} , (6.70,3.15)/{20} , (7.70,3.15)/{28} , (8.70,3.15)/{24} ,
    (5.70,2.15)/{17} , (6.70,2.15)/{21} , (7.70,2.15)/{29} , (8.70,2.15)/{25} ,
    (5.70,1.15)/{19} , (6.70,1.15)/{23} , (7.70,1.15)/{31} , (8.70,1.15)/{27} ,
    (5.70,0.15)/{18} , (6.70,0.15)/{22} , (7.70,0.15)/{30} , (8.70,0.15)/{26} }
  \draw \loc node{ \color{blue!90!black} \footnotesize{ $\txt$ }};
  
  % Draw the lines
  \draw
  % Finish drawing the grid
  [step=1.0cm,black,thin] (0,0) grid (9.0,5.0) % The Grid
  (0.0,5.0) -- (1.0,4.0) % Diagonal in the top left cell
  (1.0,4.05) -- (9.0,4.05) % Double line under top header row
  (0.95,0.0) -- (0.95,4.0) % Double line on left of header column one
  ;
  \draw
  [ultra thick] (4.97,0.0) -- (4.97,5.0) % Thick line separating the two grids
  (5.02,0.0) -- (5.02,5.0) % Thick line separating the two grids
  ;    
  \end{tikzpicture}
\end{figure}

\marginpar{It is possible to simplify a six-input circuit with a Karnaugh map, but that becomes quite challenging since the map must be simplified in four dimensions.}

On the map in Figure \ref{KM:fig:kmap_solving_for_five_variables_ex_2}, the largest group would be minterms $ 13 $-$ 15 $-$ 29 $-$ 31 $, so they should be combined first. Minterms $ 05 $-$ 13 $, $ 27 $-$ 31 $, and $ 00 $-$ 02 $ would form groups of two. Finally, even though Karnaugh maps wrap around the edges, minterm $ 00 $ will not group with minterm $ 24 $ since they are on different layers (notice that two bits, $ A $ and $ B $, change between those two minterms, so they are not adjacent); therefore, minterms $ 00 $ and $ 24 $ will not group with any other minterms. Equation \ref{KM:eq:k-map_five_variables_ex_2_simplified} is the simplified expression for this circuit.

\begin{align}
  \label{KM:eq:k-map_five_variables_ex_2_simplified}
  (A'B'C'E')+(A'CD'E)+(BCD)+(ABDE)+(ABC'D'E') &= Y
\end{align}

\section{``Don't Care'' Terms}
\label{KM:sec:dont_care_terms}

Occasionally, a circuit designer will run across a situation where the output for a particular minterm makes no difference in the circuit; so that minterm is considered ``don't care;'' that is, it can be either one or zero without having any effect on the entire circuit. As an example, consider a circuit that it designed to work with \gls{bcd} (page \pageref{MO:subsub:binary_coded_decimal}) values. In that system, minterms $ 10 $-$ 15 $ do not exist, so they would be considered ``don't care.'' On a Karnaugh map, ``don't care'' terms are indicated using several different methods, but the two most common are a dash or an ``x''.  When simplifying a Karnaugh map that contains ``don't care'' values, the designer can choose to consider those values as either zero or one, whichever makes simplifying the map easier. Consider the following Karnaugh map:

%****************************************************************
% Karnaugh map With Don't Care Terms
%****************************************************************
\begin{figure}[H]
  \caption{K-Map With ``Don't Care'' Terms, Example 1}
  \label{KM:fig:kmap_with_dont_care_terms_ex_1}
  \myfloatalign
  \begin{tikzpicture} [circuit logic US, scale=1.00]
  % make all path lines (the node shapes) a little thicker
  \tikzstyle{every path}=[line width=0.50mm]
  
  %********************************************************************
  % Adjust the settings below to display the 1's and rectangles
  %********************************************************************
  % Uncomment the appropriate lines below to insert ones where needed
  %  \node[] at (1.4,3.5) {\huge $ 1 $}; % 00
  %  \node[] at (1.4,2.5) {\huge $ 1 $}; % 01
    \node[] at (1.4,0.5) {\huge $ X $}; % 02
  %  \node[] at (1.4,1.5) {\huge $ 1 $}; % 03
  %  \node[] at (2.4,3.5) {\huge $ 1 $}; % 04
  %  \node[] at (2.4,2.5) {\huge $ 1 $}; % 05
  %  \node[] at (2.4,0.5) {\huge $ 1 $}; % 06
  %  \node[] at (2.4,1.5) {\huge $ 1 $}; % 07
    \node[] at (4.4,3.5) {\huge $ 1 $}; % 08
    \node[] at (4.4,2.5) {\huge $ 1 $}; % 09
  %  \node[] at (4.4,0.5) {\huge $ 1 $}; % 10
  %  \node[] at (4.4,1.5) {\huge $ 1 $}; % 11
    \node[] at (3.4,3.5) {\huge $ 1 $}; % 12
    \node[] at (3.4,2.5) {\huge $ X $}; % 13
  %  \node[] at (3.4,0.5) {\huge $ 1 $}; % 14
  %  \node[] at (3.4,1.5) {\huge $ 1 $}; % 15
  
  % The coords for each cell - this is used as the origin for the solution box
  \coordinate (cell00) at (1.0,3.0); \coordinate (cell01) at (1.0,2.0);
  \coordinate (cell02) at (1.0,0.0); \coordinate (cell03) at (1.0,1.0);
  
  \coordinate (cell04) at (2.0,3.0); \coordinate (cell05) at (2.0,2.0);
  \coordinate (cell06) at (2.0,0.0); \coordinate (cell07) at (2.0,1.0);
  
  \coordinate (cell12) at (3.0,3.0); \coordinate (cell13) at (3.0,2.0);
  \coordinate (cell14) at (3.0,0.0); \coordinate (cell15) at (3.0,1.0);
  
  \coordinate (cell08) at (4.0,3.0); \coordinate (cell09) at (4.0,2.0);
  \coordinate (cell10) at (4.0,0.0); \coordinate (cell11) at (4.0,1.0);
  
  % Horizontal Group
  %  \node [draw,
  %  color=red!70!black,
  %  fill=red!20!white,
  %  fill opacity=0.3,
  %  minimum height=0.95cm,
  %  minimum width=1.95cm, % Adjust to (number of cells) * 1 - 0.95
  %  rounded corners,
  %  anchor=south west] at (cell02) {}; % Enter left cell minterm
  
  % Vertical Group
  %  \node [draw,
  %  color=blue!70!black,
  %  fill=blue!20!white,
  %  fill opacity=0.3,
  %  minimum height=1.95cm, % Adjust to (number of cells) * 1 - 0.95
  %  minimum width=0.95cm,
  %  rounded corners,
  %  anchor=south west] at (cell07) {}; % Enter bottom cell minterm
  
  % Single Cell
  %  \node [draw,
  %  color=green!70!black,
  %  fill=green!20!white,
  %  fill opacity=0.3,
  %  minimum height=0.95cm,
  %  minimum width=0.95cm,
  %  rounded corners,
  %  anchor=south west] at (cell09) {}; % Enter cell minterm
  
  %********************************************************************
  % Shouldn't need to adjust anything below this point - this is just
  % the grid and the minterms.
  %********************************************************************  
  % Text in top-Left cell
  \node[] at (0.35,4.22) { \footnotesize $ \mathsf{ CD } $ }; % cd
  \node[] at (0.70,4.75) { \footnotesize $ \mathsf{ AB } $ }; % ab
  
  % Populate the top row header
  % In the following, the foreach lists a location/text pair
  % The the draw line draws the text at each location
  \foreach \loc/\txt in {
    (1.5,4.5)/{00},(2.5,4.5)/{01},(3.5,4.5)/{11},(4.5,4.5)/{10}
  }
  \draw \loc node{\large $\txt$};
  
  % Populate the header in column one
  \foreach \loc/\txt in { 
    (0.5,3.5)/{00},(0.5,2.5)/{01},(0.5,1.5)/{11},(0.5,0.5)/{10}
  }
  \draw \loc node{\large $\txt$};
  
  % Populate the minterms
  \foreach \loc/\txt in { 
    (1.75,3.15)/{00} , (2.75,3.15)/{04} , (3.75,3.15)/{12} , (4.75,3.15)/{08} ,
    (1.75,2.15)/{01} , (2.75,2.15)/{05} , (3.75,2.15)/{13} , (4.75,2.15)/{09} ,
    (1.75,1.15)/{03} , (2.75,1.15)/{07} , (3.75,1.15)/{15} , (4.75,1.15)/{11} ,
    (1.75,0.15)/{02} , (2.75,0.15)/{06} , (3.75,0.15)/{14} , (4.75,0.15)/{10} }
  \draw \loc node{ \color{blue!90!black} \footnotesize { $\txt$ }};
  
  % Draw the lines
  \draw
  % Finish drawing the grid
  [step=1.0cm,black,thin] (0,0) grid (5.0,5.0) % The Grid
  (0.0,5.0) -- (1.0,4.0) % Diagonal in the top left cell
  (1.0,4.05) -- (5.0,4.05) % Double line under top header row
  (0.95,0.0) -- (0.95,4.0) % Double line on left of header column one
  ;    
  \end{tikzpicture}
\end{figure}

On this map, minterm $ 13 $ is ``don't care.'' Since it could form a group of four with minterms $ 08 $, $ 09 $, and $ 12 $, and since groups of four are preferable to two groups of two, then minterm $ 13 $ should be considered a one and grouped with the other three minterms. However, minterm $ 02 $ does not group with anything, so it should be considered a zero and it would then be removed from the simplified expression altogether. This Karnaugh map would simplify to $ AC' $.

The Karnaugh map in Figure \ref{KM:fig:kmap_with_dont_care_terms_ex_2} is another example circuit with ``don't care'' terms:

%****************************************************************
% Karnaugh map With Don't Care Terms - Example 2
%****************************************************************
\begin{figure}[H]
  \caption{K-Map With ``Don't Care'' Terms, Example 2}
  \label{KM:fig:kmap_with_dont_care_terms_ex_2}
  \myfloatalign
  \begin{tikzpicture} [circuit logic US, scale=1.00]
  % make all path lines (the node shapes) a little thicker
  \tikzstyle{every path}=[line width=0.50mm]
  
  %********************************************************************
  % Adjust the settings below to display the 1's and rectangles
  %********************************************************************
  % Uncomment the appropriate lines below to insert ones where needed
  %  \node[] at (1.4,3.5) {\huge $ 1 $}; % 00
  %  \node[] at (1.4,2.5) {\huge $ 1 $}; % 01
  %  \node[] at (1.4,0.5) {\huge $ 1 $}; % 02
    \node[] at (1.4,1.5) {\huge $ 1 $}; % 03
  %  \node[] at (2.4,3.5) {\huge $ 1 $}; % 04
    \node[] at (2.4,2.5) {\huge $ 1 $}; % 05
    \node[] at (2.4,0.5) {\huge $ 1 $}; % 06
    \node[] at (2.4,1.5) {\huge $ 1 $}; % 07
  %  \node[] at (4.4,3.5) {\huge $ 1 $}; % 08
  %  \node[] at (4.4,2.5) {\huge $ 1 $}; % 09
    \node[] at (4.4,0.5) {\huge $ 1 $}; % 10
  %  \node[] at (4.4,1.5) {\huge $ 1 $}; % 11
  %  \node[] at (3.4,3.5) {\huge $ 1 $}; % 12
  %  \node[] at (3.4,2.5) {\huge $ 1 $}; % 13
    \node[] at (3.4,0.5) {\huge $ X $}; % 14
    \node[] at (3.4,1.5) {\huge $ X $}; % 15
  
  % The coords for each cell - this is used as the origin for the solution box
  \coordinate (cell00) at (1.0,3.0); \coordinate (cell01) at (1.0,2.0);
  \coordinate (cell02) at (1.0,0.0); \coordinate (cell03) at (1.0,1.0);
  
  \coordinate (cell04) at (2.0,3.0); \coordinate (cell05) at (2.0,2.0);
  \coordinate (cell06) at (2.0,0.0); \coordinate (cell07) at (2.0,1.0);
  
  \coordinate (cell12) at (3.0,3.0); \coordinate (cell13) at (3.0,2.0);
  \coordinate (cell14) at (3.0,0.0); \coordinate (cell15) at (3.0,1.0);
  
  \coordinate (cell08) at (4.0,3.0); \coordinate (cell09) at (4.0,2.0);
  \coordinate (cell10) at (4.0,0.0); \coordinate (cell11) at (4.0,1.0);
  
  % Horizontal Group
  %  \node [draw,
  %  color=red!70!black,
  %  fill=red!20!white,
  %  fill opacity=0.3,
  %  minimum height=0.95cm,
  %  minimum width=1.95cm, % Adjust to (number of cells) * 1 - 0.95
  %  rounded corners,
  %  anchor=south west] at (cell02) {}; % Enter left cell minterm
  
  % Vertical Group
  %  \node [draw,
  %  color=blue!70!black,
  %  fill=blue!20!white,
  %  fill opacity=0.3,
  %  minimum height=1.95cm, % Adjust to (number of cells) * 1 - 0.95
  %  minimum width=0.95cm,
  %  rounded corners,
  %  anchor=south west] at (cell07) {}; % Enter bottom cell minterm
  
  % Single Cell
  %  \node [draw,
  %  color=green!70!black,
  %  fill=green!20!white,
  %  fill opacity=0.3,
  %  minimum height=0.95cm,
  %  minimum width=0.95cm,
  %  rounded corners,
  %  anchor=south west] at (cell09) {}; % Enter cell minterm
  
  %********************************************************************
  % Shouldn't need to adjust anything below this point - this is just
  % the grid and the minterms.
  %********************************************************************  
  % Text in top-Left cell
  \node[] at (0.35,4.22) { \footnotesize $ \mathsf{ CD } $ }; % cd
  \node[] at (0.70,4.75) { \footnotesize $ \mathsf{ AB } $ }; % ab
  
  % Populate the top row header
  % In the following, the foreach lists a location/text pair
  % The the draw line draws the text at each location
  \foreach \loc/\txt in {
    (1.5,4.5)/{00},(2.5,4.5)/{01},(3.5,4.5)/{11},(4.5,4.5)/{10}
  }
  \draw \loc node{\large $\txt$};
  
  % Populate the header in column one
  \foreach \loc/\txt in { 
    (0.5,3.5)/{00},(0.5,2.5)/{01},(0.5,1.5)/{11},(0.5,0.5)/{10}
  }
  \draw \loc node{\large $\txt$};
  
  % Populate the minterms
  \foreach \loc/\txt in { 
    (1.75,3.15)/{00} , (2.75,3.15)/{04} , (3.75,3.15)/{12} , (4.75,3.15)/{08} ,
    (1.75,2.15)/{01} , (2.75,2.15)/{05} , (3.75,2.15)/{13} , (4.75,2.15)/{09} ,
    (1.75,1.15)/{03} , (2.75,1.15)/{07} , (3.75,1.15)/{15} , (4.75,1.15)/{11} ,
    (1.75,0.15)/{02} , (2.75,0.15)/{06} , (3.75,0.15)/{14} , (4.75,0.15)/{10} }
  \draw \loc node{ \color{blue!90!black} \footnotesize { $\txt$ }};
  
  % Draw the lines
  \draw
  % Finish drawing the grid
  [step=1.0cm,black,thin] (0,0) grid (5.0,5.0) % The Grid
  (0.0,5.0) -- (1.0,4.0) % Diagonal in the top left cell
  (1.0,4.05) -- (5.0,4.05) % Double line under top header row
  (0.95,0.0) -- (0.95,4.0) % Double line on left of header column one
  ;    
  \end{tikzpicture}
\end{figure}

This circuit would simplify to Equation \ref{KM:eq:dont_care_terms_ex_2}.

\begin{align}
  \label{KM:eq:dont_care_terms_ex_2}
  ACD'+BC+A'BC+A'CD &= Y 
\end{align}

\section{Karnaugh map Simplification Summary}
\label{KM:sec:karnaugh_map_simplification_summary}

Here are the rules for simplifying a Boolean equation using a four-variable Karnaugh map: 

\begin{enumerate}
  \item Create the map and plot all ones from the truth table output. 
  \item Circle all groups of $ 16 $. These will reduce to a constant output of one and the entire circuit is unnecessary. 
  \item Circle all groups of eight. These will reduce to a one-variable expression. 
  \item Circle all groups of four. These will reduce to a two-variable expression. The ones can be either horizontal, vertical, or in a square. 
  \item Circle all groups of two. These will reduce to a three-variable expression. The ones can be either horizontal or vertical. 
  \item Circle all ones that are not in any other group. These do not reduce and will result in a four-variable expression. 
  \item All ones must be circled at least one time. 
  \item Groups can overlap. 
  \item If ones are in more than one group, they can be considered part of either, or both, groups. 
  \item Groups can wrap around the edges to the other edge of the map.
  \item ``Don't Care'' terms can be considered either one or zero, whichever makes the map simpler. 
\end{enumerate}

\section{Practice Problems}
\label{KM:sec:practice_problems_karnaugh_maps}

\begin{table}[H]
  \sffamily
  \begin{center}
    \begin{tabular}{c c p{6cm} }
      \multirow{2}{*}{\textbf{1}} 
      & Expression (A,B) & $ A'B+AB'+AB $ \\
      & \cellcolor{gray!10} Simplified 
      & \cellcolor{gray!10} $ A+B $ \\
      \hline
      \multirow{2}{*}{\textbf{2}} 
      & Exression (A,B,C) & $ A'BC+AB'C'+ABC'+ABC $ \\
      & \cellcolor{gray!10} Simplified 
      & \cellcolor{gray!10} $ BC+AC' $ \\
      \hline
      \multirow{2}{*}{\textbf{3}} 
      & Exression (A,B,C) & $ A'BC'D+AB'CD $ \\
      & \cellcolor{gray!10} Simplified 
      & \cellcolor{gray!10} $ C+A'B $ \\
      \hline
      \multirow{2}{*}{\textbf{4}} 
      & Exression (A,B,C,D) & $ A'B'C'+B'CD'+A'BCD'+AB'C' $ \\
      & \cellcolor{gray!10} Simplified 
      & \cellcolor{gray!10} $ B'D'+B'C'+A'CD' $ \\
      \hline
      \multirow{2}{*}{\textbf{5}} 
      & Exression & $ \int(A,B,C,D) = \sum(0,1,6,7,12,13) $ \\
      & \cellcolor{gray!10} Simplified 
      & \cellcolor{gray!10} $ A'B'C'+ABC'+A'BC $ \\
      \hline
      \multirow{2}{*}{\textbf{6}} 
      & Exression & $ \int(A,B,C,D) = \prod(0,2,4,10) $ \\
      & \cellcolor{gray!10} Simplified 
      & \cellcolor{gray!10} $ D+BC+AC' $ \\
    \end{tabular}
  \end{center}
  \caption{Karnaugh maps Practice Problems}
  \label{KM:tab:kmap_practice_problems}
\end{table}

%***************************************************************************
% Section: Reed-Muller Logic
%***************************************************************************
\clearpage\section{Reed-M\"{u}ller Logic}
\label{KM:sec:reed-muller_logic}

\section{Introduction}
\label{KM:sec:introduction_to_reed-muller_logic}

Irving Reed and D.E. M\"{u}ller are noted for inventing various codes that self-correct transmission errors in the field of digital communications. However, they also formulated ways of simplifying digital logic expressions that do not easily yield to traditional methods, such as a Karnaugh map where the ones form a checkerboard pattern.

Consider the Truth Table \ref{03:tab:truth_table_for_checkerboard_pattern}.

% Table
\begin{table}[H]
  \sffamily
  \newcommand{\head}[1]{\textcolor{white}{\textbf{#1}}}    
  \begin{center}
    \rowcolors{2}{gray!10}{white} % Color every other line a light gray
    \begin{tabular}{ccc} 
      \rowcolor{black!75}
      \multicolumn{2}{c}{\head{Inputs}} & \head{Output} \\
      A & B & Y \\
      \hline
      0 & 0 & 0 \\
      0 & 1 & 1 \\
      1 & 0 & 1 \\
      1 & 1 & 0 
    \end{tabular}
  \end{center}
  \caption{Truth Table for Checkerboard Pattern}
  \label{03:tab:truth_table_for_checkerboard_pattern}
\end{table}

This pattern is easy to recognize as an \textsf{XOR} gate. The Karnaugh map for Truth Table \ref{03:tab:truth_table_for_checkerboard_pattern} is in Figure \ref{KM:fig:reed-muller_2-variable_example} (the zeros have been omitted and the cells with ones have been shaded to emphasize the checkerboard pattern): 

%****************************************************************
% Karnaugh map For 2-Input Circuit Showing Checkerboard Pattern
%****************************************************************
\begin{figure}[H]
  \caption{Reed-M\"{u}ller Two-Variable Example}
  \label{KM:fig:reed-muller_2-variable_example}
  \myfloatalign
  \begin{tikzpicture} [circuit logic US, scale=1.00]
  % make all path lines (the node shapes) a little thicker
  \tikzstyle{every path}=[line width=0.50mm]
  
  %********************************************************************
  % Adjust the settings below to display the 1's and rectangles
  %********************************************************************
  % Uncomment the appropriate lines below to insert ones where needed
%  \node[] at (1.4,1.5) {\huge $ 1 $}; % 00
  \node[] at (1.4,0.5) {\huge $ 1 $}; % 01
  \node[] at (2.4,1.5) {\huge $ 1 $}; % 02
%  \node[] at (2.4,0.5) {\huge $ 1 $}; % 03
  
  % The coords for each cell - this is used as the origin for the cell color
  \coordinate (cell00) at (1.0,1.0); \coordinate (cell01) at (1.0,0.0);
  \coordinate (cell02) at (2.0,1.0); \coordinate (cell03) at (2.0,0.0);

  %********************************************************************
  % Draw the Shaded Cells
  %********************************************************************
  \foreach \loc in { 
    (cell01),(cell02) 
  }
  \node [draw,
  color=green!70!black,
  fill=green!20!white,
  fill opacity=0.3,
  minimum height=0.95cm,
  minimum width=0.95cm,
  anchor=south west] at \loc {};
    
  %********************************************************************
  % Shouldn't need to adjust anything below this point - this is just
  % the grid and the minterms.
  %********************************************************************  
  % Text in top-Left cell
  \node[] at (0.35,2.22) { \footnotesize $ \mathsf{ B } $ }; % B
  \node[] at (0.70,2.75) { \footnotesize $ \mathsf{ A } $ }; % A
  
  % Populate the top row header
  % In the following, the foreach lists a location/text pair
  % The the draw line draws the text at each location
  \foreach \loc/\txt in {
    (1.5,2.5)/{0},(2.5,2.5)/{1}
  }
  \draw \loc node{\large $\txt$};
  
  % Populate the header in column one
  \foreach \loc/\txt in { 
    (0.5,1.5)/{0},(0.5,0.5)/{1}
  }
  \draw \loc node{\large $\txt$};
  
  % Populate the minterms
  \foreach \loc/\txt in { 
    (1.75,1.15)/{00} , (2.75,1.15)/{02} , (1.75,0.15)/{01} , (2.75,0.15)/{03} 
  }
  \draw \loc node{ \color{blue!90!black} \footnotesize { $\txt$ }};
  
  % Draw the lines
  \draw
  % Finish drawing the grid
  [step=1.0cm,black,thin] (0,0) grid (3.0,3.0) % The Grid
  (0.0,3.0) -- (1.0,2.0) % Diagonal in the top left cell
  (1.0,2.05) -- (3.0,2.05) % Double line under top header row
  (0.95,0.0) -- (0.95,2.0) % Double line on left of header column one
  ;
  \end{tikzpicture}
\end{figure}

Equation \ref{KM:eq:reed-muller_ex_01} describes this circuit.

\begin{align}
  \label{KM:eq:reed-muller_ex_01}
  AB'+A'B &= Y 
\end{align}

This equation cannot be simplified using common Karnaugh map simplification techniques since none of the ones are adjacent vertically or horizontally. However, whenever a Karnaugh map displays a checkerboard pattern, the circuit can be simplified using \textsf{XOR} or \textsf{XNOR} gates. 

\section{Zero In First Cell}
\label{KM:sec:zero_in_first_cell}

Karnaugh maps with a zero in the first cell (that is, in a four-variable map, $ A'B'C'D' $ is \emph{False}) are simplified in a slightly different manner than those with a one in that cell. This section describes the technique used for maps with a zero in the first cell and Section \ref{KM:sec:one_in_first_cell} (page \pageref{KM:sec:one_in_first_cell}) describes the technique for maps with a one in the first cell. 

With a zero in the first cell, the equation for the Karnaugh map is generated by:

\begin{enumerate}
  \item Grouping the ones into horizontal, vertical, or square groups if possible 
  \item Identifying the variable in each group that is \emph{True} and does not change 
  \item Combining those groups with an \textsf{XOR} gate 
\end{enumerate}

\subsection{Two-Variable Circuit}
\label{KM:subsec:two_variable_circuit}

In Karnaugh map \ref{KM:fig:reed-muller_2-variable_example}, it is not possible to group the ones. Since both $ A $ and $ B $ are \emph{True} in at least one cell, the equation for that circuit is: 

\begin{align}
  \label{KM:eq:reed-muller_2_vars_ex_1}
  A \oplus B &= Y 
\end{align}

\subsection{Three-Variable Circuit}
\label{KM:subsec:three_variable_circuit}

The Karnaugh map in Figure \ref{KM:fig:reed-muller_3-variable_example_1} is for a circuit containing three inputs.

%****************************************************************
% Reed-Muller 3-Variable Example 1
%****************************************************************
\begin{figure}[H]
  \caption{Reed-M\"{u}ller Three-Variable Example 1}
  \label{KM:fig:reed-muller_3-variable_example_1}
  \myfloatalign
  \begin{tikzpicture} [circuit logic US, scale=1.00]
  % make all path lines (the node shapes) a little thicker
  \tikzstyle{every path}=[line width=0.50mm]
  
  %********************************************************************
  % Adjust the settings below to display the 1's and rectangles
  %********************************************************************
  % Uncomment the appropriate lines below to insert ones where needed
%  \node[] at (1.4,1.5) {\huge $ 1 $}; % 00
  \node[] at (1.4,0.5) {\huge $ 1 $}; % 01
  \node[] at (2.4,1.5) {\huge $ 1 $}; % 02
%  \node[] at (2.4,0.5) {\huge $ 1 $}; % 03
  \node[] at (4.4,1.5) {\huge $ 1 $}; % 04
%  \node[] at (4.4,0.5) {\huge $ 1 $}; % 05
%  \node[] at (3.4,1.5) {\huge $ 1 $}; % 06
  \node[] at (3.4,0.5) {\huge $ 1 $}; % 07
  
  % The coords for each cell - this is used as the origin for the solution box
  \coordinate (cell00) at (1.0,1.0); \coordinate (cell01) at (1.0,0.0);
  \coordinate (cell02) at (2.0,1.0); \coordinate (cell03) at (2.0,0.0);
  
  \coordinate (cell04) at (4.0,1.0); \coordinate (cell05) at (4.0,0.0);
  \coordinate (cell06) at (3.0,1.0); \coordinate (cell07) at (3.0,0.0);
  
  %********************************************************************
  % Draw the Shaded Cells
  %********************************************************************
  \foreach \loc in { 
    (cell01),(cell02),(cell07),(cell04) 
  }
  \node [draw,
  color=green!70!black,
  fill=green!20!white,
  fill opacity=0.3,
  minimum height=0.95cm,
  minimum width=0.95cm,
  anchor=south west] at \loc {};

  %********************************************************************
  % Shouldn't need to adjust anything below this point - this is just
  % the grid and the minterms.
  %********************************************************************  
  % Text in top-Left cell
  \node[] at (0.35,2.22) { \footnotesize $ \mathsf{ C } $ }; % C
  \node[] at (0.70,2.75) { \footnotesize $ \mathsf{ AB } $ }; % ab
  
  % Populate the top row header
  % In the following, the foreach lists a location/text pair
  % The the draw line draws the text at each location
  \foreach \loc/\txt in {
    (1.5,2.5)/{00},(2.5,2.5)/{01},(3.5,2.5)/{11},(4.5,2.5)/{10}
  }
  \draw \loc node{\large $\txt$};
  
  % Populate the header in column one
  \foreach \loc/\txt in { 
    (0.5,1.5)/{0},(0.5,0.5)/{1}
  }
  \draw \loc node{\large $\txt$};
  
  % Populate the minterms
  \foreach \loc/\txt in { 
    (1.75,1.15)/{00} , (2.75,1.15)/{02} , (3.75,1.15)/{06} , (4.75,1.15)/{04} ,
    (1.75,0.15)/{01} , (2.75,0.15)/{03} , (3.75,0.15)/{07} , (4.75,0.15)/{05} }
  \draw \loc node{ \color{blue!90!black} \footnotesize { $\txt$ }};
  
  % Draw the lines
  \draw
  % Finish drawing the grid
  [step=1.0cm,black,thin] (0,0) grid (5.0,3.0) % The Grid
  (0.0,3.0) -- (1.0,2.0) % Diagonal in the top left cell
  (1.0,2.05) -- (5.0,2.05) % Double line under top header row
  (0.95,0.0) -- (0.95,2.0) % Double line on left of header column one
  ;    
  \end{tikzpicture}
\end{figure}

In this Karnaugh map, it is not possible to group the ones. Since $ A $, $ B $, and $ C $ are all \emph{True} in at least one cell, the equation for the circuit is: 

\begin{align}
  \label{KM:eq:reed-muller_3_vars_ex_1}
  A \oplus B \oplus C &= Y 
\end{align}

The Karnaugh map in Figure \ref{KM:fig:reed-muller_3-variable_example_2} is more interesting.

%****************************************************************
% Reed-Muller 3-Variable Example 2
%****************************************************************
\begin{figure}[H]
  \caption{Reed-M\"{u}ller Three-Variable Example 2}
  \label{KM:fig:reed-muller_3-variable_example_2}
  \myfloatalign
  \begin{tikzpicture} [circuit logic US, scale=1.00]
  % make all path lines (the node shapes) a little thicker
  \tikzstyle{every path}=[line width=0.50mm]
  
  %********************************************************************
  % Adjust the settings below to display the 1's and rectangles
  %********************************************************************
  % Uncomment the appropriate lines below to insert ones where needed
%  \node[] at (1.4,1.5) {\huge $ 1 $}; % 00
  \node[] at (1.4,0.5) {\huge $ 1 $}; % 01
%  \node[] at (2.4,1.5) {\huge $ 1 $}; % 02
  \node[] at (2.4,0.5) {\huge $ 1 $}; % 03
  \node[] at (4.4,1.5) {\huge $ 1 $}; % 04
%  \node[] at (4.4,0.5) {\huge $ 1 $}; % 05
  \node[] at (3.4,1.5) {\huge $ 1 $}; % 06
%  \node[] at (3.4,0.5) {\huge $ 1 $}; % 07
  
  % The coords for each cell - this is used as the origin for the solution box
  \coordinate (cell00) at (1.0,1.0); \coordinate (cell01) at (1.0,0.0);
  \coordinate (cell02) at (2.0,1.0); \coordinate (cell03) at (2.0,0.0);
  
  \coordinate (cell04) at (4.0,1.0); \coordinate (cell05) at (4.0,0.0);
  \coordinate (cell06) at (3.0,1.0); \coordinate (cell07) at (3.0,0.0);
  
  %********************************************************************
  % Draw the Shaded Cells
  %********************************************************************
  \foreach \loc in { 
    (cell01),(cell03),(cell06),(cell04) 
  }
  \node [draw,
  color=green!70!black,
  fill=green!20!white,
  fill opacity=0.3,
  minimum height=0.95cm,
  minimum width=0.95cm,
  anchor=south west] at \loc {};
  
  %********************************************************************
  % Shouldn't need to adjust anything below this point - this is just
  % the grid and the minterms.
  %********************************************************************  
  % Text in top-Left cell
  \node[] at (0.35,2.22) { \footnotesize $ \mathsf{ C } $ }; % C
  \node[] at (0.70,2.75) { \footnotesize $ \mathsf{ AB } $ }; % ab
  
  % Populate the top row header
  % In the following, the foreach lists a location/text pair
  % The the draw line draws the text at each location
  \foreach \loc/\txt in {
    (1.5,2.5)/{00},(2.5,2.5)/{01},(3.5,2.5)/{11},(4.5,2.5)/{10}
  }
  \draw \loc node{\large $\txt$};
  
  % Populate the header in column one
  \foreach \loc/\txt in { 
    (0.5,1.5)/{0},(0.5,0.5)/{1}
  }
  \draw \loc node{\large $\txt$};
  
  % Populate the minterms
  \foreach \loc/\txt in { 
    (1.75,1.15)/{00} , (2.75,1.15)/{02} , (3.75,1.15)/{06} , (4.75,1.15)/{04} ,
    (1.75,0.15)/{01} , (2.75,0.15)/{03} , (3.75,0.15)/{07} , (4.75,0.15)/{05} }
  \draw \loc node{ \color{blue!90!black} \footnotesize { $\txt$ }};
  
  % Draw the lines
  \draw
  % Finish drawing the grid
  [step=1.0cm,black,thin] (0,0) grid (5.0,3.0) % The Grid
  (0.0,3.0) -- (1.0,2.0) % Diagonal in the top left cell
  (1.0,2.05) -- (5.0,2.05) % Double line under top header row
  (0.95,0.0) -- (0.95,2.0) % Double line on left of header column one
  ;    
  \end{tikzpicture}
\end{figure}

In this Karnaugh map, the ones form two groups in a checkerboard pattern. If this map were to be simplified using common techniques it would form Equation \ref{KM:eq:reed-muller_3_vars_ex_2_ver_1}.

\begin{align}
  \label{KM:eq:reed-muller_3_vars_ex_2_ver_1}
  AC'+A'C &= Y 
\end{align}

If realized, this circuit would contain two two-input \textsf{AND} gates joined by a two-input \textsf{OR} gate. However, this equation can be simplified using the Reed-M\"{u}ller technique. For the group in the upper right corner $ A $ is a constant one and for the group in the lower left corner $ C $ is a constant one. The equation simplifies to \ref{KM:eq:reed-muller_3_vars_ex_2_ver_2}. 

\begin{align}
  \label{KM:eq:reed-muller_3_vars_ex_2_ver_2}
  A \oplus C &= Y 
\end{align}

If realized, this circuit would contain nothing more than a two-input \textsf{XOR} gate and is much simpler than the first attempt.

\subsection{Four-Variable Circuit}
\label{KM:subsec:four_variable_circuit}

The Karnaugh map in Figure \ref{KM:fig:reed-muller_4-variable_example_1} is for a circuit containing four variable inputs.

%****************************************************************
% Reed-Muller 4-Variable Example 1
%****************************************************************
\begin{figure}[H]
  \caption{Reed-M\"{u}ller Four-Variable Example 1}
  \label{KM:fig:reed-muller_4-variable_example_1}
  \myfloatalign
  \begin{tikzpicture} [circuit logic US, scale=1.00]
  % make all path lines (the node shapes) a little thicker
  \tikzstyle{every path}=[line width=0.50mm]
  
  %********************************************************************
  % Adjust the settings below to display the 1's and rectangles
  %********************************************************************
  % Uncomment the appropriate lines below to insert ones where needed
%  \node[] at (1.4,3.5) {\huge $ 1 $}; % 00
  \node[] at (1.4,2.5) {\huge $ 1 $}; % 01
  \node[] at (1.4,0.5) {\huge $ 1 $}; % 02
%  \node[] at (1.4,1.5) {\huge $ 1 $}; % 03
  \node[] at (2.4,3.5) {\huge $ 1 $}; % 04
%  \node[] at (2.4,2.5) {\huge $ 1 $}; % 05
%  \node[] at (2.4,0.5) {\huge $ 1 $}; % 06
  \node[] at (2.4,1.5) {\huge $ 1 $}; % 07
  \node[] at (4.4,3.5) {\huge $ 1 $}; % 08
%  \node[] at (4.4,2.5) {\huge $ 1 $}; % 09
%  \node[] at (4.4,0.5) {\huge $ 1 $}; % 10
  \node[] at (4.4,1.5) {\huge $ 1 $}; % 11
%  \node[] at (3.4,3.5) {\huge $ 1 $}; % 12
  \node[] at (3.4,2.5) {\huge $ 1 $}; % 13
  \node[] at (3.4,0.5) {\huge $ 1 $}; % 14
%  \node[] at (3.4,1.5) {\huge $ 1 $}; % 15
  
  % The coords for each cell - this is used as the origin for the solution box
  \coordinate (cell00) at (1.0,3.0); \coordinate (cell01) at (1.0,2.0);
  \coordinate (cell02) at (1.0,0.0); \coordinate (cell03) at (1.0,1.0);
  
  \coordinate (cell04) at (2.0,3.0); \coordinate (cell05) at (2.0,2.0);
  \coordinate (cell06) at (2.0,0.0); \coordinate (cell07) at (2.0,1.0);
  
  \coordinate (cell12) at (3.0,3.0); \coordinate (cell13) at (3.0,2.0);
  \coordinate (cell14) at (3.0,0.0); \coordinate (cell15) at (3.0,1.0);
  
  \coordinate (cell08) at (4.0,3.0); \coordinate (cell09) at (4.0,2.0);
  \coordinate (cell10) at (4.0,0.0); \coordinate (cell11) at (4.0,1.0);

  %********************************************************************
  % Draw the Shaded Cells
  %********************************************************************
  \foreach \loc in { 
    (cell01),(cell02),(cell04),(cell07),(cell13),(cell14),(cell08),(cell11) 
  }
  \node [draw,
  color=green!70!black,
  fill=green!20!white,
  fill opacity=0.3,
  minimum height=0.95cm,
  minimum width=0.95cm,
  anchor=south west] at \loc {};
  
  %********************************************************************
  % Shouldn't need to adjust anything below this point - this is just
  % the grid and the minterms.
  %********************************************************************  
  % Text in top-Left cell
  \node[] at (0.35,4.22) { \footnotesize $ \mathsf{ CD } $ }; % cd
  \node[] at (0.70,4.75) { \footnotesize $ \mathsf{ AB } $ }; % ab
  
  % Populate the top row header
  % In the following, the foreach lists a location/text pair
  % The the draw line draws the text at each location
  \foreach \loc/\txt in {
    (1.5,4.5)/{00},(2.5,4.5)/{01},(3.5,4.5)/{11},(4.5,4.5)/{10}
  }
  \draw \loc node{\large $\txt$};
  
  % Populate the header in column one
  \foreach \loc/\txt in { 
    (0.5,3.5)/{00},(0.5,2.5)/{01},(0.5,1.5)/{11},(0.5,0.5)/{10}
  }
  \draw \loc node{\large $\txt$};
  
  % Populate the minterms
  \foreach \loc/\txt in { 
    (1.75,3.15)/{00} , (2.75,3.15)/{04} , (3.75,3.15)/{12} , (4.75,3.15)/{08} ,
    (1.75,2.15)/{01} , (2.75,2.15)/{05} , (3.75,2.15)/{13} , (4.75,2.15)/{09} ,
    (1.75,1.15)/{03} , (2.75,1.15)/{07} , (3.75,1.15)/{15} , (4.75,1.15)/{11} ,
    (1.75,0.15)/{02} , (2.75,0.15)/{06} , (3.75,0.15)/{14} , (4.75,0.15)/{10} }
  \draw \loc node{ \color{blue!90!black} \footnotesize { $\txt$ }};
  
  % Draw the lines
  \draw
  % Finish drawing the grid
  [step=1.0cm,black,thin] (0,0) grid (5.0,5.0) % The Grid
  (0.0,5.0) -- (1.0,4.0) % Diagonal in the top left cell
  (1.0,4.05) -- (5.0,4.05) % Double line under top header row
  (0.95,0.0) -- (0.95,4.0) % Double line on left of header column one
  ;    
  \end{tikzpicture}
\end{figure}

In this Karnaugh map, it is not possible to group the ones. Since $ A $, $ B $, $ C $, and $ D $ are all \emph{True} in at least one cell, Equation \ref{KM:eq:reed-muller_4_vars_ex_1} would describe the circuit.

\begin{align}
  \label{KM:eq:reed-muller_4_vars_ex_1}
  A \oplus B \oplus C \oplus D &= Y 
\end{align}

Following are more interesting four-variable Karnaugh maps with groups of ones:

%****************************************************************
% Reed-Muller 4-Variable Example 2
%****************************************************************
\begin{figure}[H]
  \caption{Reed-M\"{u}ller Four-Variable Example 2}
  \label{KM:fig:reed-muller_4-variable_example_2}
  \myfloatalign
  \begin{tikzpicture} [circuit logic US, scale=1.00]
  % make all path lines (the node shapes) a little thicker
  \tikzstyle{every path}=[line width=0.50mm]
  
  %********************************************************************
  % Adjust the settings below to display the 1's and rectangles
  %********************************************************************
  % Uncomment the appropriate lines below to insert ones where needed
%  \node[] at (1.4,3.5) {\huge $ 1 $}; % 00
%  \node[] at (1.4,2.5) {\huge $ 1 $}; % 01
  \node[] at (1.4,0.5) {\huge $ 1 $}; % 02
  \node[] at (1.4,1.5) {\huge $ 1 $}; % 03
%  \node[] at (2.4,3.5) {\huge $ 1 $}; % 04
%  \node[] at (2.4,2.5) {\huge $ 1 $}; % 05
  \node[] at (2.4,0.5) {\huge $ 1 $}; % 06
  \node[] at (2.4,1.5) {\huge $ 1 $}; % 07
  \node[] at (4.4,3.5) {\huge $ 1 $}; % 08
  \node[] at (4.4,2.5) {\huge $ 1 $}; % 09
%  \node[] at (4.4,0.5) {\huge $ 1 $}; % 10
%  \node[] at (4.4,1.5) {\huge $ 1 $}; % 11
  \node[] at (3.4,3.5) {\huge $ 1 $}; % 12
  \node[] at (3.4,2.5) {\huge $ 1 $}; % 13
%  \node[] at (3.4,0.5) {\huge $ 1 $}; % 14
%  \node[] at (3.4,1.5) {\huge $ 1 $}; % 15
  
  % The coords for each cell - this is used as the origin for the solution box
  \coordinate (cell00) at (1.0,3.0); \coordinate (cell01) at (1.0,2.0);
  \coordinate (cell02) at (1.0,0.0); \coordinate (cell03) at (1.0,1.0);
  
  \coordinate (cell04) at (2.0,3.0); \coordinate (cell05) at (2.0,2.0);
  \coordinate (cell06) at (2.0,0.0); \coordinate (cell07) at (2.0,1.0);
  
  \coordinate (cell12) at (3.0,3.0); \coordinate (cell13) at (3.0,2.0);
  \coordinate (cell14) at (3.0,0.0); \coordinate (cell15) at (3.0,1.0);
  
  \coordinate (cell08) at (4.0,3.0); \coordinate (cell09) at (4.0,2.0);
  \coordinate (cell10) at (4.0,0.0); \coordinate (cell11) at (4.0,1.0);
  
  %********************************************************************
  % Draw the Shaded Cells
  %********************************************************************
  \foreach \loc in { 
    (cell02),(cell03),(cell06),(cell07),(cell08),(cell09),(cell12),(cell13) 
  }
  \node [draw,
  color=green!70!black,
  fill=green!20!white,
  fill opacity=0.3,
  minimum height=0.95cm,
  minimum width=0.95cm,
  anchor=south west] at \loc {};
  
  %********************************************************************
  % Shouldn't need to adjust anything below this point - this is just
  % the grid and the minterms.
  %********************************************************************  
  % Text in top-Left cell
  \node[] at (0.35,4.22) { \footnotesize $ \mathsf{ CD } $ }; % cd
  \node[] at (0.70,4.75) { \footnotesize $ \mathsf{ AB } $ }; % ab
  
  % Populate the top row header
  % In the following, the foreach lists a location/text pair
  % The the draw line draws the text at each location
  \foreach \loc/\txt in {
    (1.5,4.5)/{00},(2.5,4.5)/{01},(3.5,4.5)/{11},(4.5,4.5)/{10}
  }
  \draw \loc node{\large $\txt$};
  
  % Populate the header in column one
  \foreach \loc/\txt in { 
    (0.5,3.5)/{00},(0.5,2.5)/{01},(0.5,1.5)/{11},(0.5,0.5)/{10}
  }
  \draw \loc node{\large $\txt$};
  
  % Populate the minterms
  \foreach \loc/\txt in { 
    (1.75,3.15)/{00} , (2.75,3.15)/{04} , (3.75,3.15)/{12} , (4.75,3.15)/{08} ,
    (1.75,2.15)/{01} , (2.75,2.15)/{05} , (3.75,2.15)/{13} , (4.75,2.15)/{09} ,
    (1.75,1.15)/{03} , (2.75,1.15)/{07} , (3.75,1.15)/{15} , (4.75,1.15)/{11} ,
    (1.75,0.15)/{02} , (2.75,0.15)/{06} , (3.75,0.15)/{14} , (4.75,0.15)/{10} }
  \draw \loc node{ \color{blue!90!black} \footnotesize { $\txt$ }};
  
  % Draw the lines
  \draw
  % Finish drawing the grid
  [step=1.0cm,black,thin] (0,0) grid (5.0,5.0) % The Grid
  (0.0,5.0) -- (1.0,4.0) % Diagonal in the top left cell
  (1.0,4.05) -- (5.0,4.05) % Double line under top header row
  (0.95,0.0) -- (0.95,4.0) % Double line on left of header column one
  ;    
  \end{tikzpicture}
\end{figure}

In the Karnaugh map in Figure \ref{KM:fig:reed-muller_4-variable_example_2}, the group of ones in the upper right corner has a constant one for $ A $ and the group in the lower left corner has a constant one for $ C $, so Equation \ref{KM:eq:reed-muller_4_vars_ex_2} would describe the circuit.

\begin{align}
  \label{KM:eq:reed-muller_4_vars_ex_2}
  A \oplus C &= Y 
\end{align}

%****************************************************************
% Reed-Muller 4-Variable Example 3
%****************************************************************
\begin{figure}[H]
  \caption{Reed-M\"{u}ller Four-Variable Example 3}
  \label{KM:fig:reed-muller_4-variable_example_3}
  \myfloatalign
  \begin{tikzpicture} [circuit logic US, scale=1.00]
  % make all path lines (the node shapes) a little thicker
  \tikzstyle{every path}=[line width=0.50mm]
  
  %********************************************************************
  % Adjust the settings below to display the 1's and rectangles
  %********************************************************************
  % Uncomment the appropriate lines below to insert ones where needed
%  \node[] at (1.4,3.5) {\huge $ 1 $}; % 00
%  \node[] at (1.4,2.5) {\huge $ 1 $}; % 01
%  \node[] at (1.4,0.5) {\huge $ 1 $}; % 02
%  \node[] at (1.4,1.5) {\huge $ 1 $}; % 03
  \node[] at (2.4,3.5) {\huge $ 1 $}; % 04
  \node[] at (2.4,2.5) {\huge $ 1 $}; % 05
  \node[] at (2.4,0.5) {\huge $ 1 $}; % 06
  \node[] at (2.4,1.5) {\huge $ 1 $}; % 07
  \node[] at (4.4,3.5) {\huge $ 1 $}; % 08
  \node[] at (4.4,2.5) {\huge $ 1 $}; % 09
  \node[] at (4.4,0.5) {\huge $ 1 $}; % 10
  \node[] at (4.4,1.5) {\huge $ 1 $}; % 11
%  \node[] at (3.4,3.5) {\huge $ 1 $}; % 12
%  \node[] at (3.4,2.5) {\huge $ 1 $}; % 13
%  \node[] at (3.4,0.5) {\huge $ 1 $}; % 14
%  \node[] at (3.4,1.5) {\huge $ 1 $}; % 15
  
  % The coords for each cell - this is used as the origin for the solution box
  \coordinate (cell00) at (1.0,3.0); \coordinate (cell01) at (1.0,2.0);
  \coordinate (cell02) at (1.0,0.0); \coordinate (cell03) at (1.0,1.0);
  
  \coordinate (cell04) at (2.0,3.0); \coordinate (cell05) at (2.0,2.0);
  \coordinate (cell06) at (2.0,0.0); \coordinate (cell07) at (2.0,1.0);
  
  \coordinate (cell12) at (3.0,3.0); \coordinate (cell13) at (3.0,2.0);
  \coordinate (cell14) at (3.0,0.0); \coordinate (cell15) at (3.0,1.0);
  
  \coordinate (cell08) at (4.0,3.0); \coordinate (cell09) at (4.0,2.0);
  \coordinate (cell10) at (4.0,0.0); \coordinate (cell11) at (4.0,1.0);
  
  %********************************************************************
  % Draw the Shaded Cells
  %********************************************************************
  \foreach \loc in { 
    (cell04),(cell05),(cell06),(cell07),(cell08),(cell09),(cell10),(cell11) 
  }
  \node [draw,
  color=green!70!black,
  fill=green!20!white,
  fill opacity=0.3,
  minimum height=0.95cm,
  minimum width=0.95cm,
  anchor=south west] at \loc {};
  
  %********************************************************************
  % Shouldn't need to adjust anything below this point - this is just
  % the grid and the minterms.
  %********************************************************************  
  % Text in top-Left cell
  \node[] at (0.35,4.22) { \footnotesize $ \mathsf{ CD } $ }; % cd
  \node[] at (0.70,4.75) { \footnotesize $ \mathsf{ AB } $ }; % ab
  
  % Populate the top row header
  % In the following, the foreach lists a location/text pair
  % The the draw line draws the text at each location
  \foreach \loc/\txt in {
    (1.5,4.5)/{00},(2.5,4.5)/{01},(3.5,4.5)/{11},(4.5,4.5)/{10}
  }
  \draw \loc node{\large $\txt$};
  
  % Populate the header in column one
  \foreach \loc/\txt in { 
    (0.5,3.5)/{00},(0.5,2.5)/{01},(0.5,1.5)/{11},(0.5,0.5)/{10}
  }
  \draw \loc node{\large $\txt$};
  
  % Populate the minterms
  \foreach \loc/\txt in { 
    (1.75,3.15)/{00} , (2.75,3.15)/{04} , (3.75,3.15)/{12} , (4.75,3.15)/{08} ,
    (1.75,2.15)/{01} , (2.75,2.15)/{05} , (3.75,2.15)/{13} , (4.75,2.15)/{09} ,
    (1.75,1.15)/{03} , (2.75,1.15)/{07} , (3.75,1.15)/{15} , (4.75,1.15)/{11} ,
    (1.75,0.15)/{02} , (2.75,0.15)/{06} , (3.75,0.15)/{14} , (4.75,0.15)/{10} }
  \draw \loc node{ \color{blue!90!black} \footnotesize { $\txt$ }};
  
  % Draw the lines
  \draw
  % Finish drawing the grid
  [step=1.0cm,black,thin] (0,0) grid (5.0,5.0) % The Grid
  (0.0,5.0) -- (1.0,4.0) % Diagonal in the top left cell
  (1.0,4.05) -- (5.0,4.05) % Double line under top header row
  (0.95,0.0) -- (0.95,4.0) % Double line on left of header column one
  ;    
  \end{tikzpicture}
\end{figure}

In the Karnaugh map in Figure \ref{KM:fig:reed-muller_4-variable_example_3}, the group of ones in the first shaded column has a constant one for $ B $ and in the second shaded column have a constant one for $ A $, so Equation \ref{KM:eq:reed-muller_4_vars_ex_3} describes this circuit.

\begin{align}
  \label{KM:eq:reed-muller_4_vars_ex_3}
  A \oplus B &= Y 
\end{align}

%****************************************************************
% Reed-Muller 4-Variable Example 4
%****************************************************************
\begin{figure}[H]
  \caption{Reed-M\"{u}ller Four-Variable Example 4}
  \label{KM:fig:reed-muller_4-variable_example_4}
  \myfloatalign
  \begin{tikzpicture} [circuit logic US, scale=1.00]
  % make all path lines (the node shapes) a little thicker
  \tikzstyle{every path}=[line width=0.50mm]
  
  %********************************************************************
  % Adjust the settings below to display the 1's and rectangles
  %********************************************************************
  % Uncomment the appropriate lines below to insert ones where needed
%  \node[] at (1.4,3.5) {\huge $ 1 $}; % 00
  \node[] at (1.4,2.5) {\huge $ 1 $}; % 01
%  \node[] at (1.4,0.5) {\huge $ 1 $}; % 02
  \node[] at (1.4,1.5) {\huge $ 1 $}; % 03
  \node[] at (2.4,3.5) {\huge $ 1 $}; % 04
%  \node[] at (2.4,2.5) {\huge $ 1 $}; % 05
  \node[] at (2.4,0.5) {\huge $ 1 $}; % 06
%  \node[] at (2.4,1.5) {\huge $ 1 $}; % 07
%  \node[] at (4.4,3.5) {\huge $ 1 $}; % 08
  \node[] at (4.4,2.5) {\huge $ 1 $}; % 09
%  \node[] at (4.4,0.5) {\huge $ 1 $}; % 10
  \node[] at (4.4,1.5) {\huge $ 1 $}; % 11
  \node[] at (3.4,3.5) {\huge $ 1 $}; % 12
%  \node[] at (3.4,2.5) {\huge $ 1 $}; % 13
  \node[] at (3.4,0.5) {\huge $ 1 $}; % 14
%  \node[] at (3.4,1.5) {\huge $ 1 $}; % 15
  
  % The coords for each cell - this is used as the origin for the solution box
  \coordinate (cell00) at (1.0,3.0); \coordinate (cell01) at (1.0,2.0);
  \coordinate (cell02) at (1.0,0.0); \coordinate (cell03) at (1.0,1.0);
  
  \coordinate (cell04) at (2.0,3.0); \coordinate (cell05) at (2.0,2.0);
  \coordinate (cell06) at (2.0,0.0); \coordinate (cell07) at (2.0,1.0);
  
  \coordinate (cell12) at (3.0,3.0); \coordinate (cell13) at (3.0,2.0);
  \coordinate (cell14) at (3.0,0.0); \coordinate (cell15) at (3.0,1.0);
  
  \coordinate (cell08) at (4.0,3.0); \coordinate (cell09) at (4.0,2.0);
  \coordinate (cell10) at (4.0,0.0); \coordinate (cell11) at (4.0,1.0);
  
  %********************************************************************
  % Draw the Shaded Cells
  %********************************************************************
  \foreach \loc in { 
    (cell01),(cell03),(cell04),(cell06),(cell09),(cell11),(cell12),(cell14) 
  }
  \node [draw,
  color=green!70!black,
  fill=green!20!white,
  fill opacity=0.3,
  minimum height=0.95cm,
  minimum width=0.95cm,
  anchor=south west] at \loc {};
  
  %********************************************************************
  % Shouldn't need to adjust anything below this point - this is just
  % the grid and the minterms.
  %********************************************************************  
  % Text in top-Left cell
  \node[] at (0.35,4.22) { \footnotesize $ \mathsf{ CD } $ }; % cd
  \node[] at (0.70,4.75) { \footnotesize $ \mathsf{ AB } $ }; % ab
  
  % Populate the top row header
  % In the following, the foreach lists a location/text pair
  % The the draw line draws the text at each location
  \foreach \loc/\txt in {
    (1.5,4.5)/{00},(2.5,4.5)/{01},(3.5,4.5)/{11},(4.5,4.5)/{10}
  }
  \draw \loc node{\large $\txt$};
  
  % Populate the header in column one
  \foreach \loc/\txt in { 
    (0.5,3.5)/{00},(0.5,2.5)/{01},(0.5,1.5)/{11},(0.5,0.5)/{10}
  }
  \draw \loc node{\large $\txt$};
  
  % Populate the minterms
  \foreach \loc/\txt in { 
    (1.75,3.15)/{00} , (2.75,3.15)/{04} , (3.75,3.15)/{12} , (4.75,3.15)/{08} ,
    (1.75,2.15)/{01} , (2.75,2.15)/{05} , (3.75,2.15)/{13} , (4.75,2.15)/{09} ,
    (1.75,1.15)/{03} , (2.75,1.15)/{07} , (3.75,1.15)/{15} , (4.75,1.15)/{11} ,
    (1.75,0.15)/{02} , (2.75,0.15)/{06} , (3.75,0.15)/{14} , (4.75,0.15)/{10} }
  \draw \loc node{ \color{blue!90!black} \footnotesize { $\txt$ }};
  
  % Draw the lines
  \draw
  % Finish drawing the grid
  [step=1.0cm,black,thin] (0,0) grid (5.0,5.0) % The Grid
  (0.0,5.0) -- (1.0,4.0) % Diagonal in the top left cell
  (1.0,4.05) -- (5.0,4.05) % Double line under top header row
  (0.95,0.0) -- (0.95,4.0) % Double line on left of header column one
  ;    
  \end{tikzpicture}
\end{figure}

Keep in mind that groups can wrap around the edges in a Karnaugh map. The group of ones in columns 1 and 4 combine and has a constant one for $ D $. The group of ones in rows 1 and 4 combine and has a constant one for $ B $, so Equation \ref{KM:eq:reed-muller_4_vars_ex_4} describes this circuit.

\begin{align}
  \label{KM:eq:reed-muller_4_vars_ex_4}
  B \oplus D &= Y 
\end{align}

It is interesting that all of the above examples used \textsf{XOR} gates to combine the constant \emph{True} variable found in groups of ones; however, it would yield the same result if \textsf{XOR} gates combined the constant \emph{False} variable found in groups of ones. The designer could choose either, but must be consistent. For example, consider the Karnaugh map in Figure \ref{KM:fig:reed-muller_4-variable_example_5}. 

%****************************************************************
% Reed-Muller 4-Variable Example 5
%****************************************************************
\begin{figure}[H]
  \caption{Reed-M\"{u}ller Four-Variable Example 5}
  \label{KM:fig:reed-muller_4-variable_example_5}
  \myfloatalign
  \begin{tikzpicture} [circuit logic US, scale=1.00]
  % make all path lines (the node shapes) a little thicker
  \tikzstyle{every path}=[line width=0.50mm]
  
  %********************************************************************
  % Adjust the settings below to display the 1's and rectangles
  %********************************************************************
  % Uncomment the appropriate lines below to insert ones where needed
  %  \node[] at (1.4,3.5) {\huge $ 1 $}; % 00
  %  \node[] at (1.4,2.5) {\huge $ 1 $}; % 01
  \node[] at (1.4,0.5) {\huge $ 1 $}; % 02
  \node[] at (1.4,1.5) {\huge $ 1 $}; % 03
  %  \node[] at (2.4,3.5) {\huge $ 1 $}; % 04
  %  \node[] at (2.4,2.5) {\huge $ 1 $}; % 05
  \node[] at (2.4,0.5) {\huge $ 1 $}; % 06
  \node[] at (2.4,1.5) {\huge $ 1 $}; % 07
  \node[] at (4.4,3.5) {\huge $ 1 $}; % 08
  \node[] at (4.4,2.5) {\huge $ 1 $}; % 09
  %  \node[] at (4.4,0.5) {\huge $ 1 $}; % 10
  %  \node[] at (4.4,1.5) {\huge $ 1 $}; % 11
  \node[] at (3.4,3.5) {\huge $ 1 $}; % 12
  \node[] at (3.4,2.5) {\huge $ 1 $}; % 13
  %  \node[] at (3.4,0.5) {\huge $ 1 $}; % 14
  %  \node[] at (3.4,1.5) {\huge $ 1 $}; % 15
  
  % The coords for each cell - this is used as the origin for the solution box
  \coordinate (cell00) at (1.0,3.0); \coordinate (cell01) at (1.0,2.0);
  \coordinate (cell02) at (1.0,0.0); \coordinate (cell03) at (1.0,1.0);
  
  \coordinate (cell04) at (2.0,3.0); \coordinate (cell05) at (2.0,2.0);
  \coordinate (cell06) at (2.0,0.0); \coordinate (cell07) at (2.0,1.0);
  
  \coordinate (cell12) at (3.0,3.0); \coordinate (cell13) at (3.0,2.0);
  \coordinate (cell14) at (3.0,0.0); \coordinate (cell15) at (3.0,1.0);
  
  \coordinate (cell08) at (4.0,3.0); \coordinate (cell09) at (4.0,2.0);
  \coordinate (cell10) at (4.0,0.0); \coordinate (cell11) at (4.0,1.0);
  
  %********************************************************************
  % Draw the Shaded Cells
  %********************************************************************
  \foreach \loc in { 
    (cell02),(cell03),(cell06),(cell07),(cell08),(cell09),(cell12),(cell13) 
  }
  \node [draw,
  color=green!70!black,
  fill=green!20!white,
  fill opacity=0.3,
  minimum height=0.95cm,
  minimum width=0.95cm,
  anchor=south west] at \loc {};
  
  %********************************************************************
  % Shouldn't need to adjust anything below this point - this is just
  % the grid and the minterms.
  %********************************************************************  
  % Text in top-Left cell
  \node[] at (0.35,4.22) { \footnotesize $ \mathsf{ CD } $ }; % cd
  \node[] at (0.70,4.75) { \footnotesize $ \mathsf{ AB } $ }; % ab
  
  % Populate the top row header
  % In the following, the foreach lists a location/text pair
  % The the draw line draws the text at each location
  \foreach \loc/\txt in {
    (1.5,4.5)/{00},(2.5,4.5)/{01},(3.5,4.5)/{11},(4.5,4.5)/{10}
  }
  \draw \loc node{\large $\txt$};
  
  % Populate the header in column one
  \foreach \loc/\txt in { 
    (0.5,3.5)/{00},(0.5,2.5)/{01},(0.5,1.5)/{11},(0.5,0.5)/{10}
  }
  \draw \loc node{\large $\txt$};
  
  % Populate the minterms
  \foreach \loc/\txt in { 
    (1.75,3.15)/{00} , (2.75,3.15)/{04} , (3.75,3.15)/{12} , (4.75,3.15)/{08} ,
    (1.75,2.15)/{01} , (2.75,2.15)/{05} , (3.75,2.15)/{13} , (4.75,2.15)/{09} ,
    (1.75,1.15)/{03} , (2.75,1.15)/{07} , (3.75,1.15)/{15} , (4.75,1.15)/{11} ,
    (1.75,0.15)/{02} , (2.75,0.15)/{06} , (3.75,0.15)/{14} , (4.75,0.15)/{10} }
  \draw \loc node{ \color{blue!90!black} \footnotesize { $\txt$ }};
  
  % Draw the lines
  \draw
  % Finish drawing the grid
  [step=1.0cm,black,thin] (0,0) grid (5.0,5.0) % The Grid
  (0.0,5.0) -- (1.0,4.0) % Diagonal in the top left cell
  (1.0,4.05) -- (5.0,4.05) % Double line under top header row
  (0.95,0.0) -- (0.95,4.0) % Double line on left of header column one
  ;    
  \end{tikzpicture}
\end{figure}

\marginpar{To avoid confusion when simplifying these maps it is probably best to always use the constant \emph{True} terms.}

When this Karnaugh map was simplified earlier (Figure \ref{KM:fig:reed-muller_4-variable_example_2}), the constant \emph{True} for the group in the upper right corner, $ A $, and lower left corner, $ C $, was used; however, the constant \emph{False} in the lower left corner, $ A $, and upper right corner, $ C $, would give the same result. Either way yields Equation \ref{KM:eq:reed-muller_4_vars_ex_5}.

\begin{align}
  \label{KM:eq:reed-muller_4_vars_ex_5}
  A \oplus C &= Y 
\end{align}

\section{One In First Cell}
\label{KM:sec:one_in_first_cell}

Karnaugh maps with a one in the first Cell (that is, in a four-variable map, $ A'B'C'D' $ is \emph{True}) are simplified in a slightly different manner than those with a zero in that cell. When a one is present in the first cell, two of the terms must be combined with an \textsf{XNOR} rather than an \textsf{XOR} gate (though it does not matter which two are combined). To simplify these circuits, use the same technique presented above; but then select any two of the terms and change the gate from \textsf{XOR} to \textsf{XNOR}. Following are some examples. 

%****************************************************************
% Reed-Muller 4-Variable Example 6
%****************************************************************
\begin{figure}[H]
  \caption{Reed-M\"{u}ller Four-Variable Example 6}
  \label{KM:fig:reed-muller_4-variable_example_6}
  \myfloatalign
  \begin{tikzpicture} [circuit logic US, scale=1.00]
  % make all path lines (the node shapes) a little thicker
  \tikzstyle{every path}=[line width=0.50mm]
  
  %********************************************************************
  % Adjust the settings below to display the 1's and rectangles
  %********************************************************************
  % Uncomment the appropriate lines below to insert ones where needed
  \node[] at (1.4,3.5) {\huge $ 1 $}; % 00
%  \node[] at (1.4,2.5) {\huge $ 1 $}; % 01
%  \node[] at (1.4,0.5) {\huge $ 1 $}; % 02
  \node[] at (1.4,1.5) {\huge $ 1 $}; % 03
%  \node[] at (2.4,3.5) {\huge $ 1 $}; % 04
  \node[] at (2.4,2.5) {\huge $ 1 $}; % 05
  \node[] at (2.4,0.5) {\huge $ 1 $}; % 06
%  \node[] at (2.4,1.5) {\huge $ 1 $}; % 07
%  \node[] at (4.4,3.5) {\huge $ 1 $}; % 08
  \node[] at (4.4,2.5) {\huge $ 1 $}; % 09
  \node[] at (4.4,0.5) {\huge $ 1 $}; % 10
%  \node[] at (4.4,1.5) {\huge $ 1 $}; % 11
  \node[] at (3.4,3.5) {\huge $ 1 $}; % 12
%  \node[] at (3.4,2.5) {\huge $ 1 $}; % 13
%  \node[] at (3.4,0.5) {\huge $ 1 $}; % 14
  \node[] at (3.4,1.5) {\huge $ 1 $}; % 15
  
  % The coords for each cell - this is used as the origin for the solution box
  \coordinate (cell00) at (1.0,3.0); \coordinate (cell01) at (1.0,2.0);
  \coordinate (cell02) at (1.0,0.0); \coordinate (cell03) at (1.0,1.0);
  
  \coordinate (cell04) at (2.0,3.0); \coordinate (cell05) at (2.0,2.0);
  \coordinate (cell06) at (2.0,0.0); \coordinate (cell07) at (2.0,1.0);
  
  \coordinate (cell12) at (3.0,3.0); \coordinate (cell13) at (3.0,2.0);
  \coordinate (cell14) at (3.0,0.0); \coordinate (cell15) at (3.0,1.0);
  
  \coordinate (cell08) at (4.0,3.0); \coordinate (cell09) at (4.0,2.0);
  \coordinate (cell10) at (4.0,0.0); \coordinate (cell11) at (4.0,1.0);
  
  %********************************************************************
  % Draw the Shaded Cells
  %********************************************************************
  \foreach \loc in { 
    (cell00),(cell03),(cell05),(cell06),(cell09),(cell10),(cell12),(cell15) 
  }
  \node [draw,
  color=green!70!black,
  fill=green!20!white,
  fill opacity=0.3,
  minimum height=0.95cm,
  minimum width=0.95cm,
  anchor=south west] at \loc {};
  
  %********************************************************************
  % Shouldn't need to adjust anything below this point - this is just
  % the grid and the minterms.
  %********************************************************************  
  % Text in top-Left cell
  \node[] at (0.35,4.22) { \footnotesize $ \mathsf{ CD } $ }; % cd
  \node[] at (0.70,4.75) { \footnotesize $ \mathsf{ AB } $ }; % ab
  
  % Populate the top row header
  % In the following, the foreach lists a location/text pair
  % The the draw line draws the text at each location
  \foreach \loc/\txt in {
    (1.5,4.5)/{00},(2.5,4.5)/{01},(3.5,4.5)/{11},(4.5,4.5)/{10}
  }
  \draw \loc node{\large $\txt$};
  
  % Populate the header in column one
  \foreach \loc/\txt in { 
    (0.5,3.5)/{00},(0.5,2.5)/{01},(0.5,1.5)/{11},(0.5,0.5)/{10}
  }
  \draw \loc node{\large $\txt$};
  
  % Populate the minterms
  \foreach \loc/\txt in { 
    (1.75,3.15)/{00} , (2.75,3.15)/{04} , (3.75,3.15)/{12} , (4.75,3.15)/{08} ,
    (1.75,2.15)/{01} , (2.75,2.15)/{05} , (3.75,2.15)/{13} , (4.75,2.15)/{09} ,
    (1.75,1.15)/{03} , (2.75,1.15)/{07} , (3.75,1.15)/{15} , (4.75,1.15)/{11} ,
    (1.75,0.15)/{02} , (2.75,0.15)/{06} , (3.75,0.15)/{14} , (4.75,0.15)/{10} }
  \draw \loc node{ \color{blue!90!black} \footnotesize { $\txt$ }};
  
  % Draw the lines
  \draw
  % Finish drawing the grid
  [step=1.0cm,black,thin] (0,0) grid (5.0,5.0) % The Grid
  (0.0,5.0) -- (1.0,4.0) % Diagonal in the top left cell
  (1.0,4.05) -- (5.0,4.05) % Double line under top header row
  (0.95,0.0) -- (0.95,4.0) % Double line on left of header column one
  ;    
  \end{tikzpicture}
\end{figure}

In the Karnaugh map in Figure \ref{KM:fig:reed-muller_4-variable_example_6} it is not possible to group the ones. Since $ A $, $ B $, $ C $, and $ D $ are all \emph{True} in at least one cell, they would all appear in the final equation; however, since there is a one in the first cell, then two of the terms must be combined with an \textsf{XNOR} gate. Here is one possible solution:

\begin{align}
  \label{KM:eq:reed-muller_4_vars_ex_6}
  A \odot B \oplus C \oplus D &= Y 
\end{align}

%****************************************************************
% Reed-Muller 4-Variable Example 7
%****************************************************************
\begin{figure}[H]
  \caption{Reed-M\"{u}ller Four-Variable Example 7}
  \label{KM:fig:reed-muller_4-variable_example_7}
  \myfloatalign
  \begin{tikzpicture} [circuit logic US, scale=1.00]
  % make all path lines (the node shapes) a little thicker
  \tikzstyle{every path}=[line width=0.50mm]
  
  %********************************************************************
  % Adjust the settings below to display the 1's and rectangles
  %********************************************************************
  % Uncomment the appropriate lines below to insert ones where needed
  \node[] at (1.4,3.5) {\huge $ 1 $}; % 00
%  \node[] at (1.4,2.5) {\huge $ 1 $}; % 01
%  \node[] at (1.4,0.5) {\huge $ 1 $}; % 02
  \node[] at (1.4,1.5) {\huge $ 1 $}; % 03
  \node[] at (2.4,3.5) {\huge $ 1 $}; % 04
%  \node[] at (2.4,2.5) {\huge $ 1 $}; % 05
%  \node[] at (2.4,0.5) {\huge $ 1 $}; % 06
  \node[] at (2.4,1.5) {\huge $ 1 $}; % 07
  \node[] at (4.4,3.5) {\huge $ 1 $}; % 08
%  \node[] at (4.4,2.5) {\huge $ 1 $}; % 09
%  \node[] at (4.4,0.5) {\huge $ 1 $}; % 10
  \node[] at (4.4,1.5) {\huge $ 1 $}; % 11
  \node[] at (3.4,3.5) {\huge $ 1 $}; % 12
%  \node[] at (3.4,2.5) {\huge $ 1 $}; % 13
%  \node[] at (3.4,0.5) {\huge $ 1 $}; % 14
  \node[] at (3.4,1.5) {\huge $ 1 $}; % 15
  
  % The coords for each cell - this is used as the origin for the solution box
  \coordinate (cell00) at (1.0,3.0); \coordinate (cell01) at (1.0,2.0);
  \coordinate (cell02) at (1.0,0.0); \coordinate (cell03) at (1.0,1.0);
  
  \coordinate (cell04) at (2.0,3.0); \coordinate (cell05) at (2.0,2.0);
  \coordinate (cell06) at (2.0,0.0); \coordinate (cell07) at (2.0,1.0);
  
  \coordinate (cell12) at (3.0,3.0); \coordinate (cell13) at (3.0,2.0);
  \coordinate (cell14) at (3.0,0.0); \coordinate (cell15) at (3.0,1.0);
  
  \coordinate (cell08) at (4.0,3.0); \coordinate (cell09) at (4.0,2.0);
  \coordinate (cell10) at (4.0,0.0); \coordinate (cell11) at (4.0,1.0);
  
  %********************************************************************
  % Draw the Shaded Cells
  %********************************************************************
  \foreach \loc in { 
    (cell00),(cell03),(cell04),(cell07),(cell08),(cell11),(cell12),(cell15) 
  }
  \node [draw,
  color=green!70!black,
  fill=green!20!white,
  fill opacity=0.3,
  minimum height=0.95cm,
  minimum width=0.95cm,
  anchor=south west] at \loc {};
  
  %********************************************************************
  % Shouldn't need to adjust anything below this point - this is just
  % the grid and the minterms.
  %********************************************************************  
  % Text in top-Left cell
  \node[] at (0.35,4.22) { \footnotesize $ \mathsf{ CD } $ }; % cd
  \node[] at (0.70,4.75) { \footnotesize $ \mathsf{ AB } $ }; % ab
  
  % Populate the top row header
  % In the following, the foreach lists a location/text pair
  % The the draw line draws the text at each location
  \foreach \loc/\txt in {
    (1.5,4.5)/{00},(2.5,4.5)/{01},(3.5,4.5)/{11},(4.5,4.5)/{10}
  }
  \draw \loc node{\large $\txt$};
  
  % Populate the header in column one
  \foreach \loc/\txt in { 
    (0.5,3.5)/{00},(0.5,2.5)/{01},(0.5,1.5)/{11},(0.5,0.5)/{10}
  }
  \draw \loc node{\large $\txt$};
  
  % Populate the minterms
  \foreach \loc/\txt in { 
    (1.75,3.15)/{00} , (2.75,3.15)/{04} , (3.75,3.15)/{12} , (4.75,3.15)/{08} ,
    (1.75,2.15)/{01} , (2.75,2.15)/{05} , (3.75,2.15)/{13} , (4.75,2.15)/{09} ,
    (1.75,1.15)/{03} , (2.75,1.15)/{07} , (3.75,1.15)/{15} , (4.75,1.15)/{11} ,
    (1.75,0.15)/{02} , (2.75,0.15)/{06} , (3.75,0.15)/{14} , (4.75,0.15)/{10} }
  \draw \loc node{ \color{blue!90!black} \footnotesize { $\txt$ }};
  
  % Draw the lines
  \draw
  % Finish drawing the grid
  [step=1.0cm,black,thin] (0,0) grid (5.0,5.0) % The Grid
  (0.0,5.0) -- (1.0,4.0) % Diagonal in the top left cell
  (1.0,4.05) -- (5.0,4.05) % Double line under top header row
  (0.95,0.0) -- (0.95,4.0) % Double line on left of header column one
  ;    
  \end{tikzpicture}
\end{figure}

Remember that it does not matter if constant zeros or ones are used to simplify any given group, and when there is a one in the top left square it is usually easiest to look for constant zeros for that group (since that square is for input $ 0000 $). Row one of this map has a constant zero for $ C $ and $ D $, and row three has a constant one for $ C $ and $ D $. Since there is a one in the first cell, then the two terms must be combined with an \textsf{XNOR} gate: 

\begin{align}
  \label{KM:eq:reed-muller_4_vars_ex_7}
  C \odot D &= Y 
\end{align}

%****************************************************************
% Reed-Muller 4-Variable Example 8
%****************************************************************
\begin{figure}[H]
  \caption{Reed-M\"{u}ller Four-Variable Example 8}
  \label{KM:fig:reed-muller_4-variable_example_8}
  \myfloatalign
  \begin{tikzpicture} [circuit logic US, scale=1.00]
  % make all path lines (the node shapes) a little thicker
  \tikzstyle{every path}=[line width=0.50mm]
  
  %********************************************************************
  % Adjust the settings below to display the 1's and rectangles
  %********************************************************************
  % Uncomment the appropriate lines below to insert ones where needed
  \node[] at (1.4,3.5) {\huge $ 1 $}; % 00
  \node[] at (1.4,2.5) {\huge $ 1 $}; % 01
%  \node[] at (1.4,0.5) {\huge $ 1 $}; % 02
%  \node[] at (1.4,1.5) {\huge $ 1 $}; % 03
  \node[] at (2.4,3.5) {\huge $ 1 $}; % 04
  \node[] at (2.4,2.5) {\huge $ 1 $}; % 05
%  \node[] at (2.4,0.5) {\huge $ 1 $}; % 06
%  \node[] at (2.4,1.5) {\huge $ 1 $}; % 07
%  \node[] at (4.4,3.5) {\huge $ 1 $}; % 08
%  \node[] at (4.4,2.5) {\huge $ 1 $}; % 09
  \node[] at (4.4,0.5) {\huge $ 1 $}; % 10
  \node[] at (4.4,1.5) {\huge $ 1 $}; % 11
%  \node[] at (3.4,3.5) {\huge $ 1 $}; % 12
%  \node[] at (3.4,2.5) {\huge $ 1 $}; % 13
  \node[] at (3.4,0.5) {\huge $ 1 $}; % 14
  \node[] at (3.4,1.5) {\huge $ 1 $}; % 15
  
  % The coords for each cell - this is used as the origin for the solution box
  \coordinate (cell00) at (1.0,3.0); \coordinate (cell01) at (1.0,2.0);
  \coordinate (cell02) at (1.0,0.0); \coordinate (cell03) at (1.0,1.0);
  
  \coordinate (cell04) at (2.0,3.0); \coordinate (cell05) at (2.0,2.0);
  \coordinate (cell06) at (2.0,0.0); \coordinate (cell07) at (2.0,1.0);
  
  \coordinate (cell12) at (3.0,3.0); \coordinate (cell13) at (3.0,2.0);
  \coordinate (cell14) at (3.0,0.0); \coordinate (cell15) at (3.0,1.0);
  
  \coordinate (cell08) at (4.0,3.0); \coordinate (cell09) at (4.0,2.0);
  \coordinate (cell10) at (4.0,0.0); \coordinate (cell11) at (4.0,1.0);
  
  %********************************************************************
  % Draw the Shaded Cells
  %********************************************************************
  \foreach \loc in { 
    (cell00),(cell01),(cell04),(cell05),(cell10),(cell11),(cell14),(cell15) 
  }
  \node [draw,
  color=green!70!black,
  fill=green!20!white,
  fill opacity=0.3,
  minimum height=0.95cm,
  minimum width=0.95cm,
  anchor=south west] at \loc {};
  
  %********************************************************************
  % Shouldn't need to adjust anything below this point - this is just
  % the grid and the minterms.
  %********************************************************************  
  % Text in top-Left cell
  \node[] at (0.35,4.22) { \footnotesize $ \mathsf{ CD } $ }; % cd
  \node[] at (0.70,4.75) { \footnotesize $ \mathsf{ AB } $ }; % ab
  
  % Populate the top row header
  % In the following, the foreach lists a location/text pair
  % The the draw line draws the text at each location
  \foreach \loc/\txt in {
    (1.5,4.5)/{00},(2.5,4.5)/{01},(3.5,4.5)/{11},(4.5,4.5)/{10}
  }
  \draw \loc node{\large $\txt$};
  
  % Populate the header in column one
  \foreach \loc/\txt in { 
    (0.5,3.5)/{00},(0.5,2.5)/{01},(0.5,1.5)/{11},(0.5,0.5)/{10}
  }
  \draw \loc node{\large $\txt$};
  
  % Populate the minterms
  \foreach \loc/\txt in { 
    (1.75,3.15)/{00} , (2.75,3.15)/{04} , (3.75,3.15)/{12} , (4.75,3.15)/{08} ,
    (1.75,2.15)/{01} , (2.75,2.15)/{05} , (3.75,2.15)/{13} , (4.75,2.15)/{09} ,
    (1.75,1.15)/{03} , (2.75,1.15)/{07} , (3.75,1.15)/{15} , (4.75,1.15)/{11} ,
    (1.75,0.15)/{02} , (2.75,0.15)/{06} , (3.75,0.15)/{14} , (4.75,0.15)/{10} }
  \draw \loc node{ \color{blue!90!black} \footnotesize { $\txt$ }};
  
  % Draw the lines
  \draw
  % Finish drawing the grid
  [step=1.0cm,black,thin] (0,0) grid (5.0,5.0) % The Grid
  (0.0,5.0) -- (1.0,4.0) % Diagonal in the top left cell
  (1.0,4.05) -- (5.0,4.05) % Double line under top header row
  (0.95,0.0) -- (0.95,4.0) % Double line on left of header column one
  ;    
  \end{tikzpicture}
\end{figure}

The upper left corner of this map has a constant zero for $ A $ and $ C $, and the lower right corner has a constant one for $ A $ and $ C $. Since there is a one in the first cell, then the two variables must be combined with an \textsf{XNOR} gate: 

\begin{align}
  \label{KM:eq:reed-muller_4_vars_ex_8}
  A \odot C &= Y 
\end{align}


\chapter{Advanced Simplifying Methods}\label{ch07}
\section{Quine-McCluskey Simplification Method}
\subsection{Introduction}

\marginpar{This method was developed by W.V. Quine and Edward J. McCluskey and is sometimes called the method of prime implicants.} When a Boolean equation involves five or more variables it becomes very difficult to solve using standard algebra techniques or Karnaugh maps; however, the Quine-McCluskey algorithm can be used to solve these types of Boolean equations.  

The Quine-McCluskey method is based upon a simple Boolean algebra principle: if two expressions differ by only a single variable and its complement then those two expressions can be combined: 

\begin{align}
	\label{ASM:eq:quine-mccluskey_combining_complements}
	ABC+ABC' &= AB 
\end{align}

The Quine-McCluskey method looks for expressions that differ by only a single variable and combines them. Then it looks at the combined expressions to find those that differ by a single variable and combines them. The process continues until there are no expressions remaining to be combined.

\subsection{Example One}
\label{ASM:subsec:quine-mccluskey_ex_1}

\subsubsection{Step 1: Create the Implicants}
\label{ASM:subsubsec:quine-mccluskey_ex_1_step_1}

Equation \ref{ASM:eq:qm_ex_1} is the Sigma representation of a Boolean equation.

\begin{align}
	\label{ASM:eq:qm_ex_1}
	\int(A,B,C,D)=\sum(0,1,2,5,6,7,9,10,11,14) 
\end{align}

Truth Table \ref{ASM:tab:qm_ex_1_minterm_table} shows the input variables for the \emph{True} minterm values.

\begin{table}[H]
	\sffamily
	\newcommand{\head}[1]{\textcolor{white}{\textbf{#1}}}		
	\begin{center}
		\rowcolors{2}{gray!10}{white} % Color every other line a light gray
		\begin{tabular}{ccccc} 
			\rowcolor{black!75}
			\head{Minterm} & \head{A} & \head{B} & \head{C} & \head{D} \\
			0 & 0 & 0 & 0 & 0 \\
			1 & 0 & 0 & 0 & 1 \\
			2 & 0 & 0 & 1 & 0 \\
			5 & 0 & 1 & 0 & 1 \\
			6 & 0 & 1 & 1 & 0 \\
			7 & 0 & 1 & 1 & 1 \\
			9 & 1 & 0 & 0 & 1 \\
			10 & 1 & 0 & 1 & 0 \\
			11 & 1 & 0 & 1 & 1 \\
			14 & 1 & 1 & 1 & 0 
		\end{tabular}
	\end{center}
	\caption{Quine-McCluskey Ex 1: Minterm Table}
  \label{ASM:tab:qm_ex_1_minterm_table}
\end{table}

To simplify this equation, the minterms that evaluate to \emph{True} (as listed above) are first placed in a minterm table so that they form sections that are easy to combine. Each section contains only the minterms that have the same number of ones. Thus, the first section contains all minterms with zero ones, the second section contains the minterms with one one, and so forth. Truth Table \ref{ASM:tab:qm_ex_1_rearranged_table} shows the minterms rearranged appropriately.

\begin{table}[H]
	\sffamily
	\newcommand{\head}[1]{\textcolor{white}{\textbf{#1}}}		
	\begin{center}
		% \rowcolors{2}{gray!10}{white} % Color every other line a light gray
		\begin{tabular}{ccc} 
			\rowcolor{black!75}
			\head{Number of 1's} & \head{Minterm} & \head{Binary} \\
			0 & 0 & 0000 \\
			\hline
			\multirow{2}{*}{1} & 1 & 0001 \\
							   & 2 & 0010 \\
			\hline
			\multirow{4}{*}{2} & 5  & 0101 \\
							   & 6  & 0110 \\		
							   & 9  & 1001 \\		
							   & 10 & 1010 \\
			\hline
			\multirow{3}{*}{3} & 7  & 0111 \\
							   & 11 & 1011 \\		
							   & 14 & 1110 \\
			\hline
		\end{tabular}
	\end{center}
	\caption{Quine-McCluskey Ex 1: Rearranged Table}
  \label{ASM:tab:qm_ex_1_rearranged_table}
\end{table}

Start combining minterms with other minterms to create Size Two Implicants (called that since each implicant combines two minterms), but only those terms that vary by a single binary digit can be combined. When two minterms are combined, the binary digit that is different between the minterms is replaced by a dash, indicating that the digit does not matter. For example, $ 0000 $ and $ 0001 $ can be combined to form $ 000- $. The table is modified to add a Size Two Implicant column that indicates all of the combined terms. Note that every minterm must be compared to every other minterm so all possible implicants are formed. This is easier than it sounds, though, since terms in section one must be compared only with section two, then those in section two are compared with section three, and so forth, since each section differs from the next by a single binary digit. The Size Two Implicant column contains the combined binary form along with the numbers of the minterms used to create that implicant. It is also important to mark all minterms that are used to create the Size Two Implicants since allowance must be made for any not combined. Therefore, in the following table, as a minterm is used it is also struck through. Table \ref{ASM:tab:quine-mccluskey_ex_1_size_2_implicants} shows the Size Two Implicants that were found.

\begin{table}[H]
	\sffamily
	\newcommand{\head}[1]{\textcolor{white}{\textbf{#1}}}		
	\begin{center}
		% \rowcolors{2}{gray!10}{white} % Color every other line a light gray
		\begin{tabular}{ccc|l} 
			\rowcolor{black!75}
			\head{1's} & \head{Mntrm} 
				& \head{Bin} & \head{Size 2} \\
			                 0 & \sout{0} & 0000 & 000- (0,1) \\
			\cline{1-3}
			\multirow{2}{*}{1} & \sout{1}  & 0001 & 00-0 (0,2) \\
			                   & \sout{2}  & 0010 & 0-01 (1,5) \\
			\cline{1-3}
			\multirow{4}{*}{2} & \sout{5}  & 0101 & -001 (1,9) \\
			                   & \sout{6}  & 0110 & 0-10 (2,6) \\		
			                   & \sout{9}  & 1001 & -010 (2,10) \\		
			                   & \sout{10} & 1010 & 01-1 (5,7) \\
			\cline{1-3}
			\multirow{3}{*}{3} & \sout{7}  & 0111 & 011- (6,7) \\
			                   & \sout{11} & 1011 & -110 (6,14) \\		
			                   & \sout{14} & 1110 & 10-1 (9,11) \\
			\cline{1-3}
			                   &    &      & 101- (10,11) \\
			                   &    &      & 1-10 (10,14) \\
			\hline
		\end{tabular}
	\end{center}
	\caption{Quine-McCluskey Ex 1: Size 2 Implicants}
  \label{ASM:tab:quine-mccluskey_ex_1_size_2_implicants}
\end{table}

All of the Size Two Implicants can now be combined to form Size Four Implicants (those that combine a total of four minterms). Again, it is essential to only combine those with only a single binary digit difference. For this step, the dash can be considered the same as a single binary digit, as long as it is in the same place for both implicants. Thus, $ -010 $ and $ -110 $ can be combined to $ --10 $, but $ -010 $ and $ 0-00 $ cannot be combined since the dash is in different places in those numbers. It helps to match up the dashes first and then look at the binary digits. Again, as the various size-two implicants are used they are marked; but notice that a single size-four implicant actually combines four size-two implicants. Table \ref{ASM:tab:quine-mccluskey_ex_1_size_4_implicants} shows the Size Four Implicants.

\begin{table}[H]
	\sffamily
	\newcommand{\head}[1]{\textcolor{white}{\textbf{#1}}}		
	\begin{center}
		% \rowcolors{2}{gray!10}{white} % Color every other line a light gray
		\begin{tabular}{ccc|l|l} 
			\rowcolor{black!75}
			\head{1's} & \head{Mntrm} 
			& \head{Bin} & \head{Size 2} & \head{Size 4} \\
			                 0 & \sout{0}  & 0000 & 000- (0,1)  & --10 (2,10,6,14) \\
			\cline{1-3}
			\multirow{2}{*}{1} & \sout{1}  & 0001 & 00-0 (0,2)  & \\
			                   & \sout{2}  & 0010 & 0-01 (1,5)  & \\
			\cline{1-3}
			\multirow{4}{*}{2} & \sout{5}  & 0101 & -001 (1,9)  & \\
                               & \sout{6}  & 0110 & \sout{0-10 (2,6)}  & \\		
			                   & \sout{9}  & 1001 & \sout{-010 (2,10)} & \\		
			                   & \sout{10} & 1010 & 01-1 (5,7)  & \\
			\cline{1-3}
			\multirow{3}{*}{3} & \sout{7}  & 0111 & 011- (6,7)  & \\
			                   & \sout{11} & 1011 & \sout{-110 (6,14)} & \\		
			                   & \sout{14} & 1110 & 10-1 (9,11) & \\
			\cline{1-3}
			                   &           &      & 101- (10,11) & \\
			                   &           &      & \sout{1-10 (10,14)} & \\
			\hline
		\end{tabular}
	\end{center}
	\caption{Quine-McCluskey Ex 1: Size 4 Implicants}
  \label{ASM:tab:quine-mccluskey_ex_1_size_4_implicants}
\end{table}

None of the terms can be combined any further. All of the minterms or implicants that are not marked are \emph{Prime Implicants}. In the table above, for example, the Size Two Implicant $ 000- $ is a Prime Implicant. The Prime Implicants will be placed in a chart and further processed in the next step. 

\subsubsection{Step 2: The Prime Implicant Table}
\label{ASM:subsubsec:quine-mccluskey_ex_1_step_2}

A \emph{Prime Implicant Table} can now be constructed, as in Table \ref{ASM:tab:qm_ex_1_prime_implicants}. The prime implicants are listed down the left side of the table, the decimal equivalent of the minterms goes across the top, and the Boolean representation of the prime implicants is listed down the right side of the table. 

\begin{table}[H]
	\sffamily
	\newcommand{\head}[1]{\textcolor{white}{\textbf{#1}}}		
	\begin{center}
		\rowcolors{2}{gray!10}{white} % Color every other line a light gray
		\begin{adjustbox}{max width=\textwidth}
		\begin{tabular}{lccccccccccc} 
			\rowcolor{black!75}
			& \head{0} & \head{1} & \head{2} & \head{5}
			& \head{6} & \head{7} & \head{9} & \head{10}
			& \head{11} & \head{14} & \\
							%      0   1   2   5   6   7   9  10  11  14  
			$ 000-\;(0,1) $       & X & X &   &   &   &   &   &   &   &   & $ A'B'C' $ \\
			$ 00-0\;(0,2) $       & X &   & X &   &   &   &   &   &   &   & $ A'B'D' $ \\
			$ 0-01\;(1,5) $       &   & X &   & X &   &   &   &   &   &   & $ A'C'D $ \\
			$ -001\;(1,9) $       &   & X &   &   &   &   & X &   &   &   & $ B'C'D $ \\
			$ 01-1\;(5,7) $       &   &   &   & X &   & X &   &   &   &   & $ A'BD $ \\
			$ 011-\;(6,7) $       &   &   &   &   & X & X &   &   &   &   & $ A'BC $ \\
			$ 10-1\;(9,11) $      &   &   &   &   &   &   & X &   & X &   & $ AB'D $ \\
			$ 101-\;(10,11) $     &   &   &   &   &   &   &   & X & X &   & $ AB'C $ \\
			$ --10\;(2,10,6,14) $ &   &   & X &   & X &   &   & X &   & X & $ CD' $ \\
			\hline
		\end{tabular}
		\end{adjustbox}
	\end{center}
	\caption{Quine-McCluskey Ex 1: Prime Implicants}
  \label{ASM:tab:qm_ex_1_prime_implicants}
\end{table}

An \emph{X} marks the intersection where each minterm (on the top row) is used to form one of the prime implicants (in the left column). Thus, minterm $ 0 $ (or $ 0000 $) is used to form the prime implicant $ 000- (0,1) $ in row one and $ 00-0 (0,2) $ in row two. 

The Essential Prime Implicants can be found by looking for columns that contain only one \emph{X}. The column for minterm $ 14 $ has only one \emph{X}, in the last row, $ --10 (2,10,6,14) $; thus, it is an Essential Prime Implicant. That means that the term in the right column for the last row, $ CD' $, must appear in the final simplified equation. However, that term also covers the columns for $ 2 $, $ 6 $, and $ 10 $; so they can be removed from the table. The Prime Implicant table is then simplified to \ref{ASM:tab:qm_ex_1_1st_iteration}.

\begin{table}[H]
	\sffamily
	\newcommand{\head}[1]{\textcolor{white}{\textbf{#1}}}		
	\begin{center}
		\rowcolors{2}{gray!10}{white} % Color every other line a light gray
		\begin{adjustbox}{max width=\textwidth}
			\begin{tabular}{lccccccc} 
				\rowcolor{black!75}
				& \head{0} & \head{1} & \head{5}
				& \head{7} & \head{9} & \head{11} & \\
				%                      0   1   5   7   9  11   
				$ 000-\;(0,1) $       & X & X &   &   &   &   & $ A'B'C' $ \\
				$ 00-0\;(0,2) $       & X &   &   &   &   &   & $ A'B'D' $ \\
				$ 0-01\;(1,5) $       &   & X & X &   &   &   & $ A'C'D $ \\
				$ -001\;(1,9) $       &   & X &   &   & X &   & $ B'C'D $ \\
				$ 01-1\;(5,7) $       &   &   & X & X &   &   & $ A'BD $ \\
				$ 011-\;(6,7) $       &   &   &   & X &   &   & $ A'BC $ \\
				$ 10-1\;(9,11) $      &   &   &   &   & X & X & $ AB'D $ \\
				$ 101-\;(10,11) $     &   &   &   &   &   & X & $ AB'C $ \\
				\hline
			\end{tabular}
		\end{adjustbox}
	\end{center}
	\caption{Quine-McCluskey Ex 1: 1st Iteration}
  \label{ASM:tab:qm_ex_1_1st_iteration}
\end{table}

The various rows can now be combined in any order the designer desires. For example, if row $ 10-1 (9,11) $, is selected as a required implicant in the solution, then minterms $ 9 $ and $ 11 $ are accounted for in the final equation, which means that all \emph{X} marked in those columns can be removed. When that is done, then, rows $ 101- (10,11) $ and $ 10-1 (9,11) $ no longer have any marks in the table, and they can be removed. Table \ref{ASM:tab:qm_ex_1_2nd_iteration} shows the last iteration of this solution. 

\begin{table}[H]
	\sffamily
	\newcommand{\head}[1]{\textcolor{white}{\textbf{#1}}}		
	\begin{center}
		\rowcolors{2}{gray!10}{white} % Color every other line a light gray
		\begin{adjustbox}{max width=\textwidth}
			\begin{tabular}{lccccccc} 
				\rowcolor{black!75}
				& \head{0} & \head{1} & \head{5}
				& \head{7} & \\
				%                      0   1   5   7      
				$ 000-\;(0,1) $       & X & X &   &   & $ A'B'C' $ \\
				$ 00-0\;(0,2) $       & X &   &   &   & $ A'B'D' $ \\
				$ 0-01\;(1,5) $       &   & X & X &   & $ A'C'D $ \\
				$ -001\;(1,9) $       &   & X &   &   & $ B'C'D $ \\
				$ 01-1\;(5,7) $       &   &   & X & X & $ A'BD $ \\
				$ 011-\;(6,7) $       &   &   &   & X & $ A'BC $ \\
				\hline
			\end{tabular}
		\end{adjustbox}
	\end{center}
	\caption{Quine-McCluskey Ex 1: 2nd Iteration}
  \label{ASM:tab:qm_ex_1_2nd_iteration}
\end{table}

The designer next decided to select $ 01-1 (5,7) $, $ A'BD $, as a required implicant. That will include minterms $ 5 $ and $ 7 $, and those columns may be removed along with rows $ 01-1 (5,7) $, $ A'BD $, and $ 011- (6,7) $, $ A'BC $, as shown in Table \ref{ASM:tab:qm_ex_1_3rd_iteration}.

\begin{table}[H]
	\sffamily
	\newcommand{\head}[1]{\textcolor{white}{\textbf{#1}}}		
	\begin{center}
		\rowcolors{2}{gray!10}{white} % Color every other line a light gray
		\begin{adjustbox}{max width=\textwidth}
			\begin{tabular}{lccccccc} 
				\rowcolor{black!75}
				& \head{0} & \head{1} & \\
				%                       0   1         
				$ 000-\;(0,1) $       & X & X & $ A'B'C' $ \\
				$ 00-0\;(0,2) $       & X &   & $ A'B'D' $ \\
				$ 0-01\;(1,5) $       &   & X & $ A'C'D $ \\
				$ -001\;(1,9) $       &   & X & $ B'C'D $ \\
				\hline
			\end{tabular}
		\end{adjustbox}
	\end{center}
	\caption{Quine-McCluskey Ex 1: 3rd Iteration}
  \label{ASM:tab:qm_ex_1_3rd_iteration}
\end{table}

The last two minterms ($ 0 $ and $ 1 $) can be covered by the implicant $ 000- (0,1) $, and that also eliminates the last three rows in the chart. 

The original Boolean expression, then, has been simplified from ten minterms to Equation \ref{ASM:eq:qm_ex_1_solution}.

\begin{align}
	\label{ASM:eq:qm_ex_1_solution}
	A'B'C'+A'BD+AB'D+CD' = Y 
\end{align}

\subsection{Example Two}
\label{ASM:subsec:quine-mccluskey_ex_2}

\subsubsection{Step 1: Create the Implicants}
\label{ASM:subsubsec:quine-mccluskey_ex_2_step_1}

Given Equation \ref{ASM:eq:qm_ex_2}, which is a Sigma representation of a Boolean equation.

\begin{align}
\label{ASM:eq:qm_ex_2}
\int(A,B,C,D,E,F)=\sum(0,1,8,9,12,13,14,15,32,33,37,39,48,56) 
\end{align}

Truth Table \ref{ASM:tab:qm_ex_2_minterm_table} shows the \emph{True} minterm values.

\begin{table}[H]
	\sffamily
	\newcommand{\head}[1]{\textcolor{white}{\textbf{#1}}}		
	\begin{center}
		\rowcolors{2}{gray!10}{white} % Color every other line a light gray
		\begin{tabular}{ccccccc} 
			\rowcolor{black!75}
			\head{Minterm} & \head{A} & \head{B} & \head{C} 
				& \head{D} & \head{E} & \head{F} \\
			0  & 0 & 0 & 0 & 0 & 0 & 0 \\
			1  & 0 & 0 & 0 & 0 & 0 & 1 \\
			8  & 0 & 0 & 1 & 0 & 0 & 0 \\
			9  & 0 & 0 & 1 & 0 & 0 & 1 \\
			12 & 0 & 0 & 1 & 1 & 0 & 0 \\
			13 & 0 & 0 & 1 & 1 & 0 & 1 \\
			14 & 0 & 0 & 1 & 1 & 1 & 0 \\
			15 & 0 & 0 & 1 & 1 & 1 & 1 \\
			32 & 1 & 0 & 0 & 0 & 0 & 0 \\
			33 & 1 & 0 & 0 & 0 & 0 & 1 \\
			37 & 1 & 0 & 0 & 1 & 0 & 1 \\
			39 & 1 & 0 & 0 & 1 & 1 & 1 \\
			48 & 1 & 1 & 0 & 0 & 0 & 0 \\
			56 & 1 & 1 & 1 & 0 & 0 & 0 \\
		\end{tabular}
	\end{center}
	\caption{Quine-McCluskey Ex 2: Minterm Table}
  \label{ASM:tab:qm_ex_2_minterm_table}
\end{table}

To simplify this equation, the minterms that evaluate to \emph{True} are placed in a minterm table so that they form sections that are easy to combine. Each section contains only the minterms that have the same number of ones. Thus, the first section contains all minterms with zero ones, the second section contains the minterms with one one, and so forth. Table \ref{ASM:tab:qm_ex_2_rearranged_table} shows the rearranged truth table.

\begin{table}[H]
	\sffamily
	\newcommand{\head}[1]{\textcolor{white}{\textbf{#1}}}		
	\begin{center}
		% \rowcolors{2}{gray!10}{white} % Color every other line a light gray
		\begin{tabular}{ccc} 
			\rowcolor{black!75}
			\head{Number of 1's} & \head{Minterm} & \head{Binary} \\
							 0 & 0  & 000000 \\
			\hline
			\multirow{3}{*}{1} & 1  & 000001 \\
                               & 8  & 001000 \\
                               & 32 & 100000 \\
 			\hline
			\multirow{4}{*}{2} & 9  & 001001 \\
                               & 12 & 001100 \\		
							   & 33 & 100001 \\		
							   & 48 & 110000 \\
			\hline
			\multirow{3}{*}{3} & 13 & 001101 \\
							   & 14 & 001110 \\		
							   & 37 & 100101 \\		
							   & 56 & 111000 \\
			\hline
			\multirow{2}{*}{4} & 15 & 001111 \\
							   & 39 & 100111 \\		
			\hline
		\end{tabular}
	\end{center}
	\caption{Quine-McCluskey Ex 2: Rearranged Table}
  \label{ASM:tab:qm_ex_2_rearranged_table}
\end{table}

Start combining minterms with other minterms to create Size Two Implicants, as in Table \ref{ASM:tab:qm_ex_2_size_2_implicants}.

\begin{table}[H]
	\sffamily
	\newcommand{\head}[1]{\textcolor{white}{\textbf{#1}}}		
	\begin{center}
		% \rowcolors{2}{gray!10}{white} % Color every other line a light gray
		\begin{tabular}{ccc|c} 
			\rowcolor{black!75}
			\head{1's} & \head{Mntrm} & \head{Bin} & \head{Size 2} \\
							 0 & \sout{0} & 000000 & 00000- (0,1) \\
			\cline{1-3}
			\multirow{3}{*}{1} & \sout{1}  & 000001 & -000000 (0,32) \\
							   & \sout{8}  & 001000 & 00-000 (0,8) \\
							   & \sout{32} & 100000 & -00001 (1,33) \\
			\cline{1-3}
			\multirow{4}{*}{2} & \sout{9}  & 001001 & 00-001 (1,9) \\
							   & \sout{12} & 001100 & 10000- (32,33) \\		
				  			   & \sout{33} & 100001 & 1-0000 (32,48) \\		
							   & \sout{48} & 110000 & 00100- (8,9) \\
			\cline{1-3}
			\multirow{3}{*}{3} & \sout{13} & 001101 & 001-00 (8,12) \\
							   & \sout{14} & 001110 & 100-01 (33,37) \\		
							   & \sout{37} & 100101 & 001-01 (9,13) \\		
						       & \sout{56} & 111000 & 00110- (12,13) \\
			\cline{1-3}
			\multirow{2}{*}{4} & \sout{15} & 001111 & 0011-0 (12,14) \\
							   & \sout{39} & 100111 & 11-000 (48,56) \\		
							   &    &        & 1001-1 (37,39) \\		
							   &    &        & 0011-1 (13,15) \\		
							   &    &        & 00111- (14,15) \\		
			\hline
		\end{tabular}
	\end{center}
	\caption{Quine-McCluskey Ex 2: Size Two Implicants}
  \label{ASM:tab:qm_ex_2_size_2_implicants}
\end{table}

All of the Size Two Implicants can now be combined to form Size Four Implicants, as in Table \ref{ASM:tab:qm_ex_2_size_4_implicants}.

\begin{table}[H]
	\sffamily
	\newcommand{\head}[1]{\textcolor{white}{\textbf{#1}}}		
	\begin{center}
		% \rowcolors{2}{gray!10}{white} % Color every other line a light gray
		\begin{tabular}{ccc|c|c} 
			\rowcolor{black!75}
			\head{1's} & \head{Mntrm} & \head{Bin} 
				& \head{Size 2} & \head{Size 4} \\
							  0 & \sout{0} & 000000 & \sout{00000- (0,1)}   & -0000- (0,1,32,33) \\
			\cline{1-3}
			\multirow{3}{*}{1} & \sout{1}  & 000001 & \sout{-000000 (0,32)} & 00-00- (0,1,8,9)\\
							   & \sout{8}  & 001000 & \sout{00-000 (0,8)}   & 001-0- (8,9,12,13)\\
						       & \sout{32} & 100000 & \sout{-00001 (1,33)}  & 0011-- (12,13,14,15)\\
			\cline{1-3}
			\multirow{4}{*}{2} & \sout{9}  & 001001 & \sout{00-001 (1,9)}   & \\
							   & \sout{12} & 001100 & \sout{10000- (32,33)} & \\		
							   & \sout{33} & 100001 & 1-0000 (32,48)        & \\		
							   & \sout{48} & 110000 & \sout{00100- (8,9)}   & \\
			\cline{1-3}
			\multirow{3}{*}{3} & \sout{13} & 001101 & \sout{001-00 (8,12)}  & \\
							   & \sout{14} & 001110 & 100-01 (33,37)        & \\		
							   & \sout{37} & 100101 & \sout{001-01 (9,13)}  & \\		
							   & \sout{56} & 111000 & \sout{00110- (12,13)} & \\
			\cline{1-3}
			\multirow{2}{*}{4} & \sout{15} & 001111 & \sout{0011-0 (12,14)} & \\
							   & \sout{39} & 100111 & 11-000 (48,56)        & \\		
							   &    &        & 1001-1 (37,39) 		        & \\		
							   &    &        & \sout{0011-1 (13,15)}	    & \\		
							   &    &        & \sout{00111- (14,15)} 	    & \\		
			\hline
		\end{tabular}
	\end{center}
	\caption{Quine-McCluskey Ex 2: Size 4 Implicants}
  \label{ASM:tab:qm_ex_2_size_4_implicants}
\end{table}

None of the terms can be combined any further. All of the minterms or implicants that are not struck through are \emph{Prime Implicants}. In the table above, for example, $ 1-0000 $ is a Prime Implicant. The Prime Implicants are next placed in a table and further processed. 

\subsubsection{Step 2: The Prime Implicant Table}
\label{ASM:subsubsec:quine-mccluskey_ex_2_step_2}

A \emph{Prime Implicant Table} can now be constructed, as in Table \ref{ASM:tab:qm_ex_2_prime_implicants}. The prime implicants are listed down the left side of the table, the decimal equivalent of the minterms goes across the top, and the Boolean representation of the prime implicants is listed down the right side of the table. 

\begin{table}[H]
	\sffamily
	\newcommand{\head}[1]{\textcolor{white}{\textbf{#1}}}		
	\begin{center}
		\rowcolors{2}{gray!10}{white} % Color every other line a light gray
		\begin{adjustbox}{max width=\textwidth}
			\begin{tabular}{lccccccccccccccc} 
				\rowcolor{black!75}
				& \head{0} & \head{1} & \head{8} & \head{9}
				& \head{12} & \head{13} & \head{14} & \head{15}
				& \head{32} & \head{33} & \head{37} & \head{39} 
				& \head{48} & \head{56} & \\
				%      			            0   1   8   9  12  13  14  15  32  33  37  39  48  56
				$ 11-000\;(48,56) $       &   &   &   &   &   &   &   &   &   &   &   &   & X & X & $ ABD'D'F' $ \\
				$ 00-00-\;(0,1,8,9) $     & X & X & X & X &   &   &   &   &   &   &   &   &   &   & $ A'B'D'E' $ \\
				$ 1001-1\;(37,39) $       &   &   &   &   &   &   &   &   &   &   & X & X &   &   & $ AB'C'DF $ \\
				$ 1-0000\;(32,48) $       &   &   &   &   &   &   &   &   & X &   &   &   & X &   & $ AC'D'E'F' $ \\
				$ 0011--\;(12,13,14,15) $ &   &   &   &   & X & X & X & X &   &   &   &   &   &   & $ A'B'CD $ \\
				$ -0000-\;(0,1,32,33) $   & X & X &   &   &   &   &   &   & X & X &   &   &   &   & $ B'C'D'E' $ \\
				$ 001-0-\;(8,9,12,13) $   &   &   & X & X & X & X &   &   &   &   &   &   &   &   & $ A'B'CE' $ \\
				$ 100-01\;(33,37) $       &   &   &   &   &   &   &   &   &   & X & X &   &   &   & $ AB'C'E'F $ \\
				\hline
			\end{tabular}
		\end{adjustbox}
	\end{center}
	\caption{Quine-McCluskey Ex 2: Prime Implicants}
  \label{ASM:tab:qm_ex_2_prime_implicants}
\end{table}

In the above table, there are four columns that contain only one \emph{X}: $ 14 $, $ 15 $, $ 39 $, and $ 56 $. The rows that intersect the columns at that mark are \emph{Essential Prime Inplicants}, and their Boolean Expressions must appear in the final equation. Therefore, the final equation will contain, at a minimum: $ A'B'CD $ (row $ 5 $, covers minterms $ 14 $ and $ 15 $), $ AB'C'DF $ (row $ 3 $, covers minterm $ 39 $), and $ ABD'E'F' $ (row $ 1 $, covers minterm $ 56 $). Since those expressions are in the final equation, the rows that contain those expressions can be removed from the chart in order to make further analysis less confusing. 

Also, because the rows with Essential Prime Implicants are contained in the final equation, other minterms marked by those rows are covered and need no further consideration. For example, minterm $ 48 $ is covered by row one (used for minterm $ 56 $), so column $ 48 $ can be removed from the table. In a similar fashion, columns $ 12 $, $ 13 $, and $ 37 $ are covered by other minterms, so they can be removed from the table. Table \ref{ASM:tab:qm_ex_2_1st_iteration} shows the next iteration of this process.

\begin{table}[H]
	\sffamily
	\newcommand{\head}[1]{\textcolor{white}{\textbf{#1}}}		
	\begin{center}
		\rowcolors{2}{gray!10}{white} % Color every other line a light gray
		\begin{adjustbox}{max width=\textwidth}
			\begin{tabular}{lccccccc} 
				\rowcolor{black!75}
				& \head{0} & \head{1} & \head{8} & \head{9}
				& \head{32} & \head{33} & \\
				%      			            0   1   8   9  32  33 
				$ 00-00-\;(0,1,8,9) $     & X & X & X & X &   &   & $ A'B'D'E' $ \\
				$ 1-0000\;(32,48) $       &   &   &   &   & X &   & $ AC'D'E'F' $ \\
				$ -0000-\;(0,1,32,33) $   & X & X &   &   & X & X & $ B'C'D'E' $ \\
				$ 001-0-\;(8,9,12,13) $   &   &   & X & X &   &   & $ A'B'CE' $ \\
				$ 100-01\;(33,37) $       &   &   &   &   &   & X & $ AB'C'E'F $ \\
				\hline
			\end{tabular}
		\end{adjustbox}
	\end{center}
	\caption{Quine-McCluskey Ex 2: 1st Iteration}
  \label{ASM:tab:qm_ex_2_1st_iteration}
\end{table}

The circuit designer can select the next term to include in the final equation from any of the five rows still remaining in the chart; however, the first term ($ 00-00- $, or $ A'B'D'E' $) would eliminate four columns, so that would be a logical next choice. When that term is selected for the final equation, then row one, $ 00-00- $, can be removed from the chart; and columns $ 0 $, $ 1 $, $ 8 $, and $ 9 $ can be removed since those minterms are covered. 

The minterms marked for row $ 001-0- (8,9,12,13) $ are also covered, so this row can be removed. Table \ref{ASM:tab:qm_ex_2_2nd_iteration} shows the next iteration.

\begin{table}[H]
	\sffamily
	\newcommand{\head}[1]{\textcolor{white}{\textbf{#1}}}		
	\begin{center}
		\rowcolors{2}{gray!10}{white} % Color every other line a light gray
		\begin{adjustbox}{max width=\textwidth}
			\begin{tabular}{lccc} 
				\rowcolor{black!75}
				& \head{32} & \head{33} & \\
				%      			           32  33 
				$ 1-0000\;(32,48) $       & X &   & $ AC'D'E'F' $ \\
				$ -0000-\;(0,1,32,33) $   & X & X & $ B'C'D'E' $ \\
				$ 100-01\;(33,37) $       &   & X & $ AB'C'E'F $ \\
				\hline
			\end{tabular}
		\end{adjustbox}
	\end{center}
	\caption{Quine-McCluskey Ex 2: 2nd Iteration}
  \label{ASM:tab:qm_ex_2_2nd_iteration}
\end{table}

For the next simplification, row $ -0000- $ is selected since that would also cover the minterms that are marked for all remaining rows. Thus, the expression $ B'C'D'E' $ will become part of the final equation. 

When the analysis is completed, the original equation (\ref{ASM:eq:qm_ex_2}), which contained $ 14 $ minterms, is simplified into Equation \ref{ASM:eq:qm_ex_2_solution}, which contains only five terms.

\begin{align}
	\label{ASM:eq:qm_ex_2_solution}
	ABD'E'F'+A'B'D'E'+AB'C'DF+A'B'CD+B'C'D'E' = Y 
\end{align}

\subsection{Summary}
\label{ASM:subsec:quine-mccluskey_summary}

While the Quine–McCluskey method is useful for large Boolean expressions containing multiple inputs, it is also tedious and prone to error when done by hand. Also, there are some Boolean expressions (called ``Cyclic'' and ``Semi-Cyclic'' Primes) that do not reduce using this method. Finally, both Karnaugh maps and Quine-McCluskey methods become very complex when more than one output is required of a circuit. Fortunately, many automated tools are available to simplify Boolean expressions using advanced mathematical techniques. 

\subsection{Practice Problems}
\label{ASM:subsec:quine-mccluskey_practice_problems}

The following problems are presented as practice for using the Quine-McClusky method to simplify a Boolean expression. Note: designers can select different Prime Implicants so the simplified expression could vary from what is presented below. 

\subsection{Practice Problems}
\label{ASM:subsec:practice_problems_karnaugh_maps}

\begin{table}[H]
	\sffamily
	\begin{center}
		\begin{tabular}{c c p{6cm} }
			\multirow{2}{*}{\textbf{1}} 
			& Expression & $ \int(A,B,C,D) = \sum(0,1,2,5,6,7,9,10,11,14) $ \\
			& \cellcolor{gray!10} Simplified 
			& \cellcolor{gray!10} $ A'B'C'+A'BD+AB'D+CD' $ \\
			\hline
			\multirow{2}{*}{\textbf{2}} 
			& Exression & $ \int(A,B,C,D) = \sum(0,1,2,3,6,7,8,9,14,15) $ \\
			& \cellcolor{gray!10} Simplified 
			& \cellcolor{gray!10} $ A'C+BC+B'C' $ \\
			\hline
			\multirow{2}{*}{\textbf{3}} 
			& Exression & $ \int(A,B,C,D) = \sum(1,5,7,8,9,10,11,13,15) $ \\
			& \cellcolor{gray!10} Simplified 
			& \cellcolor{gray!10} $ C'D+AB'+BD $ \\
			\hline
			\multirow{2}{*}{\textbf{3}} 
			& Exression & $ \int(A,B,C,D,E) = \sum(0,4,8,9,10,11,12,13,14,15,16,20,24,28) $ \\
			& \cellcolor{gray!10} Simplified 
			& \cellcolor{gray!10} $ A'B+D'E' $ \\
		\end{tabular}
	\end{center}
	\caption{Quine-McCluskey Practice Problems}
  \label{ASM:tab:quine-mccluskey_practice_problems}
\end{table}

%***************************************************************************
% Section: Automated Tools
%***************************************************************************
\clearpage\section{Automated Tools}
\label{ASM:sec:automated_tools}

\subsection{Introduction}
\label{ASM:subsec:introduction_to_automated_tools}

There are numerous automated tools available to aid in simplifying complex Boolean equations. Many of the tools are quite expensive and intended for professionals working full time in large companies; but others are inexpensive, or even free of charge, and are more than adequate for student use. This topic introduces one such free tool: \ac{KARMA}. 

\subsection{KARMA}
\label{ASM:subsec:karma}

\subsubsection{Introduction}
\label{ASM:subsubsec:introduction_to_karma}

\ac{KARMA} is a free Java-based tool designed to help simplify Boolean expressions. Both an online and downloaded version of \ac{KARMA} is available. The application can be found at: \url{http://goo.gl/8Lmx5v}. Note: The version of \ac{KARMA} used for this text is 3.62. A newer version may be available but the instructions presented here use only the base functions and will likely be applicable even in an updated version of the software.

\begin{figure}[H]
	\centering
	\includegraphics[width=\maxwidth{.95\linewidth}]{gfx/07_01}
	\caption{Karma Initial Screen}
	\label{fig:07_01}
\end{figure}

The right side of the screen contains a row of tools available in \ac{KARMA} and the main part of the screen is a canvas where most of the work is done. The following tools are available:


\begin{itemize}
	\item Logic2Logic. Converts between two different logical representations of data; for example, a Truth Table can be converted to Boolean expressions. 
	\item Logic Equivalence. Compares two functions and determines if they are equivalent; for example, a truth table can be compared to a SOP expression to see if they are the same. 
	\item Logic Probability. Calculates the probability of any one outcome for a given Boolean expression. 
	\item Karnaugh Map. Analyzes a Karnaugh map and returns the Minimized Expression. 
	\item KM Teaching Mode. Provides drill and practice with Karnaugh maps; for example, finding adjacent minterms on a 6-variable map. 
	\item SOP and POS. Finds the SOP and POS expressions for a given function. 
	\item Exclusive-OR. Uses XOR gates to simplify an expression. 
	\item Multiplexer-Based. Realizes a function using multiplexers. 
	\item Factorization. Factors Boolean expressions. 
	\item About. Information about \ac{KARMA}. 
\end{itemize}

For this lesson, only the Karnaugh Map analyzer will be used, and the initial screen for that function is below.

\begin{figure}[H]
	\centering
	\includegraphics[width=\maxwidth{.95\linewidth}]{gfx/07_02}
	\caption{Karma K-Map Analyzer}
	\label{fig:07_02}
\end{figure}

\subsubsection{Data Entry}
\label{ASM:subsubsec:karma_data_entry}

When using \ac{KARMA}, the first step is to input some sort of information about the circuit to be analyzed. That information can be entered in several different formats, but a truth table or a Boolean expression would best match this book. 

To enter the initial data, click the \emph{Load Function} button at the top of the canvas.

By default, the Load Function screen opens with a blank screen. In the lower left corner of the Load Function window, the Source Format for the input data can be selected. There is a template available for each of the different source formats; and that template can be used to help with data entry. The best way to work with \ac{KARMA} is to click the ``Templates'' button and then select the data format of interest. Figure \ref{ASM:fig:karma_expression_one_loaded} shows the ``Expression 1'' template. 

\begin{figure}[H]
	\centering
	\includegraphics[width=\maxwidth{.95\linewidth}]{gfx/07_03}
	\caption{Karma Expression One Loaded}
	\label{fig:07_03}
\end{figure}

The designer would replace the ``inputs'' and ``onset'' lines with information for the circuit being simplified. Once the source data are entered into this window, click the \emph{Load} button at the bottom of the window to load the data into \ac{KARMA}. 

\subsubsection{Data Source Formats}
\label{ASM:subsubsec:karma_data_source_formats}

\ac{KARMA} works with input data in any of six different formats: Boolean Expression, Truth Table, Integer, Minterms, \ac{BLIF}, and \ac{BDD}. \ac{BLIF} and \ac{BDD} are programming tools that are beyond the scope of this book and will not be covered. 

%TODO: maybe expand the book to include BLIF and BDD.

\paragraph{Expression}
\label{ASM:para:karma_expression}

Boolean expressions can be defined in \ac{KARMA} using the following format.

{\small 
	\begin{verbatim}
   #Sample Expression
   (!x1*!x2*!x4)+(!x1*x2*!x3)+(x1*!x4*!x5)+(x1*x3*x4) 
	\end{verbatim}
}

Notes:

\begin{itemize}
	\item Any line that starts with a hash mark (``\#'') is a comment and will be ignored by \ac{KARMA}. 
	\item ``Not'' is indicated by a leading exclamation mark. Thus $ !x1 $ is the same as $ X1' $. 
	\item All operations are explicit. In real-number algebra the phrase ``AB'' is understood to be ``A*B.'' However, in \ac{KARMA}, since variable names can be more than one character long, all operations must be explicitly stated. \textsf{AND} is indicated by an asterisk and \textsf{OR} is indicated by a plus sign. 
	\item No space is left between operations. 
\end{itemize}

\paragraph{Truth Table}
\label{ASM:para:karma_truth_table}

A truth table can be defined in \ac{KARMA} using the following format.

\begin{verbatim}
     #Sample Truth Table
     inputs -> X, Y, Z
     000: 1
     001: 1
     010: 0
     011: 0
     100: 0
     101: 1
     110: 0
     111: 1 
\end{verbatim}

Notes: 

\begin{itemize}
	\item Any line that starts with a hash mark (``\#'') is a comment and will be ignored by \ac{KARMA}. 
	\item The various inputs are named before they are used. In the example, there are three inputs: $ X $, $ Y $, and $ Z $. 
	\item Each row in the truth table is shown, along with the output expected. So, in the example above, an input of $ 000 $ should yield an output of $ 1 $. 
	\item An output of ``-'' is permitted and means ``don't care.'' 
\end{itemize}

\paragraph{Integer}
\label{ASM:para:karma_integer}

In \ac{KARMA}, an integer can be used to define the outputs of the truth table, so it is ``shorthand'' for an entire truth table input. Here is the example of the ``integer'' type input.

\begin{verbatim}
     #Sample Integer Input
     inputs -> A, B, C, D
     onset -> E81A base 16 
\end{verbatim}

Notes: 

\begin{itemize}
	\item Any line that starts with a hash mark (``\#'') is a comment and will be ignored by \ac{KARMA}. 
	\item Input variables are defined first. In this example, there are four inputs: $ A $, $ B $, $ C $, and $ D $. 
	\item The ``onset'' line indicates what combinations of inputs should yield a \emph{True} on a truth table. In the example, the number $ E81A $ is a hexadecimal number that is written like this in binary: 
\end{itemize}

\begin{verbatim}
     1110 1000 0001 1010
      E    8    1    A 
\end{verbatim}

The least significant bit of the binary number, $ 0 $ in this example, corresponds to the output of the first row in the truth table; thus, it is false. Each bit to the left of the least significant bit corresponds to the next row, counting from $ 0000 $ to $ 1111 $. Here is the truth table generated by the hexadecimal integer $ E81A $: 

\begin{table}[H]
	\sffamily
	\newcommand{\head}[1]{\textcolor{white}{\textbf{#1}}}		
	\begin{center}
		\rowcolors{2}{gray!10}{white} % Color every other line a light gray
		\begin{tabular}{ccccc} 
			\rowcolor{black!75}
			\multicolumn{4}{c}{\head{Inputs}} & \head{Output} \\
			A & B & C & D & Y \\
			\hline
			0 & 0 & 0 & 0 & 0 \\
			0 & 0 & 0 & 1 & 1 \\
			0 & 0 & 1 & 0 & 0 \\
			0 & 0 & 1 & 1 & 1 \\
			0 & 1 & 0 & 0 & 1 \\
			0 & 1 & 0 & 1 & 0 \\
			0 & 1 & 1 & 0 & 0 \\
			0 & 1 & 1 & 1 & 0 \\
			1 & 0 & 0 & 0 & 0 \\
			1 & 0 & 0 & 1 & 0 \\
			1 & 0 & 1 & 0 & 0 \\
			1 & 0 & 1 & 1 & 1 \\
			1 & 1 & 0 & 0 & 0 \\
			1 & 1 & 0 & 1 & 1 \\
			1 & 1 & 1 & 0 & 1 \\
			1 & 1 & 1 & 1 & 1 \\
		\end{tabular}
	\end{center}
	\caption{Truth Table for KARMA}
  \label{03:tab:truth_table_for_karma}
\end{table}

The ``Output'' column contains the binary integer $ 1110\;1000\;0001\;1010 $ (or $ E81A_{16} $) from bottom to top. 

\paragraph{Terms}
\label{ASM:para:karma_terms}

Data input can be defined by using the minterms for the Boolean expression. Following is an example minterm input.

\begin{verbatim}
     #Sample Minterms
     inputs -> A, B, C, D
     onset -> 0, 1, 2, 3, 5, 10 
\end{verbatim}

\begin{itemize}
	\item Any line that starts with a hash mark (``\#'') is a comment and will be ignored by \ac{KARMA}. 
	\item The inputs, $ A $, $ B $, $ C $, and $ D $, are defined first. 
	\item The ``onset'' line indicates the minterm numbers that yield a \emph{True} output. 
	\item This is similar to a \ac{SOP} expression, and the digits in that expression could be directly entered on the onset line. For example, the onset line above would have been generated from the Sigma expression in Equation \ref{ASM:eq:KARMA Input}.
\end{itemize}

\begin{align}
	\label{ASM:eq:KARMA Input}
	\int(A,B,C,D) &= \sum(0,1,2,3,5,10)
\end{align}

\subsubsection{Truth Table and K-Map Input}
\label{ASM:subsubsec:karma_truth_table_and_kmap_input}

While \ac{KARMA} will accept a number of different input methods, as described above, one of the easiest to use is the Truth Table, and the related Karnaugh Map, and these are displayed by default when the Karnaugh Map function is selected. The value of any of the cells in the \emph{Out} column in the Truth Table, or cells in the Karnaugh Map, can be cycled through $ 0 $, $ 1 $, and ``don't care'' (indicated by a dash) on each click of the mouse in the cell. The Truth Table and Karnaugh Map are synchronized as cells are clicked. The number of input variables can be adjusted by changing the \emph{Var} setting at the top of the screen. Also, the placement of those variables on the Karnaugh Map can be adjusted as desired. 

\subsubsection{Solution}
\label{ASM:subsubsec:karma_truth_table_and_kmap_input}

\begin{figure}[H]
	\centering
	\includegraphics[width=\maxwidth{.95\linewidth}]{gfx/07_04}
	\caption{Karma Solution}
	\label{fig:07_04}
\end{figure}

To simplify the Karnaugh Map, click the \emph{Minimize} button. A number of windows will pop up (illustrated in Figure \ref{ASM:fig:karma_solution}), each showing the circuit simplification in a slightly different way. Note: the following Boolean expression was entered to generate the illustrated simplification: $ A'C + A'B + AB'C' + B'C'D' $. 

\begin{figure}[H]
	\centering
	\includegraphics[width=\maxwidth{.95\linewidth}]{gfx/07_05}
	\caption{Karma Minimized Solution}
	\label{fig:07_05}
\end{figure}

\marginpar{KARMA includes parenthesis for clarity, but the groups are obvious when the expression is written in normal Boolean form.}

In this solution, a \emph{NOT} term is identified by a leading exclamation point; thus, the minimized expression is: $ A'D' + A'B + AB'C' + A'C $. 

\paragraph{BDDeiro}
\label{ASM:para:karma_bddeiro}

The BDDeiro window is a visualization of a \ac{BDD}, which graphically represents the solution to a logic network. 

\begin{figure}[H]
	\centering
	\includegraphics[width=\maxwidth{.95\linewidth}]{gfx/07_06}
	\caption{Karma BDDeiro Map}
	\label{fig:07_06}
\end{figure}

In a \ac{BDD}, each circle represents an input and the squares at the bottom of the diagram represent the two possible outputs: \emph{False} and \emph{True}. The lines are paths from the inputs to either a \emph{False} or \emph{True} output; thus, a truth table can be viewed graphically. A \ac{BDD} is useful because it provides a compact, visual representation of a Boolean expression, and for any given Boolean expression there is one, and only one, \ac{BDD} representing it. One disadvantage to using a \ac{BDD} is its size, there are potentially two nodes for each input (except the first), and that can lead to a very large diagram.

Figure \ref{ASM:fig:karma_bddeiro_map} is actually a ``Reduced Order'' \ac{BDD} and a number of nodes and paths have been consolidated to make the diagram as simple as possible. The top node represents the start of the decision diagram: input $ a $. If that input is \emph{False}, then follow the dotted line down to node $ d $. (Note: the lines are color-coded to aid in their use; \emph{False} lines are blue and \emph{True} lines are red.) If node $ d $ is false, then that leads directly to output \emph{True}. Thus $ A'D' $ gives a \emph{True} output, and that is one of the minimized solutions. To follow one other path, if $ a $ is \emph{True} (follow the solid line down and right), $ b $ is \emph{False}, $ c $ is \emph{False}, the output is \emph{True}. Thus, $ AB'C' $ is \emph{True}. In a similar way, all four \emph{True} outputs, and three false outputs, can be traced from the top to bottom of the diagram. 

\paragraph{Quine-McCluskey}
\label{ASM:para:quine-mccluskey}

\ac{KARMA} includes the complete Quine-McCluskey solution data. Several tables display the various Implicants and show how they are derived. 

\begin{figure}[H]
	\centering
	\includegraphics[width=\maxwidth{.95\linewidth}]{gfx/07_07}
	\caption{Karma Quine-McCluskey Solution}
	\label{fig:07_07}
\end{figure}

\ac{KARMA} also displays the Covering Table for a Quine-McCluskey solution.

\begin{figure}[H]
	\centering
	\includegraphics[width=\maxwidth{.95\linewidth}]{gfx/07_08}
	\caption{Karma Quine-McCluskey Covering Table}
	\label{fig:07_08}
\end{figure}

Each of the minterms (down the left column) can be turned on or off by clicking on it. The smaller blue dots in the table indicate prime implicants and the larger red dots (if any) indicate essential prime implicants. Because this table is interactive, various different solutions can be attempted by clicking some of the colored dots to achieve the best possible simplification. 

\subsubsection{Practice Problems}
\label{ASM:subsubsec:karma_practice_problems}

The following problems are presented as practice for using \ac{KARMA} to simplify a Boolean expression. Note: designers can select different Prime Implicants so the simplified expression could vary from what is presented below.  

\begin{table}[H]
	\sffamily
	\begin{center}
		\begin{tabular}{c c p{6cm} }
			\multirow{2}{*}{\textbf{1}} 
			& Expression & $ \int(A,B,C,D) = \sum(5,6,7,9,10,11,13,14) $ \\
			& \cellcolor{gray!10} Simplified 
			& \cellcolor{gray!10} $ BC'D+A'BC+ACD'+AB'D $ \\
			\hline
			\multirow{2}{*}{\textbf{2}} 
			& Exression & $ A'BC'D+A'BCD'+A'BCD+AB'C'D+AB'CD'+AB'CD+ABC'D+ABCD' $ \\
			& \cellcolor{gray!10} Simplified 
			& \cellcolor{gray!10} $ BC'D+A'BC+ACD'+AB'D $ \\
			\hline
			\multirow{2}{*}{\textbf{3}} 
			& Exression & A 4-variable Karnaugh Map where cells 5, 6, 7, 9, and 10 are True and 13, 14 are ``Don't Care'' \\
			& \cellcolor{gray!10} Simplified 
			& \cellcolor{gray!10} $ BC'D+AC'D+A'BC+ACD' $ \\
			\hline
			\multirow{2}{*}{\textbf{3}} 
			& Exression & $ \int(A,B,C,D,E) = \sum(0, 3, 4, 12, 13, 14, 15, 24, 25, 28, 29, 30) $ \\
			& \cellcolor{gray!10} Simplified 
			& \cellcolor{gray!10} $ ABD'+A'B'C'DE+BCD'+A'BC+A'B'D'E' $ \\
		\end{tabular}
	\end{center}
	\caption{KARMA Practice Problems}
  \label{ASM:tab:karma_practice_problems}
\end{table}































% ****************************************************************
% Start of Part 2: Practice
% ****************************************************************

\cleardoublepage
\ctparttext{Once the foundations of digital logic are mastered it is time to consider creating circuits that accomplish some practical task. This part of the book begins with the simplest of combinational logic circuits and progresses to sequential logic circuits and, finally, to complex circuits that combine both combinational and sequential elements.}
\part{Practice}
\include{Chapters/08_Combinational}
\chapter{Sequential Logic Circuits}\label{ch09}
\section{Introduction}

Electronic circuits that require memory devices (like flip-flops or registers) and feedback loops are designed using is called ``Sequential Logic.'' The final output of the circuit is determined by both the inputs and internal circuit feedback loops; and the states of the various logic gates may change rapidly as the output from one stage is fed back to various input points. This makes sequential logic circuits much more dynamic, and complex, than combinational circuits, and their analysis normally requires tools like timing diagrams and state maps. Because sequential circuits are able to latch and hold output states, useful applications like flip-flops, registers, counters, and memory are possible.

%***************************************************************************
% Section: Timing Diagrams
%***************************************************************************
\section{Timing Diagrams}
\label{SL:sec:timing_diagrams}

Unlike combinational logic circuits, timing is essential in sequential circuits. Normally, a device will include a \emph{clock} \ac{IC} that generates a square wave that is used to control the sequencing of various activities throughout the device. In order to design and troubleshoot these types of circuits, a timing diagram is developed that shows the relationship between the various input, intermediate, and output signals. Figure \ref{tmg:09_01} is an example timing diagram.

\begin{figure}[H]
  \centering
  \begin{tikztimingtable}[
        timing/slope=0,         % no slope
        timing/coldist=2pt,     % column distance
        xscale=2.0,yscale=1.0,  % scale diagrams
        semithick,               % set line width
    ]
    \footnotesize \# & U     R 8{2Q} 2U     \\
    \footnotesize Clk & 17{C} \\
    %                      P 01 02 03 04 05 06 07 08
    \footnotesize In & [] {L HH LH HL LL LH LL HL HL} \\
    \footnotesize Out & []{L HH LL HH LL LL LL HH HH} \\
    \extracode % Optional
    % \fulltablegrid[]
    % \vertlines[]{}
    \tablerules[]
  \end{tikztimingtable}
  \caption{Example Timing Diagram} 
  \label{tmg:09_01}
\end{figure}

Figure \ref{tmg:09_01} is a timing diagram for a device that has three signals, a clock, an Input, and an Output. 

\begin{itemize}
  \marginpar{Signals in a timing diagram can be described as low//high, $ 0 $//$ 1 $, on//off, or any other number of ways.}

  \item \textsc{\#}: The first line of the timing diagram is only a counter that indicates the number of times the system clock has cycled from Low to High and back again. For example, the first time the clock changes from Low to High is the beginning of the first cycle. This counter is only used to facilitate a discussion about the circuit timing. 

  \item \textsc{Clk}: A clock signal regularly varies between Low and High in a predictable pattern, normally such that the amount of Low time is the same as the High time. In a circuit, a clock pulse is frequently generated by applying voltage to a crystal since the oscillation cycle of a crystal is well known and extremely stable. In Figure \ref{tmg:09_01} the clock cycle is said have a $ 50 $\% duty cycle, that is, half-low and half-high. The exact length of a single cycle would be measured in micro- or nano-seconds but for the purposes of this book all that matters is the relationship between the various signals, not the specific length of a clock cycle.
  
  \item \textsc{In}: The input is Low until Cycle \#$ 1 $, then goes High for one cycle. It then toggles between High and Low at irregular intervals.
  
  \item \textsc{Out}: The output goes High at the beginning of cycle \#$ 1 $. It then follows the input, but only at the start of a clock cycle. Notice that the input goes high halfway through cycle \#$ 2 $ but the output does not go high until the start of cycle \#$ 3 $. Most devices are manufactured to be \emph{edge triggered} and will only change their output at either the positive or negative edge of the clock cycle (this example is positive edge triggered). Thus, notice that no matter when the input toggles the output only changes when the clock transitions from Low to High.

\end{itemize}

Given the timing diagram in Figure \ref{tmg:09_01} it would be relatively easy to build a circuit to match the timing requirements. It would have a single input and output port and the output would match the input on the positive edge of a clock cycle.

Whenever the input for any device changes it takes a tiny, but measurable, amount of time for the output to change since the various transistors, capacitors, and other electronic elements must reach saturation and begin to conduct current. This lag in response is known as \emph{propagation delay} and that is important for an engineer to consider when building a circuit. \marginpar{Ideal circuits with no propagation delay are assumed in this book in order to focus on digital logic rather than engineering.} It is possible to have a circuit that is so complex that the output has still not changed from a previous input signal before a new input signal arrives and starts the process over. In Figure \ref{tmg:09_02} notice that the output goes High when the input goes Low, but only after a tiny propagation delay. 

\begin{figure}[H]
  \centering
  \begin{tikztimingtable}[
    timing/slope=0,         % no slope
    timing/coldist=2pt,     % column distance
    xscale=2.0,yscale=1.0,  % scale diagrams
    semithick,               % set line width
    ]
    \footnotesize \# & U     R 8{2Q} 2U     \\
    \footnotesize Clk & 17{C} \\
    %                      P 01 02 03 04 05 06 07 08
    \footnotesize In & [] {L 2L 1H 2L 1H 2L 1H 2L 1H 2L 1H 1L} \\
    \footnotesize Out & []{L 3.25L 0.75H 2.25L 0.75H 
                            2.25L 0.75H 2.25L 0.75H 2.25L .75H} \\
    \extracode % Optional
    % \fulltablegrid[]
    % \vertlines[]{}
    \tablerules[]
  \end{tikztimingtable}
  \caption{Example Propagation Delay} 
  \label{tmg:09_02}
\end{figure}

It is also true that a square wave is not exactly square. Due to the presence of capacitors and inductors in circuits, a square wave will actually build up and decay over time rather than instantly change. Figure \ref{drw:09_01} shows a typical charge/discharge cycle for a capacitance circuit and the resulting deformation of a square wave. It should be kept in mind, though, that the times involved for capacitance charge/discharge are quite small (measured in nanoseconds or smaller); so for all but the most sensitive of applications, square waves are assumed to be truly square.

\begin{figure}[H]
  \myfloatalign
  \begin{tikzpicture} [circuit logic US, scale=0.50]
  % make all path lines (the node shapes) a little thicker
  \tikzstyle{every path}=[line width=0.50mm]  
  
  % Draw the lines
  \draw[gray,thin] (0,0) grid [step=4] (20,4);
  \draw 
    (0,4) .. controls ( 0.05,0.50) and ( 0.50,0.05) .. (4,0)
          .. controls ( 4.05,3.50) and ( 4.55,3.95) .. (8,4)
          .. controls ( 8.05,0.50) and ( 8.50,0.05) .. (12,0)
          .. controls (12.05,3.50) and (12.55,3.95) .. (16,4)
          .. controls (16.05,0.50) and (16.50,0.05) .. (20,0)
  ;
  \end{tikzpicture}
  \caption{Capacitor Charge and Discharge}
	\label{drw:09_01}  
\end{figure}

All devices are manufactured with certain tolerance levels built-in such that when the voltage gets ``close'' to the required level then the device will react, thus mitigating the effect of the deformed square wave. Also, for applications where speed is critical, such as space exploration or medical, high-speed devices are available that both sharpen the edges of the square wave and reduce propagation delay significantly. 

%***************************************************************************
% Section: Flip-Flops
%***************************************************************************
\section{Flip-Flops}
\label{SL:sec:flip-flops}

\subsection{Introduction}
\label{SL:subsec:intro_to_flip-flops}

Flip-flops are digital circuits that can maintain an electronic state even after the initial signal is removed. They are, then, the simplest of memory devices. Flip-flops are often called \emph{latches} since they are able to ``latch'' and hold some electronic state. In general, devices are called flip-flops when sequential logic is used and the output only changes on a clock pulse and they are called latches when combinational logic is used and the output constantly reacts to the input. Commonly, flip-flops are used for clocked devices, like counters, while latches are used for storage. However, these terms are often considered synonymous and are used interchangeably.

\subsection{SR Latch}
\label{SL:subsec:sr_latch}

One of the simplest of the flip-flops is the SR (for \emph{Set-Reset}) latch. \marginpar{This latch is often also called an \emph{RS Latch}.} Figure \ref{fig:09_01} illustrates a logic diagram for an \emph{SR latch} built with \textsf{NAND} gates.

\begin{figure}[H]
	\centering
	\includegraphics[width=\maxwidth{.95\linewidth}]{gfx/09_01}
	\caption{SR Latch Using NAND Gates}
	\label{fig:09_01}
\end{figure}

Table \ref{tab:09_01} is the truth table for an \emph{SR Latch}. Notice that unlike truth tables used earlier in this book, some of the outputs are listed as ``Last Q'' since they do not change from the previous state.

\begin{table}[H]
  \sffamily
  \newcommand{\head}[1]{\textcolor{white}{\textbf{#1}}}    
  \begin{center}
    \rowcolors{2}{gray!10}{white} % Color every other line a light gray
    \begin{tabular}{ccccc} 
      \rowcolor{black!75}
      \multicolumn{3}{c}{\head{Inputs}} & \multicolumn{2}{c}{\head{Outputs}} \\
      E & S & R & Q & Q' \\
      \hline
      0 & 0 & 0 & Last Q & Last Q' \\
      0 & 0 & 1 & Last Q & Last Q' \\
      0 & 1 & 0 & Last Q & Last Q' \\
      0 & 1 & 1 & Last Q & Last Q' \\
      1 & 0 & 0 & Last Q & Last Q' \\
      1 & 0 & 1 & 0 & 1 \\
      1 & 1 & 0 & 1 & 0 \\
      1 & 1 & 1 & Not Allowed & Not Allowed \\
    \end{tabular}
  \end{center}
  \caption{Truth Table for SR Latch}
  \label{tab:09_01}
\end{table}

In Figure \ref{fig:09_01}, input \emph{E} is an enable and must be high for the latch to work; when it is low then the output state remains constant regardless of how inputs \emph{S} or \emph{R} change. When the latch is enabled, if $ S=R=0 $ then the values of \emph{Q} and \emph{Q'} remain fixed at their last value; or, the circuit is ``remembering'' how they were previously set. When input \emph{S} goes high, then \emph{Q} goes high (the latch's output, \emph{Q}, is ``set''). When input \emph{R} goes high, then \emph{Q'} goes high (the latch's output, \emph{Q}, is ``reset''). Finally, it is important that in this latch inputs \emph{R} and \emph{S} cannot both be high when the latch is enabled or the circuit becomes unstable and output \emph{Q} will oscillate between high and low as fast as the \textsf{NAND} gates can change; thus, input $ 111_2 $ must be avoided. Normally, there is an additional bit of circuit prior to the inputs to ensure \emph{S} and \emph{R} will always be different (an \textsf{XOR} gate could do that job).

An SR Latch may or may not include a clock input; if not, then the outputs change immediately when any of the inputs change, like in a combinational circuit. If the designer needs a clocked type of memory device, which is routine, then the typical choice would be a \emph{JK Flip-Flop}, covered later. Figure \ref{tmg:09_03} is a timing diagram for the SR Latch in Listing \ref{fig:09_01}.

\begin{figure}[H]
  \centering
  \begin{tikztimingtable}[
    timing/slope=0,         % no slope
    timing/coldist=2pt,     % column distance
    xscale=2.0,yscale=1.0,  % scale diagrams
    semithick,               % set line width
    ]
    \footnotesize \# & U     R 8{2Q} 2U     \\
%    \footnotesize Clk & 17{C} \\
    %                        P 01 02 03 04 05 06 07 08
    \footnotesize Ena  & [] {L LL LL HH HH HH HH HH HH} \\
    \footnotesize S    & [] {H HH LL HH LL LL HH HH HH} \\
    \footnotesize R    & [] {L LL HH LL LL HH HH LL LL} \\
    \footnotesize Q    & [] {H HH HH HH HH LL LL HH HH} \\
    \footnotesize Qn   & [] {L LL LL LL LL HH LL LL LL} \\
    \extracode % Optional
    % \fulltablegrid[]
    % \vertlines[]{}
    \tablerules[]
  \end{tikztimingtable}
  \caption{SR Latch Timing Diagram} 
  \label{tmg:09_03}
\end{figure}

At the start of Figure \ref{tmg:09_03} \emph{Enable} is low and there is no change in $ Q $ and $ Q' $ until frame 3, when \emph{Enable} goes high. At that point, $ S $ is high and $ R $ is low so $ Q $ is high and $ Q' $ is low. At frame $ 4 $ both $ S $ and $ R $ are low so there is no change in $ Q $ or $ Q' $. At frame 5 $ R $ goes high so $ Q $ goes low and $ Q' $ goes high. In frame $ 6 $ both $ S $ and $ R $ are high so both $ Q $ and $ Q' $ go low. Finally, in frame $ 7 $ $ S $ stays high and $ R $ goes low so $ Q $ goes high and $ Q' $ stays low.

\Le includes an \emph{S-R Flip-Flop} device, as illustrated in Figure \ref{fig:09_02}. 

\begin{figure}[H]
	\centering
	\includegraphics[width=\maxwidth{.95\linewidth}]{gfx/09_02}
	\caption{SR Latch}
	\label{fig:09_02}
\end{figure}

The \emph{S-R Flip-Flop} has \emph{S} and \emph{R} input ports and \emph{Q} and \emph{Q'} output ports. Also notice that there is no \emph{Enable} input but there is a \emph{Clock} input.\marginpar{On a logic diagram a clock input is indicated with a triangle symbol.} Because this is a clocked device the output would only change on the edge of a clock pulse and that makes the device a flip-flop rather than a latch. As shown, \emph{S} is high and the clock has pulsed so \emph{Q} is high, or the flip-flop is ``set.'' Also notice that on the top and bottom of the device there is an \emph{S} and \emph{R} input port that are not connected. These are ``preset'' inputs that let the designer hard-set output \emph{Q} at either one or zero, which is useful during a power-up routine. Since this device has no \emph{Enable} input it is possible to use the \emph{R} preset port as a type of enable. If a high is present on the \emph{R} preset then output \emph{Q} will go low and stay there until the \emph{R} preset returns to a low state.

\subsection{Data (D) Flip-Flop}
\label{SL:subsec:data_latch}

A Data Flip-Flop (or \emph{D Flip-Flop}) is formed when the inputs for an \emph{SR Flip-Flop} are tied together through an inverter (which also means that \emph{S} and \emph{R} cannot be high at the same time, which corrects the potential problem with two high inputs in an \emph{SR Flip-Flop}). Figure \ref{fig:09_03} illustrates an \emph{SR Flip-Flop} being used as a \emph{D Flip-Flop} in \Le. 

\begin{figure}[H]
	\centering
	\includegraphics[width=\maxwidth{.95\linewidth}]{gfx/09_03}
	\caption{D Flip-Flop Using SR Flip-Flop}
	\label{fig:09_03}
\end{figure}

In Figure \ref{fig:09_03}, the \emph{D} input (for ``Data'') is latched and held on each clock cycle. Even though Figure \ref{fig:09_03} shows the inverter external to the latch circuit, in reality, a \emph{D Flip-Flop} device bundles everything into a single package with only \emph{D} and clock inputs and \emph{Q} and \emph{Q'} outputs, as in Figure \ref{fig:09_04}. Like the \emph{RS Flip-Flop}, \emph{D Flip-Flops} also have ``preset'' inputs on the top and bottom that lets the designer hard-set output \emph{Q} at either one or zero, which is useful during a power-up routine.

\begin{figure}[H]
	\centering
	\includegraphics[width=\maxwidth{.95\linewidth}]{gfx/09_04}
	\caption{D Flip-Flop}
	\label{fig:09_04}
\end{figure}

Figure \ref{tmg:09_04} is the timing diagram for a Data Flip-Flop.

\begin{figure}[H]
  \centering
  \begin{tikztimingtable}[
    timing/slope=0,         % no slope
    timing/coldist=2pt,     % column distance
    xscale=2.0,yscale=1.0,  % scale diagrams
    semithick,               % set line width
    ]
    \footnotesize \# & U     R 8{2Q} 2U     \\
%    \footnotesize Clk & 17{C} \\
    %                     P 01 02 03 04 05 06 07 08
    \footnotesize Clk & [] {L HL HL HL HL HL HL HL HL} \\
    \footnotesize D & []   {H HH LL HH LL HL HH LH HH} \\
    \footnotesize Q & []   {H HH LL HH LL HH HH LL HH} \\
    \extracode % Optional
    % \fulltablegrid[]
    % \vertlines[]{}
    \tablerules[]
  \end{tikztimingtable}
  \caption{D Latch Timing Diagram} 
  \label{tmg:09_04}
\end{figure}

In Figure \ref{tmg:09_04} it is evident that output \emph{Q} follows input \emph{D} but only on a positive clock edge. The latch ``remembers'' the value of \emph{D} until the next clock pulse no matter how it changes between pulses. 

\subsection{JK Flip-Flop}
\label{SL:subsec:jk_flip-flop}

The \emph{JK flip-flop} is the ``workhorse'' of the flip-flop family. Figure \ref{fig:09_05} is the logic diagram for a \emph{JK flip-flop}.

\begin{figure}[H]
	\centering
	\includegraphics[width=\maxwidth{.95\linewidth}]{gfx/09_05}
	\caption{JK Flip-Flop}
	\label{fig:09_05}
\end{figure}

Internally, a \emph{JK flip-flop} is similar to an \emph{RS Latch}. However, the outputs to the circuit (\emph{Q} and \emph{Q'}) are connected to the inputs (\emph{J} and \emph{K}) in such a way that the unstable input condition (both \emph{R} and \emph{S} high) of the \emph{RS Latch} is corrected. If both inputs, \emph{J} and \emph{K}, are high then the outputs are toggled (they are ``flip-flopped'') on the next clock pulse. This toggle feature makes the \emph{JK flip-flop} extremely useful in many logic circuits.

Figure \ref{tmg:09_05} is the timing diagram for a \emph{JK flip-flop}.

\begin{figure}[H]
  \centering
  \begin{tikztimingtable}[
    timing/slope=0,         % no slope
    timing/coldist=2pt,     % column distance
    xscale=2.0,yscale=1.0,  % scale diagrams
    semithick,               % set line width
    ]
    \footnotesize \# & U     R 8{2Q} 2U     \\
    \footnotesize Clk & 17{C} \\
    %                      P 01 02 03 04 05 06 07 08
    \footnotesize J  & [] {H HH LL LL LL HL LL LL LL} \\
    \footnotesize K  & [] {L LL HH HH LL HL LL HH LL} \\
    \footnotesize Q  & [] {H HH LL LL LL HH HH LL LL} \\
    \footnotesize Q' & [] {L LL HH HH HH LL LL HH HH} \\
    \extracode % Optional
    % \fulltablegrid[]
    % \vertlines[]{}
    \tablerules[]
  \end{tikztimingtable}
  \caption{JK Flip-Flop Timing Diagram} 
  \label{tmg:09_05}
\end{figure}

Table \ref{tab:09_01} summarizes timing diagram \ref{tmg:09_05}. Notice that on clock pulse five both \emph{J} and \emph{K} are high so \emph{Q} toggles. Also notice that \emph{Q'} is not indicated in the table since it is always just the complement of \emph{Q}.

\begin{table}[H]
	\sffamily
	\newcommand{\head}[1]{\textcolor{white}{\textbf{#1}}}		
	\begin{center}
		\rowcolors{2}{gray!10}{white} % Color every other line a light gray
		\begin{tabular}{cccc} 
			\rowcolor{black!75}
			\head{Clock Pulse} & \head{J} & \head{K} & \head{Q} \\
			1 & H & L & H \\
			2 & L & H & L \\
			3 & L & H & L \\
			4 & L & L & L \\
			5 & H & H & H \\
			6 & L & L & H \\
			7 & L & H & L \\
			8 & L & L & L 
		\end{tabular}
	\end{center}
	\caption{JK Flip-Flop Timing Table}
	\label{tab:09_01}
\end{table}

\subsection{Toggle (T) Flip-Flop}
\label{SL:subsec:toggle_flip-flop}

If the J and K inputs to a \emph{JK Flip-Flop} are tied together, then when the input is high the output will toggle on every clock pulse but when the input is Low then the output remains in the previous state. This is often referred to as a \emph{Toggle Flip-Flop} (or \emph{T Flip-Flop}). \emph{T Flip-Flops} are not usually found in circuits as separate \acp{IC} since they are so easily created by soldering together the inputs of a standard \emph{JK Flip-Flop}. Figure \ref{fig:09_06} is a \emph{T Flip-Flop} in \Le.

\begin{figure}[H]
	\centering
	\includegraphics[width=\maxwidth{.95\linewidth}]{gfx/09_06}
	\caption{T Flip-Flop}
	\label{fig:09_06}
\end{figure}

Figure \ref{tmg:09_06} is the timing diagram for a \emph{T flip-flop}.

\begin{figure}[H]
  \centering
  \begin{tikztimingtable}[
    timing/slope=0,         % no slope
    timing/coldist=2pt,     % column distance
    xscale=2.0,yscale=1.0,  % scale diagrams
    semithick,               % set line width
    ]
    \footnotesize \# & U     R 8{2Q} 2U     \\
    \footnotesize Clk & 17{C} \\
    %                      P 01 02 03 04 05 06 07 08
    \footnotesize D  & [] {H HH HL LL LH HH HL LH HH} \\
    \footnotesize Q  & [] {H LL HH HH HH LL HH HH LL} \\
    \footnotesize Q' & [] {L HH LL LL LL HH LL LL HH} \\
    \extracode % Optional
    % \fulltablegrid[]
    % \vertlines[]{}
    \tablerules[]
  \end{tikztimingtable}
  \caption{Toggle Flip-Flop Timing Diagram} 
  \label{tmg:09_06}
\end{figure}

In Figure \ref{tmg:09_06} when \emph{D}, the data input, is high then \emph{Q} toggles on the positive edge of every clock cycle.

\subsection{Master-Slave Flip-Flops}
\label{SL:subsec:master-slave_flip-flops}

Master-Slave Flip-Flops are two flip-flops connected in a cascade and operating from the same clock pulse. These flip-flops tend to stabilize an input circuit and are used where the inputs may have voltage glitches (such as from a push button). Figure \ref{fig:09_07} is the logic diagram for two \emph{JK flip-flops} set up as a Master-Slave Flip-Flop.

\begin{figure}[H]
	\centering
	\includegraphics[width=\maxwidth{.95\linewidth}]{gfx/09_07}
	\caption{Master-Slave Flip-Flop}
	\label{fig:09_07}
\end{figure}

Because of the \textsf{NOT} gate on the clock signal these two flip-flops will activate on opposite clock pulses, which means the flip-flop one will enable first and read the JK inputs, then flip-flop two will enable and read the output of flip-flop one. The result is that any glitches on the input signals will tend to be eliminated.

Often, only one physical \ac{IC} is needed to provide the two flip-flops for a master-slave circuit because dual \emph{JK flip-flops} are frequently found on a single \ac{IC}. Thus, the output pins for one flip-flop could be connected directly to the input pins for the second flip-flop on the same \ac{IC}. By combining two flip-flops into a single \ac{IC} package, circuit design can be simplified and fewer components need to be purchased and mounted on a circuit board. 

%***************************************************************************
% Section: Registers
%***************************************************************************
\section{Registers}
\label{SL:sec:registers}

\subsection{Introduction}
\label{SL:subsec:intro_to_registers}

A register is a simple memory device that is composed of a series of flip-flops wired together such that they share a common clock pulse. Registers come in various sizes and types and are often used as ``scratch pads'' for devices. For example, a register can hold a number entered on a keypad until the calculation circuit is ready do something with that number or a register can hold a byte of data coming from a hard drive until the \ac{CPU} is ready to move that data someplace else. 

\subsection{Registers As Memory}
\label{SL:subsec:registers_as_memory}

Internally, a register is constructed from \emph{D Flip-Flops} as illustrated in Figure \ref{fig:09_08}.

\begin{figure}[H]
	\centering
	\includegraphics[width=\maxwidth{.95\linewidth}]{gfx/09_08}
	\caption{4-Bit Register}
	\label{fig:09_08}
\end{figure}

Data are moved from the input port, in the upper left corner, into the register; one bit goes into each of the four \emph{D Flip-Flops}. Because each latch constantly outputs whatever it contains, the output port, in the lower right corner, combines the four data bits and displays the contents of the register. Thus, a four-bit register can ``remember'' some number until it is needed, it is a four-bit memory device.

\subsubsection{74x273 Eight-Bit Register}
\label{SL:subsubsec:74x273_8-bit_register}

In reality, designers do not build memory from independent flip-flops, as shown in Figure \ref{fig:09_08}. Instead, they use a memory \ac{IC} that contains the amount of memory needed. In general, purchasing a memory \ac{IC} is cheaper and more reliable than attempting to build a memory device.

One such memory device is a $ 74x273 $ eight-bit register. This device is designed to load and store a single eight-bit number, but other memory devices are much larger, including \ac{RAM} that can store and retrieve millions of eight-bit bytes.

\subsection{Shift Registers}
\label{SL:subsec:shift_registers}

Registers have an important function in changing a stream of data from serial to parallel or parallel to serial; a function that is called ``shifting'' data. For example, data entering a computer from a network or \ac{USB} port is in serial form; that is, one bit at a time is streamed into or out of the computer. However, data inside the computer are always moved in parallel; that is, all of the bits in a word are placed on a bus and moved through the system simultaneously. Changing data from serial to parallel or vice-verse is an essential function and enables a computer to communicate over a serial device.

Figure \ref{fig:09_09} illustrates a four-bit parallel-in/serial-out shift register. The four-bit data into the register is placed on \emph{D0}-\emph{D3}, the \emph{Shift\_Write} bit is set to $ 1 $ and the clock is pulsed to write the data into the register. Then the \emph{Shift\_Write} bit is set to $ 0 $ and on each clock pulse the four bits are shifted right to the \emph{SOUT} (for ``serial out'') port.

\begin{figure}[H]
	\centering
	\includegraphics[width=\maxwidth{.95\linewidth}]{gfx/09_09}
	\caption{Shift Register}
	\label{fig:09_09}
\end{figure}

It is possible to also build other shift register configurations: serial-in/serial-out, parallel-in/parallel-out, and serial-in/parallel-out. However, universal shift registers are commonly used since they can be easily configured to work with data in either serial or parallel formats.

%***************************************************************************
% Section: Counters
%***************************************************************************
\section{Counters}
\label{SL:sec:counters}

\subsection{Introduction}
\label{SL:subsec:intro_to_counters}

Counters are a type of sequential circuit designed to count input pulses (normally from a clock) and then activate some output when a specific count is reached. Counters are commonly used as timers; thus, all digital clocks, microwave ovens, and other digital timing displays use some sort of counting circuit. However, counters can be used in many diverse applications. For example, a speed gauge is a counter. By attaching some sort of sensor to a rotating shaft and counting the number of revolutions for $ 60 $ seconds the \ac{RPM} is determined. Counters are also commonly used as frequency dividers. If a high frequency is applied to the input of a counter, but only the ``tens'' count is output, then the input frequency will be divided by ten. As one final example, counters can also be used to control sequential circuits or processes; each count can either activate or deactivate some part of a circuit that controls one of the sequential processes. 

\subsection{Asynchronous Counters}
\label{SL:subsec:asynchronous_counters}

One of the simplest counters possible is an asynchronous two-bit counter. This can be built with a two \emph{JK flip-flops} in sequence, as shown in Figure \ref{fig:09_10}.

\begin{figure}[H]
	\centering
	\includegraphics[width=\maxwidth{.95\linewidth}]{gfx/09_10}
	\caption{Asynchronous 2-Bit Counter}
	\label{fig:09_10}
\end{figure}

The J-K inputs on both flip-flops are tied high and the clock is wired into \emph{FF00}. On every positive-going clock pulse the \emph{Q} output for \emph{FF00} toggles and that is wired to the clock input of \emph{FF01}, toggling that output. This type of circuit is frequently called a ``ripple'' counter since the clock pulse must ripple through all of the flip-flops.

In Figure \ref{fig:09_10}, assume both flip-flops start with the output low (or $ 00 $ out). On the first clock pulse, \emph{Q\_FF00} will go high, and that will send a high signal to the clock input of \emph{FF01} which activates \emph{Q\_FF01}. At this point, the \emph{Q} outputs for both flip-flops are high (or $ 11 $ out). On the next clock pulse, \emph{Q\_FF00} will go low, but \emph{Q\_FF01} will not change since it only toggles when the clock goes from low to high. At this point, the output is $ 01 $. On the next clock pulse, \emph{Q\_FF00} will go high and \emph{Q\_FF01} will toggle low: $ 10 $. The next clock pulse toggles \emph{Q\_FF00} to low but \emph{Q\_FF01} does not change: $ 00 $. Then the cycle repeats. This simple circuit counts: $ 00 $, $ 11 $, $ 10 $, $ 01 $. (Note, \emph{Q\_FF00} is the low-order bit.) This counter is counting backwards, but it is a trivial exercise to add the functionality needed to reverse that count.

An asynchronous three-bit counter looks much like the asynchronous two-bit counter, except that a third stage is added.

\begin{figure}[H]
	\centering
	\includegraphics[width=\maxwidth{.95\linewidth}]{gfx/09_11}
	\caption{Asynchronous 3-Bit Counter}
	\label{fig:09_11}
\end{figure}

In Figure \ref{fig:09_11}, the ripple of the clock pulse from one flip-flop to the next is more evident than in the two-bit counter. However, the overall operation of this counter is very similar to the two-bit counter.

More stages can be added so to any desired number of outputs. Asynchronous (or ripple) counters are very easy to build and require very few parts. Unfortunately, they suffer from two rather important flaws:

\begin{itemize}
  \item \textsc{Propagation Delay}. As the clock pulse ripples through the various flip-flops, it is slightly delayed by each due to the simple physical switching of the circuitry within the flip-flop. Propagation delay cannot be prevented and as the number of stages increases the delay becomes more pronounced. At some point, one clock pulse will still be winding its way through all of the flip-flops when the next clock pulse hits the first stage and this makes the counter unstable. 

  \item \textsc{Glitches}. If a three-bit counter is needed, there will be a very brief moment while the clock pulse ripples through the flip-flops that the output will be wrong. For example, the circuit should go from $ 111 $ to $ 000 $, but it will actually go from $ 111 $ to $ 110 $ then $ 100 $ then $ 000 $ as the ``low'' ripples through the flip-flops. These glitches are very short, but they may be enough to introduce errors into a circuit.
\end{itemize}

Figure \ref{fig:09_12} is a four-bit asynchronous counter.

\begin{figure}[H]
	\centering
	\includegraphics[width=\maxwidth{.95\linewidth}]{gfx/09_12}
	\caption{Asynchronous 4-Bit Counter}
	\label{fig:09_12}
\end{figure}

Figure \ref{tmg:09_07} is the timing diagram obtained when the counter in Figure \ref{fig:09_12} is executing.

\begin{figure}[H]
  \centering
  \begin{tikztimingtable}[
    timing/slope=0,         % no slope
    timing/coldist=2pt,     % column distance
    xscale=1.0,yscale=1.0,  % scale diagrams
    semithick,               % set line width
    ]
    \footnotesize \# & U     R 18{2Q} 2U     \\
    \footnotesize Clk & 36{C} \\
    %                      P 01 02 03 04 05 06 07 08 09 10 11 12 13 14 15 16 17 18
    \footnotesize Y0 & [] {L HH LL HH LL HH LL HH LL HH LL HH LL HH LL HH LL HH LL} \\
    \footnotesize Y1 & [] {L HH HH LL LL HH HH LL LL HH HH LL LL HH HH LL LL HH HH} \\
    \footnotesize Y2 & [] {L HH HH HH HH LL LL LL LL HH HH HH HH LL LL LL LL HH HH} \\
    \footnotesize Y3 & [] {L HH HH HH HH HH HH HH HH LL LL LL LL LL LL LL LL HH HH} \\
    \extracode % Optional
    % \fulltablegrid[]
    % \vertlines[]{}
    \tablerules[]
  \end{tikztimingtable}
  \caption{4-Bit Asynchronous Counter Timing Diagram} 
  \label{tmg:09_07}
\end{figure}

Notice that \emph{Y0} counts at half the frequency of the clock and then \emph{Y1} counts at half that frequency and so forth. Each stage that is added will count at half the frequency of the previous stage.

\subsection{Synchronous Counters}
\label{SL:subsec:synchronous_counters}

Both problems with ripple counters can be corrected with a synchronous counter, where the same clock pulse is applied to every flip-flop at one time. Here is the logic diagram for a synchronous two-bit counter: 

\begin{figure}[H]
	\centering
	\includegraphics[width=\maxwidth{.95\linewidth}]{gfx/09_13}
	\caption{Synchronous 2-Bit Counter}
	\label{fig:09_13}
\end{figure}

Notice in this circuit, the clock is applied to both flip-flops and control is exercised by applying the output of one stage to both $ J $ and $ K $ inputs of the next stage, which effectively enables/disables that stage. When Q\_FF00 is high, for example, then the next clock pulse will make FF01 change states. 

\begin{figure}[H]
	\centering
	\includegraphics[width=\maxwidth{.95\linewidth}]{gfx/09_14}
	\caption{Synchronous 3-Bit Counter}
	\label{fig:09_14}
\end{figure}

A three-bit synchronous counter applies the clock pulse to each flip-flop. Notice, though, that the output from the first two stages must routed through an \textsf{AND} gate so \emph{FF03} will only change when \emph{Q\_FF00} and \emph{Q\_FF01} are high. This circuit would then count properly from $ 000 $ to $ 111 $. 

In general, synchronous counters become more complex as the number of stages increases since it must include logic for every stage to determine when that stage should activate. This complexity results in greater power usage (every additional gate requires power) and heat generation; however, synchronous counters do not have the propagation delay problems found in asynchronous counters. 

\begin{figure}[H]
	\centering
	\includegraphics[width=\maxwidth{.95\linewidth}]{gfx/09_15}
	\caption{Synchronous 4-Bit Up Counter}
	\label{fig:09_15}
\end{figure}

Figure \ref{tmg:09_08} is the timing diagram for the circuit illustrated in Figure \ref{fig:09_15}.

\begin{figure}[H]
  \centering
  \begin{tikztimingtable}[
    timing/slope=0,         % no slope
    timing/coldist=2pt,     % column distance
    xscale=1.0,yscale=1.0,  % scale diagrams
    semithick,               % set line width
    ]
    \footnotesize \# & U     R 18{2Q} 2U     \\
    \footnotesize Clk & 36{C} \\
    %                      P 01 02 03 04 05 06 07 08 09 10 11 12 13 14 15 16
   %17 18
    \footnotesize Y0 & [] {L HH LL HH LL HH LL HH LL HH LL HH LL HH LL HH LL HH LL} \\
    \footnotesize Y1 & [] {L LL HH HH LL LL HH HH LL LL HH HH LL LL HH HH LL LL HH} \\
    \footnotesize Y2 & [] {L LL LL LL HH HH HH HH LL LL LL LL HH HH HH HH LL LL LL.} \\
    \footnotesize Y3 & [] {L LL LL LL LL LL LL LL HH HH HH HH HH HH HH HH LL LL LL} \\
    \extracode % Optional
    % \fulltablegrid[]
    % \vertlines[]{}
    \tablerules[]
  \end{tikztimingtable}
  \caption{4-Bit Synchronous Counter Timing Diagram} 
  \label{tmg:09_08}
\end{figure}

The timing diagram for the asynchronous counter in Figure \ref{tmg:09_07} is the same as that for an synchronous counter in Figure \ref{tmg:09_08} since the output of the two counters are identical. The only difference in the counters is in how the count is obtained, and designers would normally opt for a synchronous \ac{IC} since that is a more efficient circuit.

\subsubsection{Synchronous Down Counters}
\label{SL:subsubsec:synchronous_down_counters}

It is possible to create a counter that counts down rather than up by using the \emph{Q'} outputs of the flip-flops to trigger the next stage. Figure \ref{fig:09_16} illustrates a four-bit down counter.

\begin{figure}[H]
	\centering
	\includegraphics[width=\maxwidth{.95\linewidth}]{gfx/09_16}
	\caption{Synchronous 4-Bit Down Counter}
	\label{fig:09_16}
\end{figure}

Figure \ref{tmg:09_09} is the timing diagram for the circuit illustrated in Figure \ref{fig:09_16}.

\begin{figure}[H]
	\centering
	\begin{tikztimingtable}[
		timing/slope=0,         % no slope
		timing/coldist=2pt,     % column distance
		xscale=1.0,yscale=1.0,  % scale diagrams
		semithick,               % set line width
		]
		\footnotesize \# & U     R 18{2Q} 2U     \\
		\footnotesize Clk & 36{C} \\
		%                      P 01 02 03 04 05 06 07 08 09 10 11 12 13 14 15 16
	 %17 18
		\footnotesize Y0 & [] {L HH LL HH LL HH LL HH LL HH LL HH LL HH LL HH LL HH LL} \\
		\footnotesize Y1 & [] {L HH HH LL LL HH HH LL LL HH HH LL LL HH HH LL LL HH HH} \\
		\footnotesize Y2 & [] {L HH HH HH HH LL LL LL LL HH HH HH HH LL LL LL LL HH HH} \\
		\footnotesize Y3 & [] {L HH HH HH HH HH HH HH HH LL LL LL LL LL LL LL LL HH HH} \\
		\extracode % Optional
		% \fulltablegrid[]
		% \vertlines[]{}
		\tablerules[]
	\end{tikztimingtable}
	\caption{4-Bit Synchronous Down Counter Timing Diagram} 
	\label{tmg:09_09}
\end{figure}


\subsection{Ring Counters}
\label{SL:subsec:ring_counters}

A ring counter is a special kind of counter where only one output is active at a time. As an example, a four-bit ring counter may output this type of pattern:

\medskip
\texttt{ $ 1000 - 0100 - 0010 - 0001 $ }
\medskip

Notice the high output cycles through each of the bit positions and then recycles. Ring counters are useful as controllers for processes where one process must follow another in a sequential manner. Each of the bits from the ring counter can be used to activate a different part of the overall process; thus ensuring the process runs in proper sequence. The circuit illustrated in Figure \ref{fig:09_17} is a 4-bit ring counter.

\begin{figure}[H]
	\centering
	\includegraphics[width=\maxwidth{.95\linewidth}]{gfx/09_17}
	\caption{4-Bit Ring Counter}
	\label{fig:09_17}
\end{figure}

When the circuit is initialized none of the flip-flops are active. Because \emph{Q'} is high for all flip-flops, \textsf{AND} gate \emph{U2} is active and that sends a high through \emph{U1} and into the data port of \emph{FF0}. The purpose of \emph{U2} is to initialize \emph{FF0} for the first count and then \emph{U2} is never activated again. On each clock pulse the next flip-flop in sequence is activated. When \emph{FF3} is active that output is fed back through \emph{U1} to \emph{FF0}, completing the ring and re-starting the process.

Figure \ref{tmg:09_10} is the timing diagram for the ring counter illustrated in Figure \ref{fig:09_17}.

\begin{figure}[H]
  \centering
  \begin{tikztimingtable}[
    timing/slope=0,         % no slope
    timing/coldist=2pt,     % column distance
    xscale=2.0,yscale=1.0,  % scale diagrams
    semithick,               % set line width
    ]
    \footnotesize \# & U     R 8{2Q} 2U     \\
    \footnotesize Clk & 17{C} \\
    %                        P 01 02 03 04 05 06 07 08
    \footnotesize Y0 & [] {L HH LL LL LL HH LL LL LL} \\
    \footnotesize Y1 & [] {L LL HH LL LL LL HH LL LL} \\
    \footnotesize Y2 & [] {L LL LL HH LL LL LL HH LL} \\
    \footnotesize Y3 & [] {L LL LL LL HH LL LL LL HH} \\
    \extracode % Optional
    % \fulltablegrid[]
    % \vertlines[]{}
    \tablerules[]
  \end{tikztimingtable}
  \caption{4-Bit Ring Counter Timing Diagram} 
  \label{tmg:09_10}
\end{figure}

On each clock pulse a different bit is toggled high, proceeding around the four-bit nibble in a ring pattern.

\subsubsection{Johnson Counters}
\label{SL:subsubsec:johnson_counters}

One common modification of a ring counter is called a Johnson, or ``Twisted Tail,'' ring counter. In this case, the counter outputs this type of pattern.

\medskip
\texttt{ $ 1000 - 1100 - 1110 - 1111 - 0111 - 0011 - 0001 - 0000 $ }
\medskip

The circuit illustrated in Figure \ref{fig:09_18} is a 4-bit Johnson counter.

\begin{figure}[H]
	\centering
	\includegraphics[width=\maxwidth{.95\linewidth}]{gfx/09_18}
	\caption{4-Bit Johnson Counter}
	\label{fig:09_18}
\end{figure}

This is very similar to the ring counter except the feedback loop is from \emph{Q'} rather than \emph{Q} of \emph{FF3} and because of that, the \textsf{AND} gate initialization is no longer needed. 

Figure \ref{tmg:09_11} is the timing diagram for the Johnson counter illustrated in Figure \ref{fig:09_18}.

\begin{figure}[H]
	\centering
	\begin{tikztimingtable}[
		timing/slope=0,         % no slope
		timing/coldist=2pt,     % column distance
		xscale=1.0,yscale=1.0,  % scale diagrams
		semithick,               % set line width
		]
		
    \footnotesize \# & U     R 18{2Q} 2U     \\
\footnotesize Clk & 36{C} \\
%                          P 01 02 03 04 05 06 07 08 09 10 11 12 13 14 15 16
%   17 18		
		\footnotesize Y0 & [] {L HH HH HH HH LL LL LL LL HH HH HH HH LL LL LL LL  HH H} \\
		\footnotesize Y1 & [] {L LL HH HH HH HH LL LL LL LL HH HH HH HH LL LL LL LL H} \\
		\footnotesize Y2 & [] {L LL LL HH HH HH HH LL LL LL LL HH HH HH HH LL LL LL L} \\
		\footnotesize Y3 & [] {L LL LL LL HH HH HH HH LL LL LL LL HH HH HH HH LL LL L} \\
		\extracode % Optional
		% \fulltablegrid[]
		% \vertlines[]{}
		\tablerules[]
	\end{tikztimingtable}
	\caption{4-Bit Johnson Counter Timing Diagram} 
	\label{tmg:09_11}
\end{figure}

\subsection{Modulus Counters}
\label{SL:subsec:modulus_counters}

Each of the counters created so far have had one significant flaw, they only count up to a number that is a power of two. A two-bit counter counts from zero to three, a three-bit counter counts from zero to seven, a four-bit counter counts from zero to $ 15 $, and so forth. With a bit more work a counter can be created that stops at some other number. These types of counters are called ``modulus counters'' and an example of one of the most common modulus counters is a decade counter that counts from zero to nine. The circuit illusted in Figure \ref{fig:09_19} is a decade counter.

\begin{figure}[H]
	\centering
	\includegraphics[width=\maxwidth{.95\linewidth}]{gfx/09_19}
	\caption{Decade Counter}
	\label{fig:09_19}
\end{figure}

This is only a four-bit counter (found in Figure \ref{fig:09_15}) but with an additional \textsf{AND} gate (\emph{U3}). The inputs for that \textsf{AND} gate are set such that when \emph{Y0}-\emph{Y3} are $ 1010 $ then the gate will activate and reset all four flip-flops.

Figure \ref{tmg:09_12} is the timing diagram obtained from the decade counter illustrated in Figure \ref{fig:09_19}.

\begin{figure}[H]
	\centering
	\begin{tikztimingtable}[
		timing/slope=0,         % no slope
		timing/coldist=2pt,     % column distance
		xscale=1.0,yscale=1.0,  % scale diagrams
		semithick,               % set line width
		]
		\footnotesize \# & U     R 18{2Q} 2U     \\
		\footnotesize Clk & 36{C} \\
		%                      P 01 02 03 04 05 06 07 08 09 10 11 12 13 14 15 16
	 %17 18
		\footnotesize Y0 & [] {L HH LL HH LL HH LL HH LL HH LL HH LL HH LL HH LL HH LL} \\
		\footnotesize Y1 & [] {L LL HH HH LL LL HH HH LL LL LL HH HH LL LL HH HH LL LL} \\
		\footnotesize Y2 & [] {L LL LL LL HH HH HH HH LL LL LL LL LL HH HH HH HH LL LL} \\
		\footnotesize Y3 & [] {L LL LL LL LL LL LL LL HH HH LL LL LL LL LL LL LL HH HH} \\
		\extracode % Optional
		% \fulltablegrid[]
		\vertlines[]{19}
		\tablerules[]
	\end{tikztimingtable}
	\caption{4-Bit Decade Counter Timing Diagram} 
	\label{tmg:09_12}
\end{figure}

The timing diagram shows the count increasing with each clock pulse (for example, at $ 8 $ the outputs are $ 1000 $) until pulse number ten, when the bits reset to $ 0000 $ and the count starts over. A vertical line was added at count ten for reference.

\subsection{Up-Down Counters}
\label{SL:subsec:up_down_counters}

In this chapter both up and down counters have been considered; however counters can also be designed to count both up and down. These counters are, of course, more complex than the simple counters encountered so far and Figure \ref{fig:09_20} illustrates an up-down counter circuit.

\begin{figure}[H]
	\centering
	\includegraphics[width=\maxwidth{.95\linewidth}]{gfx/09_20}
	\caption{Up-Down Counter}
	\label{fig:09_20}
\end{figure}

When the \emph{Up\_Dn} bit is high the counter counts up but when that bit is low then the counter counts down. This counter is little more than a merging of the up counter in Figure \ref{fig:09_15} and the down counter in Figure \ref{fig:09_16}. \emph{U1}-\emph{U3} transmit the \emph{Q} outputs from each flip-flop to the next stage while \emph{U7}-\emph{U9} transmit the \emph{Q'} outputs from each flip-flop to the next stage. The E bit activates one of those two banks of \textsf{AND} gates.

No timing diagram is provided for this circuit since that diagram would be the same as for the up counter (Figure \ref{tmg:09_08} or the down counter (Figure \ref{tmg:09_09}), depending on the setting of the \emph{Up\_Dn} bit.

\subsection{Frequency Divider}
\label{SL:subsec:frequency_divider}

Often, a designer creates a system that may need various clock frequencies throughout its subsystems. In this case, it is desirable to have only one main pulse generator, but divide the frequency from that generator so other frequencies are available where they are needed. Also, by using a common clock source all of the frequencies are easier to synchronize when needed.

The circuit illustrated in Figure \ref{fig:09_21} is the same synchronous two-bit counter first considered in Figure \ref{fig:09_13}.

\begin{figure}[H]
	\centering
	\includegraphics[width=\maxwidth{.95\linewidth}]{gfx/09_21}
	\caption{Synchronous 2-Bit Counter}
	\label{fig:09_21}
\end{figure}

The circuit in Figure \ref{fig:09_21} produces the timing diagram seen in Figure \ref{tmg:09_13}
 
\begin{figure}[H]
  \centering
  \begin{tikztimingtable}[
    timing/slope=0,         % no slope
    timing/coldist=2pt,     % column distance
    xscale=2.0,yscale=1.0,  % scale diagrams
    semithick,               % set line width
    ]
    \footnotesize \# & U     R 8{2Q} 2U     \\
    \footnotesize Clk & 17{C} \\
    %                 P 01 02 03 04 05 06 07 08
    \footnotesize Y0 & [] {L HH LL HH LL HH LL HH LL} \\
    \footnotesize Y1 & [] {L HH HH LL LL HH HH LL LL} \\
    \extracode % Optional
    % \fulltablegrid[]
    % \vertlines[]{}
    \tablerules[]
  \end{tikztimingtable}
  \caption{Frequency Divider} 
  \label{tmg:09_13}
\end{figure}

Notice that \emph{Y0} is half of the clock's frequency and \emph{Y1} is one-quarter of the clock's frequency. If the circuit designer used Y1 as a timer for some subcircuit then that circuit would operate at one-quarter of the main clock frequency. It is possible to use only one output port from a modulus counter, like the decade counter found in Figure \ref{fig:09_19}, and get any sort of division of the main clock desired.

\subsection{Counter Integrated Circuits (IC)}
\label{SL:subsec:counter_integrated_circuits}

In practice, most designers do not build counter circuits since there are so many types already commercially available. When a counter is needed, a designer can select an appropriate counter from those available on the market. Here are just a few as examples of what is available: 

\begin{table}[H]
  \sffamily
  \newcommand{\head}[1]{\textcolor{white}{\textbf{#1}}}    
  \begin{center}
    \rowcolors{2}{gray!10}{white} % Color every other line a light gray
    \begin{tabular}{ll} 
      \rowcolor{black!75}
      \head{IC} & \head{Function} \\
      74x68 & dual four-bit decade counter \\
      74x69 & dual four-bit binary counter \\
      74x90 & decade counter (divide by $ 2 $ and divide by $ 5 $ sections) \\
      74x143 & decade counter/latch/decoder/seven-segment driver \\
      74x163 & synchronous four-bit binary counter \\
      74x168 & synchronous four-bit up/down decade counter \\
      74x177 & presettable binary counter/latch \\
      74x291 & four-bit universal shift register and up/down counter
    \end{tabular}
  \end{center}
  \caption{Counter IC's}
  \label{sl:tab:counter_ics}
\end{table}

\section{Memory}
\label{SL:sec:memory}

\subsection{Read-Only Memory}
\label{SL:subsec:read-only_memory}

\acf{ROM} is an \ac{IC} that contains tens-of-millions of registers (memory locations) to store information. Typically, \ac{ROM} stores microcode (like bootup routines) and computer settings that do not change. There are several types of \ac{ROM}: mask, programmable, and erasable. 

\begin{itemize}
  \item \textsc{Mask} \acp{ROM} are manufactured such that memory locations already filled with data and that cannot be altered by the end user. They are called ``mask'' \acp{ROM} since the manufacturing process includes applying a mask to the circuit as it is being created. 
  \item \textsc{\acp{PROM}} are a type of \ac{ROM} that can be programmed by the end user, but it cannot be altered after that initial programming. This permits a designer to distribute some sort of program on a chip that is designed to be installed in some device and then never changed.
  \item \textsc{Erasable \acp{PROM}} can be programmed by the end user and then be later altered. An example of where an erasable \ac{PROM} would be used is in a computer's \ac{BIOS} (the operating system that boots up a computer), where the user can alter certain ``persistent'' computer specifications, like the device boot order. 
\end{itemize}

Two major differences between \ac{ROM} and \ac{RAM} are 1) \ac{ROM}'s ability to hold data when the computer is powered off and 2) altering the data in \ac{RAM} is much faster than in an erasable \ac{PROM}.

\subsection{Random Access Memory}
\label{SL:subsec:random_access_memory}

\acf{RAM} is an integrated circuit that contains tens-of-millions of registers (memory locations) to store information. The \ac{RAM} \ac{IC} is designed to quickly store data found at its input port and then look-up and return stored data requested by the circuit. In a computer operation, \ac{RAM} contains both program code and user input (for example, the \emph{LibreOffice Writer} program along with whatever document the user is working on). There are two types of \ac{RAM}: dynamic and static. \ac{DRAM} stores bits in such a way that it must be ``refreshed'' (or ``strobed'') every few milliseconds. \ac{SRAM} uses flip-flops to store data so it does not need to be refreshed. One enhancement to \acp{DRAM} was to package several in a single \ac{IC} and then synchronize them so they act like a single larger memory; these are called \acp{SDRAM} and they are very popular for camera and cell phone memory.

Both \ac{ROM} and \ac{RAM} circuits use registers, flip-flops, and other components already considered in this chapter, but by the millions rather than just four or so at a time. There are no example circuits or timing diagrams for these devices in this book; however, the lab manual that accompanies this book includes activities for both \ac{ROM} and \ac{RAM} devices.


















\include{Chapters/10_Simulation}

% ****************************************************************
% Start of Part 3: Labs
% ****************************************************************
%\cleardoublepage
%\ctparttext{\textsc{Laboratory Exercises} are essential to understanding digital logic. While reading about the theory and practice provide important background, completing hands-on exercises make the lessons more understandable.In this part of the book, iVerilog laboratory exercises complete the exploration of digital logic with working examples of challenging circuits.}
%\part{Laboratory Exercises}
%\include{Chapters/105_Install}                          %Include
%\include{Chapters/110_Verilog_Primer}                   %Include
%\include{Chapters/115_Verilog_Testbench}                %Include
%\include{Chapters/120_GTK_Wave}                         %Include
%\include{Chapters/125_16-bit_4-1_Mux}                   %Include
%\include{Chapters/130_4-bit_Adder_With_CO}              %Include
%\include{Chapters/135_ALU}                              %Include
%\include{Chapters/140_ALU_With_Flags}                   %Include
%\include{Chapters/145_Up_Down_Counter}                  %Include
%\include{Chapters/150_Ring_Counter}                     %Include
%\include{Chapters/155_RAM}                              %Include
%\include{Chapters/160_FSM}                              %Include

%%%%%%%%%%%%%%%%%%%%%%%%%%%%%%%%%%%%%%%%%%%%%%%%%%
%%% 4X2 K-Map
%%%%%%%%%%%%%%%%%%%%%%%%%%%%%%%%%%%%%%%%%%%%%%%%%
\begin{figure}[H]
  \myfloatalign
  \begin{tikzpicture} [circuit logic US, scale=1.00]
  % make all path lines (the node shapes) a little thicker
  \tikzstyle{every path}=[line width=0.50mm]
  
  %********************************************************************
  % Adjust the settings below to display the 1's and rectangles
  %********************************************************************
  % Uncomment the appropriate lines below to insert ones where needed
%   \node[] at (2.25,2.25) {\Huge $ 1 $}; % 00
%   \node[] at (2.25,0.75) {\Huge $ 1 $}; % 01
%   \node[] at (3.75,2.25) {\Huge $ 1 $}; % 02
%   \node[] at (3.75,0.75) {\Huge $ 1 $}; % 03
%   \node[] at (6.75,2.25) {\Huge $ 1 $}; % 04
%   \node[] at (6.75,0.75) {\Huge $ 1 $}; % 05
%   \node[] at (5.25,2.25) {\Huge $ 1 $}; % 06
%   \node[] at (5.25,0.75) {\Huge $ 1 $}; % 07
  
  % The coords for each cell - this is used as the origin for the solution box
  \coordinate (cell00) at (1.5,1.5); \coordinate (cell01) at (1.5,0.0);
  \coordinate (cell02) at (3.0,1.5); \coordinate (cell03) at (3.0,0.0);
  
  \coordinate (cell04) at (6.0,1.5); \coordinate (cell05) at (6.0,0.0);
  \coordinate (cell06) at (4.5,1.5); \coordinate (cell07) at (4.5,0.0);

	% The colored boxes enclosing adjacent ones
	% Set the ``at'' to the lower-left cell of the rectangle using 
	% the 'cellxx' defined above
	% Set the minimum height/width to (number of cells) * 1.5. 
	% May have to decrease these by 0.1 to cut the rectangle 
	% just inside the cell.
	\node [draw,
	color=red!70!black,
	fill=red!20!white,
	fill opacity=0.3,
	minimum height=1.4cm,
	minimum width=3.0cm,
	double,
	rounded corners,
	anchor=south west] at (cell02) {};
	
	\node [draw,
	color=blue!70!black,
	fill=blue!20!white,
	fill opacity=0.3,
	minimum height=2.9cm,
	minimum width=1.5cm,
	double,
	rounded corners,
	anchor=south west] at (cell07) {};


    
  %********************************************************************
  % Shouldn't need to adjust anything below this point - this is just
  % the grid and the minterms.
  %********************************************************************  
  % Text in top-Left cell
  \node[] at (0.50,3.40) { $ \mathsf{ \mathbf{C} } $ }; % C
  \node[] at (1.10,4.05) { $ \mathsf{ \mathbf{AB} } $ }; % AB
  
  % Populate the top row header
  % In the following, the foreach lists a location/text pair
  % The the draw line draws the text at each location
  \foreach \loc/\txt in {
    (2.25,3.75)/{00}, (3.75,3.75)/{01},
    (5.25,3.75)/{11}, (6.75,3.75)/{10}
  }
  \draw \loc node{\Huge $\txt$};
  
  % Populate the header in column one
  \foreach \loc/\txt in { 
    (0.75,2.25)/{0},(0.75,0.75)/{1}
  }
  \draw \loc node{\Huge $\txt$};
  
  % Populate the minterms
  \foreach \loc/\txt in { 
    (2.75,1.75)/{00} , (4.25,1.75)/{02} , (5.75,1.75)/{06} , (7.25,1.75)/{04} ,
    (2.75,0.15)/{01} , (4.25,0.15)/{03} , (5.75,0.15)/{07} , (7.25,0.15)/{05} }
  \draw \loc node{ \color{blue!90!black} \small { $\txt$ }};
  
  % Draw the lines
  \draw
  % Finish drawing the grid
  [step=1.5cm,black,thin] (0,0) grid (7.5,4.5) % The Grid
  (0.0,4.5) -- (1.5,3.0) % Diagonal in the top left cell
  (1.5,3.10) -- (7.5,3.10) % Double line under top header row
  (1.40,0.0) -- (1.40,3.0) % Double line on left of header column one
  ;    
  \end{tikzpicture}
	\caption{K-Map 2x4}
	\label{kmap:08_03}
\end{figure}



%%%%%%%%%%%%%%%%%%%%%%%%%%%%%%%%%%%%%%%%%%%%%%%%%
%%% 4X4 K-Map
%%%%%%%%%%%%%%%%%%%%%%%%%%%%%%%%%%%%%%%%%%%%%%%%%
\begin{figure}[H]
	\myfloatalign
	\begin{tikzpicture} [circuit logic US, scale=1.00]
	% make all path lines (the node shapes) a little thicker
	\tikzstyle{every path}=[line width=0.50mm]
	
	%********************************************************************
	% Adjust the settings below to display the 1's and rectangles
	%********************************************************************
	% Uncomment the appropriate lines below to insert ones where needed
	% Data Row 1
	% \node[] at (2.25,5.25) {\Huge $ 1 $}; % 00
	% \node[] at (3.75,5.25) {\Huge $ 1 $}; % 04
	% \node[] at (5.25,5.25) {\Huge $ 1 $}; % 12
	% \node[] at (6.75,5.25) {\Huge $ 1 $}; % 08
	% Data Row 2
	% \node[] at (2.25,3.75) {\Huge $ 1 $}; % 01
	% \node[] at (3.75,3.75) {\Huge $ 1 $}; % 05
	\node[] at (5.25,3.75) {\Huge $ 1 $}; % 13
	% \node[] at (6.75,3.75) {\Huge $ 1 $}; % 09
	% Data Row 3
	% \node[] at (2.25,2.25) {\Huge $ 1 $}; % 03
	% \node[] at (3.75,2.25) {\Huge $ 1 $}; % 07
	\node[] at (5.25,2.25) {\Huge $ 1 $}; % 15
	\node[] at (6.75,2.25) {\Huge $ 1 $}; % 11
	% Data Row 4
	% \node[] at (2.25,0.75) {\Huge $ 1 $}; % 02
	% \node[] at (3.75,0.75) {\Huge $ 1 $}; % 06
	% \node[] at (5.25,0.75) {\Huge $ 1 $}; % 14
	% \node[] at (6.75,0.75) {\Huge $ 1 $}; % 10
	
	% The coords for each cell - this is used to start the rectangle box
	\coordinate (cell00) at (1.50,4.50); \coordinate (cell01) at (1.50,3.00);
	\coordinate (cell02) at (1.50,0.00); \coordinate (cell03) at (1.50,1.50);
	\coordinate (cell04) at (3.00,4.50); \coordinate (cell05) at (3.00,3.00);
	\coordinate (cell06) at (3.00,0.00); \coordinate (cell07) at (3.00,1.50);
	\coordinate (cell08) at (6.00,4.50); \coordinate (cell09) at (6.00,3.00);
	\coordinate (cell10) at (6.00,0.00); \coordinate (cell11) at (6.00,1.50);
	\coordinate (cell12) at (4.50,4.50); \coordinate (cell13) at (4.50,3.00);
	\coordinate (cell14) at (4.50,0.00); \coordinate (cell15) at (4.50,1.50);
	
	% Set the ``at'' to the lower-left cell of the rectangle using 
	% the 'cellxx' defined above
	% Set the minimum height/width to (number of cells) * 1.5. 
	% May have to decrease these by 0.1 to cut the rectangle 
	% just inside the cell.
	\node [draw,
	color=red!70!black,
	fill=red!20!white,
	fill opacity=0.3,
	minimum height=1.4cm,
	minimum width=3.0cm,
	double,
	rounded corners,
	anchor=south west] at (cell15) {};
	
	\node [draw,
	color=blue!70!black,
	fill=blue!20!white,
	fill opacity=0.3,
	minimum height=2.9cm,
	minimum width=1.5cm,
	double,
	rounded corners,
	anchor=south west] at (cell15) {};
	
	%********************************************************************
	% Shouldn't need to adjust anything below this point - this is just
	% the grid and the minterms.
	%********************************************************************	
	% Text in top-Left cell
	\node[] at (0.55,6.35) { $ \mathsf{ \mathbf{CD} } $ }; % CD
	\node[] at (1.05,7.05) { $ \mathsf{ \mathbf{AB} } $ }; % AB
	
	% Populate the top row header
	% In the following, the foreach lists a location/text pair
	% The the draw line draws the text at each location
	\foreach \loc/\txt in {(2.25,6.75)/{00},(3.75,6.75)/{01},(5.25,6.75)/{11},(6.75,6.75)/{10}}
	\draw \loc node{\Huge $\txt$};
	
	% Populate the header in column one
	\foreach \loc/\txt in {(0.75,5.25)/{00},(0.75,3.75)/{01},(0.75,2.25)/{11},(0.75,0.75)/{10}}
	\draw \loc node{\Huge $\txt$};
	
	% Populate the minterms
	\foreach \loc/\txt in { (2.75,4.75)/{00} , (4.25,4.75)/{04} , (5.75,4.75)/{12} , (7.25,4.75)/{08} ,
		(2.75,3.25)/{01} , (4.25,3.25)/{05} , (5.75,3.25)/{13} , (7.25,3.25)/{09} ,
		(2.75,1.75)/{03} , (4.25,1.75)/{07} , (5.75,1.75)/{15} , (7.25,1.75)/{11} ,
		(2.75,0.25)/{02} , (4.25,0.25)/{06} , (5.75,0.25)/{14} , (7.25,0.25)/{10} }
	\draw \loc node{ \color{blue!90!black} \small{ $\txt$ }};
	
	% Draw the lines
	\draw
	% Finish drawing the grid
	[step=1.5cm,black,thin] (0,0) grid (7.5,7.5) % The Grid
	(0.0,7.5) -- (1.5,6.0) % Diagonal in the top left cell
	(1.5,6.10) -- (7.50,6.10) % Double line under top header row
	(1.40,0.0) -- (1.40,6.0) % Double line on left of header column one
	;
	\end{tikzpicture}
	\caption{K-Map 4x4}
	\label{08:fig:kmap01}
\end{figure}


% ****************************************************************
% Backmatter
% ****************************************************************
\setcounter{chapter}{0}
\renewcommand{\thechapter}{\alph{chapter}}
\cleardoublepage
\part{Appendix}
%********************************************************************
% Appendix
%*******************************************************
% If problems with the headers: get headings in appendix etc. right
%\markboth{\spacedlowsmallcaps{Appendix}}{\spacedlowsmallcaps{Appendix}}

%********************************************************************
% Summary of the various Boolean properties and functions (from Chapter 3)
%*******************************************************
\chapter{Boolean Properties and Functions}
\label{ap:ch:boolean_properties_and_functions}

\begin{tabular}{ccc}

	% AND Truth Table
	\begin{tabular}{ccc} 
		\multicolumn{3}{c}{\emph{AND Truth Table}} \\
		\multicolumn{2}{c}{Inputs} & {Output} \\
		A & B & Y \\
		\hline
		0 & 0 & 0 \\
		0 & 1 & 0 \\
		1 & 0 & 0 \\
		1 & 1 & 1 
	\end{tabular}
&
	% OR Truth Table
	\begin{tabular}{ccc} 
		\multicolumn{3}{c}{\emph{OR Truth Table}} \\
		\multicolumn{2}{c}{Inputs} & {Output} \\
		A & B & Y \\
		\hline
		0 & 0 & 0 \\
		0 & 1 & 1 \\
		1 & 0 & 1 \\
		1 & 1 & 1 
	\end{tabular}
&
	% XOR Truth Table
	\begin{tabular}{ccc} 
		\multicolumn{3}{c}{\emph{XOR Truth Table}} \\
		\multicolumn{2}{c}{Inputs} & {Output} \\
		A & B & Y \\
		\hline
		0 & 0 & 0 \\
		0 & 1 & 1 \\
		1 & 0 & 1 \\
		1 & 1 & 0 
	\end{tabular}
\\
\\
	% NAND Truth Table
	\begin{tabular}{ccc} 
		\multicolumn{3}{c}{\emph{NAND Truth Table}} \\
		\multicolumn{2}{c}{Inputs} & {Output} \\
		A & B & Y \\
		\hline
		0 & 0 & 1 \\
		0 & 1 & 1 \\
		1 & 0 & 1 \\
		1 & 1 & 0 
	\end{tabular}
&
	% NOR Truth Table
	\begin{tabular}{ccc} 
		\multicolumn{3}{c}{\emph{NOR Truth Table}} \\
		\multicolumn{2}{c}{Inputs} & {Output} \\
		A & B & Y \\
		\hline
		0 & 0 & 1 \\
		0 & 1 & 0 \\
		1 & 0 & 0 \\
		1 & 1 & 0 
	\end{tabular}
&
	% XNOR Truth Table
	\begin{tabular}{ccc} 
		\multicolumn{3}{c}{\emph{XNOR Truth Table}} \\
		\multicolumn{2}{c}{Inputs} & {Output} \\
		A & B & Y \\
		\hline
		0 & 0 & 1 \\
		0 & 1 & 0 \\
		1 & 0 & 0 \\
		1 & 1 & 1 
	\end{tabular}
\\
\\
	% NOT Truth Table
	\begin{tabular}{ccc} 
		\multicolumn{3}{c}{\emph{NOT Truth Table}} \\
		Input & Output \\
		\hline
		0 & 1 \\
		1 & 0 \\
	\end{tabular}
&
	% Buffer Truth Table
	\begin{tabular}{ccc} 
		\multicolumn{3}{c}{\emph{Buffer Truth Table}} \\
		{Input} & {Output} \\
		\hline
		0 & 0 \\
		1 & 1 
	\end{tabular}
&
\end{tabular}

\hrulefill

%******************************************************
% Univariate Properties
%******************************************************
\begin{table}[ht]
	\caption{Univariate Properties}
	\label{ap:tab:univeriate_properties}
	\sffamily
	\newcommand{\head}[1]{\textcolor{white}{\textbf{#1}}}		
	\begin{center}
		\rowcolors{2}{gray!10}{white} % Color every other line a light gray
		\begin{tabular}{ccc} 
			\rowcolor{black!75}
			{\head{Property}} & \head{OR} & \head{AND} \\
			Identity & $ A + 0 = A $ & $ 1A = A $ \\
			Idempotence & $ A + A = A $ & $ AA = A $ \\
			Annihilator & $ A + 1 = 1 $ & $ 0A = 0 $ \\
			Complement & $ A + A' = 1 $ & $ AA' = 0 $ \\
			Involution & \multicolumn{2}{c}{$ (A')' = A $} 
		\end{tabular}
	\end{center}
\end{table}

%******************************************************
% Multivariate Properties
%******************************************************
\begin{table}[ht]
	\caption{Multivariate Properties}
	\label{ap:tab:multivariate_properties}
	\sffamily
	\newcommand{\head}[1]{\textcolor{white}{\textbf{#1}}}		
	\begin{center}
		\rowcolors{2}{gray!10}{white} % Color every other line a light gray
		\begin{tabular}{ccc} 
			\rowcolor{black!75}
			{\head{Property}} & \head{OR} & \head{AND} \\
			Commutative & $ A + B = B + A $ & $ AB = BA $ \\
			Associative & $ (A + B) + C = A + (B + C) $ & $ (AB)C = A(BC) $ \\
			Distributive & $ A + (BC) = (A+B)(A+C) $ & $ A(B+C) = AB + AC $ \\
			Absorption & $ A + AB = A $ & $ A(A+B) = A $ \\
			DeMorgan & $ \overline{A+B}=\overline{A}\,\overline{B} $ 
				& $ \overline{AB}=\overline{A}+\overline{B} $ \\
			Adjacency & \multicolumn{2}{c}{$ AB + AB' = A $} 
		\end{tabular}
	\end{center}
\end{table}

\begin{table}[ht]
	\caption{Boolean Functions}
	\label{ap:tab:boolean_functions}
	\sffamily
	\newcommand{\head}[1]{\textcolor{white}{\textbf{#1}}}		
	\begin{center}
		\rowcolors{2}{gray!10}{white} % Color every other line a light gray
		\begin{tabular}{llllll}
			\textbf{A} & 0 & 0 & 1 & 1 &  \\ 
			\textbf{B} & 0 & 1 & 0 & 1 &  \\ \hline
			$ F_{0} $ & 0 & 0 & 0 & 0 
			& Zero or Clear. Always zero (Annihilation) \\ 
			$ F_{1} $ & 0 & 0 & 0 & 1 
			& Logical AND: $ A * B $  \\ 
			$ F_{2} $ & 0 & 0 & 1 & 0 
			& Inhibition: $ AB' $ or $ A>B $ \\ 
			$ F_{3} $ & 0 & 0 & 1 & 1 
			& Transfer A to Output, Ignore B \\ 
			$ F_{4} $ & 0 & 1 & 0 & 0 
			& Inhibition: $ A'B $ or $ B>A $ \\ 
			$ F_{5} $ & 0 & 1 & 0 & 1 
			& Transfer B to Output, Ignore A \\ 
			$ F_{6} $ & 0 & 1 & 1 & 0 
			& Exclusive Or (XOR): $ A \oplus B $ \\ 
			$ F_{7} $ & 0 & 1 & 1 & 1 
			& Logical OR: $ A + B $ \\ 
			$ F_{8} $ & 1 & 0 & 0 & 0 
			& Logical NOR: $ (A + B)' $ \\ 
			$ F_{9} $ & 1 & 0 & 0 & 1 
			& Equivalence: $ (A = B)' $ \\ 
			$ F_{10} $ & 1 & 0 & 1 & 0 
			& Not B and ignore A, B Complement \\ 
			$ F_{11} $ & 1 & 0 & 1 & 1 
			& Implication, $ A + B' $, $ B >= A $ \\ 
			$ F_{12} $ & 1 & 1 & 0 & 0 
			& Not A and ignore B, A Complement \\ 
			$ F_{13} $ & 1 & 1 & 0 & 1 
			& Implication, $ A' + B $, $ A >= B $ \\ 
			$ F_{14} $ & 1 & 1 & 1 & 0 
			& Logical NAND: $ (A*B)' $ \\ 
			$ F_{15} $ & 1 & 1 & 1 & 1 
			& One or Set. Always one (Identity) \\ 
		\end{tabular} 
	\end{center}
\end{table}

%*******************************************************
% Resources
%*******************************************************
%\chapter{Resources}
%\label{ap:ch:resources}
%
%The following materials were used as resources for this book.
%
%\section{Digital Logic}
%\subsection{Books}
%
%\begin{enumerate}
%  \item Gregg, John, 1998, Ones and zeros. New York : IEEE Press.
%  \item Holdsworth, B and Woods, R. C, 2002, Digital logic design. Oxford : Newnes.
%  \item Langholz, Gideon, Kandel, Abraham and Mott, Joe L, 1998, Foundations of digital logic design. Singapore : World Scientific.
%  \item M.Rafiquzzaman., 2005, Fundamentals of Digital Logic and Microcomputer Design, 5th Edition. John Wiley \& Sons.
%  \item Mano, M. Morris and Ciletti, Michael D, 2013, Digital design. Upper Saddle River, NJ : Pearson Prentice Hall.
%  \item Maxfield, Clive Max, Maxfield, Clive Max and Maxfield, Clive Max, 1998, Designus Maximus unleashed!. Boston : Newnes.
%  \item Maxfield, Clive, 2003, Bebop to the Boolean boogie. Amsterdam : Newnes.
%  \item Nisan, Noam and Schocken, Shimon, [no date], The elements of computing systems.
%  \item Petzold, Charles, 1999, Code. Redmond, Wash. : Microsoft Press.
%  \item Predko, Michael, 2005, Digital electronics demystified. New York : McGraw-Hill.
%  \item Saha, A and Manna, N, 2007, Digital principles and logic design. Hingham, Mass. : Infinity Science Press.
%  \item Tokheim, Roger L, 1999, Digital electronics. New York : Glencoe/McGraw-Hill.
%  \item Yarbrough, John M, 1997, Digital logic. Minneapolis/St. Paul : West Pub. Co.
%\end{enumerate}
%
%\subsection{Found Online}
%
%\begin{enumerate}
%  \item Cummings, Clifford, 2002, The Fundamentals of Efficient Synthesizable Finite State Machine Design using NC-Verilog and BuildGates [online]. 1. Sunburst Design, Inc. [Accessed 12  April  2016]. Available from: \url{http://www.sunburst-design.com/papers/CummingsICU2002\_FSMFundamentals.pdf}
%  \item de Pablo, S., Cebrian, J. A., Herrero, L.C. and Rey, A.B., 2016, A very simple 8-bit RISC processor for FPGA [online]. 1. [Accessed 12  April  2016]. Available from: \url{http://www.dte.eis.uva.es/Datos/Congresos/FPGAworld2006a.pdf}
%  \item Digital Circuits - Wikibooks, open books for an open world, 2016. En.wikibooks.org [online]
%  \item Gray, Jan, 2000, Designing a Simple FPGA-Optimized RISC CPU and System-on-a-Chip [online]. 1. Gray Research, LLC. [Accessed 12  April  2016]. Available from: \url{http://www.fpgacpu.org/papers/soc-gr0040-paper.pdf}
%  \item Kaur, Ramandeep, 2016, 8 Bit RISC Processor Using Verilog HDL [online]. 1. Anuj el al Int. Journal of Engineering Research and Applications. [Accessed 12  April  2016]. Available from: \url{http://www.ijera.com/papers/Vol4\_issue3/Version\%201/BV4301417422.pdf}
%  \item Liu, Jianming, Zhang, Yunjie, Xu, Lili and Liu, Pengtao, 2013, [online]. 1. 2nd International Conference on Computer Science and Electronics Engineering. [Accessed 12  April  2016]. Available from: \url{http://www.atlantis-press.com/php/pub.php?publication=iccsee-13\&frame=http\%3A//www.atlantis-press.com/php/paper-details.php\%3Fid\%3D4893}
%  \item Nyasulu, Peter and Knight, J., 2003, Introduction to Verilog [online]. 1. Carleton University. [Accessed 12  April  2016]. Available from: \url{http://www.doe.carleton.ca/~jknight/97.478/97.478\_02F/PetervrlQ.pdf}
%  \item Phelps, Andrew, 2006, Constructing an Error Correcting Code [online]. 1. Madison : University of Wisconsin. [Accessed 12  April  2016]. Available from: \url{http://pages.cs.wisc.edu/~markhill/cs552/Fall2006/handouts/ConstructingECC.pdf}
%  \item Programmable Logic - Wikibooks, open books for an open world, 2016. En.wikibooks.org [online]
%  \item User Guide, 2016. Icarus Verilog [online]
%  \item Verilog Tutorial -Table of Contents: ElectroSofts.com, 2016. Electrosofts.com [online]
%  \item Welcome To Verilog Page, 2016. Asic-world.com [online]
%\end{enumerate}


%*******************************************************
% New Appendix Here
%*******************************************************
%\chapter{New Appendix Here}
%\label{ap:ch:new_appendix_here}


% ****************************************************************
% Other Stuff in the Back
% ****************************************************************
\cleardoublepage\include{FrontBackmatter/Colophon} %\cleardoublepage%*******************************************************
% Declaration
%*******************************************************
\refstepcounter{dummy}
\pdfbookmark[0]{Style Guide}{styleguide}
\chapter*{Style Guide}
\thispagestyle{empty}

\begin{itemize}

  \item \textsc{Numbers}. 
  
  \begin{itemize}
    \item All numbers must be enclosed in a math block.
    \item Spell out numbers up to ten; thus, not ``1s,'' but ones.
    \item Larger numbers with a suffix, like ``s,'' do not use an apostrophe: not ten's, but tens.
  \end{itemize}

  \item \textsc{Variables}. 
  
  \begin{itemize}
    \item \textsc{Single-Digit}. Single digit variable names, like ``Q,'' are enclosed in a math block.
    \item \textsc{Long}. Long variable names, like \emph{i\_clk}, are enclosed in an \textbackslash emph\{\} block.
  \end{itemize}

  \item \textsc{Figures and Tables}.
  
  \begin{itemize}
    \item Figures and Tables use an [H] specification
    \item Captions go at the end of the table block so it is printed under the table to match figures and listings. (Figures and Listings place the caption under by default)
  \end{itemize}
  
  \item \textsc{Inline Expressions}. Use \lstinline[columns=fixed]|\lstinline[columns=fixed]|foo|| for in-line Boolean expressions, equations, and Verilog commands.

  \item \textsc{Verilog Commands}. Use all caps and  \lstinline[columns=fixed]|\lstinline[columns=fixed]|foo|| for Verilog commands (like CASE).

  \item \textsc{Lists}. Items in a list should have an introductory phrase using title capitalization and set in small caps. If a long variable name or Verilog command starts a list item then use the formatting appropriate for that type of item rather than a list item.
  
  \item \textsc{Gates}. Use San Seriff font and ALL CAPS for gates (like \textsf{AND} or \textsf{OR})
  
  \item \textsc{True/False}. The words \emph{True} and \emph{False} are capitalized and in an \lstinline[columns=fixed]|\emph{}| block.

\item The word: \textit{Logisim-evolution}
\item Inputs/Outputs/Single Bits: \emph{D2In} (Ctrl-Sh-E)
\item Ports: input data ports are labeled with a single letter from the beginning of the alphabet, output data ports are labeled with a single letter from the end of the alphabet, and other signals use descriptive labels
\item ICs and Gates: \textsf{OR} (Ctrl-Sh-A)
\item The word ``Flip-Flop'' is hyphenated and each part is capitalized in a title.


\end{itemize}

\section{Captions and Labels}
%  All lables are the same as the captions, but lowercase and no punctuation. 
%  Also, all spaces are filled with underscores.
%  Occasionally, labels may also need a bit of verbiage to differentiate 
%  them from each other, for example, there will be lots of ``example'' 
%  sections. In that case, put additional verbiage at the end of the label.
%  Labels are referenced book-wide, not just chapter-wide. So prefix 
%  the 2-digit chapter number at the start of the label (Appendicies 
%  will use ``ap''). Also, use the following label designations:
%	ch: -- chapter
%	sec: -- section
%	subsec: -- sub-section
%	para: -- paragraph
%	tab: -- table
%	eq: -- equation
%	fig: -- figure
%	lst: -- code listing
%	soln: -- the solution to an equation
%	A full label would look something like: \label{03:tab:truth_table_for_or}
%  The number of degrees for label position within an image node are:
%	E=0
%	NE=45
%	N=90
%	NW=135
%	W=180
%	SW=225
%	S=270
%	SE=315



% ****************************************************************
% This was used to create quiz questions that include things like
% math and graphics. Those can then be copy/pasted into Respondus.
% This would only need to be included for generating that one PDF file.
% ****************************************************************
%\include{Chapters/98_Quiz_Questions}

\blankpage
\blankpage
\blankpage
\blankpage

\end{document}
